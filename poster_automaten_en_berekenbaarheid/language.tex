alfa/LANG/4.5/2.5/8 cm/Alfabet/{$\Sigma$}/{Een alfabet $\Sigma$ is een set karakters die relevant zijn in een gegeven taal $L$.},
stri/LANG/4.5/5/8 cm/String/{$s\in\Sigma^*$}/{Een string over een alfabet $\Sigma$ is een sequentie van 0, 1 of meer karakters uit $\Sigma$. $\Sigma^*$ beschrijft de verzameling van alle strings over $\Sigma$},
lang/LANG/4.5/8/8 cm/Taal/{$L\subseteq\Sigma^*$}/{Een taal over een alfabet $\Sigma$ is een verzameling strings over $\Sigma$.},
conc/LANG/12.5/2.5/6 cm/Concatenatie van talen/{$L_1L_2$}/{Gegeven twee talen $L_1$ en $L_2$ over eenzelfde alfabet $\Sigma$, dan is de concatenatie $L_1L_2=\accl{xy|x\in L_1,y\in L_2}$. $L^n$ is de concatenatie van $n$ keer de taal $L$.},
klee/LANG/12.5/7/6 cm/Kleene-ster/{$L^*$}/{De Kleene-ster van een taal $L$ is de unie van nul, \'e\'en of meer concatenaties van de taal: $L^*=\displaystyle\cup_{i=0}^{\infty}{L^i}$.},
rege/REGEX/5.5/2.5/10 cm/Reguliere expressie over een alfabet $\Sigma$/RE/{$E$ is een reguliere expressie over $\Sigma$ indien $E$ van de vorm is:\begin{itemize}
 \item $\epsilon$
 \item $\phi$
 \item $a$ met $a\in\Sigma$
 \item $\brak{E_1E_2}$ waarbij $E_1$ en $E_2$ reguliere expressies zijn over $\Sigma$.
 \item $\brak{E_1|E_2}$ waarbij $E_1$ en $E_2$ reguliere expressies zijn over $\Sigma$.
 \item $\brak{E_1}^*$ waarbij $E_1$ een reguliere expressie is over $\Sigma$.
\end{itemize}},
rela/REGEX/18/1.5/14 cm/Een reguliere expressie bepaalt een taal/$\mbox{RE}\sim L_{\mbox{RE}}$/{Een reguliere expressie $E$ bepaalt een taal $L_E$ over hetzelfde alfabet $\Sigma$ als volgt:\begin{itemize}
 \item Als $E=\epsilon$, dan is $L_E=\accl{\epsilon}$
 \item Als $E=\phi$, dan is $L_E=\emptyset$
 \item Als $E=a$ met $a\in\Sigma$, dan is $L_E=\accl{a}$
 \item $\brak{E_1E_2}$ waarbij $E_1$ en $E_2$ reguliere expressies zijn over $\Sigma$, dan is $L_E=L_1L_2$.
 \item $\brak{E_1|E_2}$ waarbij $E_1$ en $E_2$ reguliere expressies zijn over $\Sigma$, dan is $L_E=L_1\cup L_2$.
 \item $\brak{E_1}^*$ waarbij $E_1$ een reguliere expressie is over $\Sigma$, dan is $L_E=L_{E_1}^*$.
\end{itemize}},
regl/REGEX/5.5/7.75/10 cm/Reguliere taal/$\mbox{RegLan}$/{Een taal die door een reguliere expressie bepaald wordt is een reguliere taal. De verzameling van reguliere talen duiden we aan met $\mbox{RegLan}$.},
nfa/REGEX/35/1.2/10 cm/Niet-deterministische eindige toestandsautomaat/NFA/{Een niet-deterministische eindige toestandsautomaat is een $5$-tal $\tupl{Q,\Sigma,\delta,q_s,F}$ waarbij:\begin{itemize}
 \item $Q$ een eindige verzameling toestanden is;
 \item $\Sigma$ een eindig alfabet;
 \item $\delta:Q\times\Sigma_{\epsilon}\rightarrow\powset{Q}$ de overgangsfunctie van de automaat;
 \item $q_s\in Q$ de starttoestand en een element van $Q$;
 \item $F\subseteq Q$ de verzameling eindtoestanden.
\end{itemize}},
nfaeq/REGEX/43.5/1.2/6 cm/Equivalentie van twee NFA's/$\mbox{NFA}\equiv\mbox{NFA}$/{Twee NFA's worden equivalent genoemd als ze dezelfde taal bepalen.},
nfastr/REGEX/35/6.15/10 cm/String aanvaard door een NFA/$s\in L_{\mbox{\tiny NFA}}$/{Een string $s$ wordt aanvaard door een NFA $\tupl{Q,\Sigma,\delta,q_s,F}$ indien $s$ kan geschreven worden als $a_1a_2\ldots a_n$ met $a_i\in\Sigma_{\epsilon}$, en er een rij toestanden $t_1t_2\ldots,t_{n+1}$ bestaat zodat:\begin{itemize}
 \item $t_1=q_s$;
 \item $t_{i+1}\in\fun{\delta}{t_i,a_i}$;
 \item $t_{n+1}\in F$.
\end{itemize}},
gnfa/REGEX/55/1.2/10 cm/Gegeneraliseerde niet-deterministische eindige toestandsautomaat/GNFA/{Een gegeneraliseerde niet-deterministische eindige toestandsautomaat is een NFA $\tupl{Q,\Sigma,\delta,q_s,F}$ met:\begin{itemize}
 \item slechts \'e\'en eindtoestand $F=\accl{q_f}$ zodat $q_f\neq q_s$;
 \item de bogen bevatten een reguliere expressie als label;
 \item $\forall q_i,q_j\in Q\setminus\accl{q_s,q_f}:\exists c\in\funm{RegExp}{\Sigma}:\fun{\delta}{q_i,c}=q_j$;
 \item $\forall q_i\in Q\setminus\accl{q_s,q_f}:\exists c\in\funm{RegExp}{\Sigma}:\fun{\delta}{q_s,c}=q_j$;
 \item $\forall q_i\in Q\setminus\accl{q_s,q_f}:\exists c\in\funm{RegExp}{\Sigma}:\fun{\delta}{q_i,c}=q_f$.
\end{itemize}},
cfgr/CTXFR/5.5/2.5/10 cm/Contextvrije grammatica over een alfabet $\Sigma$/CFG/{$\tupl{V,\Sigma,R,S}$ is een contextvrije grammatica over $\Sigma$ indien:\begin{itemize}
 \item $V$ een eindige verzameling niet-eindsymbolen is;
 \item $\Sigma$ een eindig alfabet van eindsymbolen disjunct met $V$;
 \item $R$ een eindige verzameling regels; een regel is een koppel van  ́\'e\'en niet-eindsymbool en een string van elementen uit $V\cup\Sigma_{\epsilon}$; we schrijven de twee delen van zulk een koppel met een $\rightarrow$ ertussen;
 \item $S$ is het startsymbool en behoort tot $V$.
\end{itemize}},
cfgder/CTXFR/18/2.5/14 cm/Afleiding met behulp van een contextvrije grammatica/$b\Rightarrow_{CFG}^{\star}f$/{Gegeven een contextvrije grammatica $\tupl{V,\Sigma,R,S}$. Een string $f$ over $V\cup\Sigma_{\epsilon}$ wordt afgeleid uit een string $b$ uit $V\cup\Sigma_{\epsilon}$ met behulp van de contextvrije grammatica indien er een eindige rij strings $s_0,s_1,\ldots,s_n$ bestaat zodat:\begin{itemize}
 \item $s_0=b\wedge s_n=f$
 \item $s_{i+1}$ verkregen wordt uit $s_i$ (voor $i<n$) door in $s_i$ een niet-eindsymbool $X$ te vervangen door de rechterkant van een regel waarin $X$ links voorkomt.
\end{itemize}
We noteren $s_i\Rightarrow s_{i+1}$ en $b\rightarrow^{\star}f$},
cfgla/CTXFR/18/7.5/14 cm/Een contextvrije grammatica bepaalt een taal/$\mbox{CFG}\sim L_{\mbox{CFG}}$/{De taal $L_{CFG}$ bepaald door een contextvrije grammatica $\tupl{V,\Sigma,R,S}$ is de verzameling strings over $\Sigma$ die kunnen afgeleid worden van het startsymbool $S$; of meer formeel: $L_{CFG}=\accl{s\in\Sigma^{\star}|S\Rightarrow^{\star}s}$.},
cfgl/CTXFR/5.5/7.75/10 cm/Contextvrije taal/$\mbox{CFL}$/{Een taal $L$ is contextvrij indien er een contextvrije grammatica is zodat $L=L_{CFG}$. De verzameling van contextvrije talen duiden we aan met $\mbox{CFL}$.},
pda/CTXFR/34/1.2/12 cm/Push-down automaat/PDA/{Een push-down automaat is een $5$-tal $\tupl{Q,\Sigma,\Gamma,\delta,q_s,F}$ waarbij:\begin{itemize}
 \item $Q$ een eindige verzameling toestanden is;
 \item $\Sigma$ een eindig inputalfabet;
 \item $\Gamma$ een eindig stapelalfabet;
 \item $\delta:Q\times\Sigma_{\epsilon}\times\Gamma_{\epsilon}\rightarrow\powset{Q\times\Gamma_{\epsilon}}$ de overgangsfunctie;
 \item $q_s\in Q$ de starttoestand en een element van $Q$;
 \item $F\subseteq Q$ de verzameling eindtoestanden.
\end{itemize}},
pdastr/CTXFR/34/6.15/12 cm/String aanvaard door een PDA/$s\in L_{\mbox{\tiny PDA}}$/{Een string $s$ wordt aanvaard door een PDA $\tupl{Q,\Sigma,\delta,q_s,F}$ indien $s$ kan worden opgesplitst in
delen $w_i,i=1\ldots m$ ($w_i\in\Sigma$), er toestanden $q_j,j=0\ldots m$ ($q_j\in Q$) zijn, en stapels $\mbox{stapel}_k,k=0\ldots m$ ($\mbox{stapel}_k\in\Gamma^{\star}$) zodat:\begin{itemize}
 \item $\mbox{stapel}_0=\epsilon\wedge q_0=q_s\wedge q_m\in F$;
 \item $\tupl{q_{i+1},y}\in\fun{\delta}{q_i,w_{i+1},x}$ waarbij $x,y\in\Gamma_{\epsilon}$ en $\mbox{stapel}_i=xt$, $\mbox{stapel}_{i+1}=yt$ met $t\in\Gamma^{\star}$.
\end{itemize}},
pdaeqcfg/CTXFR/43.5/1.2/6 cm/Equivalentie van CFG en PDA/$\mbox{CFG}\equiv\mbox{PDA}$/{Elke push-down automaat bepaalt een contextvrije taal en elke contextvrije taal wordt bepaald door een push-down automaat.},
lincfl/CTXFR/46/4.7/10 cm/Pompend lemma voor contextvrije talen/$L\in\mbox{CFL}$/{Voor een contextvrije taal $L$ bestaat een getal $d$ (de pomplengte) zodat elke string $s\in L$ met lengte minstens $d$ kan opgedeeld worden in $5$ stukken $u,v,x,y,z\in\Sigma^{\star}$ zodat $s=uvxyz$ en:
\begin{enumerate}
 \item $\forall i\geq 0:uv^ixy^iz\in L$;
 \item $\abs{vy}>0$;
 \item $\abs{vxy}\leq d$.
\end{enumerate}},
csgr/CTXSE/5.5/2.5/10 cm/Context-sensitieve grammatica over een alfabet $\Sigma$/CSG/{Een tuple $\tupl{V,\Sigma,R,S}$ is een context-sensitieve grammatica over $\Sigma$ indien:\begin{itemize}
 \item $V$ een eindige verzameling niet-eindsymbolen is;
 \item $\Sigma$ een eindig alfabet van eindsymbolen disjunct met $V$;
 \item $R$ een eindige verzameling regels; een regel is een koppel van $\alpha A\beta\rightarrow\alpha\gamma\beta$ ́met $\alpha,\beta,\gamma\in\brak{\Sigma\cup V}^{\star}$ en $A\in V$;
 \item $S$ is het startsymbool en behoort tot $V$.
\end{itemize}},
csgl/CTXSE/5.5/7.75/10 cm/Context-sensitieve taal/$\mbox{CSL}$/{Een taal $L$ is context-sensitief indien er een context-sensitieve grammatica is zodat $L=L_{CSG}$. De verzameling van contextvrije talen duiden we aan met $\mbox{CSL}$.},
lba/CTXSE/34/1.2/10 cm/Lineair begrensde automaat/LBA/{Een lineair begrensde automaat is een Turingmachine die niet leest of schrijft buiten het deel van de band dat initieel invoer bevat.},
lbaeqcsg/CTXSE/43.5/1.2/6 cm/Equivalentie van CSG en LBA/$\mbox{CSG}\equiv\mbox{LBA}$/{Elke lineair begrensde automaat bepaalt een context-sensitieve taal en elke context-sensitieve taal wordt bepaald door een lineair begrensde automaat.},
tuma/TURING/5.5/2.5/10 cm/Turingmachine/TM/{Een Turingmachine is een tuple $\tupl{Q,\Sigma,\Gamma,\delta,q_s,q_a,q_r}$ waarbij:\begin{itemize}
 \item $Q$ een eindige verzameling toestanden is;
 \item $\Sigma$ een invoeralfabet dat niet $\#$ bevat;
 \item $\Gamma$ een tape alfabet met $\#\in\Gamma$ en $\Sigma\subsetneq\Gamma$;
 \item $q_s$ de start-toestand, $q_a$ de accepterende eindtoestand, $q_r\neq q_a$ de verwerpende eindtoestand.
 \item $\funsig{\delta}{Q\times\Gamma}{Q\times\Gamma\times\acc{L,S,R}}$ de totale transitiefunctie.
\end{itemize}},
sintuma/TURING/16.5/1.5/10 cm/String die door de Turingmachine wordt geaccepteerd/$s\in L_{\mbox{\small TM}}$/{De machine wordt ge\"initialiseerd in toestand $q_s$ met de invoerstring op de tape op de overige plaatsen staat een $\#$ en de leeskop op het meest linkse invoerteken. Zolang de machine niet in $q_a$ en $q_r$ zit, wordt het karakter uitgelezen onder de leeskop. Op basis van de transitiefunctie $\delta$ wordt een nieuw karakter onder de leeskop gezet, wordt de leeskop naar links/rechts verplaatst en komt de machine in een nieuwe toestand. Als de Turingmachine na een eindig aantal stappen in $q_a$ komt wordt de string aanvaard.},
sininftuma/TURING/16.5/7.5/10 cm/String waarvoor de Turingmachine niet stopt/$s\in\infty_{\mbox{\small TM}}$/{Een string $s$ behoort tot $\infty_{\mbox{\small TM}}$ indien bij invoer van $s$, de machine nooit in $q_a$ of $q_r$ terecht komt.},
recog/TURING/27.5/1.5/10 cm/Herkennen/Recognize/{Een Turingmachine TM herkent $L_{\mbox{\small TM}}$. Een taal $L$ waarvoor zo'n Turingmachine bestaat noemt men een herkenbare taal.},
decid/TURING/27.5/4.5/10 cm/Beslissen/Decide/{Een Turingmachine TM beslist een taal $L$ indien de Turingmachine $L$ herkent en bovendien $\infty_{\mbox{\small TM}}=\emptyset$. Een taal $L$ waarvoor zo'n Turingmachine bestaat noemt men een beslisbare taal.},
enum/TURING/39/1.5/10 cm/Enumeratormachine/EM/{Een Enumeratormachine is een Turingmachine met volgende uitbreidingen:\begin{itemize}
 \item Een enumeratortoestand $q_e$;
 \item Een uitvoerband met uitvoermarker;
 \item De transitiefunctie $\funsig{\delta}{Q\times\Gamma}{Q\times\Gamma\times\Gamma_{\epsilon}\times\accl{L,S,R}}$.
\end{itemize}De machine start met lege band en lege uitvoerband, in de $q_s$ toestand en werkt als een Turingmachine. Wanneer iets op de uitvoer wordt geschreven verschuift de schrijfkop naar rechts. Wanneer de machine in toestand $q_e$ komt, wordt op de uitvoer de uitvoermarker geschreven en komt de machine opnieuw in $q_s$ terecht.},
recogenum/TURING/39/7.5/10 cm/Enumeratormachine/$\mbox{Recognize}\equiv\mbox{EM}$/{De taal door een enumerator bepaald is herkenbaar en elke herkenbare taal wordt door een enumerator ge\"enumereerd.},
recogenum/TURING/55/1.5/10 cm/Stelling van Rice/Rice/{Voor elke niet-triviale taal-invariante eigenschap $P$ van Turingmachines geldt $L_{P}=\accl{\tupl{M}|\mbox{$M$ voldoet aan $P$}}$ is niet beslisbaar.},
tcalc/TURING/55/4.375/10 cm/Turing-berekenbare functie/Turing-berekenbaar/{Een functie $f$ is Turing-berekenbaar indien er een Turingmachine bestaat die bij input $s$ uiteindelijk stopt met $\fun{f}{s}$ op de band.},
langrec/TURING/55/7.25/10 cm/Reductie van talen/$L_1\leq_mL_2$/{Een taal $L_1$ (over $\Sigma_1$) is reduceerbaar naar een taal $L_2$ (over $\Sigma_2$) indien er een afbeelding $\funsig{f}{\Sigma_1^{\star}}{\Sigma_2^{\star}}$ bestaat zodat $\fun{f}{L_1}\subseteq L_2$ en $\overline{\fun{f}{L_1}}\subseteq\overline{L_2}$ en $f$ Turing-berekenbaar is. We noteren dat door $L_1\leq_mL_2$.},
buzbev/TURING/70/5.75/10 cm/Bezige bever/$\fun{\Sigma}{n}$/{de bezige bever is een totale functie $\funsig{\Sigma}{\NN}{\NN}$ die voor invoer $n$ het maximale aantal $1$'en uitschrijft voor een Turingmachine bij lege invoer die eindigt.},
ora/ORAKEL/6.5/2.5/12 cm/Orakelmachine/$O^L$/{Een Orakelmachine $O^L$ is een Turing-machine met een orakel voor de taal $L$. Men kan dit voorstellen als een Turingmachine met een extra band (de orakel-tape) en drie toestanden $q_o$, $q_{oa}$ en $q_{or}$. De Turingmachine raadpleegt het orakel door de string op de orakelband te schrijven en in toestand $q_o$ te gaan. Het orakel brengt wanneer de string tot $L$ behoort de machine in toestand $q_{oa}$ en anders in toestand $q_{or}$.},
turrec/ORAKEL/55/7.25/10 cm/Turing-reductie van talen/$L_1\leq_TL_2$/{Een taal $L_1$ is Turingreduceerbaar naar taal $L_2$, indien $L_1$ beslisbaar is relatief ten opzichte van $L_2$, dit wil zeggen: er bestaat een orakelmachine $O^{L_2}$ die $L_1$ beslist. We schrijven $L_1\leq_TL_2$.},
pcorp/OTHER/5.5/2.5/10 cm/Post Correspondence Problem/PCP/{Gegeven een set $S$ van tuples $\accl{\alpha,\beta}$ met $\alpha,\beta\in\Sigma^{\star}$ over een alfabet $\Sigma$. Bestaat er een sequentie elementen $s_0s_1\ldots s_n$, zodat de concatenatie van het eerste element van de tuples gelijk is aan de concatenatie van het tweede element van de tuples?},
priem/OTHER/20/1.5/13 cm/Primitief recursieve functies/Primitief recursief/{Een primitief recursieve functie is inductief gedefinieerd als volgt:
\begin{itemize}
 \item De nulfunctie: $\funsig{\mbox{nul}}{\NN}{\NN}:x\mapsto 0$;
 \item De successorfunctie: $\funsig{\mbox{succ}}{\NN}{\NN}:x\mapsto x+1$;
 \item De projectiefunctie: $\funsig{p_i^n}{\NN^n}{\NN}:\tupl{x_1,x_2,\ldots,x_n}\mapsto x_i$;
 \item De compositiefunctie: $\funsig{\funmf{Cn}{f,g_1,g_2,\ldots,g_m}}{\NN^k}{\NN}:\vec{x}\mapsto\fun{f}{\fun{g_1}{\vec{x}},\fun{g_2}{\vec{x}},\ldots,\fun{g_m}{\vec{x}}}$ met $\funsig{f}{\NN^m}{\NN}$ en $\funsig{g_i}{\NN^k}{\NN}$ primitieve functies;
 \item De primitieve recursie: \[\funsig{\funmf{Pr}{f,g}}{\NN^k}{\NN}:\tupl{\vec{x},y}\mapsto\acclguard{\fun{f}{\vec{x}}&\mbox{\bf if } y=0\\\fun{g}{\vec{x},y-1,\fun{\funmf{Pr}{f,g}}{\vec{x},y-1}}&\mbox{\bf otherwise}}\] met $\funsig{f}{\NN^k}{\NN}$ en $\funsig{g}{\NN^{k+2}}{\NN}$ primitieve functies;
\end{itemize}},
ack/OTHER/33/1.5/11 cm/Ackermann-functie/Ack/{Een functie die niet kan worden voorgesteld aan de hand van primitieve recursie is de Ackermann-functie:
\[\funsig{\mbox{Ack}}{\NN^2}{\NN}:\tupl{x,y}\mapsto\acclguard{y+1&\mbox{\bf if } x=0\\\funm{Ack}{x-1,1}&\mbox{\bf if } y=0\\\funm{Ack}{x-1,\funm{Ack}{x,y-1}}&\mbox{\bf otherwise}}\]},
mu/OTHER/33/5.75/11 cm/Recursieve functies/$\mu$-recursief/{Een $\mu$ recursieve functie is een uitbreiding op primitief recursieve functies met volgende de onbegrensde minimalisatie: \[\funsig{\funmf{Mn}{f}}{\NN^k}{\NN}:\vec{x}\mapsto\acclguard{y&\mbox{\bf if }\fun{f}{\vec{x},y}=0\wedge\forall z<y:\fun{f}{\vec{x},z}\neq 0\\\mbox{\bf undef}&\mbox{\bf otherwise}}\] met $\funsig{f}{\NN^{k+1}}{\NN}$ een $\mu$-recursieve functies.},
equivla/LAMBDA/3/2.5/5 cm/Equivalente termen/$A\sim B$/{Twee termen $A$ en $B$ zijn equivalent indien er een sequentie van herschrijfregels bestaat zodat $A\rightarrow\ldots\rightarrow X$ en $B\rightarrow\ldots\rightarrow X$.},
curry/LAMBDA/9/2.5/5 cm/Currying/$f\ x$/{$f\ x$ lezen we als $f$ toegepast op argument $x$. Elke functie in $\lambda$-calculus heeft maar \'e\'en argument (maar resulteert soms in een nieuwe functie).},
cra/LAMBDA/4.5/6.5/8 cm/Church-Rosser I/CR-1/{Gegeven twee expressies $E_1$ en $E_2$. Indien $E_1\leftrightarrow^{\star}E_2$, dan bestaat er een expressie $E$ zodat $E_1\rightarrow^{\star}E$ en $E_2\rightarrow^{\star}E$.o Bijgevolg kan geen expressie geconverteerd worden naar twee verschillende normaalvormen.},
bound/LAMBDA/18/2.5/11 cm/Vrij voorkomen/$\brak{\lambda x\ .\ x}\ x$/{Een voorkomen van de variabele x is vrij in de expressie E indien:
\begin{itemize}
 \item $E$ gelijk is aan $x$;
 \item $E$ gelijk is aan $\brak{A\ B}$ en het voorkomen van $x$ is vrij in $A$ of $B$;
 \item $E$ gelijk is aan $\brak{\lambda\ y\ .\ A}$ en $y\neq x$.
\end{itemize}},
crb/LAMBDA/13.5/6.5/8 cm/Church-Rosser II/CR-2/{Gegeven twee expressies $E$ en $N$ met $N$ in normaalvorm. Dan bestaat er een reductierij in normaalorde $E\rightarrow^{ \star}N$. De normaalorde bepaalt dat de meest linkse buitenste redex eerst moet worden gereduceerd.},
alpha/LAMBDA/28/1.5/7 cm/Alfa-conversie/$\rightarrow^{\alpha}$/{De alfa-reductie hernoemt de variabele in een lambda-expressie. Formeel:\[\lambda\ x\ .\ E\rightarrow^{\alpha}\lambda\ y\ .\ E'\] In $E$ worden alle vrije voorkomens van $x$ vervangen. $y$ is geen variabele gebonden in $E$.},
betared/LAMBDA/36/1.5/7 cm/Beta-reductie/$\rightarrow^{\beta}$/{Het resultaat van de toepassing van een $\lambda$-abstractie op een argument is een kopie van het lichaam van de $\lambda$-abstractie waarin de vrije voorkomens van de formele parameter vervangen zijn door het argument. Formeel:\[\lambda\ x\ .\ E\rightarrow^{\beta}E'\]},
betaabs/LAMBDA/36/6.5/7 cm/Beta-abstractie/$\leftarrow^{\beta}$/{Vanuit een term wordt een functie afgeleid die wordt toegepast op een argument. Beta-abstractie wordt gebruikt om de equivalentie tussen termen aan te tonen. Formeel:\[\lambda\ x\ .\ E'\leftarrow^{\beta}E\]},
eta/LAMBDA/28/6/7 cm/Eta-conversie/$\rightarrow^{\eta}$/{Eta-reductie drukt uit dat aan lambda-expressie die een toepassing van $F$ uitdrukt, gelijk is aan de functie zelf. Formeel:\[\lambda\ x\ .\ F\ x\rightarrow^{\eta}F\]Met $F$ een functie waarin $x$ niet vrij voorkomt.},
nil/LAMBDA/44.5/1.5/8 cm/Natuurlijke getallen/$\NNN$/{Een natuurlijk getal $n\in\NNN$ wordt meestal voorgesteld als $c_n=\lambda\ f\ x\ .\ f^n\ x$. Volgende operaties zijn gedefinieerd:\begin{itemize}\item optelling $A_{+}=\lambda\ x\ y\ p\ q\ .\ x\ p\ \brak{y\ p\ q}$\item vermenigvuldiging $A_{*}=\lambda\ x\ y\ z\ .\ x\ \brak{y\ z}$\item exponent $A_{\mbox{\small exp}}=\lambda\ x\ y\ .\ y\ x$\end{itemize}},
ift/LAMBDA/44.5/6/8 cm/{True, False en If}/$\BBB$/{Booleaanse waarden $b\in\BBB$ worden meestal voorgesteld als volgt:\begin{itemize}\item $\mbox{\bf true}=\lambda\ x\ y\ .\ x$\item $\mbox{\bf false}=\lambda\ x\ y\ .\ y$\item $\mbox{\bf if}=\lambda\ x\ y\ z.\brak{x\ y\ z}$\end{itemize}},
lis/LAMBDA/53.5/1.5/8 cm/{Lijsten}/$\vec{x}$/{Lijsten worden meestal door het gelinkte-lijst-paradigma voorgesteld. Elke node heeft een \textsc{Hoofd} en een \textsc{Staart}. Men stelt een node voor met \textsc{Cons}:\begin{itemize}\item $\mbox{\sc Hoofd}=\lambda\ c\ .\ c\ \brak{\lambda\ a\ b\ .\ a}$\item $\mbox{\sc Staart}=\lambda\ c\ .\ c\ \brak{\lambda\ a\ b\ .\ b}$\item $\mbox{\sc Cons}=\lambda\ a\ b\ f.f\ a\ b$\end{itemize}},
pac/PROB/7/2.5/14 cm/Acceptatie/$\mbox{\textsc{A}}_{\phi}$/{$A_{\phi}=\accl{\tupl{M,s}|\mbox{$M$ is een $\phi$ die string $s$ accepteert}}$.},
phalt/PROB/18/2.5/14 cm/Stopt/$\mbox{\textsc{H}}_{\phi}$/{$H_{\phi}=\accl{\tupl{M,s}|\mbox{$M$ is een $\phi$ die stopt bij string $s$}}$.},
pe/PROB/29/2.5/14 cm/Leegheid/$\mbox{\textsc{E}}_{\phi}$/{$E_{\phi}=\accl{\tupl{M}|\mbox{$M$ herkent de lege taal}}$.},
peq/PROB/7/5/14 cm/Gelijkheid/$\mbox{\textsc{EQ}}_{\phi}$/{$EQ_{\phi}=\accl{\tupl{M_1,M_2}|\mbox{$M_1$ en $M_2$ bepalen dezelfde taal}}$.},
pes/PROB/18/5/14 cm/Lege string/$\mbox{\textsc{ES}}_{\phi}$/{$ES_{\phi}=\accl{\tupl{M}|\mbox{$M$ accepteert de lege string}}$.},
reges/PROB/29/5/14 cm/Regulier/$\mbox{\textsc{Regular}}_{\phi}$/{$ES_{\phi}=\accl{\tupl{M}|\mbox{$M$ herkent een reguliere taal}}$.},
alles/PROB/7/7.5/14 cm/Alle strings/$\mbox{\textsc{ALL}}_{\phi}$/{$ES_{\phi}=\accl{\tupl{M}|\mbox{$M$ herkent alle strings $\Sigma^{\star}$}}$.}