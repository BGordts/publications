alfa/LANG/4.5/2.5/8 cm/Alfabet/{$\Sigma$}/{Een alfabet $\Sigma$ is een set karakters die relevant zijn in een gegeven taal $L$.},
stri/LANG/4.5/5/8 cm/String/{$s\in\Sigma^*$}/{Een string over een alfabet $\Sigma$ is een sequentie van 0, 1 of meer karakters uit $\Sigma$. $\Sigma^*$ beschrijft de verzameling van alle strings over $\Sigma$},
lang/LANG/4.5/8/8 cm/Taal/{$L\subseteq\Sigma^*$}/{Een taal over een alfabet $\Sigma$ is een verzameling strings over $\Sigma$.},
conc/LANG/12.5/2.5/6 cm/Concatenatie van talen/{$L_1L_2$}/{Gegeven twee talen $L_1$ en $L_2$ over eenzelfde alfabet $\Sigma$, dan is de concatenatie $L_1L_2=\accl{xy|x\in L_1,y\in L_2}$. $L_n$ is de concatenatie van $n$ keer de taal $L$.},
klee/LANG/12.5/7/6 cm/Kleene-ster/{$L^*$}/{De Kleene-ster van een taal $L$ is de unie van nul, \'e\'en of meer concatenaties van de taal: $L^*=\displaystyle\cup_{n=0}^n{L^n}$.},
rege/REGEX/5.5/2.5/10 cm/Reguliere expressie over een alfabet $\Sigma$/RE/{$E$ is een reguliere expressie over $\Sigma$ indien $E$ van de vorm is:\begin{itemize}
 \item $\epsilon$
 \item $\phi$
 \item $a$ met $a\in\Sigma$
 \item $\brak{E_1E_2}$ waarbij $E_1$ en $E_2$ reguliere expressies zijn over $\Sigma$.
 \item $\brak{E_1|E_2}$ waarbij $E_1$ en $E_2$ reguliere expressies zijn over $\Sigma$.
 \item $\brak{E_1}^*$ waarbij $E_1$ een reguliere expressie is over $\Sigma$.
\end{itemize}},
rela/REGEX/18/2.5/14 cm/Een reguliere expressie bepaalt een taal/$\mbox{RE}\sim L_{\mbox{RE}}$/{Een reguliere expressie $E$ bepaalt een taal $L_E$ over hetzelfde alfabet $\Sigma$ als volgt:\begin{itemize}
 \item Als $E=\epsilon$, dan is $L_E=\accl{\epsilon}$
 \item Als $E=\phi$, dan is $L_E=\emptyset$
 \item Als $E=a$ met $a\in\Sigma$, dan is $L_E=\accl{a}$
 \item $\brak{E_1E_2}$ waarbij $E_1$ en $E_2$ reguliere expressies zijn over $\Sigma$, dan is $L_E=L_1L_2$.
 \item $\brak{E_1|E_2}$ waarbij $E_1$ en $E_2$ reguliere expressies zijn over $\Sigma$, dan is $L_E=L_1\cup L_2$.
 \item $\brak{E_1}^*$ waarbij $E_1$ een reguliere expressie is over $\Sigma$, dan is $L_E=L_{E_1}^*$.
\end{itemize}},
regl/REGEX/5.5/7.75/10 cm/Reguliere taal/$\mbox{RegLan}$/{Een taal die door een reguliere expressie bepaald wordt is een reguliere taal. De verzameling van reguliere talen duiden we aan met $\mbox{RegLan}$.},
nfa/REGEX/35/1.2/10 cm/Niet-deterministische eindige toestandsautomaat/NFA/{Een niet-deterministische eindige toestandsautomaat is een $5$-tal $\tupl{Q,\Sigma,\delta,q_s,F}$ waarbij:\begin{itemize}
 \item $Q$ een eindige verzameling toestanden is;
 \item $\Sigma$ een eindig alfabet;
 \item $\delta:Q\times\Sigma_{\epsilon}\rightarrow\powset{Q}$ de overgangsfunctie van de automaat;
 \item $q_s\in Q$ de starttoestand en een element van $Q$;
 \item $F\subseteq Q$ de verzameling eindtoestanden.
\end{itemize}},
nfaeq/REGEX/43.5/1.2/6 cm/Equivalentie van twee NFA's/$\mbox{NFA}\equiv\mbox{NFA}$/{Twee NFA's worden equivalent genoemd als ze dezelfde taal bepalen.},
nfastr/REGEX/35/6.15/10 cm/String aanvaard door een NFA/$s\in L_{\mbox{\tiny NFA}}$/{Een string $s$ wordt aanvaard door een NFA $\tupl{Q,\Sigma,\delta,q_s,F}$ indien $s$ kan geschreven worden als $a_1a_2\ldots a_n$ met $a_i\in\Sigma_{\epsilon}$, en er een rij toestanden $q_1q_2\ldots,q_{n+1}$ bestaat zodat:\begin{itemize}
 \item $t_1=q_s$;
 \item $t_{i+1}\in\fun{\delta}{t_i,a_i}$;
 \item $t_{n+1}\in F$.
\end{itemize}},
gnfa/REGEX/55/1.2/10 cm/Gegeneraliseerde niet-deterministische eindige toestandsautomaat/GNFA/{Een gegeneraliseerde niet-deterministische eindige toestandsautomaat is een NFA $\tupl{Q,\Sigma,\delta,q_s,F}$ met:\begin{itemize}
 \item slechts \'e\'en eindtoestand $F=\accl{q_f}$ zodat $q_f\neq q_s$;
 \item de bogen bevatten een reguliere expressie als label;
 \item $\forall q_i,q_j\in Q\setminus\accl{q_s,q_f}:\exists c\in\funm{RegExp}{\Sigma}:\fun{\delta}{q_i,c}=q_j$;
 \item $\forall q_i\in Q\setminus\accl{q_s,q_f}:\exists c\in\funm{RegExp}{\Sigma}:\fun{\delta}{q_s,c}=q_j$;
 \item $\forall q_i\in Q\setminus\accl{q_s,q_f}:\exists c\in\funm{RegExp}{\Sigma}:\fun{\delta}{q_i,c}=q_f$.
\end{itemize}},
cfgr/CTXFR/5.5/2.5/10 cm/Contextvrije grammatica over een alfabet $\Sigma$/CFG/{$\tupl{V,\Sigma,R,S}$ is een contextvrije grammatica over $\Sigma$ indien:\begin{itemize}
 \item $V$ een eindige verzameling niet-eindsymbolen is;
 \item $\Sigma$ een eindig alfabet van eindsymbolen disjunct met $V$;
 \item $R$ een eindige verzameling regels; een regel is een koppel van  ́\'e\'en niet-eindsymbool en een string van elementen uit $V\cup\Sigma_{\epsilon}$; we schrijven de twee delen van zulk een koppel met een $\rightarrow$ ertussen;
 \item $S$ is het startsymbool en behoort tot $V$.
\end{itemize}},
cfgder/CTXFR/18/2.5/14 cm/Afleiding met behulp van een contextvrije grammatica/$b\Rightarrow_{CFG}^{\star}f$/{Gegeven een contextvrije grammatica $\tupl{V,\Sigma,R,S}$. Een string $f$ over $V\cup\Sigma_{\epsilon}$ wordt afgeleid uit een string $b$ uit $V\cup\Sigma_{\epsilon}$ met behulp van de contextvrije grammatica indien er een eindige rij strings $s_0,s_1,\ldots,s_n$ bestaat zodat:\begin{itemize}
 \item $s_0=b\wedge s_n=f$
 \item $s_{i+1}$ verkregen wordt uit $s_i$ (voor $i<n$) door in $s_i$ een niet-eindsymbool $X$ te vervangen door de rechterkant van een regel waarin $X$ links voorkomt.
\end{itemize}
We noteren $s_i\Rightarrow s_{i+1}$ en $b\rightarrow^{\star}f$},
cfgla/CTXFR/18/7.5/14 cm/Een contextvrije grammatica bepaalt een taal/$\mbox{CFG}\sim L_{\mbox{CFG}}$/{De taal $L_{CFG}$ bepaald door een contextvrije grammatica $\tupl{V,\Sigma,R,S}$ is de verzameling strings over $\Sigma$ die kunnen afgeleid worden van het startsymbool $S$; of meer formeel: $L_{CFG}=\accl{s\in\Sigma^{\star}|S\Rightarrow^{\star}s}$.},
cfgl/CTXFR/5.5/7.75/10 cm/Contextvrije taal/$\mbox{CFL}$/{Een taal $L$ is contextvrij indien er een contextvrije grammatica is zodat $L=L_{CFG}$. De verzameling van contextvrije talen duiden we aan met $\mbox{CFL}$.},
pda/CTXFR/34/1.2/12 cm/Push-down automaat/PDA/{Een push-down automaat is een $5$-tal $\tupl{Q,\Sigma,\Gamma,\delta,q_s,F}$ waarbij:\begin{itemize}
 \item $Q$ een eindige verzameling toestanden is;
 \item $\Sigma$ een eindig inputalfabet;
 \item $\Gamma$ een eindig stapelalfabet;
 \item $\delta:Q\times\Sigma_{\epsilon}\times\Gamma_{\epsilon}\rightarrow\powset{Q\times\Gamma_{\epsilon}}$ de overgangsfunctie;
 \item $q_s\in Q$ de starttoestand en een element van $Q$;
 \item $F\subseteq Q$ de verzameling eindtoestanden.
\end{itemize}},
pdastr/CTXFR/34/6.15/12 cm/String aanvaard door een PDA/$s\in L_{\mbox{\tiny PDA}}$/{Een string $s$ wordt aanvaard door een PDA $\tupl{Q,\Sigma,\delta,q_s,F}$ indien $s$ kan worden opgesplitst in
delen $w_i,i=1\ldots m$ ($w_i\in\Sigma$), er toestanden $q_j,j=0\ldots m$ ($q_j\in Q$) zijn, en stapels $\mbox{stapel}_k,k=0\ldots m$ ($\mbox{stapel}_k\in\Gamma^{\star}$) zodat:\begin{itemize}
 \item $\mbox{stapel}_0=\epsilon\wedge q_0=q_s\wedge q_m\in F$;
 \item $\tupl{q_{i+1},y}\in\fun{\delta}{q_i,w_{i+1},x}$ waarbij $x,y\in\Gamma_{\epsilon}$ en $\mbox{stapel}_i=xt$, $\mbox{stapel}_{i+1}=yt$ met $t\in\Gamma^{\star}$.
\end{itemize}},
pdaeqcfg/CTXFR/43.5/1.2/6 cm/Equivalentie van CFG en PDA/$\mbox{CFG}\equiv\mbox{PDA}$/{Elke push-down automaat bepaalt een contextvrije taal en elke contextvrije taal wordt bepaald door een push-down automaat.},
lincfl/CTXFR/46/4.7/10 cm/Pompend lemma voor contextvrije talen/$L\in\mbox{CFL}$/{Voor een contextvrije taal $L$ bestaat een getal $d$ (de pomplengte) zodat elke string $s\in L$ met lengte minstens $d$ kan opgedeeld worden in $5$ stukken $u,v,x,y,z\in\Sigma^{\star}$ zodat $s=uvxyz$ en:
\begin{enumerate}
 \item $\forall i\geq 0:uv^ixy^iz\in L$;
 \item $\abs{vy}>0$;
 \item $\abs{vxy}\leq d$.
\end{enumerate}},
csgr/CTXSE/5.5/2.5/10 cm/Context-sensitieve grammatica over een alfabet $\Sigma$/CSG/{Een tuple $\tupl{V,\Sigma,R,S}$ is een context-sensitieve grammatica over $\Sigma$ indien:\begin{itemize}
 \item $V$ een eindige verzameling niet-eindsymbolen is;
 \item $\Sigma$ een eindig alfabet van eindsymbolen disjunct met $V$;
 \item $R$ een eindige verzameling regels; een regel is een koppel van $\alpha A\beta\rightarrow\alpha\gamma\beta$ ́met $\alpha,\beta,\gamma\in\brak{\Sigma\cup V}^{\star}$ en $A\in V$;
 \item $S$ is het startsymbool en behoort tot $V$.
\end{itemize}},
tuma/TURING/5.5/2.5/10 cm/Turingmachine/TM/{Een Turingmachine is een tuple $\tupl{Q,\Sigma,\Gamma,\delta,q_s,q_a,q_r}$ waarbij:\begin{itemize}
 \item $Q$ een eindige verzameling toestanden is;
 \item $\Sigma$ een invoeralfabet dat niet $\#$ bevat;
 \item $\Gamma$ een tape alfabet met $\#\in\Gamma$ en $\Sigma\subsetneq\Gamma$;
 \item $q_s$ de start-toestand, $q_a$ de accepterende eindtoestand, $q_r\neq q_a$ de verwerpende eindtoestand.
 \item $\funsig{\delta}{Q\times\Gamma}{Q\times\Gamma\times\acc{L,S,R}}$ de totale transitiefunctie.
\end{itemize}},
sintuma/TURING/16.5/1.5/10 cm/String die door de Turingmachine wordt geaccepteerd/$s\in L_{\mbox{\small TM}}$/{De machine wordt ge\"initialiseerd in toestand $q_s$ met de invoerstring op de tape op de overige plaatsen staat een $\#$ en de leeskop op het meest linkse invoerteken. Zolang de machine niet in $q_a$ en $q_r$ zit, wordt het karakter uitgelezen onder de leeskop. Op basis van de transitiefunctie $\delta$ wordt een nieuw karakter onder de leeskop gezet, wordt de leeskop naar links/rechts verplaatst en komt de machine in een nieuwe toestand. Als de Turingmachine na een eindig aantal stappen in $q_a$ komt wordt de string aanvaard.},
sininftuma/TURING/16.5/7.5/10 cm/String waarvoor de Turingmachine niet stopt/$s\in\infty_{\mbox{\small TM}}$/{Een string $s$ behoort tot $\infty_{\mbox{\small TM}}$ indien bij invoer van $s$, de machine nooit in $q_a$ of $q_r$ terecht komt.},
recog/TURING/27.5/1.5/10 cm/Herkennen/Recognize/{Een Turingmachine TM herkent $L_{\mbox{\small TM}}$.},
decid/TURING/27.5/3.5/10 cm/Beslissen/Decide/{Een Turingmachine TM beslist een taal $L$ indien de Turingmachine $L$ herkent en bovendien $\infty_{\mbox{\small TM}}=\emptyset$.},
enum/TURING/39/1.5/10 cm/Enumeratormachine/EM/{Een Enumeratormachine is een Turingmachine met volgende uitbreidingen:\begin{itemize}
 \item Een enumeratortoestand $q_e$;
 \item Een uitvoerband met uitvoermarker;
 \item De transitiefunctie $\funsig{\delta}{Q\times\Gamma}{Q\times\Gamma\times\Gammma_{\epsilon}\times\accl{L,S,R}}$.
\end{itemize}De machine start met lege band en lege outputband, in de gewone q s en begint te
werken. Telkens bij een overgang iets op de output wordt geschreven verschuift de
schrijfkop naar rechts. Telkens de machine in toestand q e komt, wordt op de out-
put de outputmarker geschreven, overgegaan naar q s en de machine loopt verder.},
pac/PROB/8/2.5/14 cm/Acceptatie/$A_{\phi}$/{$A_{\phi}=\accl{\tupl{M,s}|\mbox{$M$ is een $\phi$ die string $s$ accepteert}}$.},
phalt/PROB/23/2.5/14 cm/Stopt/$H_{\phi}$/{$H_{\phi}=\accl{\tupl{M,s}|\mbox{$M$ is een $\phi$ die stopt bij string $s$}}$.},
pe/PROB/38/2.5/14 cm/Leegheid/$E_{\phi}$/{$E_{\phi}=\accl{\tupl{M}|\mbox{$M$ is een $\phi$ die stopt bij string $s$}}$.},
peq/PROB/8/7.5/14 cm/Gelijkheid/$EQ_{\phi}$/{$EQ_{\phi}=\accl{\tupl{M_1,M_2}|\mbox{$M_1$ en $M_2$ zijn $\phi$'s die dezelfde taal bepalen}}$.},
pes/PROB/23/7.5/14 cm/Lege string/$ES_{\phi}$/{$ES_{\phi}=\accl{\tupl{M}|\mbox{$M$ bepaald de lege taal}}$.}