\chapter{Conventies en Schemas}
%\section{Circuit conventies}
%\begin{figure}[H]
%\centering
%\begin{tikzpicture}
%\node (H1) at (0,0) {\textbf{\large{Component}}};
%\node (H2) at (8,0) {\textbf{\large{Circuit Symbol}}};
%\end{tikzpicture}
%\caption{Circuit conventies.}
%\end{figure}
\section{Lijst van component interfaces}
\begin{figure}[H]
\centering
\begin{tikzpicture}[yscale=-1,circuit logic US]
\filldraw[fill=black!3,draw=black,thick] (0,0) rectangle ++(15,14);
\foreach \x in {1,2} {
  \draw[thick] (5*\x,0) -- ++(0,14);
}
\foreach \x/\y/\t in {0/0/Poorten,0/9.25/Rekenkundig,1/0/Rekenkundig,1/4.75/Andere basisschakelingen,2/0/Andere basisschakelingen} {
  \filldraw[thick,draw=black,fill=black!50] (5*\x,\y) rectangle ++(5,0.5);
  \draw[thick] (5*\x+2.5,\y+0.25) node {\textbf{\t}};
}
\foreach \x/\y/\t in {0/0.5/AND-poort,0/1.75/OR-poort,0/3/NOT-poort,0/4.25/NAND-poort,0/5.5/NOR-poort,0/6.75/XOR-poort,0/8/XNOR-poort,0/9.75/Halve Opteller,0/11.875/Volledige Opteller,1/0.5/Carry-Lookahead Opteller (CLA),1/2.625/Carry-Lookahead Opteller-Generator,1/5.25/Multiplexer,1/7.4375/Decoder,1/9.625/Demultiplexer,1/11.8125/Encoder,2/0.5/Vergelijker} {
  \draw (5*\x,\y) -- ++(5,0);
  \node[scale=0.75,anchor=north] (T) at (2.5+5*\x,\y) {\t};
}
\node[and gate] (A) at (2.5,1.25) {};
\draw (A.input 1) -- ++(-0.25,0) node[anchor=east,scale=0.75]{$x$};
\draw (A.input 2) -- ++(-0.25,0) node[anchor=east,scale=0.75]{$y$};
\draw (A.output) -- ++(0.25,0) node[anchor=west,scale=0.75]{$f$};

\node[or gate] (O) at (2.5,2.5) {};
\draw (O.input 1) -- ++(-0.25,0) node[anchor=east,scale=0.75]{$x$};
\draw (O.input 2) -- ++(-0.25,0) node[anchor=east,scale=0.75]{$y$};
\draw (O.output) -- ++(0.25,0) node[anchor=west,scale=0.75]{$f$};

\node[not gate] (N) at (2.5,3.75) {};
\draw (N.input) -- ++(-0.25,0) node[anchor=east,scale=0.75]{$x$};
\draw (N.output) -- ++(0.25,0) node[anchor=west,scale=0.75]{$f$};

\node[nand gate] (NA) at (2.5,5) {};
\draw (NA.input 1) -- ++(-0.25,0) node[anchor=east,scale=0.75]{$x$};
\draw (NA.input 2) -- ++(-0.25,0) node[anchor=east,scale=0.75]{$y$};
\draw (NA.output) -- ++(0.25,0) node[anchor=west,scale=0.75]{$f$};

\node[nor gate] (NO) at (2.5,6.25) {};
\draw (NO.input 1) -- ++(-0.25,0) node[anchor=east,scale=0.75]{$x$};
\draw (NO.input 2) -- ++(-0.25,0) node[anchor=east,scale=0.75]{$y$};
\draw (NO.output) -- ++(0.25,0) node[anchor=west,scale=0.75]{$f$};

\node[xor gate] (XO) at (2.5,7.5) {};
\draw (XO.input 1) -- ++(-0.25,0) node[anchor=east,scale=0.75]{$x$};
\draw (XO.input 2) -- ++(-0.25,0) node[anchor=east,scale=0.75]{$y$};
\draw (XO.output) -- ++(0.25,0) node[anchor=west,scale=0.75]{$f$};

\node[xnor gate] (XNO) at (2.5,8.75) {};
\draw (XNO.input 1) -- ++(-0.25,0) node[anchor=east,scale=0.75]{$x$};
\draw (XNO.input 2) -- ++(-0.25,0) node[anchor=east,scale=0.75]{$y$};
\draw (XNO.output) -- ++(0.25,0) node[anchor=west,scale=0.75]{$f$};

\node[halfadder,scale=0.75] (HA) at (2.5,11) {HA};
\draw (HA.y) -- ++(0,-0.2) node[anchor=south,scale=0.75]{$x$};
\draw (HA.x) -- ++(0,-0.2) node[anchor=south,scale=0.75]{$y$};
\draw (HA.s) -- ++(0,0.2) node[anchor=north,scale=0.75]{$s$};
\draw (HA.co) -- ++(-0.2,0) node[anchor=east,scale=0.75]{$c_o$};

\node[fulladder,scale=0.75] (FA) at (2.5,13.125) {FA};
\draw (FA.y) -- ++(0,-0.2) node[anchor=south,scale=0.75]{$x$};
\draw (FA.x) -- ++(0,-0.2) node[anchor=south,scale=0.75]{$y$};
\draw (FA.s) -- ++(0,0.2) node[anchor=north,scale=0.75]{$s$};
\draw (FA.co) -- ++(-0.2,0) node[anchor=east,scale=0.75]{$c_o$};
\draw (FA.ci) -- ++(0.2,0) node[anchor=west,scale=0.75]{$c_i$};

\node[cla,scale=0.75] (CLA) at (7.5,1.75) {};
\draw (CLA.x) -- ++(0,-0.2) node[anchor=south,scale=0.75]{$x$};
\draw (CLA.y) -- ++(0,-0.2) node[anchor=south,scale=0.75]{$y$};
\draw (CLA.g) -- ++(0,0.2) node[anchor=north,scale=0.75]{$g$};
\draw (CLA.p) -- ++(0,0.2) node[anchor=north,scale=0.75]{$p$};
\draw (CLA.c) -- ++(0.2,0) node[anchor=west,scale=0.75]{$c$};
\draw (CLA.s) -- ++(-0.2,0) node[anchor=east,scale=0.75]{$s$};

\node[clag3,scale=0.65] (CLAG) at (7.75,3.875) {};
\draw (CLAG.pa) -- ++(0,-0.2) node[anchor=south,scale=0.75]{$p_0$};
\draw (CLAG.ga) -- ++(0,-0.2) node[anchor=south,scale=0.75]{$g_0$};
\draw (CLAG.pb) -- ++(0,-0.2) node[anchor=south,scale=0.75]{$p_1$};
\draw (CLAG.gb) -- ++(0,-0.2) node[anchor=south,scale=0.75]{$g_1$};
\draw (CLAG.pc) -- ++(0,-0.2) node[anchor=south,scale=0.75]{$p_2$};
\draw (CLAG.gc) -- ++(0,-0.2) node[anchor=south,scale=0.75]{$g_2$};
\draw (CLAG.ca) -- ++(0,0.2) node[anchor=north,scale=0.75]{$c_1$};
\draw (CLAG.cb) -- ++(0,0.2) node[anchor=north,scale=0.75]{$c_2$};
\draw (CLAG.cc) -- ++(0,0.2) node[anchor=north,scale=0.75]{$c_3$};
\draw (CLAG.pab) -- ++(-0.2,0) node[anchor=east,scale=0.75]{$g_{0,2}$};
\draw (CLAG.gab) -- ++(-0.2,0) node[anchor=east,scale=0.75]{$p_{0,2}$};
\draw (CLAG.c) -- ++(0.2,0) node[anchor=west,scale=0.75]{$c$};
\node[mux4to1,scale=0.85] (MUX) at (7.5,6.59375) {};%+1.34375
\draw (MUX.data0) -- ++(0,-0.2) node[anchor=south,scale=0.75]{$d_0$};
\draw (MUX.data1) -- ++(0,-0.2) node[anchor=south,scale=0.75]{$d_1$};
\draw (MUX.data2) -- ++(0,-0.2) node[anchor=south,scale=0.75]{$d_2$};
\draw (MUX.data3) -- ++(0,-0.2) node[anchor=south,scale=0.75]{$d_3$};
\draw (MUX.selin0) -- (MUX.selin0 -| 6.65,0) node[anchor=east,scale=0.75]{$s_0$};
\draw (MUX.selin1) -- (MUX.selin1 -| 6.65,0) node[anchor=east,scale=0.75]{$s_1$};
\draw (MUX.selout0) -- (MUX.selout0 -| 8.35,0) node[anchor=west,scale=0.75]{$s_0$};
\draw (MUX.selout1) -- (MUX.selout1 -| 8.35,0) node[anchor=west,scale=0.75]{$s_1$};
\draw (MUX.output) -- ++(0,0.2) node[anchor=north,scale=0.75]{$f$};
\node[decoder2to4,scale=0.85] (DEC) at (7.5,8.25) {Decoder};
\node[demux1to4,scale=0.85] (DEM) at (7.5,10.25) {Demux};
\node[encoder4to2,scale=0.85] (ENC) at (7.5,12.25) {Encoder};
%\node[comp,scale=0.85] (COM) at (7.5,14.25) {Comp};
\end{tikzpicture}
\caption{Lijst van component interfaces.}
\end{figure}
\begin{figure}[H]
\centering
\begin{tikzpicture}[circuit logic US,yscale=-1]
\draw[thick] (0,0) rectangle (15,12);
\draw (0,0) node[anchor=north west] {\Large Basispoorten};
\begin{scope}[xshift=2.5 cm,yshift=1.5 cm]
\draw (0,-0.75) node {\underline{NOT}};
\node[not gate] (N) at (-1.25,0) {};
\draw (N.output) -- ++(0.25,0) node[anchor=west]{$z$};
\draw (N.input) -- ++(-0.25,0) node[anchor=east]{$x$};
\end{scope}
\begin{scope}[xshift=7.5 cm,yshift=1.5 cm]
\draw (0,-0.75) node {\underline{AND}};
\node[and gate] (A) at (-1.25,0) {};
\draw (A.output) -- ++(0.25,0) node[anchor=west]{$z$};
\draw (A.input 1) -- ++(-0.25,0) node[anchor=east]{$x$};
\draw (A.input 2) -- ++(-0.25,0) node[anchor=east]{$y$};
\end{scope}
\begin{scope}[xshift=12.5 cm,yshift=1.5 cm]
\draw (0,-0.75) node {\underline{OR}};
\node[or gate] (O) at (-1.25,0) {};
\draw (O.output) -- ++(0.25,0) node[anchor=west]{$z$};
\draw (O.input 1) -- ++(-0.25,0) node[anchor=east]{$x$};
\draw (O.input 2) -- ++(-0.25,0) node[anchor=east]{$y$};
\end{scope}
\draw (5,0.25) -- ++(0,2.5);
\draw (10,0.25) -- ++(0,2.5);
\draw[thick] (0,3) node[anchor=north west] {\Large{Complexe Poorten}} -- (15,3);
\begin{scope}[xshift=2.5 cm,yshift=4.5 cm]
\draw (0,-0.75) node {\underline{NAND}};
\node[nand gate] (N) at (-1.25,0) {};
\draw (N.output) -- ++(0.25,0) node[anchor=west]{};
\draw (N.input 1) -- ++(-0.25,0) node[anchor=east]{$x$};
\draw (N.input 2) -- ++(-0.25,0) node[anchor=east]{$y$};
\end{scope}
\begin{scope}[xshift=7.5 cm,yshift=4.5 cm]
\draw (0,-0.75) node {\underline{NOR}};
\node[nor gate] (A) at (-1.25,0) {};
\draw (A.output) -- ++(0.25,0) node[anchor=west]{};
\draw (A.input 1) -- ++(-0.25,0) node[anchor=east]{$x$};
\draw (A.input 2) -- ++(-0.25,0) node[anchor=east]{$y$};
\end{scope}
\begin{scope}[xshift=12.5 cm,yshift=4.5 cm]
\draw (0,-0.75) node {\underline{XOR}};
\node[xor gate] (O) at (-1.25,0) {};
\draw (O.output) -- ++(0.25,0) node[anchor=west]{};
\draw (O.input 1) -- ++(-0.25,0) node[anchor=east]{$x$};
\draw (O.input 2) -- ++(-0.25,0) node[anchor=east]{$y$};
\end{scope}
\draw (5,3.25) -- ++(0,2.5);
\draw (10,3.25) -- ++(0,2.5);
\end{tikzpicture}
\caption{Samenvattend schema: poorten en componenten (deel 1)}
\end{figure}
\section{Conventies}
\section{Poorten}
\subsection{Basispoorten}
\subsection{Complexe Poorten}
\label{ss:appendixComplexePoorten}
\section{Componenten}
\subsection{Rekenkundige schakelingen}
\subsection{Geheugen schakelingen}
\subsection{Andere schakelingen}
\subsubsection{Multiplexer}
\begin{figure}
\centering
\subfigure[Multiplexer]{
\begin{tikzpicture}[circuit logic US,rotate=-90]
\node[or gate,inputs={normal,normal,normal,normal}] (O) at (0,0) {};
\draw (O.output) -- ++(0.5,0) node[anchor=north]{$f$};
\draw[dashed] (0.75,-1.75) -- (-2.75,-2.75) -- (-2.75,2.75)  to node[gray,rotate=-90,below,sloped]{MULTIPLEXER} (0.75,1.75) -- cycle;
\draw (-2.25,-3) node[anchor=east]{$s_0$} -- (-2.25,2);
\draw (-2.5,-3) node[anchor=east]{$s_1$} -- (-2.5,2);
\node[and gate,inputs={normal,inverted,inverted}] (A0) at (-1.5,1.5) {};
\draw (A0.input 1 -| -3.25,0) node[anchor=south]{$d_0$} -- (A0.input 1);
\draw (A0.input 2 -| -2.5,0) -- (A0.input 2);
\draw (A0.input 3 -| -2.25,0) -- (A0.input 3);
\node[and gate,inputs={normal,inverted,normal}] (A1) at (-1.5,0.5) {};
\draw (A1.input 1 -| -3.25,0) node[anchor=south]{$d_1$} -- (A1.input 1);
\draw (A1.input 2 -| -2.5,0) -- (A1.input 2);
\draw (A1.input 3 -| -2.25,0) -- (A1.input 3);
\node[and gate,inputs={normal,normal,invertedl}] (A2) at (-1.5,-0.5) {};
\draw (A2.input 1 -| -3.25,0) node[anchor=south]{$d_2$} -- (A2.input 1);
\draw (A2.input 2 -| -2.5,0) -- (A2.input 2);
\draw (A2.input 3 -| -2.25,0) -- (A2.input 3);
\node[and gate,inputs={normal,normal,normal}] (A3) at (-1.5,-1.5) {};
\draw (A3.input 1 -| -3.25,0) node[anchor=south]{$d_3$} -- (A3.input 1);
\draw (A3.input 2 -| -2.5,0) -- (A3.input 2);
\draw (A3.input 3 -| -2.25,0) -- (A3.input 3);
\draw (A0.output) -- ++(0.3,0) |- (O.input 1);
\draw (A1.output) -- ++(0.2,0) |- (O.input 2);
\draw (A2.output) -- ++(0.2,0) |- (O.input 3);
\draw (A3.output) -- ++(0.3,0) |- (O.input 4);
\end{tikzpicture}
}
\subfigure[Decoder]{
\begin{tikzpicture}[circuit logic US,rotate=-90]
\draw[dashed] (-0.75,-2.5) -- (-3,-2.5) -- (-3,2.5) to node[gray,rotate=-90,below,sloped]{DECODER} (-0.75,2.5) -- cycle;
\draw (-2.25,-2) -- (-2.25,2);
\draw (-2.5,-2) -- (-2.5,2);
\draw (-2.75,-3) node[anchor=east]{enable} -- (-2.75,2);
\draw (-3.25,1) node[anchor=south]{$a_0$} -- (-2.25,1);
\draw (-3.25,-1) node[anchor=south]{$a_1$} -- (-2.5,-1);
\node[and gate,inputs={normal,inverted,inverted}] (A0) at (-1.5,1.5) {};
\draw (A0.input 1 -| -2.75,0) -- (A0.input 1);
\draw (A0.input 2 -| -2.5,0) -- (A0.input 2);
\draw (A0.input 3 -| -2.25,0) -- (A0.input 3);
\draw (A0.output) -- ++(0.75,0) node[anchor=north]{$s_0$};
\node[and gate,inputs={normal,inverted,normal}] (A1) at (-1.5,0.5) {};
\draw (A1.input 1 -| -2.75,0) -- (A1.input 1);
\draw (A1.input 2 -| -2.5,0) -- (A1.input 2);
\draw (A1.input 3 -| -2.25,0) -- (A1.input 3);
\draw (A1.output) -- ++(0.75,0) node[anchor=north]{$s_1$};
\node[and gate,inputs={normal,normal,invertedl}] (A2) at (-1.5,-0.5) {};
\draw (A2.input 1 -| -2.75,0) -- (A2.input 1);
\draw (A2.input 2 -| -2.5,0) -- (A2.input 2);
\draw (A2.input 3 -| -2.25,0) -- (A2.input 3);
\draw (A2.output)-- ++(0.75,0) node[anchor=north]{$s_2$};
\node[and gate,inputs={normal,normal,normal}] (A3) at (-1.5,-1.5) {};
\draw (A3.input 1 -| -2.75,0) -- (A3.input 1);
\draw (A3.input 2 -| -2.5,0) -- (A3.input 2);
\draw (A3.input 3 -| -2.25,0) -- (A3.input 3);
\draw (A3.output) -- ++(0.75,0) node[anchor=north]{$s_3$};
\end{tikzpicture}
}
\subfigure[Demultiplexer]{
\begin{tikzpicture}[circuit logic US,rotate=-90]
\draw[dashed] (-0.75,-2.5) -- (-3,-2.5) -- (-3,2.5) to node[gray,rotate=-90,below,sloped]{DEMUX} (-0.75,2.5) -- cycle;
\draw (-2.75,-2) -- (-2.75,2);
\draw (-2.5,-3) node[anchor=east]{$s_1$} -- (-2.5,2);
\draw (-2.25,-3) node[anchor=east]{$s_0$} -- (-2.25,2);
\draw (-3.25,0) node[anchor=south]{$f$} -- (-2.75,0);
\node[and gate,inputs={normal,inverted,inverted}] (A0) at (-1.5,1.5) {};
\draw (A0.input 1 -| -2.75,0) -- (A0.input 1);
\draw (A0.input 2 -| -2.5,0) -- (A0.input 2);
\draw (A0.input 3 -| -2.25,0) -- (A0.input 3);
\draw (A0.output) -- ++(0.75,0) node[anchor=north]{$d_0$};
\node[and gate,inputs={normal,inverted,normal}] (A1) at (-1.5,0.5) {};
\draw (A1.input 1 -| -2.75,0) -- (A1.input 1);
\draw (A1.input 2 -| -2.5,0) -- (A1.input 2);
\draw (A1.input 3 -| -2.25,0) -- (A1.input 3);
\draw (A1.output) -- ++(0.75,0) node[anchor=north]{$d_1$};
\node[and gate,inputs={normal,normal,invertedl}] (A2) at (-1.5,-0.5) {};
\draw (A2.input 1 -| -2.75,0) -- (A2.input 1);
\draw (A2.input 2 -| -2.5,0) -- (A2.input 2);
\draw (A2.input 3 -| -2.25,0) -- (A2.input 3);
\draw (A2.output)-- ++(0.75,0) node[anchor=north]{$d_2$};
\node[and gate,inputs={normal,normal,normal}] (A3) at (-1.5,-1.5) {};
\draw (A3.input 1 -| -2.75,0) -- (A3.input 1);
\draw (A3.input 2 -| -2.5,0) -- (A3.input 2);
\draw (A3.input 3 -| -2.25,0) -- (A3.input 3);
\draw (A3.output) -- ++(0.75,0) node[anchor=north]{$d_3$};
\end{tikzpicture}}
\subfigure[Encoder]{
\begin{tikzpicture}[circuit logic US,rotate=-90]
\draw[dashed] (0,-3) -- (-3,-3) -- (-3,3) to node[gray,rotate=-90,below,sloped]{ENCODER} (0,3) -- cycle;
\coordinate (S0) at (-3,1.5);
\coordinate (S1) at (-3,0.5);
\coordinate (S2) at (-3,-0.5);
\coordinate (S3) at (-3,-1.5);
\coordinate (F0) at (0,1);
\coordinate (F1) at (0,-1);
\draw (-3.25,1.5) node[anchor=south]{$s_0$} -- (S0);
\draw (-3.25,0.5) node[anchor=south]{$s_1$} -- (S1);
\draw (-3.25,-0.5) node[anchor=south]{$s_2$} -- (S2);
\draw (-3.25,-1.5) node[anchor=south]{$s_3$} -- (S3);
\draw (F0) -- ++(0.5,0) node[anchor=north]{$f_0$};
\draw (F1) -- ++(0.5,0) node[anchor=north]{$f_1$};
\node[or gate,inputs={normal,normal,normal,normal},rotate=-90] (O1) at (-2.5,-2.15) {};
\draw (O1.output) -- ++(0,-0.5) node[anchor=east]{any};
\draw (S0) -| (O1.input 1);
\draw (S1) -| (O1.input 2);
\draw (S2) -| (O1.input 3);
\draw (S3) -| (O1.input 4);
\node[or gate] (O2) at (-1.25,1) {};
\draw (S2 -| O1.input 3) |- (O2.input 2);
\draw (S3 -| O1.input 4) |- (O2.input 1);
\draw (O2.output) -- (F0);
\node[and gate, inputs={normal,inverted}] (A1) at (-1.5,0) {};
\draw (A1.input 1 -| O1.input 2) -- (A1.input 1);
\draw (A1.input 2 -| O1.input 3) -- (A1.input 2);
\node[or gate] (O3) at (-0.5,-1) {};
\draw (O1.input 4 |- O3.input 2) -- (O3.input 2);
\draw (A1.output) -- ++(0.1,0) |- (O3.input 1);
\draw (O3.output) -- (F1);
\end{tikzpicture}}
\caption{Andere combinatorische schakelingen}
\end{figure}