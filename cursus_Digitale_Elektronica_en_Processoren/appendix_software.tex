\chapter{Softwarepaketten}
\chapterquote{Mensen die bezig zijn met software, zouden hun eigen hardware moeten bouwen.}{Alan Kay, Amerikaans informaticus (1940-)}
\begin{chapterintro}
Bij het schrijven van de cursus Digitale Elektronica en Processoren werden enkele softwarepaketten geschreven. Deze software laat toe aan de lezer om zelf oefeningen te maken, oplossingen te controleren en op een geautomatiseerde manier kleine combinatorische en sequenti\"ele te realiseren alsook een processor te bouwen. In dit hoofdstuk geven we een kort overzicht van deze software. Voor het \emph{Linux} besturingssysteem bestaat ook software die helpt bij het concreet realiseren van een schakeling op bijvoorbeeld een printplaat of het simuleren van een schakeling. Deze software wordt kort besproken in enkele secties.
\end{chapterintro}
\minitoc[n]
\section{Geschreven software}
\subsection{De software installeren}
De software is geschreven in Haskell en kan worden gedownload op volgend adres: \texttt{http://goo.gl/tkyilf}. De code staat onder git-subversiebeheer. Bijdragen aan de software wordt aangemoedigd. Men kan de software gebruiken door de \texttt{Makefile} te draaien. Dit doet men door in de desbetreffende map het commando \texttt{make} te typen. Standaard wordt de software niet ge\"installeerd.