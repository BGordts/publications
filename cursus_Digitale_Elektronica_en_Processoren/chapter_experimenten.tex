\chapter{Experimenten}
\chplab{experiments}
\chapterquote{Het leven is niets dan een experiment. Hoe meer je experimenteert, hoe beter.}{Ralph Waldo Emerson, Amerikaans dichter en filosoof (1803-1882)}
\begin{chapterintro}
In dit hoofdstuk beschrijven we enkele experimenten die we kunnen uitvoeren om praktische en nuttige digitale schakelingen te realiseren voor dagelijks gebruik.
\end{chapterintro}
\minitoc[n]
\section{Oneindig vermenigvuldigen}
\'E\'en van de voordeling van delen en vermenigvuldigen met constanten, is dat de overdracht of het lenen ook beperkt is. Wanneer we een binair getal bijvoorbeeld vermenigvuldigen met $3$, kan de carry nooit groter worden dan $4$. Wanneer we dit veralgemenen naar een vermenigvuldiging in het binair stelsel met $n$, kan de overdracht nooit groter worden dan $2\cdot n-1$. We kunnen dit uitbuiten om een sequenti\"ele vermenigvuldiger te maken die over bit-stromen met oneindige lengte werkt. We realiseren dus een schakeling waar als invoer een stroom aan bits en het bijbehorende kloksignaal binnenkomt, en als uitvoer een andere stroom bits geklokt volgens hetzelfde kloksignaal maar waarbij de bits het $k$-voud voorstellen van de inkomende bitstroom. We werken volgen big-endian encodering: de eerste bit die in de schakeling binnenkomt is altijd de minst significante.
\subsection{Benodigdheden}
Dit hangt grotendeels af van de constante $k$ waarmee we willen vermenigvuldigen. Voor een constante $k=5$ zijn de benodigdheden de volgende:
\begin{enumerate}
 \item $2$ leds;
 \item $3$ npn-transistoren;
 \item $1$ pnp-transistoren;
 \item $2$ drukschakelaars.
\end{enumerate}
Samen kosten deze componenten (zonder gereedschap, printplaten en soldeersel) ongeveer $\mbox{\euro}\ ??.??$\footnote{Gebaseerd op prijzen bij Conrad.}.
\subsection{Model}
In de inleiding hebben we de werking van het component niet gespecificeerd. Dit zullen we in deze sectie doen. Wanneer een getal bit-na-bit wordt ingelezen in het circuit, dient het circuit dit getal de vermenigvuldigen. We kunnen echter de eigenschap uitbuiten dat de overdracht die bij de vermenigvuldiging optreedt, altijd eindig zal zijn. Bij voorbeeld zullen we een vermenigvuldiging met $k=5$ modelleren.
\paragraph{}
\importtikzfigure{infinitemul-carry5}{Een toestandsdiagram voor $k=5$.}
De machine begint in een toestand waar er geen overdracht is, formeler $c=0$. Dit is de toestand $q_0$ op \figref{infinitemul-carry5}. Wanneer er nu een $0$ binnenkomt, is het resultaat logischerwijs $0$ en blijft de overdracht ook op $0$ staan. We trekken dus een lus vanuit $q_0$ met boog $0/0$. Wanneer er echter een $1$ op de invoer wordt aangelegd, dan weten we dat de uitvoer-bit ook $1$ moet worden, en dat we vanaf dat moment met overdracht $c=4$ moeten rekening houden. Nu hebben we alle invoer voor $q_0$ gemodelleerd, maar $q_4$ dienen we verder in te vullen. In het algemeen geldt de regel voor een vermenigvuldiging van $k$ dat we een boog van $q_i$ naar $q_j$ trekken met label $b_i/b_o$ zodat:
\begin{eqnarray}
b_i\in\accl{0,1}\\
r=\displaystyle\frac{i}{2}+k\cdot b_i\\
b_o=r\mod 2\\
j=r-b_o
\end{eqnarray}
Voor $q_4$ betekent dit dus dat we een boog $0/0$ naar $q_2$ trekken en een boog $1/1$ naar $q_6$. We dienen vervolgens ook bogen uit deze toestanden te trekken en dit tot er geen nieuwe toestanden meer opduiken. We bekomen vervolgens een diagram zoals op \figref{infinitemul-carry5}. We kunnen dit voor iedere $k$ doen.
\subsubsection{Implementatie-specificaties}
\paragraph{Reset}
Getallen hebben altijd een eindig aantal cijfers en bijgevolg ook een eindig aantal bits. We willen de schakeling dan ook niet noodzakelijk gebruiken om een product uit te rekenen over een oneindige sequentie bits. Maar wel op een snelle manier getallen van willekeurige lengte vermenigvuldigen. We dienen echter aan het circuit duidelijk te maken dat we een nieuw getal willen vermenigvuldigen. We kunnen dit op twee manieren doen:
\begin{enumerate}
 \item We lezen de resterende bits uit door telkens $0$ op de ingang aan te leggen tot het circuit zich terug in toestand $q_0$ bevindt, dit duurt hoogstens $\fun{\log_2}{n}+1$ stappen. Dit is ook in principe de correcte methode omdat het resultaat van een vermenigvuldiging nu eenmaal een getal met meer bits kan opleveren. Een nadeel is echter wanneer we dit circuit in een processor zouden gebruiken, waarbij de registers een beperkt aantal bits hebben, we meestal ge\"interesseerd zijn in de $n$ minst significante bits\footnote{Analoog met de overloop bij bijvoorbeeld een optelling.}, en we dus klokflanken ``verspillen''.
 \item We voorzien een reset-ingang die de schakeling terug in de grondtoestand brengt. De meest significante bits gaan verloren.
\end{enumerate}
In dit voorbeeld kiezen we voor de eerste variant. Dit houdt immers ook het circuit eenvoudig en goedkoop.
\paragraph{In- en Uitvoer}
Normaal kunnen we een dergelijk circuit gebruiken in bijvoorbeeld een processor, of kunnen we het in een ethernet-kaart implementeren wanneer data moet worden aangepast. Wanneer we echter de schakeling op zichzelf willen implementeren, dienen we deze te kunnen testen. Bijgevolg zullen we ook invoer-uitvoer voorzien. Communicatie tussen mens en modules wordt meestal gerealiseerd met behulp van leds en schakelaars. Om duidelijk te maken welk signaal we zelf zullen aanleggen zullen we ook leds aan de ingang gebruiken. Een probleem die zich soms stelt is dat er een groot aantal bits aan de invoer kunnen worden aangelegd, en het dus niet eenvoudig is om met drukschakelaars een bepaalde configuratie aan te leggen. In dat geval kunnen we gebruik maken van een tuimelschakelaar. Een tuimelschakelaar kan worden aangekocht, maar ook ge\"implementeerd met behulp van een drukschakelaar, twee weerstanden, een pnp- en npn-transistor. Een dergelijke implementatie staat beschreven in \figref{toggleswitch-transistor}.
\importtikzfigure{toggleswitch-transistor}{Implementatie van een tuimelschakelaar.}
\subsection{Realisatie}
\subsubsection{Toestanden en soorten schakeling}
In de vorige subsectie hebben we besproken hoe we een diagram kunnen opstellen voor de overdracht. Dit diagram is precies het diagram voor de Mealy-machine om onze schakeling te realiseren. We kunnen het diagram dus rechtstreeks gebruiken om een toestandstabel op te stellen zoals in \tblref{infinitemul-state5}.
\importtabulartable{infinitemul-state5}{Toestandstabel van de oneindige vermenigvuldiger.}
\subsubsection{Toestandscodering}
We dienen nu enkel nog een codering te bedenken om de verschillende toestanden op te slaan. Een logische manier kan zijn om $q_i$ op te slaan als het binaire equivalent van $i/2$. Dit biedt bovendien in dit geval vrij mooie overgangen: bij twee overgangen verandert er geen bit, bij vijf overgangen slechts \'e\'en bit, \'e\'en overgang brengt twee veranderingen teweeg en \'e\'en overgang drie. We kunnen echter $q_8$ ook encoderen als $111$. Dit zorgt ervoor dat er geen overgangen met drie bits meer zijn. Dit is natuurlijk heuristisch: er is geen echte garantie dat dit effectief tot een goedkopere implementatie zal leiden, al wordt het wel verondersteld. De coderingstabel wordt dan zoals voorgesteld als in \tblref{infinitemul-code5}
\importtabulartable{infinitemul-code5}{Coderingstabel van de oneindige vermenigvuldiger.}
\subsubsection{Type flipflop}
Het is niet echt duidelijk welk type flipflop hier nodig is. We zullen de schakeling met een D-flipflop implementeren.
\importtabulartable{infinitemul-combine5}{De oneindige vermenigvuldiger.}
Wanneer we D-flipflops gebruiken bekomen dienen we combinatorische schakelingen te implementeren zoals op \tblref{ininitemul-combine5} een grafische voorstelling met behulp van Karnaugh-kaarten staat op \figref{infinitemul-karnaugh5}.
\importtikzfigure{infinitemul-karnaugh5}{Karnaugh-kaarten voor de oneindige vermenigvuldiger.}
\subsubsection{Implementatie}
Tot slot dienen we enkel nog de schakeling te implementeren. We plaatsen dus 3 flipflops op de europrintplaat en implementeren de logica tussen deze poorten zoals op \figref{infinitemul-impl5}.
\importtikzfigure{infinitemul-impl5}{Mogelijke implementatie voor de oneindige vermenigvuldiger.}
\subsubsection{Printplaat}
\section{Fietslicht}
Men kan vandaag fietslichten kopen met verschillende knippermodes. Bij wijze van experiment zullen we een knipperlicht implementeren gebaseerd op een ledlamp van \emph{Hema}.
\paragraph{}
Het fietslicht bestaat uit drie leds en kent 4 standen: uit, links-rechts, knipperen en aan. \figref{bicycle} beschrijft de verschillende modi. Men verandert de modus door een drukschakelaar in te duwen. In het geval dat men in knipper-modus staat, is de frequentie waarmee men van toestand van led verandert, slechts de helft.
\importtikzfigure{bicycle}{De verschillende standen van een \emph{Hema}-fietslicht}
\subsection{Benodigdheden}
Volgende componenten heeft men nodig om deze schakeling te implementeren:
\begin{itemize}
 \item $3$ leds
 \item $3$ NPN transistoren. Bij voorkeur \verb+BC547C+
 \item $3$ weerstanden van $1~\mbox{k}\Omega$.
 \item $1$ drukschakelaar.
 \item $1$ europrintplaat.
\end{itemize}
Alle componenten zijn verkrijgbaar bij Conrad. De richtprijs is ongeveer $\mbox{\euro}\ ??.??$
\subsection{Implementatie-specificaties}
In deze subsectie bespreken we hoe we de schakeling zullen implementeren. Dit betekent dat we onder meer de in- en uitvoer vastleggen.
\paragraph{In- en Uitvoer}
Naast de drie leds van het fietslicht is er geen uitvoer: we gaan er immers vanuit dat een gebruiker door op de drukschakelaar te duwen de werking van het licht zelf kan leren. Als invoer voorzien we \'e\'en enkele drukschakelaar. We realiseren de in- en uitvoer aan de hand van de schakeling op \figref{bicycle-io}.
\importtikzfigure{bicycle-io}{De IO-module van het fietslicht}
\paragraph{Toestanden en soort schakeling}
We dienen ook te beslissen wat soort schakeling we zullen implementeren. Vermits het aantal toestanden en de mogelijkheden nogal beperkt is, is het implementeren van een processor niet nodig: we kunnen deze schakeling met een eenvoudige sequenti\"ele schakeling realiseren.
\paragraph{}
Verder dienen we te beslissen of we een synchrone of asynchrone schakeling zullen bouwen. Er speelt natuurlijk een tijdsaspect: we willen niet dat we meteen in de volgende toestand geraken wanneer een led aan of uitgaat. Anders is het verschil tussen knipperen en aan moeilijk te zien. We kunnen echter ook het kloksignaal als een ingangsignaal beschouwen. Dit laatste pleit voor een asynchrone schakeling: er is sprake van twee soorten ingangen: de schakelaar van het fietslicht die aangeeft dat we een volgende toestand willen kiezen, en het kloksignaal die wanneer het opkomt betekent dat we de bij links-rechts en knipperen leds aan of uit moeten zetten. Een asynchrone schakeling is echter moeilijker te implementeren. Wanneer we voor een synchrone schakeling opteren zal de gebruiker wanneer hij de schakelaar induwt moeten wachten tot het volgende kloksignaal voor de volgende stand van het fietslicht wordt geactiveerd. Het tijdverlies valt echter goed mee en is meestal niet levensbedreigend. In dit boek kiezen we daarom voor een synchrone schakeling.
\paragraph{}
Verder rest ons nog het aangeven van de verschillende toestand en wat we doen wanneer een gebruiker de schakelaar indrukt. In principe zijn hier drie mogelijkheden. Iedere keer wanneer de gebruiker de schakelaar indrukt, dan komen we in de begintoestand van de stand. We kunnen ervoor opteren terecht te komen in een toestand die bijvoorbeeld globaal bepaald is: wanneer er $k$ ticks geweest zijn komen we in de $k\mod n$-de toestand terecht, met $n$ het aantal toestanden van die stand. Een laatste optie is dat het niet uitmaakt, zolang we maar in een toestand terechtkomen die geldig is voor die stand. We opteren in deze cursus voor de eerste optie, dit is ook zo bij het echte fietslicht van \emph{Hema}.
\subsection{Realisatie}
\subsubsection{Toestandstabel}
Op basis van de specificaties zullen we een toestandstabel opstellen. De uitgangssignalen van de sequenti\"ele schakeling zijn drie bits: voor elke led betekent $1$ dat de overeenkomstige led zal branden, in het geval van $0$ brandt de led natuurlijk niet. Verder voorzien we \'e\'en ingangssignaal: dit signaal is $1$ wanneer de gebruiker de schakelaar indrukt. In het geval de schakelaar niet ingedrukt is, is het signaal $0$. Dit levert ons de toestandstabel op zoals \tblref{bicycle-state0}.
\importtabulartable{bicycle-state0}{Toestandstabel van het fietslicht.}
In de tabel hebben we de toestanden geannoteerd met de stand in subscript ($0$ uit, $1$ links-rechts, $2$ knipperen, $3$ aan). In stand $2$ moest de frequentie gehalveerd worden. Dit hebben we opgelost door vier toestanden te voorzien: twee wanneer alle leds uit zijn, en twee wanneer alle leds branden. Op die manier kost het twee klokflanken vooraleer de leds veranderen.
\subsubsection{Toestandscodering}
In totaal hebben we $18$ toestanden nodig om de schakeling te realiseren. Een toestand zal dus worden ge\"encodeerd op minstens $5$ bits. We kunnen er ook voor opteren om de toestand van de leds rechtstreeks in het geheugen te encoderen. In dat geval hebben we drie bits nodig om de toestand van de leds weer te geven en vier overige bits\footnote{Het komt immers $9$ keer voor dat alle leds uit zijn. Om een onderscheid te maken tussen de verschillende toestanden hebben we dus minstens $4$ extra bits nodig}. Dit zou neerkomen op zeven bits wat vrij duur is. We zullen de toestand voorstellen aan de hand van $5$ bits.
\paragraph{}
Nadat we het aantal bits bepaald heb dienen we nog voor elke toestand een codering te voorzien. Hier proberen we de transitie- en uitvoerlogica mee te vereenvoudigen. We zien dat de links-rechts staat de meeste toestanden gebruikt. We zullen dit dan ook encoderen op zo'n manier dat de eerste bit $1$ aanwijst en de overige bits werken volgens het principe van een $4$-bit gray-teller. De overige standen zullen allemaal met een $0$ beginnen. Ook hier proberen we enige logica in de codering te stoppen. Zo spreekt het voor zich dat we de uit-toestand code $00000_a$ toewijzen: we kunnen logica voorzien die wanneer bepaalde bits passief zijn, alle leds meteen niet branden. De eerste en tweede toestand van de knipper-stand geven we dan weer de coderingen $01000_n$ en $01001_n$. Alle leds moeten branden in de laatste twee toestanden van de knipper-stand en de enige toestand van de aan-stand. We coderen deze toestanden respectievelijk met $01101_p$, $01100_q$ en $01110_r$. De volledige coderingstabel staat beschreven in \tblref{bicycle-state1}.
\importtabulartable{bicycle-state1}{Coderingstabel van het fietslicht.}
\subsubsection{Type flipflop}
Na het opstellen van de coderingstabel dienen we het type flipflop uit te kiezen.
\subsubsection{Implementatie in poortlogica}

\section{Tic tac toe-machine}
\emph{Tic tac toe} is een spel waarbij twee spelers beurtelings een zet spelen. Tijdens een zet schrijft een speler een cirkel of een kruis op het scorebord, een $3\times 3$ raster. De eerste speler zet altijd een cirkel, de tweede een kruis. Elke speler probeert een situatie te bekomen waarbij drie van de eigen symbolen op een rij staan. Het spel is simpel en kan daarom goedkoop op zelf in electronica ge\"implementeerd worden. We zullen bij dit experiment twee schakelingen ontwerpen en realiseren: een schakeling zodat twee personen tegen elkaar spelen, en een schakeling waarbij het ook mogelijk is om te spelen tegen een artifici\"ele intelligentie\footnote{Deze AI-bot zal perfect spelen: als eerste speler zal hij altijd winnen}.
\subsection{Benodigdheden}
Volgende componenten heeft men nodig om deze schakeling te implementeren.
\begin{enumerate}
 \item $22$ leds waaronder $11$ rode en $11$ groene.
 \item $11$ NPN transistoren. Bij voorkeur \verb+BC547C+.
 \item $11$ weerstanden van $1~\mbox{k}\Omega$.
 \item $6$ drukschakelaars.
 \item $1$ normale schakelaar.
 \item $1$ europrintplaat.
\end{enumerate}
Alle componenten zijn verkrijgbaar bij Conrad. De richtprijs is ongeveer $\mbox{\euro}\ ??.??$
\subsection{Implementatie-specificaties}
Alvorens we dit spel kunnen implementeren, zullen we eerst moeten specificeren hoe we bijvoorbeeld met de gebruiker(s) gaan communiceren, hoe we de toestand van het bord gaan voorstellen, hoe het spel verloopt, enzovoort.
\paragraph{Uitvoer}
De uitvoer is een $3\times 3$ bord waar op elke tegel in principe een cirkel of kruis kan worden geplaatst. Dit geeft dus ongeveer\footnote{Niet alle toestanden zijn mogelijk: wanneer bijvoorbeeld de tweede speler wint, is er altijd \'e\'en vakje niet toegekend. Een situatie waarbij alle vakjes zijn opgevuld en er drie kruiss op \'e\'en rij staan is bijgevolg niet mogelijk.} $3^9$ toestanden. De uitvoeren realiseren we aan de hand van leds. Op elk vakje brengen we twee leds aan, bijvoorbeeld een rode en groene. We maken de afspraak dat indien de rode led brandt, er een cirkel op het respectievelijke vak staat. Wanneer de groene led brandt stelt dit een kruis voor. Wanneer geen van de twee leds brandt is het vakje leeg. In het vorige hoofdstuk hebben we reeds beargumenteerd dat men niet zomaar leds aan de uitgang van een poort kan schakelen. We voorzien dus een relay-mechanimsme met behulp van transistoren\footnote{We kunnen in principe ook echte relays gebruiken. Leds hebben echter een laag verbruik waardoor de taak ook door transistoren kan worden uitgevoerd.}. We dienen dus een schakeling te realiseren zoals op \figref{ledtttmatrix}.
\paragraph{}
Om te communiceren met de spelers is meer hardware nodig. Zo is het nuttig dat de schakeling aangeeft wie aan zet is, of de gespeelde zet legaal is en wie er gewonnen heeft. Daarom voorzien we nog twee leds: een rode en een groene. Wanneer \'e\'en rode let brandt, is de eerste speler aan zet, in het geval van de groene led speelt de tweede speler. Wanneer een speler gewonnen is knippert de overeenkomstige led. Wanneer een speler een onmogelijke zet speelt, branden beide leds voor een korte periode allebei.
\importtikzfigure{ledtttmatrix}{Een led-matrix en invoer-component voor het tic tac toe spel.}
\paragraph{Invoer}
We kunnen ervoor opteren om per vak een schakelaar te voorzien. Wanneer een speler dan op de schakelaar drukt, zal de schakeling de overeenkomstige led laten branden. Dit betekent echter dat we $9$ schakelaars moeten voorzien. We hebben daarom besloten om aan de rand van de twee dimensies elk $3$ schakelaars te voorzien. $3$ schakelaars laten dus toe om de rij te specificeren, met de overige $3$ kan een gebruiker de kolom aangeven. Ook vanuit educatief standpunt is deze beslissing positief: men zal immers meer logica moeten voorzien om de invoer te interpreteren.
\paragraph{Spelverloop}
