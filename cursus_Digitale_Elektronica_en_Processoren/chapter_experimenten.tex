\chapter{Experimenten}
\chplab{experiments}
\chapterquote{Het leven is niets dan een experiment. Hoe meer je experimenteert, hoe beter.}{Ralph Waldo Emerson, Amerikaans dichter en filosoof (1803-1882)}
\begin{chapterintro}
In dit hoofdstuk beschrijven we enkele experimenten die we kunnen uitvoeren om praktische en nuttige digitale schakelingen te realiseren voor dagelijks gebruik.
\end{chapterintro}
\minitoc[n]
\section{Fietslicht}
Men kan vandaag fietslichten kopen met verschillende knippermodes. Bij wijze van experiment zullen we een knipperlicht implementeren gebaseerd op een ledlamp van \emph{Hema}.
\paragraph{}
Het fietslicht bestaat uit drie leds en kent 4 standen: uit, links-rechts, knipperen en aan. \figref{bicycle} beschrijft de verschillende modi. Men verandert de modus door een drukschakelaar in te duwen. In het geval dat men in knipper-modus staat, is de frequentie waarmee men van toestand van led verandert, slechts de helft.
\importtikzfigure{bicycle}{De verschillende standen van een \emph{Hema}-fietslicht}
\subsection{Benodigdheden}
Volgende componenten heeft men nodig om deze schakeling te implementeren:
\begin{itemize}
 \item $3$ leds
 \item $3$ NPN transistoren. Bij voorkeur \verb+BC547C+
 \item $3$ weerstanden van $1~\mbox{k}\Omega$.
 \item $1$ drukschakelaar.
 \item $1$ europrintplaat.
\end{itemize}
Alle componenten zijn verkrijgbaar bij Conrad. De richtprijs is ongeveer $\mbox{\euro}\ ??.??$
\subsection{Implementatie-specificaties}
In deze subsectie bespreken we hoe we de schakeling zullen implementeren. Dit betekent dat we onder meer de in- en uitvoer vastleggen.
\paragraph{In- en Uitvoer}
Naast de drie leds van het fietslicht is er geen uitvoer: we gaan er immers vanuit dat een gebruiker door op de drukschakelaar te duwen de werking van het licht zelf kan leren. Als invoer voorzien we \'e\'en enkele drukschakelaar. We realiseren de in- en uitvoer aan de hand van de schakeling op \figref{bicycle-io}.
\importtikzfigure{bicycle-io}{De IO-module van het fietslicht}
\paragraph{Toestanden en soort schakeling}
We dienen ook te beslissen wat soort schakeling we zullen implementeren. Vermits het aantal toestanden en de mogelijkheden nogal beperkt is, is het implementeren van een processor niet nodig: we kunnen deze schakeling met een eenvoudige sequenti\"ele schakeling realiseren.
\paragraph{}
Verder dienen we te beslissen of we een synchrone of asynchrone schakeling zullen bouwen. Er speelt natuurlijk een tijdsaspect: we willen niet dat we meteen in de volgende toestand geraken wanneer een led aan of uitgaat. Anders is het verschil tussen knipperen en aan moeilijk te zien. We kunnen echter ook het kloksignaal als een ingangsignaal beschouwen. Dit laatste pleit voor een asynchrone schakeling: er is sprake van twee soorten ingangen: de schakelaar van het fietslicht die aangeeft dat we een volgende toestand willen kiezen, en het kloksignaal die wanneer het opkomt betekent dat we de bij links-rechts en knipperen leds aan of uit moeten zetten. Een asynchrone schakeling is echter moeilijker te implementeren. Wanneer we voor een synchrone schakeling opteren zal de gebruiker wanneer hij de schakelaar induwt moeten wachten tot het volgende kloksignaal voor de volgende stand van het fietslicht wordt ge\"activeerd. Het tijdverlies valt echter goed mee en is meestal niet levensbedreigend. In dit boek kiezen we daarom voor een synchrone schakeling.
\paragraph{}
Verder rest ons nog het aangeven van de verschillende toestand en wat we doen wanneer een gebruiker de schakelaar indrukt. In principe zijn hier drie mogelijkheden. Iedere keer wanneer de gebruiker de schakelaar indrukt, dan komen we in de begintoestand van de stand. We kunnen ervoor opteren terecht te komen in een toestand die bijvoorbeeld globaal bepaald is: wanneer er $k$ ticks geweest zijn komen we in de $k\mod n$-de toestand terecht, met $n$ het aantal toestanden van die stand. Een laatste optie is dat het niet uitmaakt, zolang we maar in een toestand terechtkomen die geldig is voor die stand. We opteren in deze cursus voor de eerste optie, dit is ook zo bij het echte fietslicht van \emph{Hema}.
\subsection{Realisatie}
\subsubsection{Toestandstabel}
Op basis van de specificaties zullen we een toestandstabel opstellen. De uitgangsignalen van de sequenti\"ele schakeling zijn drie bits: voor elke led betekent $1$ dat de overeenkomstige led zal branden, in het geval van $0$ brandt de led natuurlijk niet. Verder voorzien we \'e\'en ingangsignaal: dit signaal is $1$ wanneer de gebruiker de schakelaar indrukt. In het geval de schakelaar niet ingedrukt is, is het signaal $0$. Dit levert ons de toestandstabel op zoals \tblref{bicycle-state0}.
\importtabulartable{bicycle-state0}{Toestandstabel van het fietslicht.}
In de tabel hebben we de toestanden geannoteerd met de stand in subscript ($0$ uit, $1$ links-rechts, $2$ knipperen, $3$ aan). In stand $2$ moest de frequentie gehalveerd worden. Dit hebben we opgelost door vier toestanden te voorzien: twee wanneer alle leds uit zijn, en twee wanneer alle leds branden. Op die manier kost het twee klokflanken vooraleer de leds veranderen.
\subsubsection{Toestandscodering}
In totaal hebben we $18$ toestanden nodig om de schakeling te realiseren. Een toestand zal dus worden ge\"encodeerd op minstens $5$ bits. Door de toestanden zo te encoderen verwachten we echter complexe logica nodig te hebben om $5$-bit toestanden om te zetten in aansturing voor de leds. We kunnen echter ook de toestand van de leds gebruiken als gedeeltelijke toestand. Verder dienen we dan wel nog bits te alloceren die bijhouden in welke stand het fietslicht staat of bijvoorbeeld in de hoeveelste toestand bij de stand links-rechts we staan. We kunnen dit realiseren door drie bits te alloceren.
\section{Tic tac toe-machine}
\emph{Tic tac toe} is een spel waarbij twee spelers beurtelings een zet spelen. Tijdens een zet schrijft een speler een cirkel of een kruis op het scorebord, een $3\times 3$ raster. De eerste speler zet altijd een cirkel, de tweede een kruis. Elke speler probeert een situatie te bekomen waarbij drie van de eigen symbolen op een rij staan. Het spel is simpel en kan daarom goedkoop op zelf in electronica ge\"implementeerd worden. We zullen bij dit experiment twee schakelingen ontwerpen en realiseren: een schakeling zodat twee personen tegen elkaar spelen, en een schakeling waarbij het ook mogelijk is om te spelen tegen een artifici\"ele intelligentie\footnote{Deze AI-bot zal perfect spelen: als eerste speler zal hij altijd winnen}.
\subsection{Benodigdheden}
Volgende componenten heeft men nodig om deze schakeling te implementeren.
\begin{enumerate}
 \item $22$ leds waaronder $11$ rode en $11$ groene.
 \item $11$ NPN transistoren. Bij voorkeur \verb+BC547C+.
 \item $11$ weerstanden van $1~\mbox{k}\Omega$.
 \item $6$ drukschakelaars.
 \item $1$ normale schakelaar.
 \item $1$ europrintplaat.
\end{enumerate}
Alle componenten zijn verkrijgbaar bij Conrad. De richtprijs is ongeveer $\mbox{\euro}\ ??.??$
\subsection{Implementatie-specificaties}
Alvorens we dit spel kunnen implementeren, zullen we eerst moeten specificeren hoe we bijvoorbeeld met de gebruiker(s) gaan communiceren, hoe we de toestand van het bord gaan voorstellen, hoe het spel verloopt, enzovoort.
\paragraph{Uitvoer}
De uitvoer is een $3\times 3$ bord waar op elke tegel in principe een cirkel of kruis kan worden geplaatst. Dit geeft dus ongeveer\footnote{Niet alle toestanden zijn mogelijk: wanneer bijvoorbeeld de tweede speler wint, is er altijd \'e\'en vakje niet toegekend. Een situatie waarbij alle vakjes zijn opgevuld en er drie kruiss op \'e\'en rij staan is bijgevolg niet mogelijk.} $3^9$ toestanden. De uitvoeren realiseren we aan de hand van leds. Op elk vakje brengen we twee leds aan, bijvoorbeeld een rode en groene. We maken de afspraak dat indien de rode led brandt, er een cirkel op het respectievelijke vak staat. Wanneer de groene led brandt stelt dit een kruis voor. Wanneer geen van de twee leds brandt is het vakje leeg. In het vorige hoofdstuk hebben we reeds beargumenteerd dat men niet zomaar leds aan de uitgang van een poort kan schakelen. We voorzien dus een relay-mechanimsme met behulp van transistoren\footnote{We kunnen in principe ook echte relays gebruiken. Leds hebben echter een laag verbruik waardoor de taak ook door transistoren kan worden uitgevoerd.}. We dienen dus een schakeling te realiseren zoals op \figref{ledtttmatrix}.
\paragraph{}
Om te communiceren met de spelers is meer hardware nodig. Zo is het nuttig dat de schakeling aangeeft wie aan zet is, of de gespeelde zet legaal is en wie er gewonnen heeft. Daarom voorzien we nog twee leds: een rode en een groene. Wanneer \'e\'en rode let brandt, is de eerste speler aan zet, in het geval van de groene led speelt de tweede speler. Wanneer een speler gewonnen is knippert de overeenkomstige led. Wanneer een speler een onmogelijke zet speelt, branden beide leds voor een korte periode allebei.
\importtikzfigure{ledtttmatrix}{Een led-matrix en invoer-component voor het tic tac toe spel.}
\paragraph{Invoer}
We kunnen ervoor opteren om per vak een schakelaar te voorzien. Wanneer een speler dan op de schakelaar drukt, zal de schakeling de overeenkomstige led laten branden. Dit betekent echter dat we $9$ schakelaars moeten voorzien. We hebben daarom besloten om aan de rand van de twee dimensies elk $3$ schakelaars te voorzien. $3$ schakelaars laten dus toe om de rij te specificeren, met de overige $3$ kan een gebruiker de kolom aangeven. Ook vanuit educatief standpunt is deze beslissing positief: men zal immers meer logica moeten voorzien om de invoer te interpreteren.
\paragraph{Spelverloop}
