\documentclass{book}
\title{Definitions and Theorems in Computer Science}
\author{Willem Van Onsem\\\url{https://github.com/KommuSoft/publications}}
\date{Version 0.1}
\usepackage{amsthm,amsfonts,makeidx,draftwatermark,amssymb,MnSymbol}
\usepackage[hidelinks,pdfborder=0 0 0,pdfpagelabels,plainpages=false,linktocpage=false,pdfcreator={LaTeX}]{hyperref}
\SetWatermarkScale{7}
\usepackage[cm]{fullpage}
\makeindex
\newcommand{\fun}[2]{\ensuremath{#1\left(#2\right)}}
\newcommand{\compclass}[1]{\textsc{\textbf{#1}}}
\newcommand{\bint}[4]{\ensuremath{\displaystyle\int_{#3}^{#4}{#1\ d#2}}}
\newcommand{\cd}{\ensuremath{\top\!\!\!\top}}
\newcommand{\ci}{\ensuremath{\bot\!\!\!\bot}}
\newcommand{\ffun}[1]{\fun{f}{#1}}
\newcommand{\gfun}[1]{\fun{g}{#1}}
\newcommand{\pfun}[1]{\fun{p}{#1}}
\newcommand{\pcond}[2]{\pfun{#1|#2}}
\newcommand{\txIf}{\ensuremath{\mbox{If }}}
\newcommand{\txthen}{\ensuremath{\mbox{ then }}}
\newcommand{\txwhereas}{\ensuremath{\mbox{ whereas }}}
\newcommand{\funsig}[3]{\ensuremath{#1:#2 \rightarrow #3}}
\newcommand{\abs}[1]{\ensuremath{\left|#1\right|}}
\newcommand{\dtime}[1]{\fun{\compclass{DTIME}}{#1}}
\newcommand{\ntime}[1]{\fun{\compclass{NTIME}}{#1}}
\newcommand{\bigoh}[1]{\fun{O}{#1}}
\newcommand{\bigomega}[1]{\fun{\Omega}{#1}}
\newcommand{\smallomega}[1]{\fun{\omega}{#1}}
\newcommand{\bigtheta}[1]{\fun{\Theta}{#1}}
\newcommand{\binary}{\ensuremath{\left\{0,1\right\}}}
\newcommand{\binarystrings}{\ensuremath{\left\{0,1\right\}^*}}
\newcommand{\btobfun}[1]{\funsig{#1}{\binarystrings}{\binarystrings}}
\newcommand{\btoblfun}[1]{\funsig{#1}{\binarystrings}{\binary}}
\newcommand{\smalloh}[1]{\fun{o}{#1}}
\newcommand{\gcdf}[1]{\fun{\mbox{gcd}}{#1}}
\newcommand{\phif}[1]{\fun{\phi}{#1}}
\newcommand{\ordf}[1]{\fun{\mbox{ord}}{#1}}
\newcommand{\neigh}[1]{\fun{\calN}{#1}}
\newcommand{\term}[1]{{\bf #1}\index{#1}}
\newcommand{\perm}[1]{\ensuremath{\mbox{Perm}^{#1}}}
\newcommand{\termor}[2]{{\bf #1} (also called {\bf #2})\index{#1}\index{#2|see{#1}}}
\newcommand{\termabbrev}[2]{{\bf #1} (sometimes abbreviated as {\bf #2})\index{#1}\index{#2|see{#1}}}
\newcommand{\argmax}[2][]{\ensuremath{\displaystyle\mbox{argmax}_{#1}\left(#2\right)}}
\newcommand{\calD}{\ensuremath{\mathcal{D}}}
\newcommand{\calS}{\ensuremath{\mathcal{S}}}
\newcommand{\calN}{\ensuremath{\mathcal{N}}}
\newcommand{\calX}{\ensuremath{\mathcal{X}}}
\newcommand{\calY}{\ensuremath{\mathcal{Y}}}
\newcommand{\calZ}{\ensuremath{\mathcal{Z}}}
\newcommand{\hats}{\ensuremath{\hat{s}}}
\newcommand{\soplus}{\ensuremath{\left\langle S,\oplus\right\rangle} }
\newcommand{\sotimes}{\ensuremath{\left\langle S,\otimes\right\rangle} }
\newcommand{\soplusotimes}{\ensuremath{\left\langle S,\oplus,\otimes\right\rangle} }
%\newcommand{\mod}{\ensuremath{\mbox{ mod }}}
\newcommand{\NNN}{\ensuremath{\mathbb{N}}}
\newcommand{\PPP}{\ensuremath{\mathbb{P}}}
\newcommand{\QQQ}{\ensuremath{\mathbb{Q}}}
\newcommand{\RRR}{\ensuremath{\mathbb{R}}}
\newcommand{\ZZZ}{\ensuremath{\mathbb{Z}}}
\newcommand{\ZZZstarn}{\ensuremath{\ZZZ^*_n}}
\newcommand{\slHALT}{\mbox{\sl HALT}}
\newcommand{\slUC}{\mbox{\sl UC}}
\newtheorem{defi}{Definition}
\newtheorem{theo}{Theorem}
\begin{document}
\begin{titlepage}
\maketitle
\end{titlepage}
\tableofcontents
\newpage
\chapter{Definitions}

\section{Mathematics and Formalisms}

\subsection{Operator Properties}

\begin{defi}[Commutative operation]
A binary operation $\oplus$ is \term{commutative} if $a\oplus b = b\oplus a$ for all $a,b\in S$.\cite{Oppliger:2011:CC:2049860}
\end{defi}

\begin{defi}[Associative operation]
A binary operation $\oplus$ is \term{associative} if $a\oplus\left(b\oplus c\right) = \left(a\oplus b\right)\oplus c$ for all $a,b,c\in S$.\cite{Oppliger:2011:CC:2049860}
\end{defi}

\begin{defi}[Semi-associative]
A binary operator $\otimes$ is said to be \term{semi-associative} if there is an associative operator $\oplus$ such that for any $a$, $b$, $c$, $\left(a\otimes b\right)\otimes c = a\otimes\left(b\oplus c\right)$.\cite{conf/europar/MatsuzakiHT03}
\end{defi}

\begin{defi}[Quasi-associative]
A binary operator $\oplus$ is said to be \term{quasi-associative} if there is a semi-associative operator $\otimes$ and a function $f$ such that for any $a$, $b$, $a\oplus b = a\otimes f b$.\cite{conf/europar/MatsuzakiHT03}
\end{defi}

\begin{defi}[Bi-quasi-associative]
A ternary operator $f$ is said to be \term{bi-quasi-associative} if there is a semi-associative operator $\otimes$ and two functions $f_L'$, $f_R'$ such that for any $l$, $n$, $r$, $f l n r = l \otimes f_L' n r = r \otimes f_R' n l$. We can fix a bi-quasi-associative operator $f$ by providing $\otimes$, $\oplus$ (associative operator for $\otimes$), $f_L'$ and $f_R'$, therefore, we will write $f$ with 4-tuple as $f\equiv \left[\left[\otimes,\oplus, f_L', f_R' \right]\right]$.\cite{conf/europar/MatsuzakiHT03}
\end{defi}

\subsection{Algebraic Structures}

\begin{defi}[Left identity element] Let $S$ be a set and $\oplus$ a binary operation on $S$. An element $e\in S$ is called \term{left identity element} if $e\oplus a = a$ for all $a\in S$.\cite{Oppliger:2011:CC:2049860}
\end{defi}

\begin{defi}[Right identity element] Let $S$ be a set and $\oplus$ a binary operation on $S$. An element $e\in S$ is called \term{right identity element} if $a\oplus e = a$ for all $\in S$.\cite{Oppliger:2011:CC:2049860}
\end{defi}

\begin{defi}[Identity element] Let $S$ be a set and $\oplus$ a binary operation on $S$. An element $e\in S$ is called \termor{identity element}{neutral element} if it is both a left identity element and a right identity element (i.e., $e\oplus a = a\oplus e = a$ for all $a\in S$).
\cite{Oppliger:2011:CC:2049860}
\end{defi}

\begin{defi}[Inverse element] Let $S$ be a set, $\oplus$ be a binary operation with an identity element $e$, and $a$ be an element of $S$. If there exists an element $b\in S$ with $a\oplus b = b\oplus a = e$, then $a$ is \term{invertible} and b is the \termor{inverse element}{inverse} of $a$.
\cite{Oppliger:2011:CC:2049860}
\end{defi}

\begin{defi}[Semigroup]
A \term{semigroup} is an algebraic structure $\left\langle S,\oplus\right\rangle$ that consists of a nonempty set $S$ and an associative binary operation $\oplus$. The semigroup must be closed (i.e., for all $a,b\in S$, $a\oplus b$ must also be an element of $S$).
\cite{Oppliger:2011:CC:2049860}
\end{defi}

\begin{defi}[Monoid]
A \term{monoid} is a semigroup \soplus that has an identity element $e\in S$ with respect to $\oplus$.
\cite{Oppliger:2011:CC:2049860}
\end{defi}

\begin{defi}[Group]
A \term{group} is a monoid \soplus in which every element $a\in S$ has an inverse element in $S$ (i.e., every element $a\in S$ is invertible).
\cite{Oppliger:2011:CC:2049860}
\end{defi}

\begin{defi}[Commutative group]
A group \soplus is a \term{commutative group} if the operation $\oplus$ is commutative (i.e., $a\oplus b = b\oplus a$ for all $a, b\in S$).
\cite{Oppliger:2011:CC:2049860}
\end{defi}

\begin{defi}[Finite group]
A group \soplus is a \term{finite group} if it contains only finitely many elements.
\cite{Oppliger:2011:CC:2049860}
\end{defi}

\begin{defi}[Subgroup]
A subset $H$ of a group $G$ is a \term{subgroup} of $G$ if it is closed under the operation of $G$ and also forms a group.
\cite{Oppliger:2011:CC:2049860}
\end{defi}

\begin{defi}[Left coset]
Let $G$ be a group and $H\subseteq G$ be a subset of $G$. For all $a\in G$, the sets $a\oplus H := \left\{a\oplus h | h\in H\right\}$ are called \term{left coset}s of $H$.
\cite{Oppliger:2011:CC:2049860}
\end{defi}

\begin{defi}[Right coset]
Let $G$ be a group and $H\subseteq G$ be a subset of $G$. For all $a\in G$, the sets $H\oplus a := \left\{a\oplus h | h\in H\right\}$ are called \term{right coset}s of $H$.
\cite{Oppliger:2011:CC:2049860}
\end{defi}

\begin{defi}[Coset]
Let $G$ be a (commutative) group and $H\subseteq G$. For all $a\in G$, the sets $a\oplus H$ and $H\oplus a$ are equal and are called \term{coset}s of $H$.
\cite{Oppliger:2011:CC:2049860}
\end{defi}

\begin{defi}[Ring]
A \term{ring} is an algebraic structure \soplusotimes with a set $S$ and two associative binary operations $\oplus$ and $\otimes$ that fulfill the following requirements:
\begin{enumerate}
 \item \soplus is a commutative group with identity element $e_1$ ;
 \item \sotimes is a monoid with identity element $e_2$ ;
 \item The operation $\otimes$ is distributive over the operation $\oplus$. This means that for all $a,b,c\in S$ the following two distributive laws must hold:
 \begin{equation}
 \left\{\begin{array}{c}
  a \otimes (b \oplus c)=(a \otimes b) \oplus (a \otimes c)\\
 (b \oplus c) \otimes a=(b \otimes a) \oplus (c \otimes a)
 \end{array}\right.
 \end{equation}
\end{enumerate}
\cite{Oppliger:2011:CC:2049860}
\end{defi}

\begin{defi}[Field]
A ring \soplusotimes in which $\left\langle S\setminus\left\{e_1\right\},\otimes\right\rangle$ is a group is a \term{field}.
\cite{Oppliger:2011:CC:2049860}
\end{defi}

\begin{defi}[Subfield]
A subset $H$ of a field $F$ is a \term{subfield} of $F$ if it closed under the operations of $F$ and also forms a field.
\cite{Oppliger:2011:CC:2049860}
\end{defi}

\begin{defi}[Prime field]
A \term{prime field} is a field that contains no proper subfield.
\cite{Oppliger:2011:CC:2049860}
\end{defi}

\begin{defi}[Homomorphism]
Let $A$ and $B$ be two algebraic structures. A mapping $f:A\rightarrow B$ is called a \term{homomorphism} of $A$ into $B$ if it preserves the operations of $A$. That is, if $\circ$ is an operation of $A$ and $\bullet$ an operation of $B$, then $\ffun{x\circ y}=\ffun{x}\bullet\ffun{y}$ must hold for all $x,y\in A$.
\cite{Oppliger:2011:CC:2049860}
\end{defi}

\begin{defi}[Isomorphism]
A homomorphism $f:A\rightarrow B$ is an \term{isomorphism} if it is injective (``one to one''). In this case, we say that $A$ and $B$ are isomorphic and we write $A\cong B$.
\cite{Oppliger:2011:CC:2049860}
\end{defi}

\begin{defi}[Automorphism]
An isomorphism $f:A\rightarrow A$ is an \term{automorphism}.
\cite{Oppliger:2011:CC:2049860}
\end{defi}

\begin{defi}[Permutation]
Let $S$ be a set. A map $f:S\rightarrow S$ is a \term{permutation} if $f$ is bijective (i.e., injective and surjective). The set of all permutations of $S$ is denoted by \perm{S\rightarrow S}, or $\fun{P}{S}$ in short.
\cite{Oppliger:2011:CC:2049860}
\end{defi}

\begin{defi}[Common divisors and greatest common divisor]
For $a,b\in\ZZZ_0$, $c\in\ZZZ$ is a \term{common divisor} of $a$ and $b$ if $c|a$ and $c|b$. Furthermore, $c$ is the \term{greatest common divisor}, denoted $\fun{gcd}{a,b}$, if it is the largest integer that divides $a$ and $b$.
\cite{Oppliger:2011:CC:2049860}
\end{defi}

\begin{defi}[Common multiples and least common multiple]
For $a,b\in\ZZZ_0$, $c\in\ZZZ$ is a \term{common multiple} of $a$ and $b$ if $a|c$ and $b|c$. Furthermore, $c$ is the \term{least common multiple}, denoted $\fun{lcm}{a,b}$, if it is the smallest integer that is divided by $a$ and $b$.
\cite{Oppliger:2011:CC:2049860}
\end{defi}

\begin{defi}[Prime number]
A natural number $1<n\in\NNN$ is called a \termor{prime number}{prime} if it divisible only by $1$ and itself.
\cite{Oppliger:2011:CC:2049860}
\end{defi}

\begin{defi}[Primality decision problem]
Given a positive integer $n\in\NNN$, deciding whether $n\in\PPP$ (i.e., $n$ is prime) or not (i.e., $n$ is composite) is called the \term{primality decision problem}.
\cite{Oppliger:2011:CC:2049860}
\end{defi}

\begin{defi}[B-smooth integer]
Let $B$ be an integer. An integer $n$ is a \term{B-smooth integer} if every prime factor of $n$ is less than $B$.
\cite{Oppliger:2011:CC:2049860}
\end{defi}

\begin{defi}
Let $a,b\in\ZZZ$ and $n\in\NNN$. $a$ is \term{congruent to $b$ modulo $n$}, denoted $a\equiv b \left(\mod n\right)$, if $n$ divides $a-b$ (i.e., $n|a-b$).
\cite{Oppliger:2011:CC:2049860}
\end{defi}

\begin{defi}[Polynomial]
Let $A$ be an algebraic structure with addition and multiplication (e.g., a ring or a field). A function $\fun{p}{x}$ is a \term{polynomial} in $x$ over $A$ if it is of the form
\begin{equation}
\fun{p}{x}=\displaystyle\sum_{i=0}^{n}{a_ix^i}=a_0+a_1x+a_2x^2+\ldots+a_nx^n
\end{equation}
where $n$ is a positive integer (i.e., the \term{degree} of $\fun{p}{x}$, denoted as $\fun{\mbox{deg}}{p}$), the \term{coefficient}s $a_i$ ($0\leq i\leq n$) are elements in $A$, and $x$ is a symbol not belonging to $A$.
\cite{Oppliger:2011:CC:2049860}
\end{defi}

\begin{defi}[Quadratic residue and square root]
An element $x\in\ZZZ^*_n$ is a \term{quadratic residue} modulo $n$ if there exists an element $y\in\ZZZstarn$ such that $x=y^2\left(\mod n\right)$. If such a $y$ exists, then it is called a \term{square root} of $x$ modulo $n$.
\cite{Oppliger:2011:CC:2049860}
\end{defi}

\subsection{Probability Theory}

\begin{defi}[Conditional Probability (Bayes' rule)]
The probability of event $x$ conditioned on knowing event $y$ (or more shortly, the \term{conditional probability} of $x$ given $y$) is defined as:
\begin{equation}
\pcond{x}{y}\equiv\displaystyle\frac{\pfun{x}}{\pfun{y}}
\end{equation}
If $\pfun{y}=0$, then $\pcond{x}{y}$ is not defined. This definition is also called \term{Bayes' rule}.
\cite{Barber2011}
\end{defi}

\begin{defi}[Probability Density Functions]
For a single continuous variable $x$, the \term{probability density function} $\pfun{x}$ is a function such that:
\begin{equation}
\left\{\begin{array}{l}
\pfun{x}\geq0\\
\bint{\pfun{x}}{x}{-\infty}{+\infty}=1\\
\pfun{a<x<b}=\bint{\pfun{x}}{x}{a}{b}
\end{array}\right.
\end{equation}
\cite{Barber2011}
\end{defi}

\begin{defi}[Independence]
Events $x$ and $y$ are independent if knowing one event gives no extra information about the other event. Mathematically, this is expressed by:
\begin{equation}
\pfun{x,y}=\pfun{x}\pfun{y}
\end{equation}
Provided that $\pfun{x}\neq0$ and $\pfun{y}\neq0$ \term{independence} of $x$ and $y$ is equivalent to:
\begin{equation}
\pcond{x}{y}=\pfun{x}\leftrightarrow\pcond{y}{x}=\pfun{y}
\end{equation}
If $\pcond{x}{y}=\pfun{x}$ for all states of $x$ and $y$, then the variables $x$ and $y$ are said to be independent. If
\begin{equation}
\pfun{x,y}=k\ffun{x}\gfun{y}
\end{equation}
for some constant $k$, and positive functions $f$ and $g$ then $x$ and $y$ are independent.
\cite{Barber2011}
\end{defi}

\begin{defi}[Prior likelihood and Posterior]
For data $\calD$ and variable $\theta$, Bayes' rule tells us how to update our prior beliefs about the variable $\theta$ in light of the data to a posterior belief:
\begin{equation}
\underbrace{\pcond{\theta}{\calD}}_{\mbox{\term{posterior}}}=\displaystyle\frac{\underbrace{\pcond{\calD}{\theta}}_{\mbox{\term{likelihood}}}\underbrace{\pfun{\theta}}_{\mbox{\term{prior}}}}{\underbrace{\pfun{\calD}}_{\mbox{\term{evidence}}}}
\end{equation}
The evidence is also called the \term{marginal likelihood}. The term likelihood is used for the probability that a model generates observed data. More fully, if we condition on the model $M$, we have
\begin{equation}
\pcond{\theta}{\calD,M}=\displaystyle\frac{\pcond{\calD}{\theta,M}\pcond{\theta}{M}}{\pcond{\calD}{M}}
\end{equation}
where we see the role of the likelihood $\pcond{\calD}{\theta,M}$ and marginal likelihood $\pcond{\calD}{M}$. likelihood is also called the \term{model likelihood}. The \termabbrev{most probable a posteriori}{MAP} setting is that which maximizes the posterior,
\begin{equation}
\theta_*=\argmax[\theta]{\pcond{\theta}{\calD,M}}
\end{equation}
\cite{Barber2011}
\end{defi}

\begin{defi}[Conditional Independence]
\begin{equation}
\calX\ci\calY|\calZ
\end{equation}
denotes that the two sets of variables $\calX$ and $\calY$ are independent of each other provided we know the state of the set of variables $\calZ$. For full \term{conditional independence}, $\calX$ and $\calZ$ must be independent given all states of $\calZ$. Formally, this means that
\begin{equation}
\pcond{\calX,\calY}{\calZ}=\pcond{\calX}{\calZ}\pcond{\calY}{\calZ}
\end{equation}
for all states of $\calX$, $\calY$, $\calZ$. In case the conditioning set is empty we may also write $\calX\ci\calY$ for $\calX\ci\calY|\emptyset$, in which case $\calX$ is \term{unconditionally independent} of $\calY$. If $\calX$ and $\calY$ are not conditionally independent, they are \term{conditionally dependent}. This is written:
\begin{equation}
\calX\cd\calY|\calZ
\end{equation}
\cite{Barber2011}
\end{defi}

\section{Graph Theory}

\begin{defi}[Graph]
A \term{graph} $G$ consists of \term{vertices}{nodes} and \termor{edges}{links} between the vertices. Edges may be directed (they have an arrow in a single direction) or undirected. A graph with all edges directed is called a \term{directed graph}, and one with all edges undirected is called an \term{undirected graph}.
\cite{Barber2011}
\end{defi}

\begin{defi}[Path, Ancestors, Descendants]
A \term{path} $A\mapsto B$ from node $A$ to node $B$ is a sequence of vertices $A=A_0,A1,\ldots,A_{n-1},A_n=B$, with $\left(A_i,A_{i+1}\right)$ an edge in the graph, thereby connecting $A$ to $B$. For a directed graph this means that a path is a sequence of nodes which when we follow the direction of the arrows leads us from $A$ to $B$. The vertices $A$ such that $A\mapsto B$ and $B\nmapsto A$ are the \term{ancestors} of $B$. The vertices $B$ such that $A → B$ and $B\nmapsto A$ are the \term{descendants} of $A$.
\cite{Barber2011}
\end{defi}

\begin{defi}[Directed Acyclic Graph]
A \termabbrev{Directed Acyclic Graph}{DAG} is a graph $G$ with directed edges (arrows on each link) between the vertices (nodes) such that by following a path of vertices from one node to another along the direction of each edge no path will revisit a vertex. In a DAG the ancestors of $B$ are those nodes who have a directed path ending at $B$. Conversely, the descendants of $A$ are those nodes who have a directed path starting at $A$.
\cite{Barber2011}
\end{defi}

\begin{defi}[Neighbour]
For an undirected graph G the \term{neighbour}s of $x$, $\fun{\mbox{ne}}{x}$ are those nodes directly connected to $x$.
\cite{Barber2011}
\end{defi}

\begin{defi}[Clique, Cliquo]
Given an undirected graph, a \term{clique} is a maximally connected subset of vertices. All the members of the clique are connected to each other; furthermore there is no larger clique that can be made from a clique. A non-maximal clique is sometimes called a \term{cliquo}.
\cite{Barber2011}
\end{defi}

\begin{defi}[Singly-Connected Graph, Multiply-Connected Graph, Tree, Loopy]
A \termor{singly-connected graph}{tree} is a graph where there is only one path from a vertex $a$ to another vertex $b$. Otherwise the graph is a \termor{multiply-connected graph}{loopy}. This definition applies regardless of whether or not the edges in the graph are directed.
\cite{Barber2011}
\end{defi}

\begin{defi}[Spanning Tree, Maximum Weight Spanning Tree]
A \term{spanning tree} of an undirected graph $G$ is a singly-connected subset of the existing edges such that the resulting singly-connected graph covers all vertices of $G$. A \term{maximum weight spanning tree} is a spanning tree such that the sum of all weights on the edges of the tree is larger than for any other spanning tree of $G$.
\cite{Barber2011}
\end{defi}


\section{Automata and Computability}

\begin{defi}
A \term{nondeterministic finite automaton} consists of a set $S$ of states. One of these states, $s0\in S$, is called the \term{starting state} of the automaton and a subset $F\subseteq S$ of the states are \term{accepting states}. Additionally, we have a set $T$ of \term{transitions}. Each transition $t$ connects a pair of states $s_1$ and $s_2$ and is labeled with a symbol, which is either a character $c$ from the alphabet $\Sigma$, or the symbol $\varepsilon$, which indicates an \term{epsilon-transition}. A transition from state $s$ to state $t$ on the symbol $c$ is written as $s^ct$.\cite{mogensen2009basics}
\end{defi}

\begin{defi}
Given a set $M$ of nondeterministic finite automaton states, we define the \term{$\varepsilon$-closure(M)} to be the least (in terms of the subset relation) solution to the set equation:
\begin{equation}
\fun{\varepsilon\mbox{-closure}}{M} = M\cup \left\{t|s\in \fun{\varepsilon\mbox{-closure}}{M}\mbox{ and }s^\varepsilon t\in T \right\}
\end{equation}
Where $T$ is the set of transitions in the non deterministic finite automaton.\cite{mogensen2009basics}
\end{defi}

\begin{defi}[Big-Oh notation]
If $f,g$ are two functions from $\NNN$ to $\NNN$, then we 
\begin{enumerate}
 \item say that $f=\term{\bigoh{g}}$ if there exists a constant $c$ such that $\ffun{n}\leq c\gfun{n}$ for every sufficiently large $n$,
 \item say that $f=\term{\bigomega{g}}$ if $g = \bigoh{f}$,
 \item say that $f=\term{\bigtheta{g}}$ is $f = \bigoh{g}$ and $g = \bigoh{f}$,
 \item say that $f=\term{\smalloh{g}}$ if for every $\varepsilon>0$, $\ffun{n}\leq\varepsilon\gfun{n}$ for every sufficiently large $n$, and
 \item say that $f=\term{\smallomega{g}}$ if $g=\smalloh{f}$.
\end{enumerate}
\cite{arora2009computational}
\end{defi}

\begin{defi}[Computing a function and running time]
Let $\btobfun{f}$ and let $\funsig{T}{\NNN}{\NNN}$ be some functions, and let $M$ be a Turing machine. We say that $M$ \term{computes $f$} if for every $x\in\binarystrings$, whenever $M$ is initialized to the start configuration on input $x$, then it halts with $\ffun{x}$ written on its output tape. We say $M$ \term{computes $f$ in $\fun{T}{n}$-time} if its computation on every input $x$ requires at most $\fun{T}{\left|x\right|}$ steps.
\cite{arora2009computational}
\end{defi}

\begin{defi}[The class \compclass{DTIME}]
Let $\funsig{T}{\NNN}{\NNN}$ be some function. A language $L$ is in $\fun{\term{\compclass{DTIME}}}{\fun{T}{n}}$ if and only f there is a Turing machine that runs in time $c\fun{T}{n}$ for some constant $c>0$ and decides $L$.
\end{defi}

\begin{defi}[The class \compclass{P}]
\begin{equation}
\term{\compclass{P}}=\displaystyle\cup_{c\geq1}\fun{\compclass{DTIME}}{n^c}.
\end{equation}
\cite{arora2009computational}
\end{defi}

\begin{defi}[The class \compclass{NTIME}]
For every function $\funsig{T}{\NNN}{\NNN}$ and $L\subseteq\binarystrings$, we say that $L\in\fun{\term{\compclass{NTIME}}}{\fun{T}{n}}$ if there is a constant $c>0$ and a $c\fun{T}{n}$-time non deterministic Turing Machine $M$ such that for every $x\in\binarystrings$, $x\in L\leftrightarrow\fun{M}{x}=1$.
\cite{arora2009computational}
\end{defi}

\begin{defi}[The class \compclass{NP}]
\begin{equation}
\term{\compclass{NP}}=\displaystyle\cup_{c\geq1}\fun{\compclass{NTIME}}{n^c}.
\end{equation}
\cite{arora2009computational}
\end{defi}

\begin{defi}[Reduction, \compclass{NP-hard} and \compclass{NP-complete}]
A language $L\subseteq\binarystrings$ is \term{polynomial-time Karp reducible} to a language $L'\subseteq\binarystrings$ (sometimes shortened to just ``\term{polynomial-time reducible}''), denoted by $L\leq_p L'$, if there is a polynomial-time computable function $\btobfun{f}$ such that for every $x\in\binarystrings$, $x\in L$ if and only if $\ffun{x}\in L'$. We say that $L'$ is \term{\compclass{NP-hard}} if $L\leq_p L'$ for every $L\in\compclass{NP}$. We say that $L$ is \term{\compclass{NP-complete}} if $L'$ is \compclass{NP-hard} and $L'\in\compclass{NP}$.
\cite{arora2009computational}
\end{defi}

\begin{defi}[Complement language]
If $L\subseteq\binarystrings$ is a language, then we denote by $\overline{L}$ the \term{complement language} of $L$. That is, $\overline{L}=\binarystrings\setminus L$.
\cite{arora2009computational}
\end{defi}

\begin{defi}[The class \compclass{coNP}]
\begin{equation}
\term{\compclass{coNP}}=\left\{\overline{L}:L\in\compclass{NP}\right\}.
\end{equation}
\cite{arora2009computational}
\end{defi}

\section{Artificial Intelligence}

\begin{defi}
A \term{neighborhood structure} is a function $\calN:\calS\rightarrow 2^\calS$ that assigns to every $s\in\calS$ a set of neighbors $\fun{\calN}{s}\subseteq\calS$. $\fun{\calN}{s}$ is called the neighborhood of $s$. Often, neighborhood structures are implicitly defined by specifying the changes that must be applied to a solution s in order to generate all its neighbors. The application of such an operator that produces a neighbor $s'\in\fun{\calN}{s}$ of a solution s is commonly called a \term{move}.\cite{alba05}
\end{defi}

\begin{defi}
A \term{locally minimal solution} (or \term{local minimum}) with respect to a neighborhood structure $\calN$ is a solution $\hat{s}$ such that $\forall s\in\neigh{\hat{s}}:\ffun{\hat{s}}\leq\ffun{s}$. We call $\hat{s}$ a \term{strict locally minimum} if $\forall s\in\neigh{\hat{s}}:\ffun{\hats}<\ffun{s}$.\cite{alba05}
\end{defi}


\chapter{Theorems}

\begin{theo}[Lagrange's Theorem]
If $H$ is a subgroup of $G$, then $\left|H\right||\left|G\right|$ (i.e., the order of $H$ divides the order of $G$.
\begin{proof}
If $H=G$, then $\left|H\right||\left|G\right|$ holds trivially. Consequently, we only consider the case in which $H\subset G$. For any $a\in G\setminus H$, the coset $a\oplus H$ is a subset of $G$. The following can be shown:
\begin{enumerate}
 \item \label{i} For any $a \neq a$, if $a\notin a'\oplus H$ then $\left(a\oplus H\right)\cap\left(a'\oplus H\right)=\emptyset$;
 \item \label{ii} $\left|a\oplus H\right|=\left|H\right|$.
\end{enumerate}
For (\ref{i}{}), suppose there exists a $b\in\left(a\oplus H\right)\cap\left(a'\oplus H\right)$. Then there exist $c,c'\in H$ such that $a\oplus c=b=a'\oplus c'$. Applying various group axioms, we have $a=a\oplus e= a\oplus\left(c\oplus c^{-1}\right)=b\oplus c^{-1}=\left(a'\oplus c'\right)\oplus c^{-1}=a'\oplus\left(c'\oplus c^{-1}\right)\in a'\oplus H$. This contradicts our assumption (that $a\notin a'\oplus H$).
For (\ref{ii}{}), $\left|a\oplus H\right|\leq\left|H\right|$ holds trivially (by the definition of a coset). Suppose that the inequality is rigorous. This is only possible if there are $b,c\in H$ with $b\neq c$ and $a\oplus b = a\oplus c$. Applying the inverse element of $a$ on either side of the equation, we get $b=c$, contradicting to $b\neq c$.
In summary, $G$ is partitioned by $H$ and the family of its mutually disjoint cosets, each has the size $\left|H\right|$, and hence $\left|H\right||\left|G\right|$. This proves the theorem.
\cite{Oppliger:2011:CC:2049860}
\end{proof}
\end{theo}

\begin{theo}
For all $a,b,c\in\ZZZ$, if $a|b$ and $b|c$, then $a|c$.
\begin{proof}
If $a|b$ and $b|c$, then there exist $f,g\in\ZZZ$ with $b=af$ and $c=bg$. Consequently, we can write $c=bg= \left(af\right)g = a\left(fg\right)$ to express $c$ as a multiple of $a$. The claim (i.e., $a|c$) follows directly from this equation.
\cite{Oppliger:2011:CC:2049860}
\end{proof}
\end{theo}

\begin{theo}
For all $a,b,c\in\ZZZ$, if $a|b$, then $ac|bc$ for all $c$.
\begin{proof}
If $a|b$, then there exists $f\in\ZZZ$ with $b=af$. Consequently, we can write $bc=\left(af\right)c=f\left(ac\right)$ to express $bc$ as a multiple of $ac$. The claim (i.e., $ac|bc$) follows directly from this equation.
\cite{Oppliger:2011:CC:2049860}
\end{proof}
\end{theo}

\begin{theo}
For all $a,b,c,d,e\in\ZZZ$, if $c|a$ and $c|b$, then $c|da+eb$ for all $d$ and $e$.
\begin{proof}
If $c|a$ and $c|b$, then there exist $f,g\in\ZZZ$ with $a=fc$ and $b=gc$. Consequently, we can write $da+eb=dfc+egc=\left(df+eg\right)c$ to express $da+eb$ as a multiple of $c$. The claim (i.e., $c|da+eb$) follows directly from this equation.
\cite{Oppliger:2011:CC:2049860}
\end{proof}
\end{theo}

\begin{theo}
\label{theo4}
For all $a,b\in\ZZZ$, if $a|b$ and $b\neq0$, then $\left|a\right|\leq\left|b\right|$.
\begin{proof}
If $a|b$ and $b=0$, then there exists $0\neq f\in\ZZZ$ with $b=af$. Consequently, $\left|b\right|=\left|af\right|\geq\left|a\right|$ and the claim (i.e., $\left|a\right|\leq\left|b\right|$) follows immediately.
\cite{Oppliger:2011:CC:2049860}
\end{proof}
\end{theo}

\begin{theo}
For all $a,b\in\ZZZ$, if $a|b$ and $b|a$, then $\left|a\right|=\left|b\right|$.
\begin{proof}
Let us assume that $a|b$ and $b|a$. If $a=0$ then $b=0$, and vice versa. If $a,b\neq 0$, then it follows from Theorem \ref{theo4}. that $\left|a\right|\leq\left|b\right|$ and $\left|b\right|\leq\left|a\right|$, and hence $\left|b\right|=\left|a\right|$.
\cite{Oppliger:2011:CC:2049860}
\end{proof}
\end{theo}

\begin{theo}[Euclid's division theorem]
For all $n,d\in\ZZZ_0$ there exist unique and efficiently computable $q,r\in\ZZZ$ such that $n=qd+r$ and $0\leq r\leq\left|d\right|$.
\cite{Oppliger:2011:CC:2049860}
\end{theo}

\begin{theo}[Prime density theorem]
\begin{equation}
\displaystyle\lim_{n\rightarrow\infty}\displaystyle\frac{\pi\left(n\right)\ln\left(n\right)}{n}=1
\end{equation}
\cite{Oppliger:2011:CC:2049860}
\end{theo}

\begin{theo}[Unique factorization]
Every natural number $n\in\NNN$ can be factored uniquely (up to a permutation of the prime factors):
\begin{equation}
n=\displaystyle\prod_{p\in\PPP}{p^{\fun{e_p}{n}}}
\end{equation}
In this formula, $\fun{e_p}{n}$ refers to the exponent of $p$ in the factorization of $n$. For almost all $p\in\PPP$ this value is zero, and only for finitely many primes $p$ the value $\fun{e_p}{n}$ is greater than zero.
\cite{Oppliger:2011:CC:2049860}
\end{theo}

\begin{theo}[Chinese remainder theorem]
Let
\begin{equation}
\left\{\begin{array}{l}
x\equiv a_1\left(\mod n_1\right)\\
x\equiv a_2\left(\mod n_2\right)\\
\ldots\\
x\equiv a_k\left(\mod n_k\right)
\end{array}\right.
\end{equation}
be a system of $k$ congruences with pairwise co-prime moduli $n_1,\ldots,n_k$. The system has a unique and efficiently computable solution $x$ in $\ZZZ_n$ with $n=\prod_{i=1}^kn_i$.
\cite{Oppliger:2011:CC:2049860}
\end{theo}

\begin{theo}[Fermat's Little Theorem]
If $p$ is a prime and $a\in\ZZZ^*_p$, then $a^{p-1}\left(\mod p\right)$.
\begin{proof}
Because $\fun{\phi}{p}=p-1$ for every prime number $p$, \term{Fermat's Little Theorem} is just a special case of Euler's Theorem.
\cite{Oppliger:2011:CC:2049860}
\end{proof}
\end{theo}

\begin{theo}[Euler's Theorem]
If $\gcdf{a,n}=1$, then $a^{\fun{\phi}{n}}\equiv 1\left(\mod n\right)$.
\begin{proof}
Because $\gcdf{a,n}=1$, $a \left(\mod n\right)$ must be an element in $\ZZZ^*_n$. Also, $\left|\ZZZ^*_n\right|=\fun{\phi}{n}$. According to a corollary of Lagrange's Theorem, the order of every element (in a finite group) divides the order of the group. Consequently, the order of a (i.e., $\ordf{a}$) divides $\phif{n}$, and hence if we multiply a modulo $n\phif{n}$ times we always get a value that is equivalent to $1$ modulo $n$.
\cite{Oppliger:2011:CC:2049860}
\end{proof}
\end{theo}

%\begin{theo}[Euler's criterion]
%\end{theo}TODO

\begin{theo}[Efficient universal Turing machine]
\label{theo:eutm}
There exists a Turing Machine $\mathcal{U}$ such that for every $x,\alpha\in\binarystrings$, $\fun{\mathcal{U}}{x,\alpha} = \fun{M_{\alpha}}{x}$, where $M_{\alpha}$ denotes the Turing Machine represented by $\alpha$. Moreover, if $M_{\alpha}$ halts on input $x$ within $T$ steps then $\fun{\mathcal{U}}{x,\alpha}$ halts within $CT\log T$ steps, where $C$ is a number independent of $\left|x\right|$ and depending only on $M_{\alpha}$'s alphabet size, number of tapes, and number of states.
\cite{arora2009computational}
\end{theo}

\begin{theo}
\label{theo:uncomputable}There exists a function $\btoblfun{{\sf UC}}$ that is not computable by any Turing Machine.
\begin{proof}
The function $\sl UC$ is defined as follows: For every $\alpha\in\binarystrings$, if $\fun{M_{\alpha}}{\alpha}=1$, then $\fun{\sl UC}{\alpha}=0$; otherwise (if $\fun{M_{\alpha}}{\alpha}$ outputs a different value or enters an infinite loop), $\fun{\sl UC}{\alpha}=1$. Suppose for the sake of contradiction that ${\sl UC}$ is computable and hence there exists a Turing Machine $M$ such that $\fun{M}{\alpha}=\fun{{\sl UC}}{\alpha}$ for every $\alpha\in\binarystrings$. Then, in particular, $\fun{M}{\left\lfloor M\right\rfloor}=\fun{{\sl UC}}{\left\lfloor M\right\rfloor}$. But this is impossible: By the definition of ${\sl UC}$, $\fun{{\sl UC}}{\left\lfloor M\right\rfloor}=1\leftrightarrow\fun{M}{\left\lfloor M\right\rfloor}=1$.
\cite{arora2009computational}
\end{proof}
\end{theo}

\begin{theo}
{\slHALT} is not computable by any Turing Machine.
\begin{proof}
Suppose, for the sake of contradiction, that there was a Turing Machine $M_{\slHALT}$ computing $\slHALT$. We will use $M_{\slHALT}$ to show a Turing Machine $M_{\slUC}$ computing $\slUC$, contradicting Theorem \ref{theo:uncomputable}. The Turing Machine $M_{\slUC}$ is simple: On input $\alpha$, $M_{\slUC}$ runs $\fun{M_{\slHALT}}{\alpha,\alpha}$. If the result is 0 (meaning that $M_\alpha$ does not halt on $\alpha$), then $M_{\slUC}$ outputs $1$. Otherwise, $M_{\slUC}$ uses the universal Turing Machine $\mathcal{U}$ to compute $b=\fun{M_{\alpha}}{\alpha}$. If $b=1$, then $M_{\slUC}$ outputs $0$; otherwise, it outputs $1$. Under the assumption that $\fun{M_{\slHALT}}{\alpha,\alpha}$ outputs $\fun{\slHALT}{\alpha,\alpha}$ within a finite number of steps, the Turing Machine $\fun{M_{\slUC}}{\alpha}$ will output $\fun{\slUC}{\alpha}$.
\cite{arora2009computational}
\end{proof}
\end{theo}

\begin{theo}[\term{Time Hierarchy Theorem}]
\label{theo:tht}
If $f,g$ are time-constructible functions satisfying $\ffun{n}\log\ffun{n}=\smalloh{\gfun{n}}$, then
\begin{equation}
\dtime{\ffun{n}}\subsetneq\dtime{\gfun{n}}.
\end{equation}
\begin{proof}
To showcase the essential idea of the proof of Theorem \ref{theo:tht} with minimal notation, we prove the simpler statement $\dtime{n}\subsetneq\dtime{n^{1.5}}$. Consider the following Turing machine $D$: ``On input $x$, run for $\abs{x}^{1.4}$ steps the Universal Turing Machine $\mathcal{U}$ of Theorem \ref{theo:eutm} to simulate the execution of $M_x$ on $x$. If $\mathcal{U}$ outputs some bit $b\in\binary$ in this time, then output the opposite answer (i.e., output $1-b$). Else output 0.'' Here $M_x$ is the machine represented by the string $x$. By definition, $D$ halts within $n^{1.4}$ steps and hence the language $L$ decided by $D$ is in $\dtime{n^{1.5}}$. We claim that $L\notin\dtime{n}$. For contradiction's sake, assume that there is some Turing Machine $M$ and constant $c$ such that Turing Machine $M$, given any input $x\in\binarystrings$, halts within $c\abs{x}$ steps and outputs $\fun{D}{x}$. The time to simulate $M$ by the universal Turing machine $\mathcal{U}$ on every input x is at most $c'c\abs{
x}\log\abs{x}$ for some number $c'$ that depends on the alphabet size and number of tapes and states of $M$ but is independent of $\abs{x}$. There is some number $n_0$ such that $n^{1.4} > c'cn\log n$ for every $n\geq n_0$. Let $x$ be a string representing the machine $M$ whose length is at least $n_0$ (such a string exists since $M$ is represented by infinitely many strings). Then, $\fun{D}{x}$ will obtain the output $b=\fun{M}{x}$ within $\abs{x}^{1.4}$ steps, but by definition of $D$, we have $\fun{D}{x}=1-b=\fun{M}{x}$. Thus we have derived a contradiction. The proof Theorem \ref{theo:tht} for general $f,g$ is similar and uses the observation that the slowdown in simulating a machine using $\mathcal{U}$ is at most logarithmic.
\cite{arora2009computational}
\end{proof}
\end{theo}

\begin{theo}[\term{Nondeterministic Time Hierarchy Theorem}]
\label{theo:ndtht}
If $f,g$ are time constructible functions satisfying $\ffun{n+1}= \smalloh{\gfun{n}}$, then
\begin{equation}
\ntime{\ffun{n}}\subsetneq\ntime{\gfun{n}}.
\end{equation}
\begin{proof}
We just showcase the main idea by proving $\ntime{n}\subsetneq\ntime{n^{1.5}}$. The first instinct is to duplicate the proof of Theorem \ref{theo:tht}, since there is a universal Turing machine for nondeterministic computation as well. However, this alone does not suffice because the definition of the new machine $D$ requires the ability to ``flip the answer'', in other words, to efficiently compute, given the description of an nondeterminstic Turing machine $M$ and an input $x$, the value $1-\fun{M}{x}$. It is not obvious how to do this using the universal nondeterministic machine: it is unclear how a nondeterministic machine can just ``flip the answer''. Specifically, we do not expect that that the complement of an $\ntime{n}$ language will be in $\ntime{n^{1.5}}$. Now of course, the complement of every $\ntime{n}$ language is trivially decidable in exponential time (even deterministically) by examining all the 
possibilities for the machine's nondeterministic choices, but on first sight this seems to be completely irrelevant to proving $\ntime{n}\subsetneq\ntime{n^{1.5}}$. Surprisingly, this trivial exponential simulation of a nondeterministic machine does suffice to establish a hierarchy theorem. The key idea will be lazy diagonalization, so named because the new machine $D$ is in no hurry to diagonalize and only ensures that it flips the answer of each linear time nondeterminstic Turing machine $M_i$ in only one string out of a sufficiently large (exponentially large) set of strings. Define the function $\funsig{f}{\NNN}{\NNN}$ as follows: $\ffun{1}=2$ and $\ffun{i+1}=2^{\ffun{i}^{1.2}}$. Given $n$, it's not hard to find in $\bigoh{n^1.5}$ time the number $i$ such that $n$ is sandwiched between $\ffun{i}$ and $\ffun{i+1}$. Our diagonalizing machine $D$ will try to flip the answer of $M_i$ on some input in the set $\left\{1^n:\ffun{i}<n\leq\ffun{i+1}\right\}$. $D$ is defined as follows:
\begin{quote}
``On input $x$, if $x\notin1^*$, reject. If $x=1^n$, then compute $i$ such that $\ffun{i}<n\leq\ffun{i+1}$ and
\begin{enumerate}
 \item If $\ffun{i}<n<\ffun{i+1}$ then simulate $M_i$ on input $1^{n+1}$ using nondeterminism in $n^{1.1}$ time and output its answer. (If $M_i$ has not halted in this time, then halt and accept.)
 \item If $n=\ffun{i+1}$, accept $1^n$ if and only if $M_i$ rejects $1^{\ffun{i}+1}$ in $(\ffun{i}+1)^{1.1}$ time.''
\end{enumerate}
\end{quote}
Part 2 requires going through all possible $2^(\ffun{i}+1)^{1.1}$ branches of $M_i$ on input $1^{\ffun{i}+1}$, but that is fine since the input size $\ffun{i+1}$ is $2^{\ffun{i}^{1.2}}$. Hence the nondeterminstic Turing machine $D$ runs in $\bigoh{n^{1.5}}$ time. Let $L$ be the language decided by $D$. We claim that $L\notin\ntime{n}$. Indeed, suppose for the sake of contradiction that $L$ is decided by an nondeterminstic Turing machine $M$ running in $cn$ steps (for some constant $c$). Since each nondeterminstic Turing machine is represented by infinitely many strings, we can find $i$ large enough such that $M=M_i$ and on inputs of length $n\geq\ffun{i}$, $M_i$ can be simulated in less than $n^{1.1}$ steps. This means that the two steps in the description of $D$ ensure, respectively, that 

\begin{eqnarray}
\txIf\ffun{i}<n<\ffun{i+1},&\txthen\fun{D}{1^n}=\fun{M_i}{1^{n+1}}\label{eqn:ndtht33}\\
\txwhereas&\fun{D}\neq\fun{M_i}{1^{\ffun{i}+1}}\label{eqn:ndtht34}
\end{eqnarray}

By our assumption $M_i$ and $D$ agree on all inputs $1^n$ for $n$ in the semi-open interval $\left(\ffun{i},\ffun{i+1}\right]$. Together with (\ref{eqn:ndtht33}), this implies that $\fun{D}{1^{\ffun{i+1}}}=\fun{M_i}{1^{\ffun{i}+1}}$, contradicting (\ref{eqn:ndtht34}).\cite{arora2009computational}
\end{proof}
\end{theo}

\begin{twocolumn}
\bibliographystyle{alpha}
\bibliography{definitions}
\printindex
\end{twocolumn}
\end{document}