\chapter{Artificial Intelligence}

\section{First-Order Logic and Logic Programming}

\begin{defi}[Signature]
$S$ is a \term{signature} if $S$ is a four-tuple \tuple{P,F,r,C} where:
\begin{enumerate}
\item $P$ is a set of \term{predicate symbols} \brak{P_1,P_2,\ldots,P_n},
\item $F$ is a set of \term{function symbols} \brak{F_1,F_2,\ldots,F_m},
\item $r$ is \term{arity} or \term{degree of functions} and relations. For each $P_i$ respectively $F_j$, \fun{r}{P_i} respectively \fun{r}{F_j} is a non-zero natural number denoting the arity of $P_i$ respectively $F_j$,
\item $C$ is a set of \term{constant symbols}.
\end{enumerate}
\cite{conf/fedcsis/Telnarova12}
\end{defi}

\begin{defi}[Alphabet]
An \term{alphabet} $\Sigma$ consists of the following symbols:
\begin{enumerate}
\item Signature $S=\tuple{P,F,r,C}$.
\item \term{Collection of variables} $V$.
\item \term{Operators}: $\neg$ (\term{negation}), $\wedge$ (\term{conjunction}), $\vee$ (\term{disjunction}), $\rightarrow$ (\term{implication}), $\leftrightarrow$ (\term{equivalence}).
\item \term{Quantifiers}: $\forall$ (\term{forall}), $\exists$ (\term{exists}).
\item \term{Parentheses} and \term{punctuation symbols}: $($, $)$ and $,$.
\end{enumerate}
\cite{conf/fedcsis/Telnarova12}
\end{defi}

\begin{defi}[Term]
A \term{term} is defined inductively as follows:
\begin{enumerate}
\item Variable is term.
\item Constant is term.
\item If $f$ is a function symbol ($f\in F$) with arity $m$ and $t_1,t_2,\ldots,t_m$ are terms of $\Sigma$, then \ffun{t_1,t_2,\ldots,t_m} is term of $\Sigma$.
\end{enumerate}
\cite{conf/fedcsis/Telnarova12}
\end{defi}

\begin{defi}[Atom]
If $p$ is predicate symbol with arity $m$ and $t_1,t_2,\ldots,t_m$ are terms of $\Sigma$, then $\fun{p}{t_1,t_2,\ldots,t_m}$ is an \termor{atomic formula}{atom}. An atomic formula is a \term{formula} and all occurrences of variables in an atomic formula are free.
\cite{conf/fedcsis/Telnarova12}
\end{defi}

\begin{defi}[Formula]
A \term{formula} is defined as follows:
\begin{enumerate}
 \item An atom is a formula.
 \item If $H$ and $G$ are formulas, then:
 \begin{enumerate}
  \item $\neg H$ is a formula, the occurrence of variables in $\neg H$ is free respectively bound if it is free respectively bound in $H$,
  \item $H\wedge G$ is a formula, the occurrence of variables in $H\wedge G$ is free respectively bound if it is free respectively bound in $H$ or $G$,
  \item $H\vee G$ is a formula, the occurrence of variables in $H\vee G$ is free respectively bound if it is free respectively bound in $H$ or $G$,
  \item $H\rightarrow G$ is a formula, the occurrence of variables in $H\rightarrow G$ is free respectively bound if it is free respectively bound in $H$ or $G$,
  \item $H\leftrightarrow G$ is a formula, the occurrence of variables in $H\leftrightarrow G$ is free respectively bound if it is free respectively bound in $H$ or $G$.
 \end{enumerate}
 \item If $H$ is a formula and $x$ is a variable, then $\forall x:H$ and $\exists x:H$ are formulas. All occurrences of $x$ are bound.
\end{enumerate}
\cite{conf/fedcsis/Telnarova12}
\end{defi}

\begin{defi}[Literal]
A \term{literal} $L$ is an atom or the negation of an atom.
\cite{conf/fedcsis/Telnarova12}
\end{defi}

\begin{defi}[Clause]
A \term{clause} is a formula such as $\forall\vec{x}:L_1\vee L_2\vee\ldots\vee L_m$ where each $L_i$ is a literal and $\vec{x}=\brak{x_1,x_2,\ldots,x_n}$ are all the variables occurring in $L_1\vee L_2\vee\ldots\vee L_m$.
\cite{conf/fedcsis/Telnarova12}
\end{defi}

\begin{defi}[Horn-clauses]
\term{Horn-clause}s have the form: $\forall x_1,x_2,\ldots,x_n:L_1\wedge L_2\wedge\ldots\wedge L_m\rightarrow L$ where $L,L_1\wedge L_2\wedge\ldots\wedge L_m$ are a literals and $x_1,x_2,\ldots,x_n$ are all variables having free occurrences in $L,L_1\wedge L_2\wedge\ldots\wedge L_m$.
\cite{conf/fedcsis/Telnarova12}
\end{defi}

\begin{defi}[Negation of a dilemma, Conjunction of dilemmas, Disjunction of dilemmas]
For a given set $D$ of dilemmas it is defined:
\begin{enumerate}
 \item A \term{negation of a dilemma} $d=\dilemma{u}$ as a dilemma $\neg d=\negdilemma{u}$;
 \item A \term{disjunction of the dilemmas} $d'=\dilemma{u'}$, $d''=\dilemma{u''}$ as a dilemma $d'\vee d''=\flatbrak{u'\vee u''|\neg\brak{u'\vee u''}}$;%
 \item A \term{conjunction of the dilemmas} $d'=\dilemma{u'}$, $d''=\dilemma{u''}$ as a dilemma $d'\wedge d''=\flatbrak{u'\wedge u''|\neg\brak{u'\wedge u''}}$;%
\end{enumerate}
\cite{conf/fedcsis/Kulikowski12}
\end{defi}

\begin{defi}[Equal certainty relation, Ambivalent dilemma, Equivalent dilemmas, Anti-equivalent dilemmas]
Let $D$ be a set of dilemmas. Then in $D$ a binary relation $\approx^c$ satisfying the conditions of:
\begin{enumerate}
\item Reciprocity: for each $d\in D$ it holds $d\approx^c d$;
\item Symmetry: for any $d',d''\in D$ if $d'\approx^c d''$ then also $d''\approx^c d'$ holds;
\item Reflexivity: for any $d',d''\in D$ if $d'\approx^c d''$ then also $\neg d'\approx^c\neg d''$ holds;
\item Transitivity: for any $d',d'',d'''\in D$ if $d'\approx^c d''$ and $d''\approx^c d'''$ then also $d'\approx^c d'''$ holds;
\item Fixation: for any $d',d''\in D$ if $d'\approx^c\neg d'$ and $d''\approx^c\neg d''$ then also $d'\approx^c d''$ holds,
\end{enumerate}
will be called an \term{equal certainty relation}. Any dilemma satisfying the condition $d\approx^c\neg d$ will be called an \term{ambivalent dilemma}; any dilemmas such that $d'\approx^c d''$ holds will be called \term{equivalent dilemmas}; any dilemmas such that $d'\approx^c\neg d''$ holds will be called \term{anti-equivalent dilemmas}.
\cite{conf/fedcsis/Kulikowski12}
\end{defi}

\begin{defi}[Certainty ranking]
Let $D$ be a set of dilemmas with established equal certainty relation. Then a binary relation $\preceq^c$ described in $D$ and satisfying the conditions of:
\begin{enumerate}
 \item reciprocity: for each $d\in D$ it holds $d\preceq^c d$;
 \item symmetry: for any $d',d''\in D$ $d'\preceq^c d''$ and $d''\preceq^c d'$ hold if and only if $d'\approx^c d''$ holds;
 \item anti-reflexivity: for any $d',d''\in D$ if $d'\preceq^c d''$ then $\neg d''\preceq^c\neg d'$ holds;
 \item transitivity: for any $d',d'',d'''\in D$ if $d'\preceq^c d''$ and $d''\preceq^c d'''$ then also $d'\preceq^c d'''$ holds,
\end{enumerate}
will be called a \term{certainty ranking}.
\cite{conf/fedcsis/Kulikowski12}
\end{defi}

\begin{theo}
Let $D$ be a set of dilemmas with established equal certainty and certainty ranking relations. Then for any $d',d''\in D$:
\begin{enumerate}
\item if $d'\preceq^c d''$ and not $d''\preceq^c d'$ then $d'\vee d''\preceq^c d''$;
\item if $d'\preceq^c d''$ and not $d''\preceq^c d'$ then $d'\wedge d''\preceq^c d'$;
\item if $d'\approx^c d''$ then $d'\vee d''\approx^c d'\wedge d''\approx^c d'\approx^c d''$;
\item if $d'\approx^c d''$ then $d'\wedge d''\approx^c d',d''$;
\item if $d'\approx^c d''$ then $d',d''\approx^c d'\vee d''$
\end{enumerate}
\cite{conf/fedcsis/Kulikowski12}
\end{theo}

\subsection{Logic Programming}

\begin{defi}[Domain declaration for predicate symbol $p$]
A \term{domain declaration for predicate symbol $p$} of arity n is an expression of the following form.
\begin{equation}
\mbox{domain }\fun{p}{a_1,\ldots ,a_n}
\end{equation}
where $a_i$ is either $h$ or $d$. When $a$ is equal to $h$, this means that the $i$-th argument of $p$ ranges over the Herbrand universe. Otherwise, it means that the $i$-th argument is a list of variables which ranges over $d_1$ In the following, the domains $d_i$ are finite and explicit sets of values (i.e constants).
\cite{conf/ijcai/Hentenryck87}
\end{defi}

\begin{defi}[Domain set of a logic program]
Let $dl,\ldots ,dn$ the domains appearing in the domain declarations of a logic program $PR$ and different from the Herbrand universe. We note \fun{D}{PR} the set $\accol{d|d\neq\emptyset\wedge d\in 2^{d_i}\accol{1\leq i\leq n}}$ We call it the \term{domain set of the logic program}. The domain set of a logic program contains all domains we possibly need during the computations.
\cite{conf/ijcai/Hentenryck87}
\end{defi}

\begin{defi}[Range of a term included in a domain]
We say that the \term{range of $t$ is included in a domain $d_t$} denoted $\abs{t}\in d_i$ if $t$ is a constant $\in d_t$ or a $d$-variable $x^{d_t}$ such that $d_t\subseteq d$
\cite{conf/ijcai/Hentenryck87}
\end{defi}

\begin{defi}[$d$-substitution]
A $d$-substitution $\theta$ is a finite set of the form $\accol{v_1/t_1,\ldots,v_n/t_n}$ where
\begin{enumerate}
 \item each $v_i$ is either a variable or $d$-variable
 \item $t_i$ is a term distinct from $v_i$,
 \item $v_1,\ldots,v_n$ are all distinct,
 \item if $v_i$ is a $d$-variable $v^{d_i}$, $\abs{t_i}\in d_i$
\end{enumerate}
\cite{conf/ijcai/Hentenryck87}
\end{defi}

\begin{defi}[$d$-substitutions agree on a set of variables and $d$-variables]
We say that two \term[$d$-substitutions agree on a set of variables and $d$-variables]{$d$-substitutions $\theta$ and $\lambda$ agree on a set $V$ of variables and $d$-variables}, denoted $\theta=\lambda\abs{V}$ \iffTx{} $x\theta=x\lambda$ for each $x\in V$ where $=$ denotes syntactic equality.
\cite{conf/ijcai/Hentenryck87}
\end{defi}

\begin{defi}[$d$-instance]
$\theta$ is a \term{$d$-instance} of $\lambda$ in $V$, denoted $\lambda\leq\theta$ \iffTx{} $x\theta=\delta\circ\lambda$ for some $d$-substitution $\delta$.
\cite{conf/ijcai/Hentenryck87}
\end{defi}

\begin{defi}[$d$-unifier, unifies, more general $d$-unifier, $d$-mgu]
A $d$-substitution $\sigma$ is a \term{$d$-unifier} of some non-empty and finite subset $S=\accol{t_1,\ldots,t_n}$ where $t_i$ and a literal of a term \iffTx{} $t_1\sigma=\ldots=t_n\sigma$, we also say that $\sigma$ \term{unifies} $S$. \term{\fun{UNI}{S}} is the set of all $d$-unifiers of $S$. $\sigma$ is called the \termor{more general $d$-unifier}{$d$-mgu} of $S$ \iffTx for each $\theta\in\fun{UNI}{S}$, $\theta\leq\sigma\abs{\fun{vars}{S}}$ implies $\sigma\leq\theta\abs{\fun{vars}{S}}$ where \fun{vars}{S} is the set of all variable or $d$-variable symbols in $S$.
\cite{conf/ijcai/Hentenryck87}
\end{defi}

\begin{defi}[Constraint]
Let $p$ be a $n$-ary predicate symbol, $p$ is a \term{constraint} \iffTx for any ground terms either has a successful refutation or has only finitely failed derivations.
\cite{conf/ijcai/Hentenryck87}
\end{defi}

\begin{defi}[forward checkable literal, forward-variable]
A literal $\pfun{x_1,\ldots,x_n}$ in the resolvent is \term[forward checkable literal]{forward checkable} \iffTx{} $p$ is a constraint and all it's arguments are ground but one which is a $d$-variable. This $d$-variable is called the \term{forward-variable}.
\cite{conf/ijcai/Hentenryck87}
\end{defi}

\begin{defi}[forward checking inference rule]
Let $G_1$ be the goal $\leftarrow A_1,\ldots,A_{m-1},P,A_{m+1},\ldots A_k$ and $PR$ be a logic program. $G_{i+1}$ is derived from $G_i$ using the mgu $\theta_{i+1}$ via $PR$ if the following conditions hold:
\begin{enumerate}
 \item $P$ is forward checkable and $x^d$ is the forward variable
 \item $d_{\mbox{new}}=\accol{a\in d|PR\vDash P\accol{x/a}}$ and $d_{\mbox{new}}\neq\emptyset$
 \item $\theta_{i+1}$ is
 \begin{enumerate}
  \item $\accol{x^d/e}$ if $d_{\mbox{new}}=\accol{e}$
  \item $\accol{x^d/z^{d_{\mbox{new}}}}$ where $z^d_{\mbox{new}}$ is a new variable otherwise.
 \end{enumerate}
 \item $G_{i+1}$ is the goal $\leftarrow\brak{A_1,\ldots,A_{m-1},A_{m+1},\ldots A_k}\theta_{i+1}$
\end{enumerate}
Such a derivation rule is called the \termabbrev{forward checking inference rule}{FCIR}.
\cite{conf/ijcai/Hentenryck87}
\end{defi}

\begin{defi}[Efficient computation rule with respect to forward declarations]
A computation rule is \term[Efficient computation rule with respect to forward declarations]{efficient with respect to the forward declarations}, if it selects only a predicate submitted to forward declaration when it is ground or forward checkable and if, whenever the resolvent contains literals submitted to a forward declaration which are either forward-checkable or ground, it selects one of them.
\cite{conf/ijcai/Hentenryck87}
\end{defi}

\begin{defi}[lookahead checkable literal, lookahead-variable]
A literal $\pfun{t_1,\ldots,t_n}$ in the resolvent is \term[lookahead checkable literal]{lookahead checkable} \iffTx{} $p$ is a constraint and their exists at least one $t_i$ which is a domain-variable and each other argument is either ground or a domain-variable. The domain-variables in $t_1,\ldots,t_n$ are called the \term{lookahead-variable}s.
\cite{conf/ijcai/Hentenryck87}
\end{defi}

\begin{defi}[lookahead inference rule]
Let $G_1$ be the goal $\leftarrow A_1,\ldots,A_{m-1},P,A_{m+1},\ldots A_k$ and $PR$ be a logic program. $G_{i+1}$ is derived from $G_i$ using the mgu $\theta_{i+1}$ via $PR$ if the following conditions hold:
\begin{enumerate}
 \item $P$ is lookahead checkable and $x_1,\ldots,x_n$ are the lookahead variables whose domains are $d_{x_1},\ldots d_{x_n}$.
 \item For each $x_j^{d_{x_j}}$ let:
 \begin{enumerate}
  \item $d_{z_j}=\accol{y_j\in d_{x_j}|\exists y_1\in d_{x_1},\ldots,y_{j-1}\in d_{x_{j-1}},y_{j+1}\in d_{x_{j+1}},\ldots,y_n\in d_{x_n}\mbox{ such that }PR\vDash P\theta\mbox{ with }\theta=\accol{x_1/y_1,\ldots,x_n/y_n}}$ and $d_{x_j}\neq\emptyset$.
  \item $e_j$ as
  \begin{enumerate}
   \item a new variable of domain $d_{z_j}$ if $d_{z_j}=\abs{\accol{e_1,\ldots,e_l}}>1$.
   \item the constant $e$ if $d_{z_j}$ if $d_{z_j}=\accol{e}$.
  \end{enumerate}
 \end{enumerate}
 \item $\theta_{i+1}=\accol{x_1/z_1,\ldots,x_n/z_n}$
 \item $G_{i+1}$ is the goal
 \begin{enumerate}
  \item $\leftarrow\brak{A_1,\ldots,A_{m-1},A_{m+1},\ldots A_k}\theta_{i+1}$ if at most one $z_i$ is a $d$-variable.
  \item $\leftarrow\brak{A_1,\ldots,A_{m-1},P,A_{m+1},\ldots A_k}\theta_{i+1}$ otherwise.
 \end{enumerate}
\end{enumerate}
Such a derivation rule is called the \termabbrev{lookahead inference rule}{LAIR}.
\cite{conf/ijcai/Hentenryck87}
\end{defi}

\begin{defi}[Efficient computation rule with respect to the lookahead declarations]
A computation rule is \term[efficient computation rule with respect to the lookahead declarations]{efficient with respect to the lookahead declarations}, if a literal in the resolvent submitted to a lookahead declaration is only selected if either it is lookahead checkable or all its arguments are ground.
\cite{conf/ijcai/Hentenryck87}
\end{defi}

\begin{defi}[\fun{Revise}{M,a}]
Let $a$ be a formula and M a model. \term{\fun{Revise}{M,a}} is the set of models $M'$ such that
\begin{enumerate}
\item \label{reviseIt1} $M'$ and $M$ have the same universe and agree on all functions.
\item \label{reviseIt2} $a$ and the protected formulas of $T$ are true in $M'$.
\item There is no other model $M''$ such that for some $1<i<1$,
\begin{enumerate}
\item $M''$ satisfies (\ref{reviseIt1}) and (\ref{reviseIt2});
\item $M''$ and $M'$. agree on all predicates in strata $1$ through $i-1$; and
\item the differences between M" and M on predicates in stratum $i$ are a proper subset of the differences between $M'$ and $M$ on those predicates.
\end{enumerate}
\end{enumerate}
\cite{conf/ijcai/Winslett89}
\end{defi}

\begin{defi}[Preferred model at the $i$-th stratum]
A model $\calM_1$ is \term[Preferred model at the $i$-th stratum]{preferred to model $\calM_2$ at the $i$-th stratum} (written $\calM_1<_i\calM_2$) \iffTx{}
\begin{enumerate}
\item $\calM_1$ and $\calM_2$ have identical universes;
\item $\calM_1$ and $\calM_2$ agree on all predicates and functions, except possibly those of $S_i$ and $V_i$;
\item For all predicates $P$ in stratum $S_t$, and for all $\bar{x}$ such that $\fun{P}{\bar{x}}$ is true in $\calM_1$, $\fun{P}{\bar{x}}$ is true in $\calM_2$; and
\item For some predicate $P$ in stratum $S_i$ and some $\bar{x}$, $\fun{P}{\bar{x}}$ is false in $\calM_1$ and true in $\calM_2$.
\end{enumerate}
\cite{conf/ijcai/Winslett89}
\end{defi}

\begin{defi}[Abductive problem, Background theory of the abductive problem, Abducible set of the abductive problem, Goal of the abductive problem, Solution of an abductive problem, Minimal solution of an abductive problem]
The triple $\tuple{\Pi,A,g}$ is an \term{abductive problem} \iffTx{} $\Pi$ is a set of propositional Horn Clauses (called the \term{background theory of the abductive problem}), $A$ a set of propositions (called the \term{abducible set of the abductive problem}) and $g$ is a proposition (called the \term{goal of the abductive problem}). The set of propositions $\Delta$ is a \term[solution of an abductive problem]{solution of the abductive problem} $\tuple{\Pi,A,g}$ \iffTx{}
\begin{enumerate}
\item $\Delta\subseteq A$
\item $\Delta\cup\Pi\vdash g$
\item $\Delta\cup\Pi$ is consistent
\item $\brak{a\in A \wedge \Delta+\Pi\vdash a}\rightarrow a\in\Delta$
\end{enumerate}
$\Delta$ is a \term[Minimal solution of an abductive problem]{minimal solution of} $\tuple{\Pi,A,g}$ \iffTx{} it is a solution of $\tuple{\Pi,A,g}$ and no subset of $\Delta$ is a solution of $\tuple{\Pi,A,g}$.
\cite{conf/ijcai/Eshghi93}
\end{defi}

\begin{defi}[Only-if set, \fun{only-if}{T,S}, \fun{only-if}{T}, \fun{props}{T}]
Let the clauses $p\leftarrow Q_1,p\leftarrow Q_2,\ldots,p\leftarrow Q_k$. where $Q_1,Q_2,\ldots,O_k$ are conjunctions of propositions, be all the clauses in $\Pi$ which have $p$ at their head. Let $n_{c_1},n_{c_2},...,n_{c_k}$ be the names of these clauses. Then the \term{only-if set} of p with respect to $\Pi$ is $\accol{\neg p\vee n_{c_1}\vee n_{c_2}\vee\ldots\vee n_{c_k},n_{c_1}\rightarrow Q_1,n_{c_2}\rightarrow Q_2,\ldots,n_{c_k}\rightarrow Q_k}$. For at set of propositions $S$ and the Horn clause theory $T$, we use \term{\fun{only-if}{T,S}} to denote the union of only-if sets of all the propositions in $S$ with respect to $T$. We use \term{\fun{only-if}{T}} to denote \fun{only-if}{T,\fun{props}{T}} where \term{\fun{props}{T}} is the set of all propositions in $T$.
\cite{conf/ijcai/Eshghi93}
\end{defi}

\begin{defi}[Truth-assignment, Model]
Let $C$ be a propositional clausal theory. $S$ the set of propositions in $C$, and $M$ a set of propositions. Then the \term{truth-assignment} induced by $M$ is the assignment of true to all propositions in $S$ which are in $M$. and false to those propositions in $S$ which are not in $M$. $M$ is a \term{model} of $C$ \iffTx{} the truth assignment induced by $M$ satisfies all the clauses in $C$.
\cite{conf/ijcai/Eshghi93}
\end{defi}

\begin{defi}[Model which minimizes, Minimization with respect to]
Let $C$ be a propositional clausal theory. and $A$ a set of propositions. Then $M$ is a \term[model which minimizes]{model of $C$ which minimizes $A$} \iffTx{}
\begin{enumerate}
 \item $M$ is a model of $C$
 \item There is no other model $M'$ of $C$ such that $M'\cap A\subseteq M\cap A$
\end{enumerate}
We say that $\Delta$ is a \term[minimization with respect to]{minimization of $A$ with respect to $C$} \iffTx{} there is a model $M$ of $C$ which minimizes $A$ and $\Delta=M\cap A$
\cite{conf/ijcai/Eshghi93}
\end{defi}

\begin{defi}[Unit refutable]
A propositional clausal theory $C$ is \term{unit refutable} \iffTx{}, for every set of unit clauses $U$, if $C\cup U$ is inconsistent. then the empty clause is unit-derivable from $C\cup U$.
\cite{conf/ijcai/Eshghi93}
\end{defi}

\begin{defi}[Connection graph]
Given a set of clauses $C$. the \term{connection graph} of $C$ is the graph obtained by drawing a link between each complimentary pair of literals in the set.
\cite{conf/ijcai/Eshghi93}
\end{defi}

\begin{defi}[Chain]
Given the set of clauses $C$, the sequence $\flatbrak{\brak{x_1,e_1,y_1},\brak{x_2,e_2,y_2},\ldots,\brak{x_n,e_n,y_n}}$ of $c$-triples is a \term{chain} in $C$ \iffTx{} for all $k$, $e_k$ is a clause in $C$ and $y_k=\neg x_{k+1}$.
\cite{conf/ijcai/Eshghi93}
\end{defi}

\begin{defi}[Tied chain]
$\flatbrak{\brak{x_1,e_1,y_1},\brak{x_2,e_2,y_2},\ldots,\brak{x_n,e_n,y_n}}$ is a \term{tied chain} in $C$ \iffTx{} it is a chain in $C$ and $x_1=y_n$.
\cite{conf/ijcai/Eshghi93}
\end{defi}

\begin{defi}[Subgoal clause]
Let $C$ be a set of clauses. Let $T$ be a connection tableau for $C$. Let $s_1,s_2,\ldots,s_n$, $n\geq 1$ be the subgoals of $T$ Then we call $s_1\vee s_2\vee\ldots\vee s_n$ the \term{subgoal} clause of $T$.
\cite{conf/ijcai/Fuchs99}
\end{defi}

\begin{defi}[Query tableau, Query clause]
Let $\calC$ be a set of clauses. Let $T$ be a connection tableau for $\calC$. Let $\calS\subseteq \calC$ be a set of start clauses. Let $S$ be the clause below the unlabeled root of $T$. If $S$ is an instance of a clause from $\calS$ we call$T$ a \term{query tableau} (\wrtTx{} $\calS$) and the subgoal clause of $T$ a \term{query clause} (\wrtTx{} $S$).
\cite{conf/ijcai/Fuchs99}
\end{defi}

\begin{defi}[Lemma tableau, Lemma clause]
Let $\calC$ be a clause set. Let $T$ be a connection tableau for $\calC$. Let $C=s_1\vee s_2\vee\ldots\vee s_n$ be the subgoal clause of $T$. Let $\calH$ be the set of subgoals which are immediate successors of the root. If $\calH\neq\emptyset$ we call $T$ a \term{lemma tableau}. Then, let $s+i$, $1\leq i\leq n$, be the element of $\calH$ which is left-most in $T$. We call the contrapositive $s_i\leftarrow\neg s_1\wedge\neg s_2\wedge\ldots\wedge\neg s_{i-1}\wedge\neg s_{i+1}\wedge\ldots\wedge\neg s_n$ of $C$ the \term{lemma clause} of $T$.
\cite{conf/ijcai/Fuchs99}
\end{defi}

\begin{defi}[Legal global instances]
Given an open global system $\frakG=\accol{\tuple{\varphi_1,v_1},\tuple{\varphi_2,v_2},\ldots,\tuple{\varphi_n,v_n}}$, the set of \term{legal global instances} is $\fun{Linst}{\frakG}=\accol{D\mbox{ instance of }\calR|v_i\subseteq\fun{\varphi_i}{D},i=1,\ldots,n}$.
\cite{conf/ijcai/BravoB03}
\end{defi}

\begin{defi}[Certain answer]
Given an open global system $\frakG$ and a query $\fun{Q}{X}$ to the system, a tuple $t$ is a \term{certain answer} to $Q$ in $\frakG$ if for every global instance $D\in\fun{Linst}{\frakG}$, it holds $D\vDash\funf{Q}{t}$. We denote with $\fun{Certain_{\frakG}}{Q}$ the set of certain answers to $Q$ in $\frakG$.
\cite{conf/ijcai/BravoB03}
\end{defi}

\begin{defi}[Minimal legal global instance, Minimal legal global instances]
Given a global system, $\frakG$, an instance $D$ is \term[Minimal legal global instance]{minimal} if $D\in\fun{Linst}{\frakG}$ and is minimal \wrtTx{} set inclusion, i.e. there is no other instance in \fun{Linst}{\frakG} that is a proper subset of $D$ (as a set of atoms). We denote by $\fun{Mininst}{\frakG}$ the \term[Minimal legal global instances]{set of minimal legal global instances of $\frakG$ \wrtTx{} set inclusion}.
\cite{conf/ijcai/BravoB03}
\end{defi}

\begin{defi}[Consistent global system]
A global system $\frakG$ is \term[Consistent global system]{consistent} \wrtTx{} $IC$. if for all $D\fun{Mininst}{\frakG}$, $D\vDash IC$.
\cite{conf/ijcai/BravoB03}
\end{defi}

\begin{defi}[Minimal answer, Minimal answers]
The ground tuple $\bar{a}$ is a \term{minimal answer} to a query $Q$ posed to $\frakG$ if for every $D\in\fun{Miminst}{\frakG}$, $\bar{a}\in\fun{Q}{D}$, where $\fun{Q}{D}$ is the answer set for $Q$ in $D$. The set of \term{minimal answers} is denoted by $\fun{Minimal_{\frakG}}{Q}$.
\cite{conf/ijcai/BravoB03}
\end{defi}

\begin{defi}[Database distance]
Let $D,D'$ be database instances over the same schema and domain. The \term[database distance]{distance} $\fun{\Delta}{D,D'}$, between $D$ and $D'$ is the symmetric difference $\fun{\Delta}{D,D'}=\brak{\fun{\Sigma}{D}\setminus \fun{\Sigma}{D'}}\cup\brak{\fun{\Sigma}{D'}\setminus\fun{\Sigma}{D}}$
\cite{conf/ijcai/BravoB03}
\end{defi}

\begin{defi}[Database order relation]
For database instances $D,D',D''$, we define $D'\term[Database order relation $\leq_D$]{\leq_D} D''$ if $\fun{\Delta}{D,D'}\subseteq\fun{\Delta}{D,D''}$
\cite{conf/ijcai/BravoB03}
\end{defi}

\begin{defi}[Repair of a global system]
Let $\frakG$ be a global system and $IC$ a set of global $IC$'s. A \term[repair of a global system]{repair} of $\frakG$ \wrtTx{} $IC$ is a global database instance $D'$. such that $D'\vDash IC$ and $D'$ is $\leq_D$-minimal for some $D\in\fun{Mininst}{\frakG}$.
\cite{conf/ijcai/BravoB03}
\end{defi}

\begin{defi}[Consistent answer]
Given a global system $\frakG$, a set of global integrity constraints $IC$. and a global first-order query $\fun{Q}{X}$. we say that a (ground) tuple $t$. is a \term{consistent answer} to $Q$ \wrtTx{} $IC$ \iffTx{} for every repair $D$ of $\frakG$. $D\vDash\funf{Q}{t}$.
\cite{conf/ijcai/BravoB03}
\end{defi}

\begin{defi}[Consistent answers]
We denote by $\fun{Consis_{\frakG}}{Q}$ the set of \term{consistent answers} to $Q$ in $\frakG$.
\cite{conf/ijcai/BravoB03}
\end{defi}

\begin{defi}[Program of an open global system]
\label{def:poaogs}
Given an open global system $\frakG$, the \term[Program of an open global system]{program}, $\fun{\Pi}{\frakG}$, contains the following clauses:
\begin{enumerate}
\item \label{item:poags1}Fact \dom{a} for every constant $a\in\calU$: and the fact \fun{V}{\bar{a}} whenever $\fun{V}{\bar{a}}\in v_i$ for some source extension $v_i$ in $\frakG$.
\item \label{item:poags2}For every View (source) predicate V in the system with description $\fun{V}{\bar{X}}\leftarrow\fun{P_1}{X_1}\wedge\fun{P_2}{X_2}\wedge\ldots\wedge\fun{P_n}{X_n}$, the rules $\fun{P_j}{X_j}\wedge\displaystyle\bigwedge_{X_i\in\brak{X_j\setminus\bar{X}}}\fun{F_i}{\bar{X},X_i}$, $j=1,\ldots,n$
\item \label{item:poags3}For every predicate $\fun{F_i}{\bar{X},X_i}$ introduced in \ref{item:poags2}, the rule $\fun{F_i}{\bar{X},X_i}\leftarrow\fun{V}{\bar{X}}\wedge\dom{\bar{X}}\wedge\fun{choice}{\bar{X},X_i}$.
\end{enumerate}
\cite{conf/ijcai/BravoB03}
\end{defi}

\begin{defi}[Instance associated to a stable model]
The \term{instance associated to a stable model} $\calM$ of $\fun{\Pi}{\frakG}$ is $D_{\calM}=\accol{\Pfun{a}|P\in R\wedge\pfun{a}\in\calM}$.
\cite{conf/ijcai/BravoB03}
\end{defi}

\begin{defi}[Repair program]
The \term{repair program} $\fun{\Pi}{\frakG,IC}$, of $\frakG$ \wrtTx{} $IC$ contains the following clauses:
\begin{enumerate}
\item Facts as in \defiref{poaogs} (item \ref{item:poags1}).
\item Each of the rules of item \ref{item:poags2} in \defiref{poaogs} is replaced by $\fun{P_j}{X_j,t_d}\leftarrow\fun{V}{\bar{X}}\wedge\displaystyle\bigwedge_{X_i\in\brak{X_j\setminus\bar{X}}}\fun{F_i}{\bar{X},X_i}$
\item Exactly the same rules as in item \ref{item:poags3} in \defiref{poaogs}.
\item For every predicate $P\in\calR$, the clauses:
\begin{equation}
\group{\Pfun{\bar{X},t*}\leftarrow\Pfun{\bar{X},t_d}\wedge\dom{\bar{X}}\\
\Pfun{\bar{X},t*}\leftarrow\Pfun{\bar{X},t_a}\wedge\dom{\bar{X}}\\
\Pfun{\bar{X},f*}\leftarrow\Pfun{\bar{X},f_a}\wedge\dom{\bar{X}}\\
\Pfun{\bar{X},t*}\leftarrow\dom{\bar{X}}\wedge\neg\Pfun{\bar{X},t_d}}
\end{equation}
\item For every first-order global universal $IC$ of the form
\begin{equation}
\forall\brak{\fun{Q_1}{\bar{Y}_1}\vee\fun{Q_2}{\bar{Y}_2}\vee\ldots\vee\fun{Q_n}{\bar{Y}_n}\leftarrow\fun{P_1}{\bar{X}_1}\wedge\fun{P_2}{\bar{X}_2}\wedge\ldots\wedge\fun{P_m}{\bar{X}_m}\wedge\varphi}
\end{equation}
where $P_i,Q_j\in\calR$, and $\varphi$ is a conjunction of built-in atoms. The clause $\displaystyle\bigvee_{i=1}^n\fun{P_i}{X_i,f_a}\displaystyle\bigvee_{j=1}^m\fun{Q_j}{Y_j,t_a}\leftarrow\displaystyle\bigwedge_{i=1}^n\fun{P_i}{X_i,t*}\displaystyle\bigwedge_{j=1}^m\fun{Q_j}{Y_j,f*}\wedge\dom{\bar{X}}\wedge\varphi$ where $\bar{X}$ is the tuple of all variables appearing in database
atoms in the rule.
\item For every referential $IC$ of the form $\forall x\brak{\fun{P}{x}\rightarrow\exists y\fun{Q}{x',y}}$. with $x'\subseteq$, the clauses:
\begin{equation}
\group{\fun{P}{X,f_a}\vee\fun{Q}{X',\nullmath,t_a}\leftarrow\fun{P}{X,t*}\wedge\fun{\mbox{aux}}{\bar{X'}}\wedge\neg\fun{Q}{\bar{X'},\nullmath,t_d}\wedge\dom{\bar{X}}\\
\fun{\mbox{aux}}{\bar{X'}}\leftarrow\fun{Q}{\bar{X'},Y,t_d}\wedge\neg\fun{Q}{X',Y,f_a}\wedge\dom{X',Y}\\
\fun{\mbox{aux}}{\bar{X'}}\leftarrow\fun{Q}{\bar{X'},Y,t_a}\wedge\dom{X',Y}}
\end{equation}
\item For every predicate $P\in\calR$, the interpretation clauses:
\begin{equation}
\group{\Pfun{\bar{a},f**}\leftarrow\Pfun{\bar{a},f_a}\\
\Pfun{\bar{a},f**}\leftarrow\neg\Pfun{\bar{a},t_d}\wedge\neg\Pfun{\bar{a},t_a}\\
\Pfun{\bar{a},t**}\leftarrow\neg\Pfun{\bar{a},t_a}\\
\Pfun{\bar{a},t**}\leftarrow\neg\Pfun{\bar{a},t_d}\wedge\neg\Pfun{\bar{a},f_a}\\
}
\end{equation}
\end{enumerate}
\cite{conf/ijcai/BravoB03}
\end{defi}

\begin{defi}[Instance associated to a choice model]
The \term{instance associated to a choice model} $\calM$ of \fun{\Pi}{\frakG,IC} is $D_{\calM}=\accol{\Pfun{a}|\Pfun{a,t**}\in\calM}$.
\cite{conf/ijcai/BravoB03}
\end{defi}

\begin{defi}[Normal logic program]
A \term{normal logic program} $P$ is a set of rules of the form:
\begin{equation}
a\leftarrow b_1\wedge b_2\wedge\ldots\wedge b_n
\end{equation}
Where $a$ is an atom and the $b_i$ are in $\mbox{Lit}^{\sim}$.
\cite{conf/ijcai/BrewkaK93}
\end{defi}

\begin{defi}[Closure of a set of negative literals of a logic program]
Let $L\subseteq\fun{\mbox{NEG}}{P}$ be a set of negative literals, $P$ a normal logic program. The \term[Closure of a set of negative literals of a logic program]{closure} of $L$ under $P$, $\fun{C_P}{L}$, is the smallest set such that:
\begin{enumerate}
 \item $L\subseteq\fun{C_P}{L}$,
 \item if $a\leftarrow b_1,b_2,\ldots b_n\in P$ and $b_1,b_2,\ldots,b_n\in\fun{C_P}{L}$ then $a\in\fun{C_P}{L}$.
\end{enumerate}
\cite{conf/ijcai/BrewkaK93}
\end{defi}

\begin{defi}[Normal logic program-consistent]
Let $L\subseteq\fun{\mbox{NEG}}{P}$ be it set of negative literals, $P$ a normal logic program. $L$ is \term[Normal logic program-consistent]{$P$-consistent} \iffTx{} $\fun{C_P}{L}$ is consistent.
\cite{conf/ijcai/BrewkaK93}
\end{defi}

\begin{defi}[Consistent normal logic program]
Let $P$ be a normal logic program. $P$ is \term[Consistent normal logic program]{consistent} \iffTx{} $\emptyset$ is $P$-consistent.
\cite{conf/ijcai/BrewkaK93}
\end{defi}

\begin{defi}[Logic program cover of a closure]
Let $P$ be a logic program, $H\subseteq\fun{\mbox{NEG}}{P}$, and $C$ a $P$-closure of $H$. The \term[Logic program cover of a closure]{$P$-cover} of $C$, \fun{\mbox{COV}_P}{C}, is the set
\begin{equation}
H\cup\accol{\neg a\in\fun{\mbox{NEG}}{P}|a\in C}
\end{equation}
\cite{conf/ijcai/BrewkaK93}
\end{defi}

\begin{defi}[Extension of a logic program, Extension base of a logic program]
Let $P$ be a logic program, $H\subseteq\fun{\mbox{NEG}}{P}$, and $C$ a $P$-closure of $H$. $C$ is an \term[Extension of a logic program]{extension of P} \iffTx{}
\begin{enumerate}
 \item $C$ is consistent,
 \item there is no $H'\subseteq\fun{\mbox{NEG}}{P}$ with consistent closure $C'$ such that $\fun{\mbox{COV}_P}{C}\subsetneq\fun{\mbox{COV}_P}{C'}$.
\end{enumerate}
if $C$ is an extension, $H$ is called an \term[Extension base of a logic program]{extension base of $P$}.
\cite{conf/ijcai/BrewkaK93}
\end{defi}

\begin{defi}[General logic program]
A \term{general logic program} $P$ is a set of rules of the form:
\begin{equation}
c_1\vee c_2\vee\ldots\vee c_m\leftarrow b_1\wedge b_2\wedge\ldots\wedge b_n
\end{equation}
where all the literals $c_i$ and $b_j$ are in $\mbox{Lit}^+$.
\cite{conf/ijcai/BrewkaK93}
\end{defi}

\begin{defi}[General logic program closure of a set of literals]%8
Let $L$ be a set of literals from $\fun{\mbox{NEG}}{P}$. $C$ is a \term[General logic program closure]{$P$-closure of $L$} if it is a smallest set such that:
\begin{enumerate}
 \item $L\subseteq C$,
 \item if $l$ and $\neg l$ or $a$ and $\neg a$ are in $C$, then $C=\mbox{Lit}^+$
 \item if $c_1\vee c_2\vee\ldots\vee c_m\leftarrow b_1\wedge b_2\wedge\ldots\wedge b_n\in P$ and $b_1,b_2,\ldots,b_n\in C$ then for at least one $c_i$, $c_i\in C$.
\end{enumerate}
\cite{conf/ijcai/BrewkaK93}
\end{defi}

\begin{defi}[Extension of a default theory]
Let $\fun{\Gamma}{S}$ be any least set such that:
\begin{enumerate}
 \item $W\subseteq\fun{\Gamma}{S}$
 \item $\fun{\Gamma}{S}$ is deductively closed
 \item If $\alpha\in\fun{\Gamma}{S}$ and all $\beta_i$ are consistent with $S$, then one of $\gamma_i$ is in $\fun{\Gamma}{S}$.
\end{enumerate}
An \term[Extension of a default theory]{extension of $\tuple{W,D}$} is any set $S$ that is equal to some $\fun{\Gamma}{S}$.
\cite{conf/ijcai/BrewkaK93}
\end{defi}

\begin{defi}[Cover with respect to a default theory]
Let $C$ be some $\fun{\Gamma}{S}$ the \term[Cover with respect to a default theory]{cover \fun{\mbox{COV}_\Delta}{C} with respect to $\Delta=\tuple{W,D}$} is the set:
\begin{equation}
\bar{S}\cup C
\end{equation}
Where $\bar{S}$ are all the sentences not in $S$.
\cite{conf/ijcai/BrewkaK93}
\end{defi}

\begin{defi}[Abductive extension of a default theory]
Let $C$ be some $\fun{\Gamma}{S}$. $C$ is an \term[Abductive extension of a default theory]{abductive extension of $\Delta$} \iffTx{}
\begin{enumerate}
 \item $C\subseteq S$,
 \item $\fun{\mbox{COV}_\Delta}{S}$ is maximal.
\end{enumerate}
\cite{conf/ijcai/BrewkaK93}
\end{defi}

\begin{defi}[Moderately-grounded extension of a set of set of sentences]
Let $A$ be a set of sentences of $\calL$. Let $U$ be a subset of $\calL_0$ and $\bar{U}$ its complement $\calL_0\setminus U$. $U$ is a \term[Moderately-grounded extension of a set of set of sentences]{moderately-grounded extension of $A$} \iffTx{} it satisfies the equation:
\begin{equation}
U=\accol{\phi\in\calL_0|A\cup\neg L\bar{U}\vdash_{\mbox{K45}}\phi}
\end{equation}
\cite{conf/ijcai/BrewkaK93}
\end{defi}

\begin{defi}[Cover of an $A$ theory under hypotheses]
The \term[[Cover of an $A$ theory under hypotheses]{cover of an $AE$ theory $A$ under hypotheses $S$}, \fun{\mbox{COV}_A}{\fun{C_A}{S}}, is given by:
\begin{equation}
S\cup\accol{\neg L\phi|\phi\in\fun{C_A}{S}}
\end{equation}
\cite{conf/ijcai/BrewkaK93}
\end{defi}

\begin{defi}[Abductive extension of an $AE$ theory]
The closure of an $AE$ theory $A$, \fun{C_A}{S}, is an \term[Abductive extension of an $AE$ theory]{abductive extension of $A$} \iffTx{}
\begin{enumerate}
 \item \fun{C_A}{S} is consistent
 \item \fun{\mbox{COV}_A}{\fun{C_A}{S}} is maximal.
\end{enumerate}
\cite{conf/ijcai/BrewkaK93}
\end{defi}


\section{Game Playing}

\begin{defi}[Maxset, Minset]
The \term{maxset} $X^M$ is that subset of $X$ at which it is MAX's turn to make a move. The \term{minset} $X^m$ is that subset of $X$ at which it is MIN's turn to make a move.
\cite{conf/ijcai/Boffey73}
\end{defi}

\begin{defi}[Evaluation function, Better position]
An \termabbrev{evaluation function}{EF} over a game graph $\tuple{X,\Gamma}$ is a single valued function $\funsig{f}{X}{\RRR}$. If $\ffun{x}>\ffun{y}$ then $x$ is said to be a \term{better position} than $y$ under $f$.
\cite{conf/ijcai/Boffey73}
\end{defi}

\begin{defi}[Strategy $\calS_o$]
$\calS_o$ is that \term[Strategy $\calS_o$]{strategy} which leads to move $xy*$ being made from position $x$ where:
\begin{equation}
\group{\ffun{x*}=\displaystyle\max_{y\in\Gamma_x}\ffun{y}\mbox{ if }x\in X^M\\
\ffun{x*}=\displaystyle\min_{y\in\Gamma_x}\ffun{y}\mbox{ if }x\in X^m}
\end{equation}
Ties for $y*$ are broken by some subsidiary rule.
\cite{conf/ijcai/Boffey73}
\end{defi}

\begin{defi}[Locally equivalent evaluation functions, Locally perfect evaluation function]
Evaluation functions $f$ and $g$ are \term[locally equivalent evaluation functions]{locally equivalent} if
\begin{equation}
\ffun{x}>\ffun{y}\mbox{ whenever }\gfun{x}>\gfun{y}
\end{equation}
where $x,y\in\Gamma_w$ for some $w\in X$. $f$ is \term[locally perfect evaluation function]{locally perfect} if it is locally equivalent to $\lambda_{mm}^w$.
\cite{conf/ijcai/Boffey73}
\end{defi}

\begin{defi}[Probabilistic evaluation function]
A \termabbrev{probabilistic evaluation function}{PEF} over a game graph $\tuple{X,\Gamma}$ is a bounded function $\funsig{p}{X\times\RRR}{\RRR}$ with $\pfun{x,r}\geq0$, $\displaystyle\int{\pfun{x,r}\ dr}=1$ and $\pfun{x,r}=0$ outside some finite range $\fun{r_1}{x}\leq r\leq \fun{r_2}{x}$.
\cite{conf/ijcai/Boffey73}
\end{defi}

\begin{defi}[Strategy $\calS_o'$]
$\calS_o'$ is that \term[Strategy $\calS_o'$]{strategy} which leads to move $xy*$ being made from position $x$ where:
\begin{equation}
\group{\fun{v}{x*}_p=\displaystyle\max_{y\in\Gamma_x}\fun{v}{y}_p\mbox{ if }x\in X^M\\
\fun{v}{x*}_p=\displaystyle\min_{y\in\Gamma_x}\fun{v}{y}_p\mbox{ if }x\in X^m}
\end{equation}
\cite{conf/ijcai/Boffey73}
\end{defi}

\begin{defi}[(Constant) decision rule, Preference function, Preferred move]
A (constant) \termor{decision rule}{preference function} for MAX is a single valued function \funsig{\sigma_M}{X}{X} with $\fun{\sigma_M}{x}\in\Gamma_x$. $x\mapsto\fun{\sigma_M}{x}$ is MAX's \term{preferred move} from $x$. A preference function $\sigma_m$ for MIN can be defined in an analogous way.
\cite{conf/ijcai/Boffey73}
\end{defi}

\begin{defi}[(Constant) preference pair]
If $\sigma_M$ and $\sigma_m$ are preference functions for MAX and MIN respectively then $\sigma=\brak{\sigma_M,\sigma_m}$ is termed a (constant) \term{preference pair} on $\brak{X,\Gamma}$.
\cite{conf/ijcai/Boffey73}
\end{defi}

\begin{defi}[Strategy order relation $\leq_\sigma$]
If $u,v\in X$ and $\sigma$ is a preference pair on $\brak{X,\Gamma}$ then \term[Strategy order relation $\leq_\sigma$]{$u\geq_\sigma v$} \iffTx{} there exists an element $t\in X$ such that
\begin{enumerate}
 \item $u,v\in\Gamma^t$
 \item $u=\fun{\sigma_M}{t}$ or $v=\fun{\sigma_m}{t}$
\end{enumerate}
\cite{conf/ijcai/Boffey73}
\end{defi}

\begin{defi}[Evaluation function reproduces a preference pair]
An evaluation function $f$ over $\brak{X,\Gamma}$ \term[Evaluation function reproduces]{reproduces} a preference pair $\sigma$ (with respect to a strategy $\calS_0$) if $\ffun{x}>\ffun{y}$ whenever $x>_\sigma y$.
\cite{conf/ijcai/Boffey73}
\end{defi}

\begin{defi}[Strategy order relation $\leq_\sigma^*$]
If $u,v\in X$ and $\sigma$ is a preference pair on $\brak{X,\Gamma}$ then \term[Strategy order relation $\leq_\sigma^*$]{$u\geq_\sigma^* v$} \iffTx{} there exists is a sequence $u=w_0,w_1,\ldots,w_s=v$ such that $w_i>_\sigma w_{i-1}$, $i=1,2,\ldots,s$
\cite{conf/ijcai/Boffey73}
\end{defi}

\begin{defi}[Fully consistent strategy]
$\sigma$ is \term[Fully consistent strategy]{fully consistent} if $C_{\pi_1}=C_{\pi_2}$ for any pair of paths $\pi_1,\pi_2$ with the same endpoints.
\cite{conf/ijcai/Boffey73}
\end{defi}

\begin{defi}[(Variable) preference function]
A \term{(Variable) preference function} for MAX is a single-valued function \funsig{\tau_M}{X^M}{\Gamma\times X^M\setminus I} where $I$ is the unit interval $I=\accol{x|0\leq x\leq 1}$ and
\begin{enumerate}
 \item $\fun{\tau_M}{x,u}=0$ if $u\notin\Gamma_x$
 \item $\fun{\tau_M}{x,u}\geq 0$ if $u\in\Gamma_x$
 \item $\displaystyle\sum_{u\in\Gamma_x}{\fun{\tau_M}{x,u}}=1$
\end{enumerate}
\fun{\tau_M}{x,u} can be thought of as the probability that, when in position $x$, MAX will choose to move to position $u$. A variable preference rule $\tau_m$ for MIN can be defined in an analogous way.
\cite{conf/ijcai/Boffey73}
\end{defi}

\begin{defi}[(Variable) preference pair]
If $\tau_M$ and $\tau_m$ are variable preference functions for MAX and MIN respectively then $\tau=\brak{\tau_M,\tau_m}$ is termed a \term{(variable) preference pair} on \brak{X,\Gamma}.
\cite{conf/ijcai/Boffey73}
\end{defi}

\section{Knowledge Representation}

\begin{defi}[Belief change scenario]
A \term{belief change scenario} is a triple $B=\tuple{K,R,C}$, where $K$, $R$ and $C$ are sets of formula over a fixed propositional language $\calLcalP$. Informally, $K$ is a \term{knowledge base} which is to be modified in such a way that the resulting knowledge base includes all elements from $R$ and does not include any element from $C$. The modified knowledge base corresponding to $B$ will be denoted by $K\dotplus R\dotminus C$.
\cite{conf/fedcsis/KorpusikLM12}
\end{defi}

\begin{defi}[Belief change extension, Unique (inconsistent) belief change extension]
Let $B=\tuple{K,R,C}$ be a belief change scenario over $\calLcalP$. Define a new set $\calP'$ of atoms, isomorphic with $\calP$, given by $\calP'=\accol{p':p\in\calP}$. Let $K'$ be a knowledge base obtained from $K$ by replacing any $p\in\calP$ by $p'\in\calP'$. Let $EQ$ be a maximal (with respect to set inclusion) set of equivalences $\accol{p\Leftrightarrow p'|p\in\calP}$ such that $\fun{Th}{K'\cup EQ\cup R}\cap\brak{C\cup\bot}=\emptyset$. The set $\fun{Th}{K'\cup EQ\cup R}\cap\calLcalP$ is called a \term{belief change extension} of $B$. If there is no such set $EQ$, then $B$ is inconsistent and $\calLcalP$ is a \term{unique (inconsistent) belief change extension} of $B$.
\cite{conf/fedcsis/KorpusikLM12}
\end{defi}

\begin{defi}[Class of all belief change extensions]
Let $\accol{E_i}_{i\in I}$ be the \term{class of all belief change extensions} of $B=\tuple{K,R,C}$. Then
\begin{equation}
K\dotplus R\dotminus C=\displaystyle\bigcap_{i\in I}E_i
\end{equation}
\cite{conf/fedcsis/KorpusikLM12}
\end{defi}

\begin{defi}[Knowledge base, Observation, Defeasible statement, Domain axiom]
A \term{knowledge base} is a triple $KB=\tuple{OB,DS,DA}$, where $OB$, $DS$ and $DA$ are finite sets of formulas. These sets are referred to as \term{observation}s, \term{defeasible statement}s and \term{domain axiom}s, respectively.
\cite{conf/fedcsis/KorpusikLM12}
\end{defi}

\begin{defi}[Revision formula, Revision]
Let $KB=\tuple{OB,DS,DA}$ be a knowledge base and suppose that $\alpha$ is a \term{revision formula} representing a new observation. A \term{revision} of $KB$ by $\alpha$, written $KB\ast\alpha$, is a new knowledge base given by $\tuple{OB_1,DS,DA}$, where $OB_1=OB\oplus\accol{\alpha}\ominus\accol{\neg DA\hat{}}$. Here $\accol{\alpha}\ominus\accol{\neg DA\hat{}}$ is a finite representation of the modified knowledge base corresponding to belief change scenario $\tuple{OB,\accol{\alpha},\accol{\neg DA\hat{}}}$.
\cite{conf/fedcsis/KorpusikLM12}
\end{defi}

\begin{defi}[Belief set corresponding to]
Let $KB=\tuple{OB,DS,DA}$ be a knowledge base. A \term{belief set corresponding to} $KB$, written $B_{KB}$, is given
by $DS\dotplus\brak{OB\cup DA}$.
\cite{conf/fedcsis/KorpusikLM12}
\end{defi}

\begin{defi}[Prioritized belief revision of $KB$ by $\alpha$]
Let $KB=\tuple{OB,DS,DA}$ be a knowledge base and $\alpha$ be a revision formula. Let $OB_1=OB\oplus\accol{\alpha}\ominus\accol{\neg DA\hat{}}$. The \term{prioritized belief revision of $KB$ by $\alpha$} with respect to priorities $DS_1<DS_2<\ldots<DS_n$, written $KB\ast^{\flatbrak{DS_1<DS_2<\ldots<DS_n}}\alpha$, is the formula
\begin{equation}
DS_1\dotplus\brak{DS_2\oplus\ldots\brak{DS_n\oplus (OB_1\cup DA)\ldots}}.
\end{equation}
\cite{}
\end{defi}

\begin{defi}[Finite representation of the modified knowledge base corresponding to belief change scenario]
Let $KB=\tuple{OB,DS,DA}$ and suppose that $\alpha$ is a revision formula representing a domain axiom. The revised knowledge base is defined by $KB\ast\alpha=\tuple{OB1,DS,DA\cup\accol{α}}$, where $OB_1=OB\oplus\accol{\top}\ominus\accol{\neg\brak{DA\cup\accol{\alpha}}\hat{}}$. Here $OB\oplus\accol{\top}\ominus\accol{\neg\brak{DA\cup\accol{\alpha}}\hat{}}$ is a \term{finite representation of the modified knowledge base corresponding to belief change scenario}, $\tuple{OB,\accol{\top},\accol{\neg\brak{DA\cup\accol{\alpha}}\hat{}}}$.
\cite{conf/fedcsis/KorpusikLM12}
\end{defi}

\begin{defi}
Let $KB=\tuple{OB,DS,DA}$ and suppose that $\alpha$ is a revision formula representing a defeasible statement. The revised knowledge base is defined by $KB\ast\alpha=\tuple{OB,DS_1,DA}$, where $DS_1=DS\oplus\accol{\alpha}$.
\cite{conf/fedcsis/KorpusikLM12}
\end{defi}

\section{Machine Learning and Probabilistic Reasoning}

\begin{defi}[Learning algorithm for a meta-domain]%Problem: SxG??
Formally, an algorithm $A$ is a \term{learning algorithm for a meta-domain} $M$ in a hypothesis space $H$ with respect to a set of problem distributions $T$, if for any domain $D\in M$, any choice of a problem distribution $P$ in $T$, and any target problem solver $f\in H$,
\begin{enumerate}
\item $A$ takes as input the specification of a domain $D\in M$, an error parameter $E$, and a confidence parameter $\sigma$,
\item $A$ may call \alg{SolvedProblem}, which returns examples $\brak{x,\ffun{x}}$ for $D$, where $x$ is chosen with probability $\Pfun{x}$ from $S\times G$. The number of oracle calls of $A$ and its running time must be polynomial in the maximum problem size and the length of its input.
\item For all $D\in M$ and distributions $P\in T$, with probability at least $\brak{1-\delta}A$ outputs a program $f'$ that approximates $f$ in the sense that
\begin{equation}
\displaystyle\sum_{x\in\Delta}{\Pfun{x}}\leq\epsilon
\end{equation}
where $\Delta=\accol{x | f'\mbox{ fails on }x\mbox{ while }f\mbox{ succeeds}}$.
\item There is a polynomial $R$ such that, for a maximum problem size $n,\frac{1}{\epsilon},|frac{1}{\delta}$, maximum length $I$ and maximum step length $r$ of any solution output by \alg{SolvedProblem}, and an upper bound $t$ on the running times of programs in $D$ on inputs of size $n$, if $A$ outputs $f'$, the run time of $f'$ is bounded by $\fun{R}{n,l,r,t,\frac{1}{\epsilon},\frac{1}{\delta}}$
\end{enumerate}
\cite{conf/ijcai/Tadepalli91}
\end{defi}

\begin{defi}[Satisfying a spare solution space bias]
A problem solver $f$ for a domain $D$ and a problem distribution $P$ \term[Satisfying a sparse solution space bias]{satisfies a sparse solution space bias} if there is a set of operator sequences $m_f$ such that, on any problem $x\in D$ such that $\Pfun{x}>0$, $\ffun{x}\in m_f$ and $\abs{m_f}$ is bounded by a polynomial $Q$ in the problem size $n$.
\cite{conf/ijcai/Tadepalli91}
\end{defi}

\begin{defi}[Satisfying a macro table bias]
A problem solver $f$ \term[Satisfying a macro table bias]{satisfies a macro table bias} for a domain $D$ in $M$ if there is a feature ordering $O=\brak{l,\ldots,n}$ such that,
\begin{enumerate}
 \item $D$ is serially decomposable for $O$, and
 \item $f$ constructs all its solutions using a macro table $M$ as follows: for each feature $i$ from $1$ to $n$, macros $M_{j,i}$ are successively applied, where $j$ is the value of feature $i$ in the state before applying the macro.
\end{enumerate}
\cite{conf/ijcai/Tadepalli91}
\end{defi}

\begin{defi}[Beliefs in Conjoint Analysis]
We allow weighted beliefs with a weight parameter coming from \ocinterval{0}{1} where $1$ means full truth degree (complete certainty, the perfect belief), while a value $\alpha\in\oointerval{0}{1}$ describe a regular belief that can be doubted.
\begin{enumerate}
 \item \term{Regular belief}s such as:
 \begin{equation}
(\fun{A_1}{a_1}\wedge\ldots\wedge\fun{A_t}{a_t}):\alpha
 \end{equation}
 \item \term{Indifference belief}s such as:
 \begin{equation}
 \left(L\leftrightarrow R\right):1
 \end{equation}
 Indifference beliefs are always have full truth because we claim that if the respondent would distinguish degrees of truth then she is able to express preference.
 \item \term{Negative belief}s such as:
 \begin{equation}
 \left(\neg F\right):1
 \end{equation}
\end{enumerate}
where $A_i$ are attribute predicates and $L$, $R$, $F$ are regular atom conjunctions. Again, it is obvious in conjoint to don't ask user to express thoughts on negative information. As such there are no real negative beliefs such as $F:0$. Moreover, the reader may notice that we adopt the intuitionistic logic approach i.e., there is no assumption on any kind of law of excluded middle, as we don't necessarily assume $F:0\Leftrightarrow\left(\neg F\right):1$.
\cite{conf/fedcsis/GiurcaSB12}
\end{defi}

\subsection{Explanation Based Learning}

\begin{defi}[Explanation structure, Internal structure of an explanation structure, External structure of an explanation structure, Instantiation of an explanation structure, Uninstanstiation of an explanation structure, Binary resolvent of two explanation structures, Unresolution of an explanation structure, Remainder of an explanation structure, Generalization of an explanation structure]
\begin{enumerate}
 \item For a horn clause $C=A\leftarrow B_1\wedge B_2\wedge\ldots\wedge B_n$ let it's equivalent copy be $C'=A'\leftarrow B_1'\wedge B_2'\wedge\ldots\wedge B_n'$. Then the following expression is an \term{explanation structure}.
 \begin{equation}
  A':\brak{A\leftarrow \brak{B_1:B_1'}\wedge\brak{B_2:B_2'}\wedge\ldots\wedge\brak{B_n:B_n'}}
 \end{equation}
 $C$ and $C'$ are called its \term[Internal structure of an explanation structure]{internal} and \term[External structure of an explanation structure]{external structure} respectively.
 \item For an explanation structure $T$, let $C'$ be its external structure and $\sigma$ be a substitution. Then, the expression $T\sigma$, which has the same internal structure as $T$ and has
the external structure $C'\sigma$ is called an \term[Instantiation of an explanation structure]{instantiation of $T$}. Inversely, $T$ is called an \term[Uninstanstiation of an explanation structure]{uninstantiation of $T\sigma$}.
 \item For two explanation structures $S$ and $T$, let $A'\leftarrow B_1'\wedge B_2'\wedge\ldots\wedge B_n'$ and $B_i'\leftarrow C_1'\wedge C_2'\wedge\ldots\wedge C_m'$ be their external structures respectively. Then, the expression $U$ connected $S$ and $T$ at $B_i'$ is called a \term[Binary resolvent of two explanation structures]{(binary) resolvent of $S$ and $T$}. Inversely $S$ is called the \term[Unresolution of an explanation structure]{unresolution of $U$} and $T$ is called the \term[Remainder of an explanation structure]{remainder}.
 \item An explanation structure $S$ is \term[Generalization of an explanation structure]{more general} than $T$ denoted by $S\preceq T$ \iffTx{} there exists a sequence of uninstantiations and unresolutions from $T$ to $S$.
\end{enumerate}
\cite{conf/ijcai/YamamuraK91}
\end{defi}

\begin{defi}[Obliged generalization of an explanation structure, EBG, EBG Macro]
\begin{enumerate}
 \item An explanation $T$ is an \term{obliged generalization}, \iffTx{} the internal structure of T does not include any facts of a certain example and its external structure is a definite clause.
 \item Obliged generalizations of an explanation for some example are called \term{EBG}. For an EBG, its external structure is called an \term{EBG macro}.
\end{enumerate}
\cite{conf/ijcai/YamamuraK91}
\end{defi}

\begin{defi}[More general explanation, Generalization space]
\begin{enumerate}
 \item For an explanation $T\in\calT$, let $T_0$ be an uninstantiation of $T$. Then $\calT'=\brak{\calT\setminus\accol{T}}\cup\accol{T_0}$ is an uninstantiation of $T$. Similarly let $T_0$ be an unresolution and $T_0,T_1,\ldots,T_n$ are its remainders. Then $\calT'=\brak{\calT\setminus\accol{T}}\cup\accol{T_1,T_2,\ldots,T_n}$ is an unresolution of $\calT$.
 \item An explanation $S$ is \term[More general explanation]{more general} than $T$, denoted by $S<T$ \iffTx{} there exists a sequence of uninstantiations and unresolutions from $T$ to $S$.
 \item A class of all generalizations of the obliged general of an example set is called a \term{generalization space}.
\end{enumerate}
\cite{conf/ijcai/YamamuraK91}
\end{defi}

\begin{defi}[Common EGB, Least EBG]
For EBGs $T_1,\ldots,T_n$, $T$ is a \term{common EBG} \iffTx{} $T\preceq T_1,T\preceq T_2,\ldots,T\preceq T_n$. The least generalization in common EBGs is called the \term{least EBG}.
\cite{conf/ijcai/YamamuraK91}
\end{defi}

\begin{defi}[Least EGB generalization of an explanation, More general by least EBG, Least EBG space]
\begin{enumerate}
 \item For $\calS\subseteq\calT$, let $T_0$ be the least EBG of $\calS$ and $T_1,T_2,\ldots,T_n$ be its remainder. Then $\calT'=\brak{\calT\subseteq S}\cup\accol{T_0,T_1,\ldots,T_n}$ is called a \term[Least EGB generalization of an explanation]{least EBG generalization} of $\calT$.
 \item $\calS$ is \term{more general by least EBG than} $\calT$ \iffTx{} there exists a sequence of least EBG generalizations from $\calT$ to $\calS$.
 \item A class of all least EBG generalizations of the obliged generalization of an example set is called its \term{least EBG space}.
\end{enumerate}
\cite{conf/ijcai/YamamuraK91}
\end{defi}

\begin{defi}[Usage degree, Backtrack number \fun{\calB}{\calT,\calE}, Operationally criteria for generalizations]
\begin{enumerate}
 \item For a generalization $\calS=\brak{\calT\setminus\accol{T}}\cup\accol{T_0,T_1,\ldots,T_n}$, its \term{usage degree} is a function \stTx{}
 \begin{equation}
  \fun{\calU}{\calS,\calT}=\displaystyle\sum_{i=0}^{n}{\left\{\begin{array}{ll}
  \abs{T_i}&\mbox{if }T_i\in\calT\\0&\mbox{otherwise}\end{array}\right.}
 \end{equation}
 Where $\abs{T_i}$ denotes the number of clauses used in $T_i$. In general, when $\calS\preceq\calT$, the usage degree sums up all usage degrees through some generalization path.
 \item For an example set $\calE$ and its generalization $\calT$, the \term{backtrack number \fun{\calB}{\calT,\calE}} is the minimum of the sum of goal failures which occur during a macrotable of $\calT$ reconstructs all explanations of $\calE$.
 \item  For two generalizations $\calS$ and $\calT$ of an example set $\calE$, $\calS$ is \term[Operationally criteria for generalizations]{more operational than} $\calT$ \iffTx{} and $\fun{\calU}{\calS,\calE}\geq\fun{\calU}{\calT.\calE}$ and $\fun{\calB}{\calS,\calE}\geq\fun{\calB}{\calT.\calE}$.
\end{enumerate}
\cite{conf/ijcai/YamamuraK91}
\end{defi}

\subsection{Agent Systems, Decision Making and Q-Learning}

\begin{defi}[Sequence of decision elements, decision element, participation, positive set, neutral set, negative set, rate of return, degree of secure of rate $Z$, date of knowledge]
The structure decision $P$ of finite set of decision elements $E=\accol{e_1,e_2,\ldots,e_Y}$ is called as \term{sequence of decision elements}: $P=\tuple{\accol{EW^+},\accol{EW^{\pm}},\accol{EW^-},Z,SP,DT}$, Where:
\begin{enumerate}
 \item $EW^+=\tuple{e_0,pe_0},\tuple{e_q,pe_q},\ldots,\tuple{e_p,pe_p}$; couple $\tuple{e_x,pe_x}$, where $e_x\in E$ and $pe_x\in\ccinterval{0}{1}$. Denote a \term{decision element} and \term{participation} this element in set $EW^+$; decision element $e_x\in EW^+$ will be denoted as $e_x^+$; The set $EW^+$ is called \term{positive set}, in other words it is a set of decision elements, about which the agent knows that these elements are in the environment.
 \item $EW^{\pm}=\tuple{e_r,pe_r},\tuple{e_s,pe_s},\ldots,\tuple{e_t,pe_t}$; couple $\tuple{e_x,pe_x}$, where $e_x\in E$ and $pe_x\in\ccinterval{0}{1}$. Denote a decision element and participation this element in set $EW^+$; decision element $e_x\in EW^+$ will be denoted as $e_x^{\pm}$; The set $EW^{\pm}$ is called \term{neutral set}, in other words it is a set of decision elements, about which the agent does not know that these elements are in the environment.
 \item $EW^-=\tuple{e_r,pe_r},\tuple{e_s,pe_s},\ldots,\tuple{e_t,pe_t}$; couple $\tuple{e_x,pe_x}$, where $e_x\in E$ and $pe_x\in\ccinterval{0}{1}$. Denote a decision element and participation this element in set $EW^-$; decision element $e_x\in EW^+$ will be denoted as $e_x^-$; The set $EW^{\pm}$ is called a \term{negative set}, in other words it is a set of decision elements, about which the agent knows that these elements are not in the environment.
 \item $Z\in\ccinterval{0}{1}$ -- \term{rate of return}
 \item $SP\in\ccinterval{0}{1}$ -- \term{degree of secure of rate $Z$}
 \item $DT$ -- \term{date of knowledge}.
\end{enumerate}
\cite{conf/fedcsis/Sobieska-KarpinskaH12}
\end{defi}

\begin{defi}[Profile]
Set of decision elements $E=\accol{e_1,e_2,\ldots,e_Y}$ is given. A \term{profile} $A=\accol{A^{(1)},A^{(2)},\ldots,A^{(M)}}$ is called set of $M$ decisions of finite set of decision elements $E$, such that:
\begin{equation}
\group{
A^{(1)}=\tuple{\accol{EW^+}^{(1)},\accol{EW^{\pm}}^{(1)},\accol{EW^-}^{(1)},Z^{(1)},SP^{(1)},DT^{(1)}}\\\\
A^{(2)}=\tuple{\accol{EW^+}^{(2)},\accol{EW^{\pm}}^{(2)},\accol{EW^-}^{(2)},Z^{(2)},SP^{(2)},DT^{(2)}}\\\\
\vdots\\\\
A^{(M)}=\tuple{\accol{EW^+}^{(M)},\accol{EW^{\pm}}^{(M)},\accol{EW^-}^{(M)},Z^{(M)},SP^{(M)},DT^{(M)}}
}
\end{equation}
\cite{conf/fedcsis/Sobieska-KarpinskaH12}
\end{defi}

\section{Optimization Problems}

\begin{defi}
A \term{neighborhood structure} is a function $\calN:\calS\rightarrow 2^\calS$ that assigns to every $s\in\calS$ a set of neighbors $\fun{\calN}{s}\subseteq\calS$. $\fun{\calN}{s}$ is called the neighborhood of $s$. Often, neighborhood structures are implicitly defined by specifying the changes that must be applied to a solution s in order to generate all its neighbors. The application of such an operator that produces a neighbor $s'\in\fun{\calN}{s}$ of a solution s is commonly called a \term{move}.\cite{alba05}
\end{defi}

\begin{defi}
A \term{locally minimal solution} (or \term{local minimum}) with respect to a neighborhood structure $\calN$ is a solution $\hat{s}$ such that $\forall s\in\neigh{\hat{s}}:\ffun{\hat{s}}\leq\ffun{s}$. We call $\hat{s}$ a \term{strict locally minimum} if $\forall s\in\neigh{\hat{s}}:\ffun{\hats}<\ffun{s}$.\cite{alba05}
\end{defi}

\section{Toxicology}

\begin{defi}[Performance comparison]
The predictive accuracies of two theories are statistically equivalent then the theory with better explanatory power has \term[Performance comparison]{better performance}. Otherwise the one with higher accuracy has better performance.
\cite{conf/ijcai/SrinivasanKMS97}
\end{defi}

