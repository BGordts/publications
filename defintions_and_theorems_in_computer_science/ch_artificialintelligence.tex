\chapter{Artificial Intelligence}

\section{First-Order Logic and Logic Programming}

\begin{defi}[Signature]
$S$ is a \term{signature} if $S$ is a four-tuple \tuple{P,F,r,C} where:
\begin{enumerate}
\item $P$ is a set of \term{predicate symbols} \brak{P_1,P_2,\ldots,P_n},
\item $F$ is a set of \term{function symbols} \brak{F_1,F_2,\ldots,F_m},
\item $r$ is \term{arity} or \term{degree of functions} and relations. For each $P_i$ respectively $F_j$, \fun{r}{P_i} respectively \fun{r}{F_j} is a non-zero natural number denoting the arity of $P_i$ respectively $F_j$,
\item $C$ is a set of \term{constant symbols}.
\end{enumerate}
\cite{conf/fedcsis/Telnarova12}
\end{defi}

\begin{defi}[Alphabet]
An \term{alphabet} $\Sigma$ consists of the following symbols:
\begin{enumerate}
\item Signature $S=\tuple{P,F,r,C}$.
\item \term{Collection of variables} $V$.
\item \term{Operators}: $\neg$ (\term{negation}), $\wedge$ (\term{conjunction}), $\vee$ (\term{disjunction}), $\rightarrow$ (\term{implication}), $\leftrightarrow$ (\term{equivalence}).
\item \term{Quantifiers}: $\forall$ (\term{forall}), $\exists$ (\term{exists}).
\item \term{Parentheses} and \term{punctuation symbols}: $($, $)$ and $,$.
\end{enumerate}
\cite{conf/fedcsis/Telnarova12}
\end{defi}

\begin{defi}[Term]
A \term{term} is defined inductively as follows:
\begin{enumerate}
\item Variable is term.
\item Constant is term.
\item If $f$ is a function symbol ($f\in F$) with arity $m$ and $t_1,t_2,\ldots,t_m$ are terms of $\Sigma$, then \ffun{t_1,t_2,\ldots,t_m} is term of $\Sigma$.
\end{enumerate}
\cite{conf/fedcsis/Telnarova12}
\end{defi}

\begin{defi}[Atom]
If $p$ is predicate symbol with arity $m$ and $t_1,t_2,\ldots,t_m$ are terms of $\Sigma$, then $\fun{p}{t_1,t_2,\ldots,t_m}$ is an \termor{atomic formula}{atom}. An atomic formula is a \term{formula} and all occurrences of variables in an atomic formula are free.
\cite{conf/fedcsis/Telnarova12}
\end{defi}

\begin{defi}[Formula]
A \term{formula} is defined as follows:
\begin{enumerate}
 \item An atom is a formula.
 \item If $H$ and $G$ are formulas, then:
 \begin{enumerate}
  \item $\neg H$ is a formula, the occurrence of variables in $\neg H$ is free respectively bound if it is free respectively bound in $H$,
  \item $H\wedge G$ is a formula, the occurrence of variables in $H\wedge G$ is free respectively bound if it is free respectively bound in $H$ or $G$,
  \item $H\vee G$ is a formula, the occurrence of variables in $H\vee G$ is free respectively bound if it is free respectively bound in $H$ or $G$,
  \item $H\rightarrow G$ is a formula, the occurrence of variables in $H\rightarrow G$ is free respectively bound if it is free respectively bound in $H$ or $G$,
  \item $H\leftrightarrow G$ is a formula, the occurrence of variables in $H\leftrightarrow G$ is free respectively bound if it is free respectively bound in $H$ or $G$.
 \end{enumerate}
 \item If $H$ is a formula and $x$ is a variable, then $\forall x:H$ and $\exists x:H$ are formulas. All occurrences of $x$ are bound.
\end{enumerate}
\cite{conf/fedcsis/Telnarova12}
\end{defi}

\begin{defi}[Literal]
A \term{literal} $L$ is an atom or the negation of an atom.
\cite{conf/fedcsis/Telnarova12}
\end{defi}

\begin{defi}[Clause]
A \term{clause} is a formula such as $\forall\vec{x}:L_1\vee L_2\vee\ldots\vee L_m$ where each $L_i$ is a literal and $\vec{x}=\brak{x_1,x_2,\ldots,x_n}$ are all the variables occurring in $L_1\vee L_2\vee\ldots\vee L_m$.
\cite{conf/fedcsis/Telnarova12}
\end{defi}

\begin{defi}[Horn-clauses]
\term{Horn-clause}s have the form: $\forall x_1,x_2,\ldots,x_n:L_1\wedge L_2\wedge\ldots\wedge L_m\rightarrow L$ where $L,L_1\wedge L_2\wedge\ldots\wedge L_m$ are a literals and $x_1,x_2,\ldots,x_n$ are all variables having free occurrences in $L,L_1\wedge L_2\wedge\ldots\wedge L_m$.
\cite{conf/fedcsis/Telnarova12}
\end{defi}

\begin{defi}[Negation of a dilemma, Conjunction of dilemmas, Disjunction of dilemmas]
For a given set $D$ of dilemmas it is defined:
\begin{enumerate}
 \item A \term{negation of a dilemma} $d=\dilemma{u}$ as a dilemma $\neg d=\negdilemma{u}$;
 \item A \term{disjunction of the dilemmas} $d'=\dilemma{u'}$, $d''=\dilemma{u''}$ as a dilemma $d'\vee d''=\flatbrak{u'\vee u''|\neg\brak{u'\vee u''}}$;%
 \item A \term{conjunction of the dilemmas} $d'=\dilemma{u'}$, $d''=\dilemma{u''}$ as a dilemma $d'\wedge d''=\flatbrak{u'\wedge u''|\neg\brak{u'\wedge u''}}$;%
\end{enumerate}
\cite{conf/fedcsis/Kulikowski12}
\end{defi}

\begin{defi}[Equal certainty relation, Ambivalent dilemma, Equivalent dilemmas, Anti-equivalent dilemmas]
Let $D$ be a set of dilemmas. Then in $D$ a binary relation $\approx^c$ satisfying the conditions of:
\begin{enumerate}
\item Reciprocity: for each $d\in D$ it holds $d\approx^c d$;
\item Symmetry: for any $d',d''\in D$ if $d'\approx^c d''$ then also $d''\approx^c d'$ holds;
\item Reflexivity: for any $d',d''\in D$ if $d'\approx^c d''$ then also $\neg d'\approx^c\neg d''$ holds;
\item Transitivity: for any $d',d'',d'''\in D$ if $d'\approx^c d''$ and $d''\approx^c d'''$ then also $d'\approx^c d'''$ holds;
\item Fixation: for any $d',d''\in D$ if $d'\approx^c\neg d'$ and $d''\approx^c\neg d''$ then also $d'\approx^c d''$ holds,
\end{enumerate}
will be called an \term{equal certainty relation}. Any dilemma satisfying the condition $d\approx^c\neg d$ will be called an \term{ambivalent dilemma}; any dilemmas such that $d'\approx^c d''$ holds will be called \term{equivalent dilemmas}; any dilemmas such that $d'\approx^c\neg d''$ holds will be called \term{anti-equivalent dilemmas}.
\cite{conf/fedcsis/Kulikowski12}
\end{defi}

\begin{defi}[Certainty ranking]
Let $D$ be a set of dilemmas with established equal certainty relation. Then a binary relation $\preceq^c$ described in $D$ and satisfying the conditions of:
\begin{enumerate}
 \item reciprocity: for each $d\in D$ it holds $d\preceq^c d$;
 \item symmetry: for any $d',d''\in D$ $d'\preceq^c d''$ and $d''\preceq^c d'$ hold if and only if $d'\approx^c d''$ holds;
 \item anti-reflexivity: for any $d',d''\in D$ if $d'\preceq^c d''$ then $\neg d''\preceq^c\neg d'$ holds;
 \item transitivity: for any $d',d'',d'''\in D$ if $d'\preceq^c d''$ and $d''\preceq^c d'''$ then also $d'\preceq^c d'''$ holds,
\end{enumerate}
will be called a \term{certainty ranking}.
\cite{conf/fedcsis/Kulikowski12}
\end{defi}

\begin{theo}
Let $D$ be a set of dilemmas with established equal certainty and certainty ranking relations. Then for any $d',d''\in D$:
\begin{enumerate}
\item if $d'\preceq^c d''$ and not $d''\preceq^c d'$ then $d'\vee d''\preceq^c d''$;
\item if $d'\preceq^c d''$ and not $d''\preceq^c d'$ then $d'\wedge d''\preceq^c d'$;
\item if $d'\approx^c d''$ then $d'\vee d''\approx^c d'\wedge d''\approx^c d'\approx^c d''$;
\item if $d'\approx^c d''$ then $d'\wedge d''\approx^c d',d''$;
\item if $d'\approx^c d''$ then $d',d''\approx^c d'\vee d''$
\end{enumerate}
\cite{conf/fedcsis/Kulikowski12}
\end{theo}

\subsection{Logic Programming}

\begin{defi}[Domain declaration for predicate symbol $p$]
A \term{domain declaration for predicate symbol $p$} of arity n is an expression of the following form.
\begin{equation}
\mbox{domain }\fun{p}{a_1,\ldots ,a_n}
\end{equation}
where $a_i$ is either $h$ or $d$. When $a$ is equal to $h$, this means that the $i$-th argument of $p$ ranges over the Herbrand universe. Otherwise, it means that the $i$-th argument is a list of variables which ranges over $d_1$ In the following, the domains $d_i$ are finite and explicit sets of values (i.e constants).
\cite{conf/ijcai/Hentenryck87}
\end{defi}

\begin{defi}[Domain set of a logic program]
Let $dl,\ldots ,dn$ the domains appearing in the domain declarations of a logic program $PR$ and different from the Herbrand universe. We note \fun{D}{PR} the set $\accol{d|d\neq\emptyset\wedge d\in 2^{d_i}\accol{1\leq i\leq n}}$ We call it the \term{domain set of the logic program}. The domain set of a logic program contains all domains we possibly need during the computations.
\cite{conf/ijcai/Hentenryck87}
\end{defi}

\begin{defi}[Range of a term included in a domain]
We say that the \term{range of $t$ is included in a domain $d_t$} denoted $\abs{t}\in d_i$ if $t$ is a constant $\in d_t$ or a $d$-variable $x^{d_t}$ such that $d_t\subseteq d$
\cite{conf/ijcai/Hentenryck87}
\end{defi}

\begin{defi}[$d$-substitution]
A $d$-substitution $\theta$ is a finite set of the form $\accol{v_1/t_1,\ldots,v_n/t_n}$ where
\begin{enumerate}
 \item each $v_i$ is either a variable or $d$-variable
 \item $t_i$ is a term distinct from $v_i$,
 \item $v_1,\ldots,v_n$ are all distinct,
 \item if $v_i$ is a $d$-variable $v^{d_i}$, $\abs{t_i}\in d_i$
\end{enumerate}
\cite{conf/ijcai/Hentenryck87}
\end{defi}

\begin{defi}[$d$-substitutions agree on a set of variables and $d$-variables]
We say that two \term[$d$-substitutions agree on a set of variables and $d$-variables]{$d$-substitutions $\theta$ and $\lambda$ agree on a set $V$ of variables and $d$-variables}, denoted $\theta=\lambda\abs{V}$ \iffTx{} $x\theta=x\lambda$ for each $x\in V$ where $=$ denotes syntactic equality.
\cite{conf/ijcai/Hentenryck87}
\end{defi}

\begin{defi}[$d$-instance]
$\theta$ is a \term{$d$-instance} of $\lambda$ in $V$, denoted $\lambda\leq\theta$ \iffTx{} $x\theta=\delta\circ\lambda$ for some $d$-substitution $\delta$.
\cite{conf/ijcai/Hentenryck87}
\end{defi}

\begin{defi}[$d$-unifier, unifies, more general $d$-unifier, $d$-mgu]
A $d$-substitution $\sigma$ is a \term{$d$-unifier} of some non-empty and finite subset $S=\accol{t_1,\ldots,t_n}$ where $t_i$ and a literal of a term \iffTx{} $t_1\sigma=\ldots=t_n\sigma$, we also say that $\sigma$ \term{unifies} $S$. \term{\fun{UNI}{S}} is the set of all $d$-unifiers of $S$. $\sigma$ is called the \termor{more general $d$-unifier}{$d$-mgu} of $S$ \iffTx for each $\theta\in\fun{UNI}{S}$, $\theta\leq\sigma\abs{\fun{vars}{S}}$ implies $\sigma\leq\theta\abs{\fun{vars}{S}}$ where \fun{vars}{S} is the set of all variable or $d$-variable symbols in $S$.
\cite{conf/ijcai/Hentenryck87}
\end{defi}

\begin{defi}
Let $p$ be a $n$-ary predicate symbol, $p$ is a constraint \iffTx for any ground terms either
has a successful refutation or
) has
only finitely failed derivations.
S. F o r w a r d checking in logic p r o g r a m ming.
Forward checking is often considered as one of the most
efficient procedures for solving CSPs. Intuitively, a constraint
can be used in forward checking as soon as at most one
variable occurs In it. In this case, the set of possible values
for the variable Is reduced to the set of values which satisfy
the constraint. Thus, a program based on forward checking
gives a value to a variable, uses all constraints which contains
\cite{conf/ijcai/Hentenryck87}
\end{defi}
%
% 1\begin{defi}
% Definition 13s A computation rule is efficient wrt the
% forward declarations, if it selects only a predicate
% submitted to forward declaration when it is ground or
% forward checkable and if, whenever the resolvent contains
% literals submitted to a forward declaration which are
% either forward-checkable or ground, it selects one of them.
% \cite{conf/ijcai/Hentenryck87}
% \end{defi}
%
% 1\begin{defi}
% Definition 20: A computation rule is efficient wrt the
% lookahead declarations, if a l i t e r a l i n the
% resolvent submitted to a lookahead declaration is only
% selected if either it is lookahead checkable or all its
% arguments are ground.
% A efficient computation rule wrt lookahead declarations gives
% UB few informations about when to select a lookahead
% constraint. It is clear that selecting it too early can induce
% some unproductive work (no new informations are inferred)
% and that a late selection reduces the pruning of the search
% space. It is not difficult to define efficient computation rules
% which select only lookahead constraints which are likely to
% produce new informations. The definition of the LAIR should
% not be seen as suggesting a particular implementation. Actual
% implementations should be based, for instance, on
% generalisation* of AC-3 |ll) or AC-4 |12).
% Example: Consider a vision problem in a three-faced vertex
% world.
% is the set of possible labels for the
% vertex. Constraints in the problem are given by the "so-called"
% fork, L, T and arrow junctions. For instance, the fork junction
% can be defined as follows.
% \cite{conf/ijcai/Hentenryck87}
% \end{defi}

\section{Knowledge Representation}

\begin{defi}[Belief change scenario]
A \term{belief change scenario} is a triple $B=\tuple{K,R,C}$, where $K$, $R$ and $C$ are sets of formula over a fixed propositional language $\calLcalP$. Informally, $K$ is a \term{knowledge base} which is to be modified in such a way that the resulting knowledge base includes all elements from $R$ and does not include any element from $C$. The modified knowledge base corresponding to $B$ will be denoted by $K\dotplus R\dotminus C$.
\cite{conf/fedcsis/KorpusikLM12}
\end{defi}

\begin{defi}[Belief change extension, Unique (inconsistent) belief change extension]
Let $B=\tuple{K,R,C}$ be a belief change scenario over $\calLcalP$. Define a new set $\calP'$ of atoms, isomorphic with $\calP$, given by $\calP'=\accol{p':p\in\calP}$. Let $K'$ be a knowledge base obtained from $K$ by replacing any $p\in\calP$ by $p'\in\calP'$. Let $EQ$ be a maximal (with respect to set inclusion) set of equivalences $\accol{p\Leftrightarrow p'|p\in\calP}$ such that $\fun{Th}{K'\cup EQ\cup R}\cap\brak{C\cup\bot}=\emptyset$. The set $\fun{Th}{K'\cup EQ\cup R}\cap\calLcalP$ is called a \term{belief change extension} of $B$. If there is no such set $EQ$, then $B$ is inconsistent and $\calLcalP$ is a \term{unique (inconsistent) belief change extension} of $B$.
\cite{conf/fedcsis/KorpusikLM12}
\end{defi}

\begin{defi}[Class of all belief change extensions]
Let $\accol{E_i}_{i\in I}$ be the \term{class of all belief change extensions} of $B=\tuple{K,R,C}$. Then
\begin{equation}
K\dotplus R\dotminus C=\displaystyle\bigcap_{i\in I}E_i
\end{equation}
\cite{conf/fedcsis/KorpusikLM12}
\end{defi}

\begin{defi}[Knowledge base, Observation, Defeasible statement, Domain axiom]
A \term{knowledge base} is a triple $KB=\tuple{OB,DS,DA}$, where $OB$, $DS$ and $DA$ are finite sets of formulas. These sets are referred to as \term{observation}s, \term{defeasible statement}s and \term{domain axiom}s, respectively.
\cite{conf/fedcsis/KorpusikLM12}
\end{defi}

\begin{defi}[Revision formula, Revision]
Let $KB=\tuple{OB,DS,DA}$ be a knowledge base and suppose that $\alpha$ is a \term{revision formula} representing a new observation. A \term{revision} of $KB$ by $\alpha$, written $KB\ast\alpha$, is a new knowledge base given by $\tuple{OB_1,DS,DA}$, where $OB_1=OB\oplus\accol{\alpha}\ominus\accol{\neg DA\hat{}}$. Here $\accol{\alpha}\ominus\accol{\neg DA\hat{}}$ is a finite representation of the modified knowledge base corresponding to belief change scenario $\tuple{OB,\accol{\alpha},\accol{\neg DA\hat{}}}$.
\cite{conf/fedcsis/KorpusikLM12}
\end{defi}

\begin{defi}[Belief set corresponding to]
Let $KB=\tuple{OB,DS,DA}$ be a knowledge base. A \term{belief set corresponding to} $KB$, written $B_{KB}$, is given
by $DS\dotplus\brak{OB\cup DA}$.
\cite{conf/fedcsis/KorpusikLM12}
\end{defi}

\begin{defi}[Prioritized belief revision of $KB$ by $\alpha$]
Let $KB=\tuple{OB,DS,DA}$ be a knowledge base and $\alpha$ be a revision formula. Let $OB_1=OB\oplus\accol{\alpha}\ominus\accol{\neg DA\hat{}}$. The \term{prioritized belief revision of $KB$ by $\alpha$} with respect to priorities $DS_1<DS_2<\ldots<DS_n$, written $KB\ast^{\flatbrak{DS_1<DS_2<\ldots<DS_n}}\alpha$, is the formula
\begin{equation}
DS_1\dotplus\brak{DS_2\oplus\ldots\brak{DS_n\oplus (OB_1\cup DA)\ldots}}.
\end{equation}
\cite{}
\end{defi}

\begin{defi}[Finite representation of the modified knowledge base corresponding to belief change scenario]
Let $KB=\tuple{OB,DS,DA}$ and suppose that $\alpha$ is a revision formula representing a domain axiom. The revised knowledge base is defined by $KB\ast\alpha=\tuple{OB1,DS,DA\cup\accol{α}}$, where $OB_1=OB\oplus\accol{\top}\ominus\accol{\neg\brak{DA\cup\accol{\alpha}}\hat{}}$. Here $OB\oplus\accol{\top}\ominus\accol{\neg\brak{DA\cup\accol{\alpha}}\hat{}}$ is a \term{finite representation of the modified knowledge base corresponding to belief change scenario}, $\tuple{OB,\accol{\top},\accol{\neg\brak{DA\cup\accol{\alpha}}\hat{}}}$.
\cite{conf/fedcsis/KorpusikLM12}
\end{defi}

\begin{defi}
Let $KB=\tuple{OB,DS,DA}$ and suppose that $\alpha$ is a revision formula representing a defeasible statement. The revised knowledge base is defined by $KB\ast\alpha=\tuple{OB,DS_1,DA}$, where $DS_1=DS\oplus\accol{\alpha}$.
\cite{conf/fedcsis/KorpusikLM12}
\end{defi}

\section{Machine Learning and Probabilistic Reasoning}

\begin{defi}[Beliefs in Conjoint Analysis]
We allow weighted beliefs with a weight parameter coming from \ocinterval{0}{1} where $1$ means full truth degree (complete certainty, the perfect belief), while a value $\alpha\in\oointerval{0}{1}$ describe a regular belief that can be doubted.
\begin{enumerate}
 \item \term{Regular belief}s such as:
 \begin{equation}
(\fun{A_1}{a_1}\wedge\ldots\wedge\fun{A_t}{a_t}):\alpha
 \end{equation}
 \item \term{Indifference belief}s such as:
 \begin{equation}
 \left(L\leftrightarrow R\right):1
 \end{equation}
 Indifference beliefs are always have full truth because we claim that if the respondent would distinguish degrees of truth then she is able to express preference.
 \item \term{Negative belief}s such as:
 \begin{equation}
 \left(\neg F\right):1
 \end{equation}
\end{enumerate}
where $A_i$ are attribute predicates and $L$, $R$, $F$ are regular atom conjunctions. Again, it is obvious in conjoint to don't ask user to express thoughts on negative information. As such there are no real negative beliefs such as $F:0$. Moreover, the reader may notice that we adopt the intuitionistic logic approach i.e., there is no assumption on any kind of law of excluded middle, as we don't necessarily assume $F:0\Leftrightarrow\left(\neg F\right):1$.
\cite{conf/fedcsis/GiurcaSB12}
\end{defi}

\subsection{Agent Systems, Decision Making and Q-Learning}

\begin{defi}[Sequence of decision elements, decision element, participation, positive set, neutral set, negative set, rate of return, degree of secure of rate $Z$, date of knowledge]
The structure decision $P$ of finite set of decision elements $E=\accol{e_1,e_2,\ldots,e_Y}$ is called as \term{sequence of decision elements}: $P=\tuple{\accol{EW^+},\accol{EW^{\pm}},\accol{EW^-},Z,SP,DT}$, Where:
\begin{enumerate}
 \item $EW^+=\tuple{e_0,pe_0},\tuple{e_q,pe_q},\ldots,\tuple{e_p,pe_p}$; couple $\tuple{e_x,pe_x}$, where $e_x\in E$ and $pe_x\in\ccinterval{0}{1}$. Denote a \term{decision element} and \term{participation} this element in set $EW^+$; decision element $e_x\in EW^+$ will be denoted as $e_x^+$; The set $EW^+$ is called \term{positive set}, in other words it is a set of decision elements, about which the agent knows that these elements are in the environment.
 \item $EW^{\pm}=\tuple{e_r,pe_r},\tuple{e_s,pe_s},\ldots,\tuple{e_t,pe_t}$; couple $\tuple{e_x,pe_x}$, where $e_x\in E$ and $pe_x\in\ccinterval{0}{1}$. Denote a decision element and participation this element in set $EW^+$; decision element $e_x\in EW^+$ will be denoted as $e_x^{\pm}$; The set $EW^{\pm}$ is called \term{neutral set}, in other words it is a set of decision elements, about which the agent does not know that these elements are in the environment.
 \item $EW^-=\tuple{e_r,pe_r},\tuple{e_s,pe_s},\ldots,\tuple{e_t,pe_t}$; couple $\tuple{e_x,pe_x}$, where $e_x\in E$ and $pe_x\in\ccinterval{0}{1}$. Denote a decision element and participation this element in set $EW^-$; decision element $e_x\in EW^+$ will be denoted as $e_x^-$; The set $EW^{\pm}$ is called a \term{negative set}, in other words it is a set of decision elements, about which the agent knows that these elements are not in the environment.
 \item $Z\in\ccinterval{0}{1}$ -- \term{rate of return}
 \item $SP\in\ccinterval{0}{1}$ -- \term{degree of secure of rate $Z$}
 \item $DT$ -- \term{date of knowledge}.
\end{enumerate}
\cite{conf/fedcsis/Sobieska-KarpinskaH12}
\end{defi}

\begin{defi}[Profile]
Set of decision elements $E=\accol{e_1,e_2,\ldots,e_Y}$ is given. A \term{profile} $A=\accol{A^{(1)},A^{(2)},\ldots,A^{(M)}}$ is called set of $M$ decisions of finite set of decision elements $E$, such that:
\begin{equation}
\group{
A^{(1)}=\tuple{\accol{EW^+}^{(1)},\accol{EW^{\pm}}^{(1)},\accol{EW^-}^{(1)},Z^{(1)},SP^{(1)},DT^{(1)}}\\\\
A^{(2)}=\tuple{\accol{EW^+}^{(2)},\accol{EW^{\pm}}^{(2)},\accol{EW^-}^{(2)},Z^{(2)},SP^{(2)},DT^{(2)}}\\\\
\vdots\\\\
A^{(M)}=\tuple{\accol{EW^+}^{(M)},\accol{EW^{\pm}}^{(M)},\accol{EW^-}^{(M)},Z^{(M)},SP^{(M)},DT^{(M)}}
}
\end{equation}
\cite{conf/fedcsis/Sobieska-KarpinskaH12}
\end{defi}

\section{Optimization Problems}

\begin{defi}
A \term{neighborhood structure} is a function $\calN:\calS\rightarrow 2^\calS$ that assigns to every $s\in\calS$ a set of neighbors $\fun{\calN}{s}\subseteq\calS$. $\fun{\calN}{s}$ is called the neighborhood of $s$. Often, neighborhood structures are implicitly defined by specifying the changes that must be applied to a solution s in order to generate all its neighbors. The application of such an operator that produces a neighbor $s'\in\fun{\calN}{s}$ of a solution s is commonly called a \term{move}.\cite{alba05}
\end{defi}

\begin{defi}
A \term{locally minimal solution} (or \term{local minimum}) with respect to a neighborhood structure $\calN$ is a solution $\hat{s}$ such that $\forall s\in\neigh{\hat{s}}:\ffun{\hat{s}}\leq\ffun{s}$. We call $\hat{s}$ a \term{strict locally minimum} if $\forall s\in\neigh{\hat{s}}:\ffun{\hats}<\ffun{s}$.\cite{alba05}
\end{defi}