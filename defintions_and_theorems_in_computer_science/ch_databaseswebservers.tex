\chapter{Databases and Web applications}
\begin{defi}[Web application]
A \term{web application} $\calA$ is a tuple \tuple{W,S,I,A,W_0,W_{\epsilon},M,T}, where:
\begin{enumerate}
\item $S$, $I$, $A$ are disjoint finite sets of state, input, and action values. Constant values can be shared among these sets.
\item $W$ is finite set of the representations of \term{web pages},
\item $W_0\in W$ is the representation of \term{home page}, $W_{\epsilon}\notin W$ is the the representation of \term{error page},
\item $M$ is a set of \term{business metadata} values,
\item $T$ is a set of \term{technical (non-model) metadata} values
\end{enumerate}
In the testing process tester has to pass through all the test case scenario steps defined in selected test case scenario. A specification of an interaction of the tester with system under test is defined in each of test case scenario step. Instructions describing the test case scenario step are defined clearly without ambiguities. For brevity we will limit the user interaction with the SUT to only one action that is captured in the test scenario step.
\cite{conf/fedcsis/FrajtakBJ12}
\end{defi}

\begin{defi}[Test case scenario]
A \term{test case scenario} is a finite sequence $\accol{\tuple{V_i,S_i,I_i,A_i,R_i,TD_i}}_{i\geq 0}$ where $V_i$ is a web page page representation used in step $i$, $S_i\subseteq S$ is set of state values used in step $i$, $I_i\subseteq I$ is set of values of input elements used in step $i$, $A_i\in A$ is an action taken in step $i$, $R_i$ is set of restricting constraints rules, $TD_i\subseteq S_i\cup I_i\cup M\cup T$ is finite set of state, input, business and technical metadata values.
\cite{conf/fedcsis/FrajtakBJ12}
\end{defi}

\begin{defi}[Inbound client-side constraint rule]
An \term{inbound client-side constraint rule} is a function \funsig{\phi_{\mbox{in}}}{I}{\binary}.
\cite{conf/fedcsis/FrajtakBJ12}
\end{defi}

\begin{defi}[Inbound server-side constraint rule]
An \term{inbound server-side constraint rule} is a function \funsig{\Phi_{\mbox{in}}}{S\cup I\cup M\cup T}{\binary}. The outbound server-side constraint rules are the rules validating the state on the server side, any metadata and database right after the server action is executed and before the data is sent back to client.
\cite{conf/fedcsis/FrajtakBJ12}
\end{defi}

\begin{defi}[Outbound server-side constraint rule]
An \term{outbound server-side constraint rule} is a function \funsig{\Phi_{\mbox{out}}}{S\cup I\cup M\cup T}{\binary}.
\cite{conf/fedcsis/FrajtakBJ12}
\end{defi}

\begin{defi}[Passed test case scenario]
Test case scenario step $i$ with configuration \tuple{V_i,S_i,I_i,A_i,R_i,TD_i} where $V_i$ is a web page page representation used in step $i$, $S_i\subseteq S$ is set of state values used in step $i$, $I_i\subseteq I$ is set of values of input elements used in step $i$, $A_i\in A$ is an action taken in step $i$, $R_i$ is set of restricting constraints rules, $TD_i\subseteq S_i\cup I_i\cup M\cup T$ is finite set of state, input, business and technical metadata values used in step $i$, is called \term{passed} when for all rule constraints $f \in R_i$:
\begin{enumerate}
 \item if $f$ is client-side inbound rule, then $\ffun{j}=1$ where $j\in I_i$
 \item if $f$ is client-side outbound rule or server-side inbound rule, then $\ffun{k}=1$ where $k \in TD_i$
 \item $V_{i+1}$ is not error web page $W_{\epsilon}$
 \item execution of action $A_i$ results in success
\end{enumerate}
\cite{conf/fedcsis/FrajtakBJ12}
\end{defi}

\begin{defi}[Test scenario]
Let $\calW =\tuple{W,S,I,A,W_0,W_{\epsilon},M,T}$ be a web application. A \term{test scenario} of $W$ is an finite sequence of test case scenario step configurations $\accol{\tuple{V_i,S_i,I_i,A_i,R_i,TD_i}}_{i\geq0}$ where $V_i$ is a web page page representation used in step $i$, $S_i\subseteq S$ is set of state values used in step $i$, $I_i\subseteq I$ is set of values of input elements used in step $i$, $A_i\in A$ is an action taken in step $i$, $R_i$ is set of restricting constraints rules, $TD_i\subseteq Si\cup I_i\cup M\cup T$ is finite set of state, input, business and technical metadata values used in step $i$, we say that given test case scenario is \term{passed} when each test scenario step $K_i$ is passed for each $i\geq 0$. Our target is to find errors left in the system that were not discovered by the unit testing. To have higher confidence about the correctness of the implementation of the system the SUT must be covered by our test case scenarios--every single action the user can execute must be used in at least one test case scenario step and this test case scenario must be successfully completed by the tester without problems.
\cite{conf/fedcsis/FrajtakBJ12}
\end{defi}

\begin{defi}[Covered test scenario step]
A test scenario step $K_i$ with test case scenario step configuration \tuple{V_i,S_i,I_i,A_i,R_i,TD_i} where $V_i$ is a web page page representation used in step $i$, $S_i\subseteq S$ is set of state values used in step $i$, $I_i\subseteq I$ is set of values of input elements used in step $i$, $A_i\in A$ is an action taken in step $i$, $R_i$ is set of restricting constraints rules, $TDi\subseteq Si\cup I_i\cup M\cup T$ is finite set of state, input, business and technical metadata values used in step $i$, then action $A_i$ , respectively input element $I_i$ , respectively action $A_i$ , is \term{covered} by $K_i$ when $K_i$ is passed.
\cite{conf/fedcsis/FrajtakBJ12}
\end{defi}

\begin{defi}[Test case action verification ratio]
Given the finite set $A$ of action symbols of the web application $\calA$, then
\begin{equation}
a_{\mbox{verify}}=\displaystyle\frac{\abs{A_{\mbox{verified}}}}{\abs{A}}
\end{equation}
is called \term{test case action verification ratio} of the SUT
\begin{equation}
A_{\mbox{verified}}=\displaystyle\bigcup_{T\in E}\displaystyle\bigcup_{\tau\in T}\accol{a|a\in A_i\wedge \tau\mbox{ is passed}}
\end{equation}
is the finite set of actions covered by all passed test scenarios. $E$ is the set of \term{executed test case scenarios}, $T$ denotes \term{test case scenario test}, $τ$ is $i$-th single step in test case scenario
with configuration \tuple{V_i,S_i,I_i,A_i,R_i,TD_i}. We then use $R_a = 100\%\cdot a_{\mbox{verify}}$ to express the SUT \term{test case action verification percentage}. The test case state, respectively input, verification ratio of the SUT are defined in a similar way.
\cite{conf/fedcsis/FrajtakBJ12}
\end{defi}