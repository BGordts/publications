\chapter{Automata and Computability}

\begin{defi}
A \term{nondeterministic finite automaton} consists of a set $S$ of states. One of these states, $s0\in S$, is called the \term{starting state} of the automaton and a subset $F\subseteq S$ of the states are \term{accepting states}. Additionally, we have a set $T$ of \term{transitions}. Each transition $t$ connects a pair of states $s_1$ and $s_2$ and is labeled with a symbol, which is either a character $c$ from the alphabet $\Sigma$, or the symbol $\varepsilon$, which indicates an \term{epsilon-transition}. A transition from state $s$ to state $t$ on the symbol $c$ is written as $s^ct$.\cite{mogensen2009basics}
\end{defi}

\begin{defi}
Given a set $M$ of nondeterministic finite automaton states, we define the \term{$\varepsilon$-closure(M)} to be the least (in terms of the subset relation) solution to the set equation:
\begin{equation}
\fun{\varepsilon\mbox{-closure}}{M} = M\cup \left\{t|s\in \fun{\varepsilon\mbox{-closure}}{M}\mbox{ and }s^\varepsilon t\in T \right\}
\end{equation}
Where $T$ is the set of transitions in the non deterministic finite automaton.\cite{mogensen2009basics}
\end{defi}

\begin{defi}[Big-Oh notation]
If $f,g$ are two functions from $\NNN$ to $\NNN$, then we 
\begin{enumerate}
 \item say that $f=\term{\bigoh{g}}$ if there exists a constant $c$ such that $\ffun{n}\leq c\gfun{n}$ for every sufficiently large $n$,
 \item say that $f=\term{\bigomega{g}}$ if $g = \bigoh{f}$,
 \item say that $f=\term{\bigtheta{g}}$ is $f = \bigoh{g}$ and $g = \bigoh{f}$,
 \item say that $f=\term{\smalloh{g}}$ if for every $\varepsilon>0$, $\ffun{n}\leq\varepsilon\gfun{n}$ for every sufficiently large $n$, and
 \item say that $f=\term{\smallomega{g}}$ if $g=\smalloh{f}$.
\end{enumerate}
\cite{arora2009computational}
\end{defi}

\begin{defi}[Computing a function and running time]
Let $\btobfun{f}$ and let $\funsig{T}{\NNN}{\NNN}$ be some functions, and let $M$ be a Turing machine. We say that $M$ \term{computes $f$} if for every $x\in\binarystrings$, whenever $M$ is initialized to the start configuration on input $x$, then it halts with $\ffun{x}$ written on its output tape. We say $M$ \term{computes $f$ in $\fun{T}{n}$-time} if its computation on every input $x$ requires at most $\fun{T}{\left|x\right|}$ steps.
\cite{arora2009computational}
\end{defi}

\begin{defi}[The class \compclass{DTIME}]
Let $\funsig{T}{\NNN}{\NNN}$ be some function. A language $L$ is in $\fun{\term{\compclass{DTIME}}}{\fun{T}{n}}$ if and only f there is a Turing machine that runs in time $c\fun{T}{n}$ for some constant $c>0$ and decides $L$.
\end{defi}

\begin{defi}[The class \compclass{P}]
\begin{equation}
\term{\compclass{P}}=\displaystyle\cup_{c\geq1}\fun{\compclass{DTIME}}{n^c}.
\end{equation}
\cite{arora2009computational}
\end{defi}

\begin{defi}[The class \compclass{NTIME}]
For every function $\funsig{T}{\NNN}{\NNN}$ and $L\subseteq\binarystrings$, we say that $L\in\fun{\term{\compclass{NTIME}}}{\fun{T}{n}}$ if there is a constant $c>0$ and a $c\fun{T}{n}$-time non deterministic Turing Machine $M$ such that for every $x\in\binarystrings$, $x\in L\leftrightarrow\fun{M}{x}=1$.
\cite{arora2009computational}
\end{defi}

\begin{defi}[The class \compclass{NP}]
\begin{equation}
\term{\compclass{NP}}=\displaystyle\cup_{c\geq1}\fun{\compclass{NTIME}}{n^c}.
\end{equation}
\cite{arora2009computational}
\end{defi}

\begin{defi}[Reduction, \compclass{NP-hard} and \compclass{NP-complete}]
A language $L\subseteq\binarystrings$ is \term{polynomial-time Karp reducible} to a language $L'\subseteq\binarystrings$ (sometimes shortened to just ``\term{polynomial-time reducible}''), denoted by $L\leq_p L'$, if there is a polynomial-time computable function $\btobfun{f}$ such that for every $x\in\binarystrings$, $x\in L$ if and only if $\ffun{x}\in L'$. We say that $L'$ is \term{\compclass{NP-hard}} if $L\leq_p L'$ for every $L\in\compclass{NP}$. We say that $L$ is \term{\compclass{NP-complete}} if $L'$ is \compclass{NP-hard} and $L'\in\compclass{NP}$.
\cite{arora2009computational}
\end{defi}

\begin{defi}[Complement language]
If $L\subseteq\binarystrings$ is a language, then we denote by $\overline{L}$ the \term{complement language} of $L$. That is, $\overline{L}=\binarystrings\setminus L$.
\cite{arora2009computational}
\end{defi}

\begin{defi}[The class \compclass{coNP}]
\begin{equation}
\term{\compclass{coNP}}=\left\{\overline{L}:L\in\compclass{NP}\right\}.
\end{equation}
\cite{arora2009computational}
\end{defi}

\begin{theo}[Efficient universal Turing machine]
\label{theo:eutm}
There exists a Turing Machine $\mathcal{U}$ such that for every $x,\alpha\in\binarystrings$, $\fun{\mathcal{U}}{x,\alpha} = \fun{M_{\alpha}}{x}$, where $M_{\alpha}$ denotes the Turing Machine represented by $\alpha$. Moreover, if $M_{\alpha}$ halts on input $x$ within $T$ steps then $\fun{\mathcal{U}}{x,\alpha}$ halts within $CT\log T$ steps, where $C$ is a number independent of $\left|x\right|$ and depending only on $M_{\alpha}$'s alphabet size, number of tapes, and number of states.
\cite{arora2009computational}
\end{theo}

\begin{theo}
\label{theo:uncomputable}There exists a function $\btoblfun{{\sf UC}}$ that is not computable by any Turing Machine.
\begin{proof}
The function $\sl UC$ is defined as follows: For every $\alpha\in\binarystrings$, if $\fun{M_{\alpha}}{\alpha}=1$, then $\fun{\sl UC}{\alpha}=0$; otherwise (if $\fun{M_{\alpha}}{\alpha}$ outputs a different value or enters an infinite loop), $\fun{\sl UC}{\alpha}=1$. Suppose for the sake of contradiction that ${\sl UC}$ is computable and hence there exists a Turing Machine $M$ such that $\fun{M}{\alpha}=\fun{{\sl UC}}{\alpha}$ for every $\alpha\in\binarystrings$. Then, in particular, $\fun{M}{\left\lfloor M\right\rfloor}=\fun{{\sl UC}}{\left\lfloor M\right\rfloor}$. But this is impossible: By the definition of ${\sl UC}$, $\fun{{\sl UC}}{\left\lfloor M\right\rfloor}=1\leftrightarrow\fun{M}{\left\lfloor M\right\rfloor}=1$.
\cite{arora2009computational}
\end{proof}
\end{theo}

\begin{theo}
{\slHALT} is not computable by any Turing Machine.
\begin{proof}
Suppose, for the sake of contradiction, that there was a Turing Machine $M_{\slHALT}$ computing $\slHALT$. We will use $M_{\slHALT}$ to show a Turing Machine $M_{\slUC}$ computing $\slUC$, contradicting Theorem \ref{theo:uncomputable}. The Turing Machine $M_{\slUC}$ is simple: On input $\alpha$, $M_{\slUC}$ runs $\fun{M_{\slHALT}}{\alpha,\alpha}$. If the result is 0 (meaning that $M_\alpha$ does not halt on $\alpha$), then $M_{\slUC}$ outputs $1$. Otherwise, $M_{\slUC}$ uses the universal Turing Machine $\mathcal{U}$ to compute $b=\fun{M_{\alpha}}{\alpha}$. If $b=1$, then $M_{\slUC}$ outputs $0$; otherwise, it outputs $1$. Under the assumption that $\fun{M_{\slHALT}}{\alpha,\alpha}$ outputs $\fun{\slHALT}{\alpha,\alpha}$ within a finite number of steps, the Turing Machine $\fun{M_{\slUC}}{\alpha}$ will output $\fun{\slUC}{\alpha}$.
\cite{arora2009computational}
\end{proof}
\end{theo}

\begin{theo}[\term{Time Hierarchy Theorem}]
\label{theo:tht}
If $f,g$ are time-constructible functions satisfying $\ffun{n}\log\ffun{n}=\smalloh{\gfun{n}}$, then
\begin{equation}
\dtime{\ffun{n}}\subsetneq\dtime{\gfun{n}}.
\end{equation}
\begin{proof}
To showcase the essential idea of the proof of Theorem \ref{theo:tht} with minimal notation, we prove the simpler statement $\dtime{n}\subsetneq\dtime{n^{1.5}}$. Consider the following Turing machine $D$: ``On input $x$, run for $\abs{x}^{1.4}$ steps the Universal Turing Machine $\mathcal{U}$ of Theorem \ref{theo:eutm} to simulate the execution of $M_x$ on $x$. If $\mathcal{U}$ outputs some bit $b\in\binary$ in this time, then output the opposite answer (i.e., output $1-b$). Else output 0.'' Here $M_x$ is the machine represented by the string $x$. By definition, $D$ halts within $n^{1.4}$ steps and hence the language $L$ decided by $D$ is in $\dtime{n^{1.5}}$. We claim that $L\notin\dtime{n}$. For contradiction's sake, assume that there is some Turing Machine $M$ and constant $c$ such that Turing Machine $M$, given any input $x\in\binarystrings$, halts within $c\abs{x}$ steps and outputs $\fun{D}{x}$. The time to simulate $M$ by the universal Turing machine $\mathcal{U}$ on every input x is at most $c'c\abs{
x}\log\abs{x}$ for some number $c'$ that depends on the alphabet size and number of tapes and states of $M$ but is independent of $\abs{x}$. There is some number $n_0$ such that $n^{1.4} > c'cn\log n$ for every $n\geq n_0$. Let $x$ be a string representing the machine $M$ whose length is at least $n_0$ (such a string exists since $M$ is represented by infinitely many strings). Then, $\fun{D}{x}$ will obtain the output $b=\fun{M}{x}$ within $\abs{x}^{1.4}$ steps, but by definition of $D$, we have $\fun{D}{x}=1-b=\fun{M}{x}$. Thus we have derived a contradiction. The proof Theorem \ref{theo:tht} for general $f,g$ is similar and uses the observation that the slowdown in simulating a machine using $\mathcal{U}$ is at most logarithmic.
\cite{arora2009computational}
\end{proof}
\end{theo}

\begin{theo}[\term{Nondeterministic Time Hierarchy Theorem}]
\label{theo:ndtht}
If $f,g$ are time constructible functions satisfying $\ffun{n+1}= \smalloh{\gfun{n}}$, then
\begin{equation}
\ntime{\ffun{n}}\subsetneq\ntime{\gfun{n}}.
\end{equation}
\begin{proof}
We just showcase the main idea by proving $\ntime{n}\subsetneq\ntime{n^{1.5}}$. The first instinct is to duplicate the proof of Theorem \ref{theo:tht}, since there is a universal Turing machine for nondeterministic computation as well. However, this alone does not suffice because the definition of the new machine $D$ requires the ability to ``flip the answer'', in other words, to efficiently compute, given the description of an nondeterminstic Turing machine $M$ and an input $x$, the value $1-\fun{M}{x}$. It is not obvious how to do this using the universal nondeterministic machine: it is unclear how a nondeterministic machine can just ``flip the answer''. Specifically, we do not expect that that the complement of an $\ntime{n}$ language will be in $\ntime{n^{1.5}}$. Now of course, the complement of every $\ntime{n}$ language is trivially decidable in exponential time (even deterministically) by examining all the 
possibilities for the machine's nondeterministic choices, but on first sight this seems to be completely irrelevant to proving $\ntime{n}\subsetneq\ntime{n^{1.5}}$. Surprisingly, this trivial exponential simulation of a nondeterministic machine does suffice to establish a hierarchy theorem. The key idea will be lazy diagonalization, so named because the new machine $D$ is in no hurry to diagonalize and only ensures that it flips the answer of each linear time nondeterminstic Turing machine $M_i$ in only one string out of a sufficiently large (exponentially large) set of strings. Define the function $\funsig{f}{\NNN}{\NNN}$ as follows: $\ffun{1}=2$ and $\ffun{i+1}=2^{\ffun{i}^{1.2}}$. Given $n$, it's not hard to find in $\bigoh{n^1.5}$ time the number $i$ such that $n$ is sandwiched between $\ffun{i}$ and $\ffun{i+1}$. Our diagonalizing machine $D$ will try to flip the answer of $M_i$ on some input in the set $\left\{1^n:\ffun{i}<n\leq\ffun{i+1}\right\}$. $D$ is defined as follows:
\begin{quote}
``On input $x$, if $x\notin1^*$, reject. If $x=1^n$, then compute $i$ such that $\ffun{i}<n\leq\ffun{i+1}$ and
\begin{enumerate}
 \item If $\ffun{i}<n<\ffun{i+1}$ then simulate $M_i$ on input $1^{n+1}$ using nondeterminism in $n^{1.1}$ time and output its answer. (If $M_i$ has not halted in this time, then halt and accept.)
 \item If $n=\ffun{i+1}$, accept $1^n$ if and only if $M_i$ rejects $1^{\ffun{i}+1}$ in $(\ffun{i}+1)^{1.1}$ time.''
\end{enumerate}
\end{quote}
Part 2 requires going through all possible $2^(\ffun{i}+1)^{1.1}$ branches of $M_i$ on input $1^{\ffun{i}+1}$, but that is fine since the input size $\ffun{i+1}$ is $2^{\ffun{i}^{1.2}}$. Hence the nondeterminstic Turing machine $D$ runs in $\bigoh{n^{1.5}}$ time. Let $L$ be the language decided by $D$. We claim that $L\notin\ntime{n}$. Indeed, suppose for the sake of contradiction that $L$ is decided by an nondeterminstic Turing machine $M$ running in $cn$ steps (for some constant $c$). Since each nondeterminstic Turing machine is represented by infinitely many strings, we can find $i$ large enough such that $M=M_i$ and on inputs of length $n\geq\ffun{i}$, $M_i$ can be simulated in less than $n^{1.1}$ steps. This means that the two steps in the description of $D$ ensure, respectively, that 

\begin{eqnarray}
\txIf\ffun{i}<n<\ffun{i+1},&\txthen\fun{D}{1^n}=\fun{M_i}{1^{n+1}}\label{eqn:ndtht33}\\
\txwhereas&\fun{D}\neq\fun{M_i}{1^{\ffun{i}+1}}\label{eqn:ndtht34}
\end{eqnarray}

By our assumption $M_i$ and $D$ agree on all inputs $1^n$ for $n$ in the semi-open interval $\left(\ffun{i},\ffun{i+1}\right]$. Together with (\ref{eqn:ndtht33}), this implies that $\fun{D}{1^{\ffun{i+1}}}=\fun{M_i}{1^{\ffun{i}+1}}$, contradicting (\ref{eqn:ndtht34}).\cite{arora2009computational}
\end{proof}
\end{theo}