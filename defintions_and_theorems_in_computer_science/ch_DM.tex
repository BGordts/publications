\chapter{Discrete Mathematics}

\section{Operator Properties}

\begin{defi}[Commutative operation]
A binary operation $\oplus$ is \term{commutative} if $a\oplus b = b\oplus a$ for all $a,b\in S$.\cite{Oppliger:2011:CC:2049860}
\end{defi}

\begin{defi}[Associative operation]
A binary operation $\oplus$ is \term{associative} if $a\oplus\left(b\oplus c\right) = \left(a\oplus b\right)\oplus c$ for all $a,b,c\in S$.\cite{Oppliger:2011:CC:2049860}
\end{defi}

\begin{defi}[Semi-associative]
A binary operator $\otimes$ is said to be \term{semi-associative} if there is an associative operator $\oplus$ such that for any $a$, $b$, $c$, $\left(a\otimes b\right)\otimes c = a\otimes\left(b\oplus c\right)$.\cite{conf/europar/MatsuzakiHT03}
\end{defi}

\begin{defi}[Quasi-associative]
A binary operator $\oplus$ is said to be \term{quasi-associative} if there is a semi-associative operator $\otimes$ and a function $f$ such that for any $a$, $b$, $a\oplus b = a\otimes f b$.\cite{conf/europar/MatsuzakiHT03}
\end{defi}

\begin{defi}[Bi-quasi-associative]
A ternary operator $f$ is said to be \term{bi-quasi-associative} if there is a semi-associative operator $\otimes$ and two functions $f_L'$, $f_R'$ such that for any $l$, $n$, $r$, $f l n r = l \otimes f_L' n r = r \otimes f_R' n l$. We can fix a bi-quasi-associative operator $f$ by providing $\otimes$, $\oplus$ (associative operator for $\otimes$), $f_L'$ and $f_R'$, therefore, we will write $f$ with 4-tuple as $f\equiv \left[\left[\otimes,\oplus, f_L', f_R' \right]\right]$.\cite{conf/europar/MatsuzakiHT03}
\end{defi}

\section{Algebraic Structures}

\begin{defi}[Left identity element] Let $S$ be a set and $\oplus$ a binary operation on $S$. An element $e\in S$ is called \term{left identity element} if $e\oplus a = a$ for all $a\in S$.\cite{Oppliger:2011:CC:2049860}
\end{defi}

\begin{defi}[Right identity element] Let $S$ be a set and $\oplus$ a binary operation on $S$. An element $e\in S$ is called \term{right identity element} if $a\oplus e = a$ for all $\in S$.\cite{Oppliger:2011:CC:2049860}
\end{defi}

\begin{defi}[Identity element] Let $S$ be a set and $\oplus$ a binary operation on $S$. An element $e\in S$ is called \termor{identity element}{neutral element} if it is both a left identity element and a right identity element (i.e., $e\oplus a = a\oplus e = a$ for all $a\in S$).
\cite{Oppliger:2011:CC:2049860}
\end{defi}

\begin{defi}[Inverse element] Let $S$ be a set, $\oplus$ be a binary operation with an identity element $e$, and $a$ be an element of $S$. If there exists an element $b\in S$ with $a\oplus b = b\oplus a = e$, then $a$ is \term{invertible} and b is the \termor{inverse element}{inverse} of $a$.
\cite{Oppliger:2011:CC:2049860}
\end{defi}

\begin{defi}[Semigroup]
A \term{semigroup} is an algebraic structure $\left\langle S,\oplus\right\rangle$ that consists of a nonempty set $S$ and an associative binary operation $\oplus$. The semigroup must be closed (i.e., for all $a,b\in S$, $a\oplus b$ must also be an element of $S$).
\cite{Oppliger:2011:CC:2049860}
\end{defi}

\begin{defi}[Monoid]
A \term{monoid} is a semigroup \soplus that has an identity element $e\in S$ with respect to $\oplus$.
\cite{Oppliger:2011:CC:2049860}
\end{defi}

\begin{defi}[Group]
A \term{group} is a monoid \soplus in which every element $a\in S$ has an inverse element in $S$ (i.e., every element $a\in S$ is invertible).
\cite{Oppliger:2011:CC:2049860}
\end{defi}

\begin{defi}[Commutative group]
A group \soplus is a \term{commutative group} if the operation $\oplus$ is commutative (i.e., $a\oplus b = b\oplus a$ for all $a, b\in S$).
\cite{Oppliger:2011:CC:2049860}
\end{defi}

\begin{defi}[Finite group]
A group \soplus is a \term{finite group} if it contains only finitely many elements.
\cite{Oppliger:2011:CC:2049860}
\end{defi}

\begin{defi}[Subgroup]
A subset $H$ of a group $G$ is a \term{subgroup} of $G$ if it is closed under the operation of $G$ and also forms a group.
\cite{Oppliger:2011:CC:2049860}
\end{defi}

\begin{defi}[Left coset]
Let $G$ be a group and $H\subseteq G$ be a subset of $G$. For all $a\in G$, the sets $a\oplus H := \left\{a\oplus h | h\in H\right\}$ are called \term{left coset}s of $H$.
\cite{Oppliger:2011:CC:2049860}
\end{defi}

\begin{defi}[Right coset]
Let $G$ be a group and $H\subseteq G$ be a subset of $G$. For all $a\in G$, the sets $H\oplus a := \left\{a\oplus h | h\in H\right\}$ are called \term{right coset}s of $H$.
\cite{Oppliger:2011:CC:2049860}
\end{defi}

\begin{defi}[Coset]
Let $G$ be a (commutative) group and $H\subseteq G$. For all $a\in G$, the sets $a\oplus H$ and $H\oplus a$ are equal and are called \term{coset}s of $H$.
\cite{Oppliger:2011:CC:2049860}
\end{defi}

\begin{defi}[Ring]
A \term{ring} is an algebraic structure \soplusotimes with a set $S$ and two associative binary operations $\oplus$ and $\otimes$ that fulfill the following requirements:
\begin{enumerate}
 \item \soplus is a commutative group with identity element $e_1$ ;
 \item \sotimes is a monoid with identity element $e_2$ ;
 \item The operation $\otimes$ is distributive over the operation $\oplus$. This means that for all $a,b,c\in S$ the following two distributive laws must hold:
 \begin{equation}
 \left\{\begin{array}{c}
  a \otimes (b \oplus c)=(a \otimes b) \oplus (a \otimes c)\\
 (b \oplus c) \otimes a=(b \otimes a) \oplus (c \otimes a)
 \end{array}\right.
 \end{equation}
\end{enumerate}
\cite{Oppliger:2011:CC:2049860}
\end{defi}

\begin{defi}[Field]
A ring \soplusotimes in which $\left\langle S\setminus\left\{e_1\right\},\otimes\right\rangle$ is a group is a \term{field}.
\cite{Oppliger:2011:CC:2049860}
\end{defi}

\begin{defi}[Subfield]
A subset $H$ of a field $F$ is a \term{subfield} of $F$ if it closed under the operations of $F$ and also forms a field.
\cite{Oppliger:2011:CC:2049860}
\end{defi}

\begin{defi}[Prime field]
A \term{prime field} is a field that contains no proper subfield.
\cite{Oppliger:2011:CC:2049860}
\end{defi}

\begin{defi}[Homomorphism]
Let $A$ and $B$ be two algebraic structures. A mapping $f:A\rightarrow B$ is called a \term{homomorphism} of $A$ into $B$ if it preserves the operations of $A$. That is, if $\circ$ is an operation of $A$ and $\bullet$ an operation of $B$, then $\ffun{x\circ y}=\ffun{x}\bullet\ffun{y}$ must hold for all $x,y\in A$.
\cite{Oppliger:2011:CC:2049860}
\end{defi}

\begin{defi}[Isomorphism]
A homomorphism $f:A\rightarrow B$ is an \term{isomorphism} if it is injective (``one to one''). In this case, we say that $A$ and $B$ are isomorphic and we write $A\cong B$.
\cite{Oppliger:2011:CC:2049860}
\end{defi}

\begin{defi}[Automorphism]
An isomorphism $f:A\rightarrow A$ is an \term{automorphism}.
\cite{Oppliger:2011:CC:2049860}
\end{defi}

\begin{defi}[Permutation]
Let $S$ be a set. A map $f:S\rightarrow S$ is a \term{permutation} if $f$ is bijective (i.e., injective and surjective). The set of all permutations of $S$ is denoted by \perm{S\rightarrow S}, or $\fun{P}{S}$ in short.
\cite{Oppliger:2011:CC:2049860}
\end{defi}

\begin{defi}[Common divisors and greatest common divisor]
For $a,b\in\ZZZ_0$, $c\in\ZZZ$ is a \term{common divisor} of $a$ and $b$ if $c|a$ and $c|b$. Furthermore, $c$ is the \term{greatest common divisor}, denoted $\fun{gcd}{a,b}$, if it is the largest integer that divides $a$ and $b$.
\cite{Oppliger:2011:CC:2049860}
\end{defi}

\begin{defi}[Common multiples and least common multiple]
For $a,b\in\ZZZ_0$, $c\in\ZZZ$ is a \term{common multiple} of $a$ and $b$ if $a|c$ and $b|c$. Furthermore, $c$ is the \term{least common multiple}, denoted $\fun{lcm}{a,b}$, if it is the smallest integer that is divided by $a$ and $b$.
\cite{Oppliger:2011:CC:2049860}
\end{defi}

\begin{defi}[Prime number]
A natural number $1<n\in\NNN$ is called a \termor{prime number}{prime} if it divisible only by $1$ and itself.
\cite{Oppliger:2011:CC:2049860}
\end{defi}

\begin{defi}[Primality decision problem]
Given a positive integer $n\in\NNN$, deciding whether $n\in\PPP$ (i.e., $n$ is prime) or not (i.e., $n$ is composite) is called the \term{primality decision problem}.
\cite{Oppliger:2011:CC:2049860}
\end{defi}

\begin{defi}[B-smooth integer]
Let $B$ be an integer. An integer $n$ is a \term{B-smooth integer} if every prime factor of $n$ is less than $B$.
\cite{Oppliger:2011:CC:2049860}
\end{defi}

\begin{defi}
Let $a,b\in\ZZZ$ and $n\in\NNN$. $a$ is \term{congruent to $b$ modulo $n$}, denoted $a\equiv b \left(\mod n\right)$, if $n$ divides $a-b$ (i.e., $n|a-b$).
\cite{Oppliger:2011:CC:2049860}
\end{defi}

\begin{defi}[Polynomial]
Let $A$ be an algebraic structure with addition and multiplication (e.g., a ring or a field). A function $\fun{p}{x}$ is a \term{polynomial} in $x$ over $A$ if it is of the form
\begin{equation}
\fun{p}{x}=\displaystyle\sum_{i=0}^{n}{a_ix^i}=a_0+a_1x+a_2x^2+\ldots+a_nx^n
\end{equation}
where $n$ is a positive integer (i.e., the \term{degree} of $\fun{p}{x}$, denoted as $\fun{\mbox{deg}}{p}$), the \term{coefficient}s $a_i$ ($0\leq i\leq n$) are elements in $A$, and $x$ is a symbol not belonging to $A$.
\cite{Oppliger:2011:CC:2049860}
\end{defi}

\begin{defi}[Quadratic residue and square root]
An element $x\in\ZZZ^*_n$ is a \term{quadratic residue} modulo $n$ if there exists an element $y\in\ZZZstarn$ such that $x=y^2\left(\mod n\right)$. If such a $y$ exists, then it is called a \term{square root} of $x$ modulo $n$.
\cite{Oppliger:2011:CC:2049860}
\end{defi}

\begin{theo}[Lagrange's Theorem]
If $H$ is a subgroup of $G$, then $\left|H\right||\left|G\right|$ (i.e., the order of $H$ divides the order of $G$.
\begin{proof}
If $H=G$, then $\left|H\right||\left|G\right|$ holds trivially. Consequently, we only consider the case in which $H\subset G$. For any $a\in G\setminus H$, the coset $a\oplus H$ is a subset of $G$. The following can be shown:
\begin{enumerate}
 \item \label{i} For any $a \neq a$, if $a\notin a'\oplus H$ then $\left(a\oplus H\right)\cap\left(a'\oplus H\right)=\emptyset$;
 \item \label{ii} $\left|a\oplus H\right|=\left|H\right|$.
\end{enumerate}
For (\ref{i}{}), suppose there exists a $b\in\left(a\oplus H\right)\cap\left(a'\oplus H\right)$. Then there exist $c,c'\in H$ such that $a\oplus c=b=a'\oplus c'$. Applying various group axioms, we have $a=a\oplus e= a\oplus\left(c\oplus c^{-1}\right)=b\oplus c^{-1}=\left(a'\oplus c'\right)\oplus c^{-1}=a'\oplus\left(c'\oplus c^{-1}\right)\in a'\oplus H$. This contradicts our assumption (that $a\notin a'\oplus H$).
For (\ref{ii}{}), $\left|a\oplus H\right|\leq\left|H\right|$ holds trivially (by the definition of a coset). Suppose that the inequality is rigorous. This is only possible if there are $b,c\in H$ with $b\neq c$ and $a\oplus b = a\oplus c$. Applying the inverse element of $a$ on either side of the equation, we get $b=c$, contradicting to $b\neq c$.
In summary, $G$ is partitioned by $H$ and the family of its mutually disjoint cosets, each has the size $\left|H\right|$, and hence $\left|H\right||\left|G\right|$. This proves the theorem.
\cite{Oppliger:2011:CC:2049860}
\end{proof}
\end{theo}

\begin{theo}
For all $a,b,c\in\ZZZ$, if $a|b$ and $b|c$, then $a|c$.
\begin{proof}
If $a|b$ and $b|c$, then there exist $f,g\in\ZZZ$ with $b=af$ and $c=bg$. Consequently, we can write $c=bg= \left(af\right)g = a\left(fg\right)$ to express $c$ as a multiple of $a$. The claim (i.e., $a|c$) follows directly from this equation.
\cite{Oppliger:2011:CC:2049860}
\end{proof}
\end{theo}

\begin{theo}
For all $a,b,c\in\ZZZ$, if $a|b$, then $ac|bc$ for all $c$.
\begin{proof}
If $a|b$, then there exists $f\in\ZZZ$ with $b=af$. Consequently, we can write $bc=\left(af\right)c=f\left(ac\right)$ to express $bc$ as a multiple of $ac$. The claim (i.e., $ac|bc$) follows directly from this equation.
\cite{Oppliger:2011:CC:2049860}
\end{proof}
\end{theo}

\begin{theo}
For all $a,b,c,d,e\in\ZZZ$, if $c|a$ and $c|b$, then $c|da+eb$ for all $d$ and $e$.
\begin{proof}
If $c|a$ and $c|b$, then there exist $f,g\in\ZZZ$ with $a=fc$ and $b=gc$. Consequently, we can write $da+eb=dfc+egc=\left(df+eg\right)c$ to express $da+eb$ as a multiple of $c$. The claim (i.e., $c|da+eb$) follows directly from this equation.
\cite{Oppliger:2011:CC:2049860}
\end{proof}
\end{theo}

\begin{theo}
\label{theo4}
For all $a,b\in\ZZZ$, if $a|b$ and $b\neq0$, then $\left|a\right|\leq\left|b\right|$.
\begin{proof}
If $a|b$ and $b=0$, then there exists $0\neq f\in\ZZZ$ with $b=af$. Consequently, $\left|b\right|=\left|af\right|\geq\left|a\right|$ and the claim (i.e., $\left|a\right|\leq\left|b\right|$) follows immediately.
\cite{Oppliger:2011:CC:2049860}
\end{proof}
\end{theo}

\begin{theo}
For all $a,b\in\ZZZ$, if $a|b$ and $b|a$, then $\left|a\right|=\left|b\right|$.
\begin{proof}
Let us assume that $a|b$ and $b|a$. If $a=0$ then $b=0$, and vice versa. If $a,b\neq 0$, then it follows from Theorem \ref{theo4}. that $\left|a\right|\leq\left|b\right|$ and $\left|b\right|\leq\left|a\right|$, and hence $\left|b\right|=\left|a\right|$.
\cite{Oppliger:2011:CC:2049860}
\end{proof}
\end{theo}

\begin{theo}[Euclid's division theorem]
For all $n,d\in\ZZZ_0$ there exist unique and efficiently computable $q,r\in\ZZZ$ such that $n=qd+r$ and $0\leq r\leq\left|d\right|$.
\cite{Oppliger:2011:CC:2049860}
\end{theo}

\begin{theo}[Prime density theorem]
\begin{equation}
\displaystyle\lim_{n\rightarrow\infty}\displaystyle\frac{\pi\left(n\right)\ln\left(n\right)}{n}=1
\end{equation}
\cite{Oppliger:2011:CC:2049860}
\end{theo}

\begin{theo}[Unique factorization]
Every natural number $n\in\NNN$ can be factored uniquely (up to a permutation of the prime factors):
\begin{equation}
n=\displaystyle\prod_{p\in\PPP}{p^{\fun{e_p}{n}}}
\end{equation}
In this formula, $\fun{e_p}{n}$ refers to the exponent of $p$ in the factorization of $n$. For almost all $p\in\PPP$ this value is zero, and only for finitely many primes $p$ the value $\fun{e_p}{n}$ is greater than zero.
\cite{Oppliger:2011:CC:2049860}
\end{theo}

\begin{theo}[Chinese remainder theorem]
Let
\begin{equation}
\left\{\begin{array}{l}
x\equiv a_1\left(\mod n_1\right)\\
x\equiv a_2\left(\mod n_2\right)\\
\ldots\\
x\equiv a_k\left(\mod n_k\right)
\end{array}\right.
\end{equation}
be a system of $k$ congruences with pairwise co-prime moduli $n_1,\ldots,n_k$. The system has a unique and efficiently computable solution $x$ in $\ZZZ_n$ with $n=\prod_{i=1}^kn_i$.
\cite{Oppliger:2011:CC:2049860}
\end{theo}

\begin{theo}[Fermat's Little Theorem]
If $p$ is a prime and $a\in\ZZZ^*_p$, then $a^{p-1}\left(\mod p\right)$.
\begin{proof}
Because $\fun{\phi}{p}=p-1$ for every prime number $p$, \term{Fermat's Little Theorem} is just a special case of Euler's Theorem.
\cite{Oppliger:2011:CC:2049860}
\end{proof}
\end{theo}

\begin{theo}[Euler's Theorem]
If $\gcdf{a,n}=1$, then $a^{\fun{\phi}{n}}\equiv 1\left(\mod n\right)$.
\begin{proof}
Because $\gcdf{a,n}=1$, $a \left(\mod n\right)$ must be an element in $\ZZZ^*_n$. Also, $\left|\ZZZ^*_n\right|=\fun{\phi}{n}$. According to a corollary of Lagrange's Theorem, the order of every element (in a finite group) divides the order of the group. Consequently, the order of a (i.e., $\ordf{a}$) divides $\phif{n}$, and hence if we multiply a modulo $n\phif{n}$ times we always get a value that is equivalent to $1$ modulo $n$.
\cite{Oppliger:2011:CC:2049860}
\end{proof}
\end{theo}

\section{Probability Theory}

\begin{defi}[Conditional Probability (Bayes' rule)]
The probability of event $x$ conditioned on knowing event $y$ (or more shortly, the \term{conditional probability} of $x$ given $y$) is defined as:
\begin{equation}
\pcond{x}{y}\equiv\displaystyle\frac{\pfun{x}}{\pfun{y}}
\end{equation}
If $\pfun{y}=0$, then $\pcond{x}{y}$ is not defined. This definition is also called \term{Bayes' rule}.
\cite{Barber2011}
\end{defi}

\begin{defi}[Probability Density Functions]
For a single continuous variable $x$, the \term{probability density function} $\pfun{x}$ is a function such that:
\begin{equation}
\left\{\begin{array}{l}
\pfun{x}\geq0\\
\bint{\pfun{x}}{x}{-\infty}{+\infty}=1\\
\pfun{a<x<b}=\bint{\pfun{x}}{x}{a}{b}
\end{array}\right.
\end{equation}
\cite{Barber2011}
\end{defi}

\begin{defi}[Independence]
Events $x$ and $y$ are independent if knowing one event gives no extra information about the other event. Mathematically, this is expressed by:
\begin{equation}
\pfun{x,y}=\pfun{x}\pfun{y}
\end{equation}
Provided that $\pfun{x}\neq0$ and $\pfun{y}\neq0$ \term{independence} of $x$ and $y$ is equivalent to:
\begin{equation}
\pcond{x}{y}=\pfun{x}\leftrightarrow\pcond{y}{x}=\pfun{y}
\end{equation}
If $\pcond{x}{y}=\pfun{x}$ for all states of $x$ and $y$, then the variables $x$ and $y$ are said to be independent. If
\begin{equation}
\pfun{x,y}=k\ffun{x}\gfun{y}
\end{equation}
for some constant $k$, and positive functions $f$ and $g$ then $x$ and $y$ are independent.
\cite{Barber2011}
\end{defi}

\begin{defi}[Prior likelihood and Posterior]
For data $\calD$ and variable $\theta$, Bayes' rule tells us how to update our prior beliefs about the variable $\theta$ in light of the data to a posterior belief:
\begin{equation}
\underbrace{\pcond{\theta}{\calD}}_{\mbox{\term{posterior}}}=\displaystyle\frac{\underbrace{\pcond{\calD}{\theta}}_{\mbox{\term{likelihood}}}\underbrace{\pfun{\theta}}_{\mbox{\term{prior}}}}{\underbrace{\pfun{\calD}}_{\mbox{\term{evidence}}}}
\end{equation}
The evidence is also called the \term{marginal likelihood}. The term likelihood is used for the probability that a model generates observed data. More fully, if we condition on the model $M$, we have
\begin{equation}
\pcond{\theta}{\calD,M}=\displaystyle\frac{\pcond{\calD}{\theta,M}\pcond{\theta}{M}}{\pcond{\calD}{M}}
\end{equation}
where we see the role of the likelihood $\pcond{\calD}{\theta,M}$ and marginal likelihood $\pcond{\calD}{M}$. likelihood is also called the \term{model likelihood}. The \termabbrev{most probable a posteriori}{MAP} setting is that which maximizes the posterior,
\begin{equation}
\theta_*=\argmax[\theta]{\pcond{\theta}{\calD,M}}
\end{equation}
\cite{Barber2011}
\end{defi}

\begin{defi}[Conditional Independence]
\begin{equation}
\calX\ci\calY|\calZ
\end{equation}
denotes that the two sets of variables $\calX$ and $\calY$ are independent of each other provided we know the state of the set of variables $\calZ$. For full \term{conditional independence}, $\calX$ and $\calZ$ must be independent given all states of $\calZ$. Formally, this means that
\begin{equation}
\pcond{\calX,\calY}{\calZ}=\pcond{\calX}{\calZ}\pcond{\calY}{\calZ}
\end{equation}
for all states of $\calX$, $\calY$, $\calZ$. In case the conditioning set is empty we may also write $\calX\ci\calY$ for $\calX\ci\calY|\emptyset$, in which case $\calX$ is \term{unconditionally independent} of $\calY$. If $\calX$ and $\calY$ are not conditionally independent, they are \term{conditionally dependent}. This is written:
\begin{equation}
\calX\cd\calY|\calZ
\end{equation}
\cite{Barber2011}
\end{defi}

\section{Number Theory}

\section{Statistics}

\begin{defi}[Related data]
If data $d_1,d_2,\ldots,d_m$ describe a common phenomenon altogether, or they refer to the same behavior simultaneously, then they can be treated as \term{related data}.
\cite{conf/ijcai/ZhaoN95}
\end{defi}

\begin{defi}[Qualitative correlations among related data]
If $d_i$ and $d_j$ are two related data, then the presence of $d_i$ somewhat (qualitatively) enhances the presence of $d_j$, and the absence of $d_i$ somewhat (qualitatively) depresses the presence of $d_j$. The above effects are called \term{qualitative correlations among related data}.
\cite{conf/ijcai/ZhaoN95}
\end{defi}

\begin{defi}[Support coefficient Function (SCF)]
If there are $m-1$ data related to $d_i$, then the \termabbrev{support coefficient function}{SCF} of $d_i$ calculates the total effects from the related data by considering the qualitative correlations between $d_i$ and each of its related data.
\cite{conf/ijcai/ZhaoN95}
\end{defi}

\begin{defi}[Shift interval]
\term{Shift interval} is a dynamic region for inaccurate data. Given a standard fuzzy region for inaccurate $d_i$, the shift interval of $d_i$ varies around the standard fuzzy region on the basis of $SCF_i$. When $SCF_i$ shows that the related data support $d_i$ the shift interval of $d_i$ becomes wider than the standard fuzzy region. On the other hand, when $SCF_i$ shows that the related data do not support $d_i$, the shift interval of $d_i$ becomes narrower than the standard fuzzy region.
\cite{conf/ijcai/ZhaoN95}
\end{defi}

\begin{defi}[Evidence based on a Support Coefficient Function (SCF)]
$SCF_i$ determines the shift interval of $d_i$, that is, $SCF_i$ determines how widely $d_i$ is allowed to shift. The wider the shift interval, the more easily $d_i$ is identified. Therefore, SCFi provides confirmatory or disconfirmatory \term[Evidence based on a Support Coefficient Function]{evidence} for identifying $d_j$.
\cite{conf/ijcai/ZhaoN95}
\end{defi}

\begin{defi}[Rate of correctness (RC)]
the \term[Rate of correctness]{rate} that the identified partial component set is \termstyle{exactly the same} as the partial component set in the correct solutions.
\cite{conf/ijcai/ZhaoN95}
\end{defi}

\begin{defi}[Rate of identification (RI)]
The \term[Rate of identification]{rate} that the partial components in the correct solutions are \termstyle{identified}.
\cite{conf/ijcai/ZhaoN95}
\end{defi}