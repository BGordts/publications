\documentclass[handout]{beamer}
\usepackage[dutch]{babel}
\usepackage{tikz,listings}
%\usepackage{courrier}
\usetheme{Antibes}
\setcounter{tocdepth}{2}
\title{Gegevensstructuren en Algoritmen 1:\\ \texttt{Array}, \texttt{ArrayList} en \texttt{LinkedList}}
\author{Willem Van Onsem}
\AtBeginSection[]{\begin{frame}{Inhoud}
 \tableofcontents[currentsection]
\end{frame}}
% \lstset {
% basicstyle=\footnotesize\ttfamily,
% numberstyle=\tiny,
% breakatwhitespace=false,
% breaklines=true,
% captionpos=b,
% extendedchars=true,
% keywordstyle=\bffamily
% language=java
% }
\lstloadlanguages{Java}
\lstset{language=Java,basicstyle=\tiny}
\newcommand{\term}[1]{\textbf{#1}}
\begin{document}
\begin{frame}[plain]
\maketitle
\end{frame}
\begin{frame}
\tableofcontents
\end{frame}
\section{Interface}
\begin{frame}[fragile]{Interface}
\begin{definition}[Interface]
Een interface is een lijst van methodes die klasses die de interface implementeren moeten aanbieden.
\end{definition}
\begin{example}[\texttt{Praat}-interface]
\begin{lstlisting}
public interface Praat {

  void praat ();

}
\end{lstlisting}
\end{example}
\end{frame}
\subsection{Implementeren}
\begin{frame}[fragile]{Implementeren van een Interface}
Alle klasses die de interface \term{implementeren}, moeten de methodes invullen:
\begin{example}[Gert-klasse]
\begin{lstlisting}
public class Gert implements Praat {

  public void praat () {
    System.out.println("Hallo, mijn naam is Gert.");
  }

}
\end{lstlisting}
\end{example}
\begin{example}[Samson-klasse]
\small{\begin{lstlisting}
public class Samson implements Praat {

  public void praat () {
    System.out.println("Mwoa en ik ben Samson.");
  }

}
\end{lstlisting}}
\end{example}
\end{frame}
\section{Array}
\section{ArrayList}
\section{LinkedList}
\end{document}
