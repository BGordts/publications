\subsection{Implementatie}
\begin{frame}[fragile]{\dsarraylist{}: Implementatie}
Een \dsarraylist{} is een klasse in Java die werkt op basis van een \dsarray{} en een teller.
\begin{figure}
\centering
\begin{tikzpicture}
\genobj{alo}{elementen,aantal/3}{eenArrayList}
\begin{scope}[xshift=4 cm,yshift=0.5 cm]
\tikzarrayptr{9}{aloarray}
\pointrtlebridge{aloptrelementen}{aloarray.west}
\end{scope}
\begin{scope}[xshift=5.5 cm,yshift=2 cm]
\strobj{elem0}{Samson}
\end{scope}
\begin{scope}[xshift=6.5 cm,yshift=1.25 cm]
\strobj{elem2}{Octaaf}
\end{scope}
\begin{scope}[xshift=6 cm,yshift=-0.25 cm]
\strobj{elem3}{Alberto}
\end{scope}
\pointarraytopleft{aloarray1}{elem0}
\pointarraytopleft{aloarray3}{elem2}
\pointarraybotleft{aloarray2}{elem3}
\end{tikzpicture}
\caption{Structuur van een \dsarraylist{}-object}
\end{figure}
Telkens wanneer de \dsarray{} vol zit en we willen een element toevoegen, maken we een nieuwe (grotere) \dsarray{} aan en kopi\"eren we de oude elementen naar de nieuwe \dsarray{}.
\end{frame}
\begin{frame}{\dsarraylist{}: Implementatie Overzicht}
We zullen de implementatie van volgende methodes bespreken:
\begin{itemize}
 \item \texttt{setCapaciteit (int capaciteit)}: een methode die de capaciteit (lengte van de \dsarray{}) aanpast. In het geval we deze moeten aanpassen worden de elementen gekopieerd.
 \item \texttt{voegToe (Object object)}
 \item \texttt{voegToe (int index, Object object)}
 \item \texttt{verwijder (int index)}
 \item \texttt{verwijder (int index, Object object)}
\end{itemize}
\end{frame}
\subsubsection{\texttt{setCapaciteit(int capaciteit)}}
\begin{frame}[fragile]{\dsarraylist{}: \texttt{setCapaciteit (int capaciteit)}}
\begin{methodexample}[\texttt{setCapaciteit (int capaciteit)}]
\begin{center}
\begin{animateinline}{1}
\methodexec{arraylistsetcapaciteit}{8}{this}{
\genobj{alo}{elementen,aantal/3}{eenArrayList}
\begin{scope}[xshift=4 cm,yshift=0.5 cm]
\tikzarrayptr{9}{aloarray}
\pointrtlebridge{aloptrelementen}{aloarray.west}
\end{scope}
\begin{scope}[xshift=5.5 cm,yshift=2 cm]
\strobj{elem0}{Samson}
\end{scope}
\begin{scope}[xshift=6.5 cm,yshift=1.25 cm]
\strobj{elem2}{Octaaf}
\end{scope}
\begin{scope}[xshift=6 cm,yshift=-0.25 cm]
\strobj{elem3}{Alberto}
\end{scope}
\pointarraytopleft{aloarray1}{elem0}
\pointarraytopleft{aloarray3}{elem2}
\pointarraybotleft{aloarray2}{elem3}
}
\end{animateinline}
\end{center}
\end{methodexample}
\end{frame}