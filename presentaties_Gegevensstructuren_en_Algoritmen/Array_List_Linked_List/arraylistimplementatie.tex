\subsection{Implementatie}
\begin{frame}[fragile]{\dsarraylist{}: Implementatie}
Een \dsarraylist{} is een klasse in Java die werkt op basis van een \dsarray{} en een teller.
\begin{figure}
\centering
\begin{tikzpicture}
\genobj{alo}{elementen,aantal/3}{eenArrayList}
\begin{scope}[xshift=4 cm,yshift=0.5 cm]
\tikzarrayptr{4}{aloarray}
\pointrtlebridge{aloptrelementen}{aloarray.west}
\end{scope}
\begin{scope}[xshift=5.5 cm,yshift=2 cm]
\strobj{elem0}{Samson}
\end{scope}
\begin{scope}[xshift=6.5 cm,yshift=1.25 cm]
\strobj{elem2}{Octaaf}
\end{scope}
\begin{scope}[xshift=6 cm,yshift=-0.25 cm]
\strobj{elem3}{Alberto}
\end{scope}
\pointarraytopleft{aloarray1}{elem0}
\pointarraytopleft{aloarray3}{elem2}
\pointarraybotleft{aloarray2}{elem3}
\end{tikzpicture}
\caption{Structuur van een \dsarraylist{}-object}
\end{figure}
Telkens wanneer de \dsarray{} vol zit en we willen een element toevoegen, maken we een nieuwe (grotere) \dsarray{} aan en kopi\"eren we de oude elementen naar de nieuwe \dsarray{}.
\end{frame}
\begin{frame}{\dsarraylist{}: Implementatie Overzicht}
We zullen de implementatie van volgende methodes bespreken:
\begin{itemize}
 \item \texttt{setCapaciteit (int cap)}: een methode die de capaciteit (lengte van de \dsarray{}) aanpast. In het geval we deze moeten aanpassen worden de elementen gekopieerd.
 \item \texttt{voegToe (Object obj)}
 \item \texttt{voegToe (int index, Object obj)}
 \item \texttt{verwijder (int index)}
 \item \texttt{verwijder (int index, Object obj)}
\end{itemize}
\end{frame}
\subsubsection{\texttt{setCapaciteit(int cap)}}
\begin{frame}[fragile]{\dsarraylist{}: \texttt{setCapaciteit (int cap)}}
\begin{methodexample}[\texttt{setCapaciteit (int cap)}]
\begin{center}
\begin{animateinline}{1}
\setCapaciteitframe{0}{1}{this,cap/6}\newframe
\setCapaciteitframe{1}{2}{this,cap/6}\newframe
\setCapaciteitframe{2}{5}{this,cap/6,nieuw}\newframe
\setCapaciteitframe{3}{6}{this,cap/6,nieuw,temp}\newframe
\setCapaciteitframe{4}{7}{this,cap/6,nieuw,temp,i/0}\newframe
\setCapaciteitframe{5}{8}{this,cap/6,nieuw,temp,i/0}\newframe
\setCapaciteitframe{6}{9}{this,cap/6,nieuw,temp,i/0}\newframe
\setCapaciteitframe{7}{7}{this,cap/6,nieuw,temp,i/1}\newframe
\setCapaciteitframe{8}{8}{this,cap/6,nieuw,temp,i/1}\newframe
\setCapaciteitframe{9}{9}{this,cap/6,nieuw,temp,i/1}\newframe
\setCapaciteitframe{10}{7}{this,cap/6,nieuw,temp,i/2}\newframe
\setCapaciteitframe{11}{8}{this,cap/6,nieuw,temp,i/2}\newframe
\setCapaciteitframe{12}{9}{this,cap/6,nieuw,temp,i/2}\newframe
\setCapaciteitframe{13}{7}{this,cap/6,nieuw,temp,i/3}\newframe
\setCapaciteitframe{14}{11}{this,cap/6,nieuw,temp}
\end{animateinline}
\end{center}
\end{methodexample}
\end{frame}
\subsubsection{\texttt{voegToe(Object obj)}}
\begin{frame}[fragile]{\dsarraylist{}: \texttt{voegToe(Object obj)}}
\begin{methodexample}[\texttt{voegToe(Object obj)}]
\begin{center}
\begin{animateinline}{1}
\voegToeAframe{0}{1}{3}\newframe
\voegToeAframe{1}{2}{3}\newframe
\voegToeAframe{2}{5}{3}\newframe
\voegToeAframe{3}{6}{4}\newframe
\voegToeAframe{4}{1}{4}\newframe
\voegToeAframe{5}{2}{4}\newframe
\voegToeAframe{6}{3}{4}\newframe
\voegToeAframe{7}{5}{4}\newframe
\voegToeAframe{8}{6}{5}
\end{animateinline}
\end{center}
\end{methodexample}
\end{frame}
\subsubsection{\texttt{voegToe(int index, Object obj)}}
\begin{frame}[fragile]{\dsarraylist{}: \texttt{voegToe(int index, Object obj)}}
\begin{methodexample}[\texttt{voegToe(int index, Object obj)}]
\begin{center}
\begin{animateinline}{1}
\voegToeBframe{0}{1}{5}{}\newframe
\voegToeBframe{1}{2}{5}{}\newframe
\voegToeBframe{2}{5}{5}{}\newframe
\voegToeBframe{3}{8}{5}{,temp}\newframe
\voegToeBframe{4}{9}{5}{,temp,i/5}\newframe
\voegToeBframe{5}{10}{5}{,temp,i/5}\newframe
\voegToeBframe{6}{11}{5}{,temp,i/5}\newframe
\voegToeBframe{7}{9}{5}{,temp,i/4}\newframe
\voegToeBframe{8}{10}{5}{,temp,i/4}\newframe
\voegToeBframe{9}{11}{5}{,temp,i/4}\newframe
\voegToeBframe{10}{9}{5}{,temp,i/3}\newframe
\voegToeBframe{11}{10}{5}{,temp,i/3}\newframe
\voegToeBframe{12}{11}{5}{,temp,i/3}\newframe
\voegToeBframe{13}{9}{5}{,temp,i/2}\newframe
\voegToeBframe{14}{10}{5}{,temp,i/2}\newframe
\voegToeBframe{15}{11}{5}{,temp,i/2}\newframe
\voegToeBframe{16}{9}{5}{,temp,i/1}\newframe
\voegToeBframe{17}{13}{5}{,temp}\newframe
\voegToeBframe{18}{14}{6}{,temp}
\end{animateinline}
\end{center}
\end{methodexample}
\end{frame}
\subsubsection{\texttt{verwijder(int index)}}
\begin{frame}[fragile]{\dsarraylist{}: \texttt{verwijder(int index)}}
\begin{methodexample}[\texttt{verwijder(int index)}]
\begin{center}
\begin{animateinline}{1}
\verwijderAframe{0}{1}{6}{}\newframe
\verwijderAframe{1}{2}{6}{}\newframe
\verwijderAframe{2}{5}{6}{,outp}\newframe
\verwijderAframe{3}{6}{6}{,outp,temp}\newframe
\verwijderAframe{4}{7}{6}{,outp,temp,i/2}\newframe
\verwijderAframe{5}{8}{6}{,outp,temp,i/2}\newframe
\verwijderAframe{6}{9}{6}{,outp,temp,i/2}\newframe
\verwijderAframe{7}{7}{6}{,outp,temp,i/3}\newframe
\verwijderAframe{8}{8}{6}{,outp,temp,i/3}\newframe
\verwijderAframe{9}{9}{6}{,outp,temp,i/3}\newframe
\verwijderAframe{10}{7}{6}{,outp,temp,i/4}\newframe
\verwijderAframe{11}{8}{6}{,outp,temp,i/4}\newframe
\verwijderAframe{12}{9}{6}{,outp,temp,i/4}\newframe
\verwijderAframe{13}{7}{6}{,outp,temp,i/5}\newframe
\verwijderAframe{14}{11}{6}{,outp,temp}\newframe
\verwijderAframe{15}{12}{5}{,outp,temp}\newframe
\verwijderAframe{16}{13}{5}{,outp,temp}
\end{animateinline}
\end{center}
\end{methodexample}
\end{frame}
\subsubsection{\texttt{verwijder(Object obj)}}
\begin{frame}[fragile]{\dsarraylist{}: \texttt{verwijder(Object obj)}}
\begin{methodexample}[\texttt{verwijder(Object obj)}]
\begin{center}
\begin{animateinline}{1}
\verwijderBframe{0}{1}{5}{}\newframe
\verwijderBframe{1}{2}{5}{,temp}\newframe
\verwijderBframe{2}{3}{5}{,temp,i/0}\newframe
\verwijderBframe{3}{4}{5}{,temp,i/0}\newframe
\verwijderBframe{4}{5}{5}{,temp,i/0}\newframe
\verwijderBframe{5}{3}{5}{,temp,i/1}\newframe
\verwijderBframe{6}{4}{5}{,temp,i/1}\newframe
\verwijderBframe{7}{5}{5}{,temp,i/1}\newframe
\verwijderBframe{8}{3}{5}{,temp,i/2}\newframe
\verwijderBframe{9}{4}{5}{,temp,i/2}\newframe
\verwijderBframe{10}{5}{5}{,temp,i/2}\newframe
\verwijderBframe{11}{6}{4}{,temp,i/2}\newframe
\verwijderBframe{12}{7}{4}{,temp,i/2}\newframe
\verwijderBframe{13}{1}{4}{}\newframe
\verwijderBframe{14}{2}{4}{,temp}\newframe
\verwijderBframe{15}{3}{4}{,temp,i/0}\newframe
\verwijderBframe{16}{4}{4}{,temp,i/0}\newframe
\verwijderBframe{17}{5}{4}{,temp,i/0}\newframe
\verwijderBframe{18}{3}{4}{,temp,i/1}\newframe
\verwijderBframe{19}{4}{4}{,temp,i/1}\newframe
\verwijderBframe{20}{5}{4}{,temp,i/1}\newframe
\verwijderBframe{21}{3}{4}{,temp,i/2}\newframe
\verwijderBframe{22}{4}{4}{,temp,i/2}\newframe
\verwijderBframe{23}{5}{4}{,temp,i/2}\newframe
\verwijderBframe{24}{3}{4}{,temp,i/3}\newframe
\verwijderBframe{25}{4}{4}{,temp,i/3}\newframe
\verwijderBframe{26}{5}{4}{,temp,i/3}\newframe
\verwijderBframe{27}{3}{4}{,temp,i/4}\newframe
\verwijderBframe{28}{10}{4}{,temp}
\end{animateinline}
\end{center}
\end{methodexample}
\end{frame}