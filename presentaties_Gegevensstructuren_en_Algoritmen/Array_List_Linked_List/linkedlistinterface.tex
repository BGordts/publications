\subsection{Definitie}
\begin{frame}{\dslinkedlist{}: definitie}
\begin{definition}[\dslinkedlist{}]
Een \term{\dslinkedlist{}} is een gegevensstructuur die bestaat uit een opeenvolging van elementen. Een \dslinkedlist{} heeft een variabele \term{lengte}. Een \dslinkedlist{} kan sneller elementen toevoegen op het einde en aan het begin van de lijst dan een \dsarraylist{}. De toegang tot een willekeurig element hangt af van de index. (dit noemen we \term{sequential access}).
\end{definition}
Visueel zullen we een \dslinkedlist{} in deze presentatie voorstellen als een opeenvolging van knopen (cirkels), verbonden met elkaar en met een open begin en einde. Zoals hier:
\begin{figure}
\begin{tikzpicture}
\tikzlinkedlist{10}{a,b,c,d,e,f,g,h,i,j}
\end{tikzpicture}
\caption{Voorstelling van een \dslinkedlist{}}
\end{figure}
\end{frame}
\subsection{In Java}
\subsubsection{Declaratie}
\begin{frame}[fragile]{\dslinkedlist{} in Java: declaratie}
Een \dslinkedlist{} is een klasse die uit de Java-bibliotheek komt. Deze klasse is \texttt{java.util.LinkedList}. Daarom kunnen we een \dslinkedlist{} aanmaken zoals andere objecten. Een \dslinkedlist{} is een \term{generische klasse}: we kunnen een type-parameter tussen scheve haken (\texttt{<>}).
\examplecode{Declaratie \dslinkedlist{}}{linkedlistdeclaration0}
\end{frame}
\subsubsection{Toevoegen en verwijderen}
\begin{frame}[fragile]{\dslinkedlist{} in Java: toevoegen/verwijderen van elementen}
\dslinkedlist{} biedt methodes aan om elementen toe te voegen en te verwijderen:
\small{\begin{itemize}
 \item \texttt{public boolean add (E element)}: voegt een element toe op het einde van de \dslinkedlist{}
 \item \texttt{public void add (int index, E element)}: voegt een element toe op plaats \texttt{index}.%, de overige elementen worden naar rechts opgeschoven
 \item \texttt{public E remove (int index)}: verwijdert het element op \texttt{index}. Het element die oorspronkelijk op deze plaats stond, wordt teruggegeven.%De elementen erna worden naar links opgeschoven.
 \item \texttt{public boolean remove (Object o)}: het opgegeven element wordt verwijdert. Geeft \texttt{true} terug indien dit element aanwezig was.
\end{itemize}}
\begin{letop}
Merk op dat de methodes volledig overeenkomen met de methodes uit de \dsarraylist{}.
\end{letop}
\end{frame}
\begin{frame}[fragile]{\dslinkedlist{} in Java: \texttt{add (E element)}}
\begin{methodexample}[\texttt{add (E element)}]
\begin{center}
\begin{animateinline}{1}
\methodexec{linkedlistadd0me}{1}{}{
}\newframe

\methodexec{linkedlistadd0me}{2}{namen}{
\begin{scope}[xshift=4 cm]
\tikzlinkedlistptr{0}{namenlist}
\pointleftbridge{ptrnamen}{offsetnamenlist}
\end{scope}
}\newframe
\methodexec{linkedlistadd0me}{3}{namen}{
\begin{scope}[xshift=1 cm,yshift=1 cm]
\strobj{elem0}{Samson}
\end{scope}
\begin{scope}[xshift=4 cm]
\tikzlinkedlistptr{1}{namenlist}
\pointleftbridge{ptrnamen}{offsetnamenlist}
\pointarraytopright{namenlist1}{elem0}
\end{scope}
}\newframe

\methodexec{linkedlistadd0me}{4}{namen}{
\begin{scope}[xshift=1 cm,yshift=1 cm]
\strobj{elem0}{Samson}
\end{scope}
\begin{scope}[xshift=2 cm,yshift=-0.5 cm]
\strobj{elem1}{Gert}
\end{scope}
\begin{scope}[xshift=4 cm]
\tikzlinkedlistptr{2}{namenlist}
\pointleftbridge{ptrnamen}{offsetnamenlist}
\pointarraytopright{namenlist1}{elem0}
\pointarraybotright{namenlist2}{elem1}
\end{scope}
}\newframe

\methodexec{arraylistadd0me}{5}{namen}{
\begin{scope}[xshift=1 cm,yshift=1 cm]
\strobj{elem0}{Samson}
\end{scope}
\begin{scope}[xshift=2 cm,yshift=-0.5 cm]
\strobj{elem1}{Gert}
\end{scope}
\begin{scope}[xshift=3 cm,yshift=1.5 cm]
\strobj{elem2}{Octaaf}
\end{scope}
\begin{scope}[xshift=4 cm]
\tikzlinkedlistptr{3}{namenlist}
\pointleftbridge{ptrnamen}{offsetnamenlist}
\pointarraytopright{namenlist1}{elem0}
\pointarraybotright{namenlist2}{elem1}
\pointarraytopright{namenlist3}{elem2}
\end{scope}
}
\end{animateinline}
\end{center}
\end{methodexample}
\end{frame}
\begin{frame}[fragile]{\dsarraylist{} in Java: \texttt{add (int index, E element)}}
\begin{methodexample}[\texttt{add (int index, E element)}]
\begin{center}
\begin{animateinline}{1}
\methodexec{arraylistadd1me}{5}{namen}{
\begin{scope}[xshift=1 cm,yshift=1 cm]
\strobj{elem0}{Samson}
\end{scope}
\begin{scope}[xshift=2 cm,yshift=-0.5 cm]
\strobj{elem1}{Gert}
\end{scope}
\begin{scope}[xshift=3 cm,yshift=1.5 cm]
\strobj{elem2}{Octaaf}
\end{scope}
\begin{scope}[xshift=4 cm]
\tikzlinkedlistptr{3}{namenlist}
\pointleftbridge{ptrnamen}{offsetnamenlist}
\pointarraytopright{namenlist1}{elem0}
\pointarraybotright{namenlist2}{elem1}
\pointarraytopright{namenlist3}{elem2}
\end{scope}
}\newframe

\methodexec{arraylistadd1me}{6}{namen}{
\begin{scope}[xshift=1 cm,yshift=1 cm]
\strobj{elem0}{Samson}
\end{scope}
\begin{scope}[xshift=2 cm,yshift=-0.5 cm]
\strobj{elem1}{Gert}
\end{scope}
\begin{scope}[xshift=3 cm,yshift=1.5 cm]
\strobj{elem2}{Octaaf}
\end{scope}
\begin{scope}[xshift=4 cm,yshift=-1 cm]
\strobj{elem3}{Alberto}
\end{scope}
\begin{scope}[xshift=4 cm]
\tikzlinkedlistptr{2}{namenlist}
\end{scope}
\begin{scope}[xshift=5.5 cm,yshift=-0.75 cm]
\tikzlinkedlistptr{1}{nbmenlist}
\end{scope}
\begin{scope}[xshift=6 cm]
\tikzlinkedlistptr{1}{ncmenlist}
\llconcat{namenlist}{nbmenlist}
\llconcat{nbmenlist}{ncmenlist}
\pointleftbridge{ptrnamen}{offsetnamenlist}
\pointarraytopright{namenlist1}{elem0}
\pointarraybotright{namenlist2}{elem1}
\pointarraybotright{nbmenlist1}{elem3}
\pointarraytopright{ncmenlist1}{elem2}
\end{scope}
}
\end{animateinline}
\end{center}
\end{methodexample}
\end{frame}
\begin{frame}[fragile]{\dsarraylist{} in Java: \texttt{remove (int index)}}
\begin{methodexample}[\texttt{remove (int index)}]
\begin{center}
\begin{animateinline}{1}
\methodexec{arraylistremove0me}{6}{namen}{
\begin{scope}[xshift=3 cm,yshift=1 cm]
\strobj{elem0}{Samson}
\end{scope}
\begin{scope}[xshift=4 cm,yshift=-0.5 cm]
\strobj{elem1}{Gert}
\end{scope}
\begin{scope}[xshift=5 cm,yshift=1.5 cm]
\strobj{elem2}{Octaaf}
\end{scope}
\begin{scope}[xshift=6 cm,yshift=-1 cm]
\strobj{elem3}{Alberto}
\end{scope}
\begin{scope}[xshift=7 cm]
\tikzarraylistptr{4}{namenlist}
\pointleftbridge{ptrnamen}{offsetnamenlist}
\pointarraytopright{namenlist1}{elem0}
\pointarraybotright{namenlist2}{elem1}
\pointarraytopright{namenlist4}{elem2}
\pointarraybotright{namenlist3}{elem3}
\end{scope}
}\newframe

\methodexec{arraylistremove0me}{7}{namen,verw}{
\begin{scope}[xshift=3 cm,yshift=1 cm]
\strobj{elem0}{Samson}
\end{scope}
\begin{scope}[xshift=4 cm,yshift=-0.5 cm]
\strobj{elem1}{Gert}
\end{scope}
\begin{scope}[xshift=5 cm,yshift=1.5 cm]
\strobj{elem2}{Octaaf}
\end{scope}
\begin{scope}[xshift=6 cm,yshift=-1 cm]
\strobj{elem3}{Alberto}
\end{scope}
\begin{scope}[xshift=7 cm]
\tikzarraylistptr{3}{namenlist}
\pointleftbridge{ptrnamen}{offsetnamenlist}
\pointarraytopright{namenlist1}{elem0}
\pointarraytopright{namenlist3}{elem2}
\pointarraybotright{namenlist2}{elem3}
\pointlertbridge{ptrverw}{elem1}
\end{scope}
}\newframe

\methodexec{arraylistremove0me}{8}{namen,verw}{
\begin{scope}[xshift=3 cm,yshift=1 cm]
\strobj{elem0}{Samson}
\end{scope}
% \begin{scope}[xshift=4 cm,yshift=-0.5 cm]
% \strobj{elem1}{Gert}
% \end{scope}
\begin{scope}[xshift=5 cm,yshift=1.5 cm]
\strobj{elem2}{Octaaf}
\end{scope}
\begin{scope}[xshift=6 cm,yshift=-1 cm]
\strobj{elem3}{Alberto}
\end{scope}
\begin{scope}[xshift=7 cm]
\tikzarraylistptr{2}{namenlist}
\pointleftbridge{ptrnamen}{offsetnamenlist}
\pointarraytopright{namenlist1}{elem0}
\pointarraybotright{namenlist2}{elem3}
\pointlertbridge{ptrverw}{elem2}
\end{scope}
}
\end{animateinline}
\end{center}
\end{methodexample}
\end{frame}
\begin{frame}[fragile]{\dsarraylist{} in Java: \texttt{remove (Object o)}}
\begin{methodexample}[\texttt{remove (Object o)}]
\begin{center}
\begin{animateinline}{1}
\methodexec{arraylistremove1me}{8}{namen,verw}{
\begin{scope}[xshift=3 cm,yshift=1 cm]
\strobj{elem0}{Samson}
\end{scope}
% \begin{scope}[xshift=4 cm,yshift=-0.5 cm]
% \strobj{elem1}{Gert}
% \end{scope}
\begin{scope}[xshift=5 cm,yshift=1.5 cm]
\strobj{elem2}{Octaaf}
\end{scope}
\begin{scope}[xshift=6 cm,yshift=-1 cm]
\strobj{elem3}{Alberto}
\end{scope}
\begin{scope}[xshift=7 cm]
\tikzarraylistptr{2}{namenlist}
\pointleftbridge{ptrnamen}{offsetnamenlist}
\pointarraytopright{namenlist1}{elem0}
\pointarraybotright{namenlist2}{elem3}
\pointlertbridge{ptrverw}{elem2}
\end{scope}
}\newframe

\methodexec{arraylistremove1me}{9}{namen,verw,wasin/false}{
\begin{scope}[xshift=3 cm,yshift=1 cm]
\strobj{elem0}{Samson}
\end{scope}
% \begin{scope}[xshift=4 cm,yshift=-0.5 cm]
% \strobj{elem1}{Gert}
% \end{scope}
\begin{scope}[xshift=5 cm,yshift=1.5 cm]
\strobj{elem2}{Octaaf}
\end{scope}
\begin{scope}[xshift=6 cm,yshift=-1 cm]
\strobj{elem3}{Alberto}
\end{scope}
\begin{scope}[xshift=7 cm]
\tikzarraylistptr{2}{namenlist}
\pointleftbridge{ptrnamen}{offsetnamenlist}
\pointarraytopright{namenlist1}{elem0}
\pointarraybotright{namenlist2}{elem3}
\pointlertbridge{ptrverw}{elem2}
\end{scope}
}\newframe

\methodexec{arraylistremove1me}{10}{namen,verw,wasin/true}{
\begin{scope}[xshift=3 cm,yshift=1 cm]
\strobj{elem0}{Samson}
\end{scope}
% \begin{scope}[xshift=4 cm,yshift=-0.5 cm]
% \strobj{elem1}{Gert}
% \end{scope}
\begin{scope}[xshift=5 cm,yshift=1.5 cm]
\strobj{elem2}{Octaaf}
\end{scope}
% \begin{scope}[xshift=6 cm,yshift=-1 cm]
% \strobj{elem3}{Alberto}
% \end{scope}
\begin{scope}[xshift=7 cm]
\tikzarraylistptr{1}{namenlist}
\pointleftbridge{ptrnamen}{offsetnamenlist}
\pointarraytopright{namenlist1}{elem0}
%\pointarraybotright{namenlist2}{elem3}
\pointlertbridge{ptrverw}{elem2}
\end{scope}
}
\end{animateinline}
\end{center}
\end{methodexample}
\end{frame}