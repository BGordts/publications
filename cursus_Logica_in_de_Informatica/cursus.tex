\documentclass[titlepage,a4paper]{book}
\usepackage[dutch]{babel}
\usepackage{glossaries}
\usepackage{fullpage}
\usepackage{tikz}
\usepackage{index}
\usepackage{subfigure}
\usepackage{graphicx}
\usepackage{framed}
\usepackage{tabularx}
\usepackage{wrapfig}
\usepackage{listings}
\usepackage{multicol}
\usepackage{multirow}
\usepackage{stmaryrd}
\usepackage{amsthm}
\usepackage{amsfonts}
\usepackage{amsmath}
\usepackage{bbding}
\usepackage{array}
\usepackage{../SharedData/commusoftScripts,../SharedData/brackets}
\makeindex
\title{Logica in de Informatica}
\author{Willem M. A. Van Onsem, BSc.}
\date{mei 2014}

\newcommand{\prolog}{\textsc{ProLog}}
\newcommand{\problog}{\textsc{ProbLog}}
\newcommand{\idp}{\textsc{Idp}}
\newcommand{\ld}[1]{\textbf{\ensuremath{#1}}}
\newcommand{\ldm}[1]{\ld{\mbox{#1}}}
\newcommand{\func}[2]{\fun{\ldm{#1}}{#2}}
\newcommand{\pred}[2]{\fun{\ldm{#1}}{#2}}

\theoremstyle{definition}
\newtheorem{definition}{Definitie}[chapter]
\AtEndEnvironment{definition}{\null\hfill$\diamond$}
\theoremstyle{remark}
\newtheorem{example}{Voorbeeld}[chapter]

\begin{document}
\frontmatter
\begin{titlepage}
\maketitle
\end{titlepage}
\tableofcontents
\chapter*{Notities vooraf}
Op verschillende vlakken komt een student informatica in aanraking met logica. Allereerst bestaan de meeste programma's uit een grote hoeveelheid logische expressies. Dit noemt men vaak de ``business logic'' of ``application logic'' van het programma. Anderzijds bestaan er een grote hoeveelheid programma's die logica beschouwen en manipuleren. We denken hierbij bijvoorbeeld aan \prolog{} en \idp{}. Logica wordt dan ook vaak als de fundamentele bouwsteen gezien van de Artifici\"ele intelligentie.
\paragraph{}
Verschillende vormen van logica worden in verschillende opleidingsonderdelen onderwezen. Dit resulteert soms in een eerder chaotisch parcours met verschillende notaties, terminologie en semantiek. Het leidt er meestal toe dat studenten kostbare uren verspillen aan het opnieuw analyseren van een gelijkaardige inleiding. Deze ``cursus'' wenst hierop een antwoord te bieden. In plaats van \'e\'en opleidingsonderdeel uit te spitten worden alle opleidingsonderdelen in verband met logica samengevat in \'e\'en werk. Sommige cursussen worden slechts gedeeltelijk behandelt: dit omdat slechts in een deel van de cursus logica aan bod komt.
\paragraph{}
Het samenbrengen van verschillende opleidingsonderdelen brengt traditioneel enige overhead met zich mee: men deelt een boek in in hoofdstukken en sommige hoofdstukken zullen meer beschouwen dan het strikt noodzakelijk. Veel inspanningen werden ge\"investeerd in het opsplitsen van de kennis in een structuur die deze overhead zo laag mogelijk houdt. We hopen dat we in deze opzet geslaagd zijn en dat lezers die toch met deze overhead worden geconfronteerd de bijkomende kennis interessant zullen vinden.
\paragraph{Ondersteunde opleidingsonderdelen} Volgende opleidingsonderdelen worden volledig of gedeeltelijk ondersteund. Onderdelen die worden aangeduid met een ster ($\star$) worden gedeeltelijk ondersteund:
\begin{multicols}{2}
\begin{itemize}
 \item Logic as a Foundation of A.I. (H02A2)
 \item Logica en Argumentatieleer (C01B6, C01X5)
 \item Logica, met oefeningen (W0AA3)
 \item Logica, Verzamelingen en Relaties (S1WI1)
 \item Computational Logic (H02A9)
 \item Logica voor Informatici (G0T43)
 \item Wiskundige Logica (G0A98)
\end{itemize}
\end{multicols}
\paragraph{}
Het is niet altijd eenvoudig om de kennis homogeen uit te drukken. Een negatie in eerste orde logica heeft immers een andere betekenis dan een negatie in bijvoorbeeld logisch programmeren, in dit laatste zal men meestal werken met principes zoals \emph{negatie als eindige faling} of de \emph{gesloten wereld assumptie}. Inspanningen werden geleverd om de pure logica gescheiden te houden van logisch programmeren. We roepen echter de lezer op dit verschil indachtig te blijven.
\mainmatter
\chapter{Terminologie}
\chapterquote{Hij gaf de dingen andere namen. Zo noemde hij een stoel een krant. De tafel noemde hij een deurknop. Een briefje noemde hij een hand.}{Herman van Veen}
In het eerste hoofdstuk introduceren we enkele termen om de verschillende componenten in logica te kunnen benoemen. Zo hebben verschillende logische symbolen een eigen naam samen met de syntaxmatige structuren in logische programma's. Verder delen we de logica in propositie- en predicatenlogica.
\section{Symbolen}
Een logische taal bestaat doorgaans uit \termen{taal-onafhankelijke taalelementen} en \termen{taal-afhankelijke elementen}. De taal-onafhankelijke elementen zijn elementen die men in elke instantie van het gebruik van logica terug zal vinden. Deze symbolen hebben bijgevolg een vaste betekenis. Taal-afhankelijke elementen hangen af van het probleem zelf. Er wordt dan ook een betekenis ad-hoc aan toegekend.
\paragraph{}
Onder de taal-onafhankelijke elementen rekenen we:
\begin{enumerate}
 \item \termen{variabelen} zoals $x$, $y$, $Lx$,...
 \item \termen{propositionele constanten} zoals \xtrue{} en \xfalse{}. In het geval van driewaardige of vierwaardige logica kunnen hier nog meer constanten bijkomen.
 \item \termen{leestekens} zoals de \termen{haakjes} (\verb+(+, \verb+)+) en de \termen{komma} (\verb+,+)
 \item \termen{logische connectieven}: $\neg$, $\wedge$, $\vee$, $\Rightarrow$ en $\Leftrightarrow$.
 \item \termen{kwantoren}: de \termen{universele kwantor} ($\forall$), de \termen{existenti\"ele kwantor} ofwel \termen{existentiekwantor} ($\exists$) en in mindere mate de \termen{unieke existenti\"ele kwantor} ($\exists!$)
\end{enumerate}
\paragraph{}
Het taal-afhankelijke gedeelte bevat dan weer
\begin{enumerate}
 \item \termen{constanten} zoals \ldm{maarten}, \ldm{agnes}, ...
 \item \termen{functiesymbolen} ofwel \termen{functoren} zoals \ldm{hart/1}, \ldm{+/2}, ...
 \item \termen{predicaat-symbolen} zoals \ldm{mens/1}, \ldm{het\_regent/0} en \ldm{fan\_van/2}, ...
\end{enumerate}
Bij functors en predicaat-symbolen noteren we ook de \termen{ariteit}: het aantal argumenten die de functor of het predicaat vereist. De expressie \ldm{predicaat/ariteit} noemen we de \termen{signatuur} van het predicaat. De ariteit is een eigenschap van een predicaat of functor. \ldm{min/2} is dus een andere functor dan bijvoorbeeld \ldm{min/3}. Constanten zijn in wezen functoren met een ariteit gelijk aan 0. \ldm{maarten} is dus in principe gelijk aan \ldm{maarten/0}. Bij een functor of predicaat met ariteit gelijk aan 0 hoeven we geen haakjes te schrijven.
\paragraph{}
In deze cursus maken we de afspraak dat taal-afhankelijke elementen (constanten, functors, predicaten) in het vetjes zullen worden geschreven. Dit is echter louter om de expressies makkelijker te kunnen begrijpen en is zeker niet verplicht. Voorbeelden hiervan zijn \ldm{jan}, \ldm{hart/1}, \ldm{regen/0}. Taal-onafhankelijke elementen zoals variabelen schrijven we dan weer cursief zoals $x$, $y$ en $Lx$.
\section{Grammatica}
In de vorige sectie hebben we de verschillende symbolen die we in de taal kunnen gebruiken gedefinieerd. We kunnen deze symbolen echter niet in om het even welke volgorde gebruiken: er zijn regels hoe een logische formule er moet uitzien. Om deze regels te specificeren zullen we verdere terminologie invoeren.
\subsection{Term}
We beginnen met een \termen{term}. Een term defini\"eren we inductief:
\begin{definition}[Term]
Volgende expressies zijn termen:
\begin{enumerate}
 \item Een constante
 \item Een variabele
 \item \label{itm:compoundtermf} $\fun{\ldm{f}}{t_1,t_2,\ldots,t_n}$ met \ldm{f/$n$} een functor met ariteit $n$ en $t_1,t_2,\ldots,t_n$ allemaal termen.
\end{enumerate}
Termen van de vorm zoals in \ref{itm:compoundtermf} noemen we ook \termen{samengestelde termen} ofwel \termen{compound terms}.
\end{definition}
\begin{example}[Termen van hartelijke mensen]
Stel constanten $\accl{\ldm{maarten},\ldm{agnes},\ldm{ruben}}$, variabelen $\accl{x,y,z}$ en functoren $\accl{\ldm{hart/1}, \ldm{vader/1}, \ldm{gelijkaardig/2}}$ dan zijn volgende expressies termen:
\begin{multicols}{2}
\begin{itemize}
 \item $y$
 \item $\ldm{maarten}$
 \item $\func{hart}{\ldm{maarten}}$
 \item $\func{hart}{\func{vader}{\ldm{maarten}}}$
 \item $\func{hart}{\func{vader}{\func{vader}{x}}}$
 \item $\func{vader}{\func{hart}{\func{vader}{x}}}$
 \item $\func{gelijkaardig}{x,y}$
 \item $\func{gelijkaardig}{x,\ldm{maarten}}$
\end{itemize}
\end{multicols}
\end{example}
Men dient op te merken dat termen niet noodzakelijk semantisch zinvol zijn. Zo is \func{vader}{\func{hart}{x}} een term, terwijl semantisch gezien deze expressie niet zinvol is: een hart heeft namelijk geen vader.
\subsection{Atoom}
Naast termen bestaan er ook atomen. Een \termen{atoom} is als volgt gedefinieerd:
\begin{definition}[Atoom]
Een expressie \pred{p}{t_1,t_2,\ldots,t_m} is een atoom indien \ldm{p/$m$} een predicaat is met ariteit $m$ en $t_1,t_2,\ldots,t_m$ allemaal termen zijn.
\end{definition}
Syntactisch gezien is het onderscheid tussen een atoom en een term quasi niet te maken: men dient kennis te hebben van de taal-afhankelijke aspecten namelijk wat een predicaat is en wat een functor alvorens men het verschil kan zien. Verder zijn er ook allerhande syntactische constructies mogelijk die eigenlijk volgens de regels van atomen en termen niet mogelijk zijn: denk bijvoorbeeld aan het gebruik van een atoom in een term. Dit is niet toegelaten maar niet louter aan de hand van syntax te detecteren.
\subsection{Formule}
Atomen vormen op zich weer bouwstenen van een \termen{goed gevormde formule} ofwel \termen{well formed formula} die wiskundig gezien zinvol is:
\begin{definition}[Goed gevormde formule]
Een \emph{goed gevormde formule} wordt inductief als volgt gedefinieerd:
\begin{itemize}
 \item Een atoom is een goed gevormde formule.
 \item Stel dat $\phi$ en $\psi$ goed gevormde formules zijn, dan zijn volgende formules ook goed gevormde formules:
 \begin{multicols}{5}
  \begin{itemize}
   \item $\brak{\phi\vee\psi}$
   \item $\brak{\phi\wedge\psi}$
   \item $\brak{\neg\phi}$
   \item $\brak{\phi\Rightarrow\psi}$
   \item $\brak{\phi\Leftrightarrow\psi}$
  \end{itemize}
 \end{multicols}
\end{itemize}
\end{definition}
Herinner echter dat we bijvoorbeeld met behulp van termen al expressies kunnen vormen die wiskundig correct zijn, maar semantisch gezien niet zinnig zijn.
\paragraph{Haakjes en ambigu\"iteit}
De formule voor goed gevormde formules introduceert veel haakjes. Dit leidt tot bijvoorbeeld volgende expressie:
\begin{equation}
\brak{\brak{\brak{\brak{\ldm{p}\wedge\brak{\neg\ldm{r}}}\wedge\ldm{r}}\vee\ldm{p}}\vee\brak{\brak{\brak{\brak{\ldm{p}\wedge\brak{\neg\ldm{r}}}\wedge\ldm{r}}\wedge\ldm{p}}}}
\end{equation}
Voor bijvoorbeeld een compiler zijn haakjes een grote steun, mensen zijn echter in staat om snel de structuur van een expressie in te zien, haakjes zijn dan ook ruis die meestal tot verwarring leiden.
\subsection{In logische programma's}
Logische programma's zoals \prolog{} en \problog{} programma's kenmerken zich door een vaste vorm: expressies worden in een vast formaat geschreven. Alle mogelijke vormen van logisch programmeren overlopen is geen sinecure. In deze sectie zullen we de terminologie rond logische programma's echter zo algemeen mogelijk bespreken.
\subsection{Samenvatting}
\section{Soorten logica}
\section{Vrije variabelen en de scope van kwantoren}
\chapter{Epistemische logica}
\chapterquote{Nu mobiele telefoons en het Internet het epistemische selectieve landschap hebben veroverd, moet elke religie zijn verdedigingen laten mee\"evolueren of uitsterven.}{Daniel Dennett}
\chapter{Compiler-implementatie}
\chapterquote{Iedere serieuze software ingenieur zal zeggen dat een compiler en interpreter uitwisselbaar zijn.}{Tim Berners-Lee}
Soms wenst men een programma te ontwikkelen dat als invoer \'e\'en of meer logische expressies neemt. Hierop worden dan vervolgens operaties uitgevoerd zoals de $T_P$ operator. In dit hoofdstuk zullen we enkele aspecten bespreken rond de implementatie van een compiler of interpreter dat op een of andere wijze te maken krijgt met logica. Dit hoofdstuk is niet exhaustief: voor de implementatie van bijvoorbeeld een volledig operationeel \prolog{} systeem is een meer gesofisticeerde implementatie vereist.
\section{Syntax: Lexer en Parser}
Niet alle programma's die met logica te maken hebben, zullen noodzakelijkerwijs te maken krijgen met invoer van logische expressies. Toch is het een vaak voorkomend fenomeen. In dat geval dient men eerst de invoer te analyseren, iets wat men bewerkstelligt met een \termen{lexer} en \termen{parser}.
\subsection{Lexer}
Een lexer is een component die een stroom van karakters omzet in een stroom van tokens. Een \termen{token} is een reeks karakters die eerder op zichzelf staat. In een programmeertaal is dit typisch een codewoord zoals \verb+if+ of \verb+while+, operatoren zoals \verb+=+ of \verb+!=+ enzovoort.
\paragraph{}
Alle programmeer en specificatie-talen lenen zich doorgaans goed tot het omzetten van reeksen karakters in tokens. Meestal stelt men dat bij een dergelijke taal, de strings die onder een token vallen moeten kunnen worden gespecificeerd aan de hand van een reguliere expressie.
\paragraph{}
We kunnen in dit referentiewerk natuurlijk niet de concrete grammatica voor een logisch programma voorstellen: dit hangt af van het te ontwikkelen programma. Indien logische specificatie bijvoorbeeld slechts een onderdeel vormt rond een grotere taal is het mogelijk dat er beperkingen worden gezet op de vorm van een predicaat. We zullen echter illustratief de verschillende tokens bespreken die doorgaans een rol spelen bij het interpreteren van een logische expressie.
\subsection{Parser}
\section{Semantiek: Symbooltabellen en Typesystemen}
\section{Processing: logische inferentie}
\backmatter
\bibliographystyle{alpha}
\nocite{*}
\bibliography{biblio}
\printindex
\end{document}