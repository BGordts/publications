\chapter*{Introduction}

\begin{defi}[Definition, Definiendum, Definiens]
A \term{definition} is a statement that explains the meaning of a term (a word, phrase, or other set of symbols). The term to be defined is the \term{definiendum}. The term may have many different senses and multiple meanings. For each meaning, a \term{definiens} is a cluster of words that defines that term (and clarifies the speaker's intention).
\cite{wiki:definition}
\end{defi}

This publication aims to compile a large number of common definitions and theorems in a single reference guide.

\section*{Advantage of a List of Definitions}

We expect the number of definitions in computer science to be around 100'000. The major problem with such definitions is that papers become quite inaccessible due to the fact that they use concepts who are defined elsewhere. This publication aims to tackle this problem by providing a collection of these definitions. Possible other advantages are a reduction of definition collisions: two meanings for the same term and preventing people from reinventing the wheel over and over again. A final advantage is that knowledge of a large amount of concepts enriches ones global understanding of computer science.

\section*{How is this Document Compiled?}

The definitions, theorems and lemma's are extracted from a massive amount of papers from conferences, proceedings, journals, etc. Since no individual can read this amount of papers in a lifetime, some intelligent scripts look for patterns who look interesting together with extracting the actual content. The extracted content is then reviewed by the author and after a minimal amount of modifications added to the right chapter.
\paragraph{}
The current processing rate is five definitions per day, but we hope by improving the data mining scripts, we will improve this number. At the moment our scripts already use optical character recognition together with error correction. We are currently working on technology that can generate \LaTeX{} code for a specific formula.
\paragraph{}
People who want to help to improve the mining scripts, provide papers or do some post-processing can propose a pull-request on \url{http://goo.gl/jZeAG} or contact the author at \href{mailto:vanonsem.willem@gmail.com}{\nolinkurl{vanonsem.willem@gmail.com}}.