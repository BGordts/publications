\chapter{Formal Languages and Automata Theory}

\begin{defi}
A \term{nondeterministic finite automaton} consists of a set $S$ of states. One of these states, $s0\in S$, is called the \term{starting state} of the automaton and a subset $F\subseteq S$ of the states are \term{accepting states}. Additionally, we have a set $T$ of \term{transitions}. Each transition $t$ connects a pair of states $s_1$ and $s_2$ and is labeled with a symbol, which is either a character $c$ from the alphabet $\Sigma$, or the symbol $\varepsilon$, which indicates an \term{epsilon-transition}. A transition from state $s$ to state $t$ on the symbol $c$ is written as $s^ct$.\cite{mogensen2009basics}
\end{defi}

\begin{defi}
Given a set $M$ of nondeterministic finite automaton states, we define the \term{$\varepsilon$-closure(M)} to be the least (in terms of the subset relation) solution to the set equation:
\begin{equation}
\fun{\varepsilon\mbox{-closure}}{M} = M\cup \left\{t|s\in \fun{\varepsilon\mbox{-closure}}{M}\mbox{ and }s^\varepsilon t\in T \right\}
\end{equation}
Where $T$ is the set of transitions in the non deterministic finite automaton.\cite{mogensen2009basics}
\end{defi}