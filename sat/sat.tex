\documentclass{article}
\title{The SAT problem:\\An in depth analysis}
\author{Willem Van Onsem}
\begin{document}
\maketitle
\tableofcontents
\newtheorem{problem}{Problem}
\newtheorem{definition}{Definition}
\section{Number of Problems}
An interesting question is how many unique SAT-problems we can generate given the number of variables $n_x$. The solution to this question is quite straightforward. Since their are $n$ variables, we can give a value to each configuration. Therefore the number of SAT-problems is:
\begin{equation}
n\left(\mbox{problems with $n$ variables}\right)=\displaystyle\frac{2^{2^n}}{2^n\cdot n!}
\end{equation}
\paragraph{$k$-expressions}
A more interesting question is therefore how expressive SAT is: how many problems we can represent with $k$ expressions. A problem with this question, is that we can make expressions as long as we want and therefore as long we can write a single expression, the number of SAT-problems is again $2^{2^n}/2^n\cdot n!$. If we restrict our notation to $3$-SAT however, the problem is more complex.
\subsection{With 3-SAT expressiveness}
We can formalize the problem as:
\begin{problem}[Expressiveness of 3-SAT]
Given $k$ expressions in the 3-SAT problem, what are the numbers of $3$-SAT problems we can specify?
\end{problem}
A first remark is that we can only use $3k$ variables. One could of course argue that the number of expressions, doesn't state that the number of variables are restricted. However defining variables who are not used is quite useless.
\subsubsection{The expressiveness of a single expression}
A first problem we have to tackle is the expressiveness of a single expression.
\subsection{Number of Local Maxima}
\end{document}