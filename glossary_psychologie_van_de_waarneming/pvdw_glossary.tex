\documentclass[a4paper]{article}
\usepackage[utf8x]{inputenc}
\title{Glossary ``Perception and Sensation''\\Chapters 1,3-8}
\author{Willem Van Onsem}
\date{June 2012}
\newcounter{tmpC}
\newcommand{\glos}[2]{
  \newglossaryentry{\arabic{tmpC}}{
    name={#1},
    description={#2}
  }
  \addtocounter{tmpC}{1}
}
\usepackage[toc,nonumberlist]{glossaries}
\usepackage{fullpage}
\usepackage{draftwatermark}
\SetWatermarkScale{6}
\makeglossaries
\begin{document}
\maketitle
\glos{adaption}{A reduction in response caused by prior or continuing simulation.}
\glos{sensory transducer}{A receptor that converts physical energy from the environment into neural activity.}
\glos{nativism}{The idea that the mind produces ideas that are not derived from external sources and have abilities that are innate and not learned.}
\glos{dualism}{The idea that both mind and body exist.}
\glos{monism}{The idea that mind and matter are formed from, or reducible to, a single ultimate substance or principle of being.}
\glos{materialism}{The idea that physical matter is the only reality, and everything including the mind can be explained in terms of matter and physical phenomena. A type of monism.}
\glos{mentalism}{The idea that the mind is the true reality and the objects exist only as aspects of the mind's awareness. is a type of monoism.}
\glos{Mind-body dualism}{Originated by Ren\'e Descartes, the idea positing the existence of two distinct principles of being in the universe: spirit/soul and matter/body.}
\glos{Empiricism}{The idea that experience from the senses is the only source of knowledge.}
\glos{Panpsychism}{The idea that all mater has consciousness.}
\glos{Psychophysics}{The science of defining quantitative relationships between physical and psychological (subjective) events.}
\glos{Two-point touch threshold}{The minimum distance at which two stimuli (e.g. two simultaneous touches) are just perceptible as separate.}
\glos{Just noticeable difference (JND) (or difference threshold)}{The smallest detectable difference between two stimuli, or the minimum change in stimulus that enables it to be correctly judged as different fro a reference stimulus.}
\glos{Weber fraction}{The constant of proportionality in Weber's law.}
\glos{Weber's law}{The principle that the just noticeable difference (JND) is a constant fraction of comparison stimulus.}
\glos{Fechner's law}{The principle describing the relationship between stimulus and resulting sensation such that the magnitude of subjective sensation increases proportionally to the logarithm of the stimulus intensity.}
\glos{Absolute threshold}{The minimum amount of stimuli necessary for a person to detect a stimulus 50\% of the time.}
\glos{Method of constant stimuli}{A psychophysical method in which many stimuli, ranging from rarely to almost always perceivable (or rarely to almost perceivable different from a reference stimulus), are presented one at a time. Participants respond to each presentation: ``yes/no'', ``same/different'', and so on.}
\glos{Method of limits}{A psychophysical method in which the particular dimension of a stimulus, or the difference between two stimuli, is varied incrementally until the participant responds differently.}
\glos{Method of adjustment}{The method of limits for which the subject controls the change of the stimulus.}
\glos{Receiver Operating Characteristics (ROC) curve}{In studies of signal detection, the graphical plot of the hit rate as a function of the false alarm rate. If these are the same, points fall on the diagonal, indicating that the observer cannot tell the difference between the presence and the absence of the signal. As the observers sensitivity increases the curve bows upward toward the upper left corner. That point represents a perfect ability to distinguish signal from noise (100\% hits, 0\% false alarms).}
\glos{Signal detection theory}{A psychophysical theory that quantifies the response of an observer to the presentation of the signal in the presence of noise. Measure attained from a series of presentations are sensitivity (d') and criterion of the observer.}
\glos{Magnitude estimation}{A psychophysical method in which the participant assigns values according to perceived magnitudes of the stimuli.}
\glos{Cross-modality matching}{The ability to match the intensities of sensation that come from different sensory modalities. This ability enables insight into sensory differences. For example, a listener might adjust the brightness of a light until it matches the loudness of a tone.}
\glos{Steven's power law}{A principle describing the relationship between stimulus and resulting sensation, such that the magnitude of subjective sensation is proportional to the stimulus magnitude raised to an exponent.}
\glos{Doctrine of the specific nerve energies}{A doctrine formulated by Johannes M\"uller stating that the nature of a sensation depends on which sensory fibers are simulated, not on how the fibers are simulated.}
\glos{Cranial nerves}{Twelve pair of nerves (one for each side of the body) that originate in the brain stem and reach sense organs and muscles through openings in the skull.}
\glos{Olfactory nerves}{The first pair of cranial nerves, which conduct impulses from the mucous membranes of the nose to the olfactory bulb.}
\glos{Optic nerves}{The second pair of cranial nerves which arise from retina and carry visual information to the thalamus and other parts of the brain.}
\glos{Auditory nerves}{The eight pair of cranial nerves which connect the inner ear with the brain. Transmitting impulses concerned with hearing and balance. The nerve is composed of the cochlear nerve and the vestibular nerve and therefore is sometimes referred to as the vestibulocochlear nerve.}
\glos{Oculomotor nerves}{The third pair of cranial nerves which innervate all the excentric muscles of the eye except the lateral rectums and the superior oblique muscles, and which innervate the elevator muscle of the upper eyelid, the ciliary muscle, and the sphincter muscle of the pupil.}
\glos{Trochlear nerves}{The fourth pair of cranial nerves which innervate the superior oblique muscles of the eyeballs.}
\glos{Abduncens nerves}{The sixth pair of cranial nerves, which innervate the lateral rectus muscle on each eye.}
\glos{Polysensory}{Blending multiple sensory systems.}
\glos{Vitalism}{The idea that ``vital forces'' are active within living organisms, and these forces cannot be explained by physical processes of matter more generally.}
\glos{Synapse}{The junction between neurons that permits information transfer.}
\glos{Neurotransmitter}{A chemical substance used in neuronal communication at synapses.}
\glos{Wave}{An oscillation that travels through a medium by transferring energy from one particle or point to another without causing any permanent displacement of the medium.}
\glos{Photon}{A quantum of visible light or other form of electromagnetic radiation demonstrating both particle and wave properties.}
\glos{Absorb}{To take up light, noise, or energy and not transmit it at all.}
\glos{Scatter}{To disperse light in an irregular fashion.}
\glos{Reflect}{To redirect something that strikes a surface - especially light, sound, or heat -  usally back towards it's point of origin.}
\glos{Transmit}{To convey something (e.g., light) from one place or thing to another.}
\glos{Refract}{1. To alter the course of a wave of energy that passes into something from another medium, as water does to light entering from the air. 2. To mesure the degree of refraction in a lens or eye.}
\glos{Image}{A picture or likeness.}
\glos{Cornea}{The transparent ``window'' into the eyeball.}
\glos{Transparent}{Allowing light to pass through with no interruption so that objects on the other side can be clearly seen.}
\glos{Aqueous humor}{The watery fluid in the anterior chamber of the eye.}
\glos{Crystalline lens}{The lens inside the eye that enables changing focus.}
\glos{Pupil}{The dark circular opening at the center of the iris in the eye, where light enters the eye.}
\glos{Iris}{The colored part of the eye consisting of a muscular diaphragm surrounding the pupil and regulating the light entering the eye by expanding and contracting the pupil.}
\glos{Vitreous humor}{The transparent fluid that fills the vitreous chamber in the posterior part of the eye.}
\glos{Retina}{A light-sensitive membrane in the back of the eye that contains rods and cones, which receive an image from the lens and sens it to the brain through the optic nerve.}
\glos{Accommodation}{The process by which the eye changes it's focus (in which the lens gets fatter as gaze is directed towards nearer objects).}
\glos{Presbyopia}{Literally ``old sight''. The loss of near vision because of insufficient accommodation.}
\glos{Cataract}{Opacity of the crystalline lens.}
\glos{Emmetropia}{The condition in which there is no refractive error. Because the refractive power of the eye is perfect matched to the length of the eyeball.}
\glos{Myopia}{A common condition in which light entering the eye is focused in front of the retina and distant objects cannot be seen sharply.}
\glos{Astigmatism}{A visual defect caused by the unequal curving of one or more of the refractive surfaces of the eye, usually the cornea.}
\glos{Transduced}{Referring to the conversion from one form energy (e.g., light) to another (e.g., electricity).}
\glos{Fundus}{The back layer of the retina - what the eye doctor sees through an ophthalmoscope.}
\glos{Photoreceptors}{Light-sensitive receptors in the retina.}
\glos{Rods}{Photoreceptors specialized for night vision.}
\glos{Cones}{Photoreceptors specialized for daylight vision, fine visual acuity and color.}
\glos{Duplex}{In reference to the retina, consisting of two parts: the rods and the cones, which operate under different conditions.}
\glos{Outer segment}{The part of a photoreceptor that contains photopigment molecules.}
\glos{Inner segment}{The part of a photoreceptor that lies between the outer segment and the cell nucleus.}
\glos{Synaptic terminal}{The location where axons terminate at the synapse for transmission of the information by release of a chemical transmitter.}
\glos{Chromophore}{The light-catching part of at the visual pigments in the retina.}
\glos{Rhodopsin}{The visual pigment found in rods.}
\glos{Photoactivation}{Activation by light.}
\glos{Hyperpolarization}{A increase in membrane potential where the inner membrane surface becomes more negatie than the outer membrane surface.}
\glos{Graded ptential}{A electrical potential that can vary continuously in amplitude.}
\glos{Eccentricity}{The distance between the retina image and the fovea.}
\glos{Horizontal cells}{Specialized retinal cells that contact both photoreceptor and bipolar cells.}
\glos{Lateral inhibition}{Antagonistic neural interaction between adjacent regions of the retina.}
\glos{Amacrine cells}{Retinal cells found in the inner synaptic layer that make synaptic contact with bipolar cells, ganglion cells, and one other.}
\glos{Bipolar cells}{Retinal cells that synapse with either rods or cones (not both) and with horizontal cells and then pass signals on the ganglion cells.}
\glos{Diffuse bipolar cells}{Bipolar retinal cells whose processes are spread out to receive input from multiple cones.}
\glos{Sensitivity}{1. The ability to perceive via the sense organs. 2. Extreme responsiveness to radiation, especially to light of a specific wavelength. 3. The ability to respond to transmitted signals.}
\glos{Visual acuity}{A measure of the finest detail that can be resolved by the eyes.}
\glos{Midget bipolar cells}{Small cone bipolar cells in the central retina that receive input from a single cone.}
\glos{ON bipolar cells}{Bipolar cells that respond to an increase in light captured by the cones.}
\glos{OFF bipolar cells}{Bipolar cells that respond to an decrease in light captured by the cones.}
\glos{Ganglion cells}{Retinal cells that receive visual information from photoreceptors via two intermediate neuron types (bipolar cells and amacrine cells) and transmits information to the brain and midbrain.}
\glos{P ganglion cells}{Small ganglion cells that receive excitatory input from single midget bipolar cells in the central retina and feed the parvocellular layer of the lateral geniculate nucleus.}
\glos{M ganglion cells}{Ganglion cells resembling little umbrellas that receive excitatory input from diffuse bipolar cells and feed the magnocellular layer of the lateral geniculate nucleus.}
\glos{Receptive field}{The region on the retina in which visual stimuli influence a neuron's firing rate.}
\glos{ON-center cell}{A cell that depolarizes in response to an increase in light intensity in it's receptive field center.}
\glos{OFF-center cell}{A cell that depolarizes in response to an decrease in light intensity in it's receptive field center.}
\glos{filter}{An acoustic, electrical, electronic, or optical device, instrument, computer program, or neuron that allows pasage of some frequencies or digital elements and blocks the passage of others.}
\glos{Contrast}{The difference in luminance between an object and the background, or between lighter and darker parts of the same object.}
\glos{Retinis pigmentosa (RP)}{A progressive degeneration of the retina that affects night vision and peripheral vision. It runs commonly in families and can be caused by defects in a number of different genes that have recently been identified.}
\glos{Middle (midlevel) vision}{A loosely defined stage of visual processing that comes after basic features have been extracted from the image (early vision) and before object recognition and scene understanding (high-level vision).}
\glos{Illusory Contour}{A contour that is perceived, even though nothing changes from one side of te contour to the other in the image.}
\glos{Structuralism}{A school of thought believing that complex objects or perceptions could be understood by analysis of the components.}
\glos{Gestalt}{In German literally ``form''. In perception, the name of a school of thought stressing that the perceptual whole could be greater than the apperent sum of the parts.}
\glos{Gestalt grouping rules}{A set of rules describing which elements in an image will appear to group together. The original list was assembled by members of the Gestalt school of thought.}
\glos{Good continuation}{A Gestalt grouping rule stating that two elements will tend to group together if they seem to lie on the same contour.}
\glos{Texture segmentation}{Carving an image into regions of common texture properties.}
\glos{Similarity}{A Gestalt grouping rule stating that the tendency of features to group together will increase as the similarity between them increases.}
\glos{Proximity}{A Gestalt grouping rule stating that the tendency of features to group together will increase as the distance between them decreases.}
\glos{Parallelism}{A rule for figure-ground assignment stating that parallel contours are likely to belong to the same figure.}
\glos{Symmetry}{A rule for figure-ground assignment stating that the symmetrical regions are more likely to be seen as figure.}
\glos{Common region}{A gestalt grouping rule stating that two features will tend to group together if they appear to be part of the same larger region.}
\glos{Connectedness}{A Gestalt grouping rule stating that two items will tend to group together if they are connected.}
\glos{Ambiguous figure}{A visual stimulus that give rise to two or more interpretations of its identity or structure.}
\glos{Necker cube}{An outline that is perceptually bi-stable. Unlike the situation with most stimuli, two interpretations continually battle for perceptual dominance.}
\glos{Accidental viewpoint}{A viewing position that produces some regularity in the visual image that is not present in the world (e.g., the sides of two independent objects lining up perfectly).}
\glos{Figure-ground assignment}{The process of determining that some regions of an image belong to a foreground object (figure) and other regions are part of the background (ground).}
\glos{Surroundedness}{A rule for figure-ground assignment stating that if one region is entirely surrounded by another, it is likely that the surrounded region is the figure.}
\glos{Relatability}{The degree of which two line segments appear to be part of the same contour.}
\glos{Heuristic}{A mental shortcut.}
\glos{Nonaccidental Feature}{A feature of an object that is not dependent on the exact (or accidental) viewing position of the observer.}
\glos{Global superiority effect}{The finding in various experiments that the properties that whole objects take precedence over the properties of parts of the object.}
\glos{Naive template theory}{The proposal that the visual system recognizes objects by matching the neural representation of the image with a stored representation of the same ``shape'' in the brain.}
\glos{Structural description}{A description of an object in terms of the nature of its consituent parts and the relationships between those parts.}
\glos{Geons}{In Biederman's ``Recognition by components'' model, the ``geometric ions'' out which perceptual objects are build.}
\glos{``Recognition by components'' model}{Biederman's model of object recognition, which holds that objects are recognized by the identities and relationships of their component parts.}
\glos{Viewpoint invariance}{1. A property of an object that does not change when observer viewpoint changes. 2. A class of theories of object recognition that proposes representations of objects that do not change when viewpoint changes.}
\glos{Entry-level category}{For an object, the label to come to mind most quickly when we identify the object (e.g., bird). The object might be more specially named: that's the ``subordinate level'' (e.g., eagle). The object might be more generally named; that's the ``superordinate level'' (e.g., animal).}
\glos{Prosopagnosia}{An inability to recognize faces.}
\glos{Double dissocation}{The phenomenon which one of two functions such as being and sight can be damaged without harm to the other, and vice versa.}
\glos{Extrastriate cortex}{The region of cortex bordering the primary visual cortex and containing multiple areas involved in visual processing.}
\glos{Lesion}{In neuropsychology: 1. A region of damaged brain. 2. To destroy a section of the brain.}
\glos{Agnosia}{A failure to recognize objects in spite of the ability to see them. Agnosia is typically due to brain damage.}
\glos{Inferotemporal (IT) cortex}{Part of the cerebral cortex in the low portion of the temporal lobe. Important in object recognition.}
\glos{Homologous regions}{Brain regions that appear to have the same function in different species.}
\glos{Feed-forward process}{A process that caries out computation (e.g., object recognition) one nneural stap after another, without need for feedback from a later stage o an earlier stage.}
\glos{Problem of Univariance}{The fact that an infinite set of different wavelength-intensity combinations can be elicit exactly the same response from a single type of photonreceptor. One photoreceptor type cannot make color discriminations based on wavelength.}
\glos{Scotopic}{Light intensities that are bright enough to stimulate the rod receptors but to dim to stimulate the cone receptors. Compare sctopic and mescopic.}
\glos{Photopic}{Light intensities that are bright enough to stimulate the cone receptors and bright enough to ``saturate'' the rod receptors (i.e,. drive them to their maximum response) Compare scotopic en mescopic.}
\glos{S-cone}{A cone that is preferentially sensitive to short wavelengths; colloquially (but not entirely accurately) known as a ``blue cone''.}
\glos{M-cone}{A cone that is preferentially sensitive to middle wavelengths; colloquially (but not entirely accurately) known as a ``green cone''.}
\glos{L-cone}{A cone that is preferentially sensitive to long wavelengths; colloquially (but not entirely accurately) known as a ``red cone''.}
\glos{Trichromatic theory of color vision (or trichometry)}{The theory that color of any light is defined in our visual system by the relationships of three numbers, the outputs of three receptor types now known to be the three cones. Also known as the Young-Helmholtz theory.}
\glos{Metamers}{Different mixtures of wavelengths that look identical. More generally any pairs of stimuli that are perceived as indentical in spite of physical differences.}
\glos{Addition color mixture}{A mixture of lights. If light A and light B are both reflected from a surface to the eye, in the perception of color the effects of those two lights are add together.}
\glos{Subtractive color mixture}{A mixture of pigments. If pigments A and B mix, some of the light shining on the surface will be subtracted by A, and some by B. Only the remainder contributes to the perception of color.}
\glos{Color space}{The Three-dimensional space, established because color perception is based on the output of three cone types that describe the set of all colors.}
\glos{Hue}{The chromatic (colorful) aspect of color (red, blue, green, yellow, and so on).}
\glos{Saturation}{The chromatic strength of a hue. White has zero saturation, pink is more saturated, and red is fully saturated.}
\glos{Brightness}{The distance from black (zero brightness) in color space.}
\glos{Lateral Geniculate Nucleaus (LGN)}{A structure in the thalamus, part of the middle brain, that receives input from the retinal ganglion cells and has input and output connections to the visual cortex.}
\glos{Color-opponent cell}{A neuron whose output is based on a difference between sets of cones.}
\glos{Opponent color theory}{The theory that perception of color is based on the output of three mechanisms, each of them based on an opponency between two colors: red-green, blue-yellow and black-white.}
\glos{Unique blue}{A blue that has no red or green tint.}
\glos{Unique hue}{Any of four colors that can be described with only a single color term: red, yellow, green, blue. Other colors (e.g., purple or orange) can be described as compounds (reddish blue, reddish yellow).}
\glos{Afterimage}{A visual image seen after the stimulus has been removed.}
\glos{Adaptive stimulus}{A stimulus whose removal produces a change in visual perception or sensitivity.}
\glos{Negative afterimage}{An afterimage whose polarity is the opposite of the original stimulus. Light stimuli produce dark negative afterimages. Colors are complementary; for example red produces green; yellow produces blue.}
\glos{Neutral point}{The point at which an opponent color mechanism is generating no signal. If red-green and blue-yellow machanisms are at their neutral points, a stimulus will appear archromatic (The black-white process has no neutral point.)}
\glos{Achromatopsia}{An inability to perceive colors that is caused by damage to the central nervous system.}
\glos{Deuteranope}{An individual who suffers from color blindness that is due to the absence of M-cones.}
\glos{Proteranope}{An individual who suffers from color blindness that is due to the absence of L-cones.}
\glos{Tritanope}{An individual who suffers from color blindness that is due to the absence of S-cones.}
\glos{Color-anomalous}{A better term for what is usually called ``color-blind''. ``Most color-blind'' individuals can still make discriminations based on wavelength. Those discrimations are different from normal -- that is anomalous.}
\glos{Cone-monochromat}{An individual with only one cone type. Cone monochromats are truly color-blind.}
\glos{Rod-monochromat}{An individual with no cones of any type. In addition to being truly color-blind, rod monochromats are badly visually impaired in bright light.}
\glos{Agnosia}{A failure to recognize objects in spite of the ability to see them. Agnosia is typically due to brain damage.}
\glos{Anomia}{An inability to name objects in spite of the ability to see and recognize them (as shown by usage). Anomia is typically due to brain damage.}
\glos{Cultural Relativism}{In sensation and perception, the idea that basic perceptual experiences (e.g., color perception) may be determined in part by the cultural environment.}
\glos{Unrelated color}{A color that can be experiences in isolation.}
\glos{Related color}{A color, such as brown or gray, that is seen only in relation to colors. A ``gray'' patch in complete darkness appears white.}
\glos{Illuminant}{The light that illuminates a surface.}
\glos{Spectral reflectance function}{The function relating the wavelength of light to the percentage of the wavelength that is reflected from a surface.}
\glos{Spectral power distribution}{The physical energy in alight as function of wavelength.}
\glos{Color contancy}{The tendency of surface to appear the same color under a fairly wide range of illuminants.}
\glos{Reflectance}{The percentage of light hitting the surface that is reflected and not absorbed into the surface. Typically reflectance is given as a function of wavelength.}
\glos{Realism}{A philosophical position arguing that there is a real world to sense.}
\glos{Positivism}{A philosophical position arguing that all we really have to go on is the evidence of the senses, so the world might be nothing more than elaborate hallucination.}
\glos{Euclidean}{Referring to the geometry of the world, so named in honor of Euclid, the ancient Greek geometer of the third century BCE. In Euclidean geometry parallel lines remain parallel as they are extended in space, objects maintain the same size and shape as they move around in space, the internal angles of a triangle always add up to 180 degrees, and so forth.}
\glos{Binocular summation}{The combination (or summation) of signals from each eye in ways that make performance on many tasks better with both eyes than with either eye alone.}
\glos{Binocular disparity}{The differences between the two retinal image of the same scene. Disparity is the basis of stereopsis, a vivid perception of the three-dimensionality of the world that is not available with monocular vision.}
\glos{Monocular}{With one eye.}
\glos{Stereopsis}{The ability to use binocular disparity as a cue to depth.}
\glos{Depth cues}{Information about the third dimension (dept) of visual space. Depth cues may be monocular or binocular.}
\glos{Monocular depth cue}{A depth cue that is available even when the world is viewed with one eye alone.}
\glos{Binocular depth cue}{A depth cue that relies on information from both eyes. Stereopsis is the basic example in humans, but convergence and the ability of an object to see more of an object than one eye sees are also binocular depth cues.}
\glos{Occlusion}{A cue to relative depth order in which, for example, one object obstructs the view of part of another object.}
\glos{Nonmetrical depth cue}{A depth cue that provides information about the depth order (relative depth), but not depth magnitude (e.g., his nose is in front of his face).}
\glos{Metrical depth cue}{A depth cue that provides quantitative information about the distance in the third dimension.}
\glos{Projective geometry}{For purpose of studying perception of the three-dimensional world, the geometry that describes that transformations that occur when the three dimensional world is projected onto a two dimensional surface. For example, parallel lines do not converge in the world, but they do in the two-dimensional projection.}
\glos{Relative size}{A comparison of size between objects without knowing the absolute size of either one.}
\glos{Texture gradient}{A depth cue based on the geometric fact that items of the same size from smaller images when they are farther away. An array of items that change in size across the image will appear to form a surface in depth.}
\glos{Relative height}{As depth cue, the observation that objects at different distances from their view on the ground plane will form images at different heights in the retinal image. Object further away will be seen as higher in the image.}
\glos{Familiar size}{A depth cue based on the knowledge of the typical size of objects like humans and pennies.}
\glos{Relative metrical depth cue}{A depth cue that could specify, for example, that object A was twice as far away as object B without providing information about the absolute distance to either A or B.}
\glos{Absolute metrical depth cue}{A depth cue that provides absolute information about the distance in the third dimension (e.g., his nose sticks out 4 centimeter in front of his face).}
\glos{Arial perspective (or haze)}{ A depth cue based on the implicit understanding that light is scattered by the atmosphere. More light is scattered by the atmosphere. Thus more distant objects are subject to more scatter and appear fainter, bluer, and less distinct.}
\glos{Linear perspective}{A depth cue based on the fact that lines that are parallel in the three dimensional world will appear to converge in a two-dimensional image.}
\glos{Vanish point}{The apparent point to which parallel lines receding in depth converge.}
\glos{Pictoral depth cue}{A cue to distance or depth used by artist to depict three dimensional depth in two-dimensional pictures.}
\glos{Anamorphosis (or anamorphosis projection)}{Use of the rule of linear respective to create a two-dimensional image so distorted that it looks correct only when viewed from special angle or with a mirror that conters the distoration.}
\glos{Motion parallax}{An important depth cue that is based on the head movement. The geometric information obtained from an eye in two different positions at two different times is simular to the information from tow eyes in different positions in the head at the same time.}
\glos{Accommodation}{The process by which the eye changes it's focus (in which the lens gets fatter as gaze is directed towards nearer objects).}
\glos{Convergence}{The ability of the two eyes to turn inward, often used to place the two images of a feature in the world on corresponding location in the two retinal images (typically on the fovea of each eye). Convergence reduces the disparity of that feature to zero (or near zero).}
\glos{Divergence}{The ability of the two eyes to turn outward, often used to place the two images of a feature in the world on corresponding location in the two retinal images (typically on the fovea of each eye). Convergence reduces the disparity of that feature to zero (or near zero).}
\glos{Corresponding retinal points}{A geometric concept stating that points on the retina of each eye were the monocular retinal images of a single object are formed are at the same distance from the fovea in each eye. The two foveas are also corresponding points.}
\glos{Vieth-Muller circle}{The location of objects whose images fall on the geometrically corresponding points in the two retinas. If life were simple, this circle would be the horopter, but life is not simple.}
\glos{Horopter}{The location of objects whose images lie on corresponding points. The surface of zero disparity.}
\glos{Diplopia}{Double vision. If visible in both eyes, stimuli falling outside the Panum's fusional area will appear diplopic.}
\glos{Panum's fusional area}{The region of space in from and behind the horopther. Within which binocular single vision is possible.}
\glos{Crossed disparity}{The sign of disparity created by objects in from of the place of fixation (the horopther). The term crossed is used because images of objects located in from of the horopther appear to be displaced to the left in the right eye, and to the right in the left eye.}
\glos{Uncrossed disparity}{The sign of disparity created by objects behind the place of fixation (the horopther). The term uncrossed is used because images of objects located in from of the horopther appear to be displaced to the right in the right eye, and to the left in the left eye.}
\glos{Stereoscope}{A device for presenting one image to one eye and another image to the other eye. Stereopscos can be used to present dichoptic stimuli for stereopsis and binocular rivalry.}
\glos{Free fusion}{The technique of converging (crossing) or diverging the eyes in order to view a stereogram without stereoscope.}
\glos{Correspondance problem (binocular vision)}{The problem of figuring out which bit of image in the left eye should be matched with which bit in the right eye. The problem is particulary vexing when the images consist of thousands of similar features like dots in random dot stereograms.}
\glos{Uniqueness constraint}{In stereopsis, the observation that a feature of the world is represented exactly once in each retinal image. The constraint simplifies the correspondence problem.}
\glos{Continuity constraint}{In stereopsis, the observation that except the edges of objects, neighboring points in the world lie at a similar distances from the viewer. This is one of the several constraints that have been proposed to solve the correspondence problem.}
\glos{Absolute disparity}{A difference in the actual retinal coordinates in the left and the right eyes of the images of a feature in a visual scene.}
\glos{Relative disparity}{The difference in absolute disparity of two elements in a visual scene.}
\glos{Bayesian approach}{A statistical model based on Revend Thomas Bayes insight that prior knowledge could influence our estimates of the probability of current event.}
\glos{Ideal observer}{A theoretical observer witch complete access to the best available information and the ability to combine different sources of information in the optimal manner. It can be usefull to compare human performance to that of an ideal observer.}
\glos{Binocular rivalry}{The competition between tow eyes to control of vision perception, which is evident when completely different stimuli are presented to the two eyes.}
\glos{Stereoacuity}{A measure of the smallest binocular disparity that can generate a sensation of depth.}
\glos{Diochoptic}{Reffering to the presentation of two stimuli, one to each eye. Different from binocular presentation, which could involve both eyes looking at a single stimulus.}
\glos{Critical period}{In the study of development, a period of time when the organism is particulary susceptible to develop mental change. There are critical periods in the development of binocular vision, human language and so on.}
\glos{Strabismus}{A misalginment of the two eyes such that a single object in space is imaged on the fovea of one eye and on a nonfovea area of the other (turned eye).}
\glos{Estropia}{Strabismus wich one eye deviates inward.}
\glos{Exotropia}{Strabismus wich one eye deviates outward.}
\glos{Suppression}{In vision, the inhibition of unwanted image. Suppression occurs frequently in persons with strabismus.}
\glos{Motion Aftereffect (MAE)}{The illusion of motion of a stationary object that occurs after prolonged exposure to a moving object.}
\glos{Apparent Motion}{The (illusory) impression of smooth motion resulting from the rapid alternation of objects that appear in different locations in rapid succession.}
\glos{Aperture}{An opening that allows only a partial view of the object.}
\glos{Corresponding problem (motion)}{The problem faced by the motion detection system of knowing which feature in frame 2 corresponding to a particular feature in frame 1.}
\glos{Aperture problem}{The fact when a moving object is viewed through an apperture (or receptive field), the direction of motion on a local feature or part of the object may be ambiguous.}
\glos{Middle temporal lobe (MT)}{An area of the brain thought to be important in the perception of motion.}
\glos{Interocular transfer}{The transfer of an effect (such as adaption) from one eye to the other.}
\glos{First-order motion}{The motion of an object that is defined by changes in luminance.}
\glos{Luminance-defined object}{An object that is delineated by changes in reflexted light.}
\glos{Second-order motion}{The motion of an object that is defined by changes in the contrast or texture, but not by luminiance.}
\glos{Texture-defined (contrast-defined) object}{An object that is defined by changes in constrast or texture, but not by luminance.}
\glos{Optic array}{The collection of light rays that interact with objects in the world in front of a viewer. Term coined by J.J. Gibson.}
\glos{Optic flow}{The changing angular position of points in a perspective image that we experience as we move through the world.}
\glos{Focus of expansion}{The point of the center of the horizontal form which, when we are in motion (e.g., driving on the highway), all points in the perspective image seem to emenate. The focus of expansion is one aspect of optic flow.}
\glos{Biological motion}{The pattern of movement of living beings (humans and animals).}
\glos{Time to collision (TTC)}{The time required for a moving object (such as a cricket ball) to hit a stationary object (such as a batsman's head). TTC=distance/rate.}
\glos{Tau}{Information in the optic flow that could signal TTC (time to collision) without the necessity of estimation either absolute distances or rates. The ratio of the retinal image size at any moment to the rate which the image is expanding is tau, and TTC is proportional to tau.}
\glos{Smooth pursuit}{A type of eye movement in which the eye moves smoothly to following a moving object.}
\glos{Superio colliculus}{A structure in the midbrain that is important in intiating and guiding eye movements.}
\glos{Vergence}{A type of eye movement in which the two eyes move in opposite directions -- for example, both eyes turn toward the noce (convergence) or away from the nose (divergence).}
\glos{Saccade}{A rapid movement of the eyes that changes fixation from one object or location to another.}
\glos{Reflexive eye movements}{A movement of the eye that is automatic and involuntary.}
\glos{Saccadic supression}{The reduction of visual sensitivity that occurs when we make saccadic eye movements. Saccadic suppression eliminates the smear from retinal image motion during an eye movement.}
\glos{Comparator}{An area of the visual system that receives one copy of the order issued by the motor system when the eye move (the other copy goes to the eye muscle). The comparator can compensate for the image changes caused by the eye movement.}
\glos{Akinetopsia}{A rare neuropsychological disorder in which the affected individual has no perception of motion.}
\glos{Attention}{Any of the vary large set of selective processes in the brain. To deal with the impossibility of handling all inputs at once, the nervous system has evolved machanisms that are able to restrict processing to a subset of things, places, ideas, or moments in time.}
\glos{Selective attention}{The form of attention involved when processing is restricted to a subset of possible stimuli.}
\glos{Reaction Time (RT)}{A measure of the time from the onset of a stimulus to a response.}
\glos{Cue}{A stimulus that might indicate wether (or what) a subsequent stimulus will be. Cues can be valid (correct information), invalid (incorrect), or neutral (uninformative).}
\glos{Stimulus onset asynchrony (SOA)}{The time between the onset of one stimulus and the onset of another.}
\glos{Visual Search}{Looking for a target in a display containing distract elements.}
\glos{Target}{The goal of a visual search.}
\glos{Distractor}{In visual search, any stimulus other than the target.}
\glos{Set size}{The number of items in a visual display.}
\glos{Feature search}{Search for a target defined by a signal attribute, such as a salient color or orientation.}
\glos{Salience}{The vividness of a stimulus related to its neighbors.}
\glos{Parallel}{In visual attention, referring to the processing of multiple stimuli at the same time.}
\glos{Serial self-terminating search}{A search from item to item, ending when a target is found.}
\glos{Guided search}{Search in which attention can be restricted to a subset of possible items on the basis of information about the target item's basic features (e.g., its color).}
\glos{Conjunction search}{Search for a target defined by the presence of two or more attributes (e.g., a red vertical target among red horizontal and blue vertical distractors).}
\glos{Biding problem}{The challenge of tying different attributes of visual stimuli (e.g., color, orientation, motion), which are handled by different brain circuits, to the appropriate objects so that we perceive a unified object (e.g., red, vertical, moving right).}
\glos{Preattentive stage}{The processing of a stimulus that occurs before selective attention is deployed to that stimulus.}
\glos{Feature integration theory}{Anne Treisman's theory of visual attention which holds that a limited set of basic features can be processed in parallel preattentively, but that other properties, including the correct binding of features to objects, require attention.}
\glos{Illuory conjuction}{An erroneous combination of two features in a visual scene - for example, seeing red X when the display contains red letters and Xs but no red Xs.}
\glos{Rapid sterial visual presentation (RSVP)}{An experimental procedure in which stimuli appear in a stream at one location (typically the point of fixation) at a rapid rate (typically about eight per second).}
\glos{Attentinal blink}{The difficulty in perceiving and responding to the second of two target stimuli amid a rapid stream of distracting stimuli if the observer has responded to the first target stimulus within 200 to 500 milliseconds before the second stimulus is presented.}
\glos{Repitation blindness}{A failure to detect the second occurrence of a letter, word, or picture in a rapidly presented stream of stimuli when the second occurence falls within 200 to 500 milliseconds of the first.}
\glos{Fusiform face area}{An area in the fusiform gyris of human extrastriate cortex that responds preferentially to faces in fMRI studies.}
\glos{Parahippocampal place area}{A region of cortex in the temporal lobe of humans that appear to respond with particular strength to images of places (as opposed to isolated objects).}
\glos{Response enhancement}{An effect of attention on the response of a neuron in which the neuron responding to an attended stimulus gives a bigger response.}
\glos{Sharper tuning}{An effect of attention on the response of a neuron in which the neuron responds more precely. For example a neuron that responds to line with orientation from -20 degrees to +20 degrees might come to respond to +- 10 degree lines.}
\glos{Visual-field defect}{A portion of the visual field with no vision or with abnormal vision, typically resulting from damage to the vision nervous system.}
\glos{Partietal lobe}{In each cerebral hemisphere a lobe that lies towards the top of the brain between the frontal and occipital lobes.}
\glos{Neglect}{In visual attention, the inability to attend to or respond to stimuli in the contralesional visual field (typically, neglect of the left field after right parietal damage). Also neglect of half of the body or half of an object.}
\glos{Contralesional field}{The visual field on the side opposite a brain lesion (e.g., point to the left of fixation are contralesional to damage to the right hemisphere of the brain).}
\glos{Extinct}{In visual attention, the inability to perceive a stimuli to one side of the point of fixation (e.g., the right) in the presence of another stimulus, typically in a comparable position in the other visual field (e.g., the left side).}
\glos{Ipsilesional field}{The visual field on the same side as the brain lesion.}
\glos{Simultagnosia}{An inability to perceive more than one object at a time. Simultagnosia is consequence of bilateral damage to the parietal lobes (Balint syndrome).}
\glos{Change blindess}{The failure to notice a change between two scenes. If the change does not alter the gist, or meaning, of the scene, quite large changes can pass unnoticed.}
\glos{Convert attentional Shift}{A shift of attention in the absence of corresponding movements of the eyes.}
\glos{Overt attentional shift}{A shift of attention accompanied by corresponding movements of the eyes.}
\glos{Spatial layout}{The description of the structure of a scene (e.g., enclosed, open, rough, smooth) without reference to the identity of specific objects in the scene.}
\glsaddall
\printglossaries
\end{document}

