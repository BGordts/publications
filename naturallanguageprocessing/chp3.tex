\section{Morphology}
\begin{df}{Morphology}
\sb{} is the way words are built up from smaller meaning-bearing words called \flv{Morpheme}s.
\end{df}
\begin{df}{Morpheme}
A \sb{} is a part of a word with a specific meaning. \sb{}s are categorized in the following way:
\begin{itemize}
 \item \flv{Stem}s: the main part of a word
 \item \flv{Affix}es:
 \begin{itemize}
  \item \flv{Prefix}es
  \item \flv{Suffix}es
  \item \flv{Infix}es
  \item \flv{Circumfix}es
 \end{itemize}
\end{itemize}
\end{df}
\begin{df}{Stem}
%TODO
\end{df}
\begin{df}{Affix}
An \sb{} is a part of a word that is not part of the \\flv{Stem}. \sb{}s are used to give a \flv{Stem} a different meaning or for the declension of the \flv{Stem}. Sometimes a word has more than one affix. Especially \flv{agglutinative languages} tend to string \sb{}es together.
\end{df}
\begin{df}{Prefix}
A \sb{} is a part of a word that proceeds the \flv{Stem}. In English, prefixes play an important role in the meaning of a word. For instance \stc{un-buckle}, \stc{un-necessary}, \stc{ir-regular}, \stc{il-legal}.
\end{df}
\begin{df}{Suffix}
A \sb{} is a part of a word that follows the \flv{Stem}. In English, the suffix plays an important role in the declension of nouns and verbs. For instance \stc{cat-s}, \stc{bush-es}, \stc{eat-s}, \stc{try-ing}.
\end{df}
\begin{df}{Infix}
A \sb{} is a part of a word in the middle of the \flv{Stem}. These are rare in English. For instance \stc{asbo-bloody-lutely}.
\end{df}
\begin{df}{Circumfix}
A \sb{} is a part of a word surrounding the \flv{Stem}. Circumfixes don't exist in English but are used for past participles in Dutch and German. For instance \stc{ge-zeg-d} and \stc{ge-sag-t}.
\end{df}
\begin{df}{Morphotactics}
\sb{} are a set of rules that describe how to combine different affixes.
\end{df}
\begin{df}{Inflection}
\sb{} is a method where one combines a \flv{Stem} and a grammatical morpheme. This morpheme has the same \pos{} class as the \flv{Stem} and fills some \flv{syntactical function}.
\end{df}
\begin{df}{English plural nouns declension}
\sb{} are a form of \flv{inflection}. A plural noun uses a suffix \stc{s}. Sometimes \flv{Orthographic rules} should be applied, for instances \stc{thrush-es}. Furthermore some irregular nouns exist like \stc{mouse/mice} and \stc{ox/oxen}.
\end{df}
\begin{df}{English verb inflection}
\sb{} are a form of \flv{inflection}. Regular verbs can be processed easily. If the third person is used, one uses a suffix \stc{s}, when the past tense is used, the suffix \stc{ed} is used. Finally the gerund of a verb is the \flv{Stem} together with the \stc{ing} suffix. Irregular verbs can be processed by using a set of idiosyncratic rules of inflection.
\end{df}
\begin{df}{Derivation}
\sb{} is a method where one combines a \flv{Stem} and a grammatical morpheme. This morpheme has a different \pos{} class than the \flv{Stem}. Therefore the meaning is hard to predict. Examples are the suffixes \stc{-ation}, \stc{-ee}, \stc{-er} and \stc{-ness} for nouns and \stc{-al}, \stc{-able} and \stc{-less} for adjectives.
\end{df}
\begin{df}{Compounding}
\sb{} is a method where one combines multiple word \flv{Stem}s to form a new word. Usually the result has the same class as that of the head. For instance \stc{doghouse=dog (modifier)+house (head)}. Compounding can be implemented recursively.
\end{df}
\begin{df}{Clitic}
A clitic is a morpheme that acts syntactically as a word but is reduced in form. It is attached phonologically or orthographically (for instance with an accent). A table of \sb{}s and their corresponding full form is listed in \tblref{clitics}.
\end{df}
\begin{df}{Cliticization}
\sb{} is a method where one combines a \flv{Stem} with a \flv{clitic}. If the clitic is placed before the word, we call this a \flv{proclitic}. For instance \stc{\underline{l'}opera}. If the clitic is placed behind the word, we call this an enclitic. For instance \stc{I\underline{'ve}}.
\end{df}
\begin{df}{Agreement}
\sb{} is a concept where for instance a subject and noun must agree in \flv{number} (person). In French nouns and adjectives must also agree in \flv{gender}.
\end{df}
\begin{df}{Sentence Detection}
\sb{} is a task where a sy\flv{Stem} must detect the begin and end of a sentence in a text. Most sy\flv{Stem}s use punctuations to process this. However headlines for instance often don't have punctuations. Furthermore abbreviations use punctuations. Therefore it's not that easy to disambiguate.
\end{df}
\begin{df}{Lexicon}
A \sb{} is a repository of words.
\end{df}
\begin{df}{Morphotactics}
Morphotactics are a set of rules that tell us how \flv{Stem}s and affixes fit together.
\end{df}
\begin{df}[Finite-State Lexicon]{Computational lexicon}
\sb{} is a finite state model that describes the different rules of morphotactics using a \flv{Finite State Automaton}. Examples of such sy\flv{Stem}s can be found on \figref{nominalinflection,verbalinflection,adjectivalderivations}.
\end{df}
\importfsm{nominalinflection}{Nominal inflection}
\importfsm{verbalinflection}{Verbal inflection}
\importfsm{adjectivalderivations}{Adjectival derivations}
\begin{df}{Porter Stemmer}
The \sb{} is a transducer who transforms a word to its \flv{Stem}. The transducer is lexicon free and is based on three rules: \conv{ational}{ate}, \conv{ing}{$\epsilon$} and \conv{sses}{ss}. The \flv{Stem}mer can improve performance but is error-prone.
\end{df}
\importfsm{orthographictransducer}{Orthographic transducer}