\section{Statical parsing}
\begin{df}[PCFG,Stochastic context-free grammar]{Probabilistic context-free grammar}
A \sb{} is 4-tuple $\tupl{N,\Sigma,R,n_0}$ where $N$ is a set of \flv{Non-terminal symbol}s (of \flv{Variable}s). $\Sigma$ a set of \flv{Terminal symbol}s (disjoint from $N$), $R$ a set of \flv{Production rule}s each of the form $a\rightarrow\funf{\beta}{p}$ where $a$ is a \flv{Non-terminal symbol}, $\beta$ a string of symbols from the infinite set of strings $\klee{\brak{\Sigma\cup N}}$ and $p\in\fbrk{0,1}$ expressing $\Prob{\beta|a}=\Prob{a\rightarrow\beta}$. $n_0\in N$ is the \flv{Start symbol}. Since each non-terminal eventually derives the following condition must hold:
\begin{equation}
\forall a:\displaystyle\sum_{\beta\in\klee{\brak{\Sigma\cup N}}}\Prob{a\rightarrow\beta}=1
\end{equation}
The joint probability of a \flv{Syntax tree} $T$ and a sentence $\vec{w}$ is defined as the product of the probabilities of all the $k$ rules used to expand eacht of the $n$ non-terminal nodes in the parse tree, where each rule $i$ is expressed as $a_i\rightarrow\beta_i$:
\begin{equation}
\Prob{T,\vec{w}}=\displaystyle\prod_{i=1}^{k}{\Prob{a_i\rightarrow\beta_i}}=\Prob{T}\cdot\Prob{\vec{w}|T}
\end{equation}
Since the parse tree includes all the words of the sentence, one can state that:
\begin{equation}
\Prob{\vec{w}|T}=1
\end{equation}
We can use a \sb{} for \flv{Syntactic disambiguation} by accepting the most probable tree:
\begin{equation}
\fun{\hat{T}}{\vec{w}}=\argmax_T\Prob{T|\vec{w}}=\argmax_T\displaystyle\frac{\Prob{T|\vec{w}}}{\Prob{S}}=\argmax_T\Prob{T,\vec{w}}=\argmax_T\Prob{T}
\end{equation}
\end{df}
\begin{df}[PCKY]{Probabilistic Cocke-Kasami-Younger algorithm}
The\sb{} is a \flv{Bottum-up parsing} algorithm that produces the most probable tree. Instead of the \flv{Cocke-Kasami-Younger algorithm}, it uses a $n+1\times n+1\times V$ matrix. With $V$ the number of \flv{Non-terminal symbol}s. The entries contain the probability for the non-terminal contituent that spans positions $i$ through $j$ of the output. The entire algorithm is listed in \algref{probcykparse}.
\end{df}
\importalgo{probcykparse}{The Probabilistic Cocke-Kasami-Younger algorithm}