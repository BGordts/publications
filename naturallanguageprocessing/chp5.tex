\section{Part-of-Speech Tagging and Shallow Parsing}
\begin{df}{Part-of-Speech tagging}
\sb{} is a process where one assigns a \pos{} or other syntactic class marker to each word in a corpus. This tagging requires tokenization. Punctuations are tagged as well.
\end{df}
\begin{df}{Part-of-Speech tag}
A \sb{} specifies the meaning or function of a word in the corpus. The tag is based on both the syntactical and morphological function. The tag can be extracted by the words that occur nearby or the affixes they use. \sb{} are grouped in \flv{Class}es.
\end{df}
\begin{df}{Part-of-Speech class}
A \sb{} is a group of different \flv{Part-of-Speech tag}s. \sb{}es come in two flavors: \flv{Closed class}es and \flv{Open class}es. \flv{Closed class}es have fixed membership (no new words), they mainly consist out of articles, conjugations and prepositions and function words (\stc{of}, \stc{it}, \stc{and}, \stc{you},...). \flv{Open class}es can contain new (introduced) words. They contain \flv{Noun}s, \flv{Verb}s, \flv{Adjective}s and \flv{Adverb}s.
\end{df}
\begin{df}{Noun}
A \sb{} is a \flv{Syntactic class} in which the word occurs. Semantically it points to people, places and things. A \sb{} can has a plural declension, can occur with determiners and takes possessives.
\end{df}
\begin{df}{Proper noun}
\sb{} are names of specific persons or entities. For instance \stc{Regina}, \stc{Leuven} and \stc{IBM}. Usually \sb{} are no articles and are capitalized.
\end{df}
\begin{df}[Common name]{Common noun}
A \sb{} is a noun who is not a specific person or entity. \sb{} are subdivided into \flv{Count noun}s and \flv{Mass noun}s.
\end{df}
\begin{df}{Count noun}
A \sb{} is a \flv{Noum} which allows enumeration. \sb{}s require an article and have a singular and plural form. Examples are \stc{goat/goats} and \stc{relationship}.
\end{df}
\begin{df}{Mass noun}
A \sb{} is a \flv{Noun} which is conceptualized as a homogeneous group. Examples are \stc{snow} and \stc{communism}.
\end{df}
\begin{df}{Verb}
A \sb{} is a \flv{Syntactic class}. \sb{}s describe \flv{Action}s and \flv{Process}es. They have different \flv{Morphological form}s. Subclasses of \sb{} are \flv{Auxiliaries} and \flv{Modal verb}s.
\end{df}
\begin{df}{Auxiliaries}
\sb{} is a closed subclass of \flv{Verb}s. They mark semantic features of the main verb (tense, aspect, polarity, mood). In general it connects a \flv{Subject} with \flv{Predicate}s. \sb{}s include \stc{to have} and \stc{to be}, \stc{to do} and \flv{Modal verb}s.
\end{df}
\begin{df}[Modal,Modal auxiliary verb, Modal auxiliary]{Modal verb}
A \sb{} is a type of \flv{Verb} that is used to indicate modality: likelihood, ability, permission, and obligation. Examples are \stc{shall/shoud}, \stc{will/would}. \tblref{modalverbs} lists a set of \sb{}s.
\end{df}
\begin{df}{Adjective}
An \sb{} is a \flv{Part-of-Speech class} which describes properties or qualities of an object. For instance its color, age, value. Other adjectives describe some degrees of comparison. For instance \stc{good}, \stc{better}, \stc{nicest}. Not all languages have \sb{}s.
\end{df}
\begin{df}{Adverb}
An \sb{} is a diverse \flv{Part-of-Speech class} which modifies something, usually a verb. Adverbs come in different flavors: \flv{Directional adverb}, \flv{Degree adverb}, \flv{Manner adverb}, \flv{Temporal adverb}.
\end{df}
\begin{df}[Locative adverb]{Directional adverb}
A \sb{} is a type of \flv{Adverb} describing the location where something takes place. For instance \stc{home}, \stc{here} and \stc{downhill}.
\end{df}
\begin{df}{Degree adverb}
A \sb{} is a type of \flv{Adverb} describing the severity of the action. For instance \stc{extremely}, \stc{very} and \stc{somewhat}.
\end{df}
\begin{df}{Manner adverb}
A \sb{} is a type of \flv{Adverb} describing in which manner the action takes place. For instance \stc{slowly}, \stc{delicately}.
\end{df}
\begin{df}{Temporal adverb}
A \sb{} is a type of \flv{Adverb} describing when an action takes place. For instance \stc{yesterday}, \stc{Monday}. Some of these adverbs are \flv{Noun}s as well.
\end{df}
\begin{df}{Preposition}
A \sb{} is a \flv{Part-of-Speech class} that comes before \flv{Noun} phrases. They express \flv{Spatial relation}s or \flv{Temporal relation}s. Since \sb{}s are a closed class, \tblref{prepositions} lists all \sb{}s.
\end{df}
\begin{df}{Particle}
A \sb{} is a \flv{Part-of-Speech class}. They resemble \flv{Preposition}s and are combined with a verb. Often this verb has a different meaning. Examples of such combinations are \stc{turn down} and \stc{rule out}. Since \sb{}s are a closed group, \tblref{particles} lists all the single word \sb{}s.
\end{df}
\begin{df}{Determiner}
A \sb{} is a \flv{Part-of-Speech class} that occurs with nouns. \sb{}s mark the beginning of a \flv{Noun phrase}. A subtype of a \sb{} is a \flv{Article}. Other \sb{}s are \stc{this} and \stc{that}.
\end{df}
\begin{tm}{Marker hypotheses}
There is a set of words in every language that marks boundaries of phrases in a sentence.
\end{tm}
\begin{df}{Article}
An \sb{} is a subclass of a \flv{Determiner}. \sb{}s come in two flavors: \flv{Definite article}s like \stc{the} and \flv{Indefinite article}s like \stc{a} and \stc{an}.
\end{df}
\begin{df}{Conjugation}
A \sb{} is a \flv{Part-of-Speech class} who joins two \flv{Phrase}s, \flv{Clause}s or \flv{Sentence}s. \sb{}s come in different flavors: \flv{Coordinating conjugation}s and \flv{Subordinate conjugation}s. \tblref{conjugations} lists a set of \sb{}s.
\end{df}
\begin{df}{Coordinating conjugation}
A \sb{} is a subclass of \flv{Conjugation}s. They join elements of equal status. Examples are \stc{and}, \stc{or} and \stc{but}.
\end{df}
\begin{df}{Subordinate conjugation}
A \sb{} is a subclass of \flv{Conjugation}s. They join elements where one element is of embedded status. An example is \stc{I thought \underline{that} you might like that}. \sb{}s that link a \flv{Verb} to its \flv{Argument}s are called \flv{Complementizer}s.
\end{df}
\begin{df}{Pronoun}
A \sb{} is a \flv{Part-of-Speech class} which is a shorthand for referring to a \flv{Noun phrase}, \flv{Entity} or \flv{Event}. \sb{}s come in different flavors: \flv{Personal pronouns}, \flv{Possessive pronouns} and \flv{Wh-pronouns}. \tblref{pronouns} lists a set of \sb{}s.
\end{df}
\begin{df}{Personal pronoun}
\sb{}s are \flv{Pronoun}s who refers to \flv{Person}s, \flv{Entitie}s.
\end{df}
\begin{df}{Possessive pronoun}
\sb{}s are \flv{Pronoun}s who express actual \flv{Possession} or an abstract relation between a \flv{Person} and an \flv{Object}.
\end{df}
\begin{df}{Wh-pronoun}
\sb{}s are \flv{Pronoun}s used in \flv{Question}s. They can act as \flv{Complementizer}s.
\end{df}
\begin{df}{Interjection}
An \sb{} is a \flv{Part-of-Speech class} which contains for instance \stc{oh}, \stc{ah}, \stc{yes} and \stc{uh}.
\end{df}
\begin{df}{Negative}
A \sb{} is a \flv{Part-of-Speech class} which contains for instance \stc{no} and \stc{not}.
\end{df}
\begin{df}{Politeness marker}
A \sb{} is a \flv{Part-of-Speech class} which contains for instance \stc{thank you} and \stc{please}.
\end{df}
\begin{df}{Greeting}
A \sb{} is a \flv{Part-of-Speech class} which contains for instance \stc{hello} and \stc{goodbye}.
\end{df}
\begin{df}{Existential there}
The \sb{} is a \flv{Part-of-Speech class} which contains markers for the fact that some object exists. For instance \stc{\underline{there} are two apples on the table}.
\end{df}
\begin{df}{Rule-based Part-of-Speech tagging}
\sb{} is a method where \flv{Part-of-Speech tagging} is handled by using a \flv{Dictionary} to assing a list of potential tags to each word, and use a large list of hand-written disambiguation rules. The result is a fragment where each word has one \flv{Part-of-Speech tag} per word. For different tag sets ore languages, one has to use different rules.
\end{df}
\begin{df}[Voutilainen]{EngCGTagger}
\sb{} is a \flv{Rule-based Part-of-Speech tagger} that uses a set of $56'000$ word stems and performs a \flv{Morphological analysis}. The analysis is done by running the text through a two level \flv{Lexicon transducer}. Next the tagger rules out incorrect \flv{Part of Speech tags} using \flv{Constraint}s.
\end{df}
\begin{tm}{Bayes rule}
\sb{} is a rule in \flv{Probability theory} stating the following:
\begin{equation}
\Prob{x|y}=\displaystyle\frac{\Prob{y|x}\cdot\Prob{x}}{\Prob{y}}
\end{equation}
If one is only interested in the maximum likelihood of of $x$, one may drop the denominator since it's the same of each item:
\begin{equation}
x^{\star}=\argmax_{x\in X}\Prob{y|x}\cdot\Prob{x}
\end{equation}
\end{tm}
\begin{tm}{Part-of-Speech simplification assumption}
In order to perform \flv{Part-of-Speech tagging}, most systems use the following two assumptions:
\begin{itemize}
 \item The probability of a word only depends on its own \flv{Part-of-Speech tag}:
 \begin{equation}
  \Prob{\vec{w}|\vec{t}}\approx\displaystyle\prod_{i=1}^n{\Prob{w_i|t_i}}
 \end{equation}
 \item The probability of a tag only depends on the previous tag (this is sometimes called the \flv{Bigram assumption}):
 \begin{equation}
  \Prob{\vec{t}}\approx\displaystyle\prod_{i=1}^{n}\Prob{t_i|t_{i-1}}
 \end{equation}
\end{itemize}
\end{tm}
\begin{df}{Hidden Markov Model Part-of-Speech tagging}
\sb{} is a form of \flv{Part-of-Speech tagging} where one uses a \flv{Hidden Markov Model}. This system is based on the \flv{Part-of-Speech simplification assumption}s (otherwise one could model this with a \flv{Hidden Markov Model}). One can simply train the network by counting tuples in a \flv{Training set}:
\begin{eqnarray}
\Prob{t_i|t_{i-1}}&=&\displaystyle\frac{\Count{t_{i-1},t_i}}{\Count{t_{i-1}}}\\
\Prob{w_i|t_i}&=&\displaystyle\frac{\Count{t_i,w_i}}{\Count{t_i}}
\end{eqnarray}
In this context, the \flv{emission probabilities} are sometimes called the \flv{Lexical likelihood}. Unknown words are in most cases handled by using a \stc{unknown word} tag. In other cases the probabilities are equally distributed over the different \flv{Part-of-Speech tag}s who represent \flv{Open class}es. In this case, the result mainly depends on the previous tag. A better system is using \flv{Morphology}: analyzing the word in order to determine its type. Finally one may allso consider all final letter sequences of all words and for each suffix compute the probability of a tag $t$ given the suffix\footnote{This method works fine for most European languages, but is not considered to be universally applicable.}.
\end{df}
\begin{df}[Brill tagging]{Hybrid Part-of-Speech tagging}
\sb{} is a method to perform \flv{Part-of-Speech tagging} using \flv{Transformation-based Part-of-Speech tagging}. By mixing \flv{Rule-based Part-of-Speech tagging} and \flv{Stochastic Part-of-Speech tagging} one hopes to generate better results. Rules specify which tags should be assigned to what words and these rules are automatically induced from data (by \flv{Supervised learning}). When the system tags a fragment, the rules are ordered so the most broadest rule is applied first. The algorithm then chooses more specific rules modifying a smaller number of tags. The rules are learned automatically but limited by a template.
\end{df}
\begin{df}{Combination Part-of-Speech tagging}
\sb{} is a \flv{Part-of-Speech tagging} mechanism where a couple of taggers are used concurrently. The results are compared and an algorithm decides about the final \flv{Part-of-Speech tag}s for each word.
\end{df}
\begin{df}[Chunking]{Shallow parsing}
\sb{} is a process where one groups \flv{Word}s in linguistically meaningful \flv{Chunk}s. By applying this process recursively we get a \flv{Shallow parse tree}. This process of courses requires \flv{Part-of-Speech tag}s. \sb{} is based on the \flv{Marker hypotheses}.
\end{df}
\begin{df}{Machine learning chunker}
A \sb{} is a system that performs \flv{Shallow parsing}. Therefore each word is either: the \flv{Beginning of a chunk}, \flv{End of a chunk} or nor the beginning or end of a chunk. No \flv{Part-of-Speech tagging} is required, but it can help. The program learns the rules by using a traing set. For a chunk $X$, the beginning is annotated as $\mbox{\texttt{B}}X$ and the end as $\mbox{\texttt{E}}X$. Machine learnings don't consider \flv{Crossing chunk}s: if one marks the end of a chunk $X$, all embedded chunks end as well.
\end{df}