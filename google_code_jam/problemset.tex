\documentclass[titlepage]{article}
\title{Problem sets of the\\Google Code Jam}
\author{}
\date{2013}
\begin{document}

\begin{titlepage}
\maketitle
\end{titlepage}
\tableofcontents
\newpage

\section{First round}

\subsection{Tic-Tac-Toe-Tomek}

\subsubsection{Problem}
Tic-Tac-Toe-Tomek is a game played on a $4\times4$ square board. The board starts empty, except that a single \verb+'T'+ symbol may appear in one of the $16$ squares. There are two players: \verb+X+ and \verb+O+. They take turns to make moves, with \verb+X+ starting. In each move a player puts her symbol in one of the empty squares. Player \verb+X+'s symbol is \verb+'X'+, and player \verb+O+'s symbol is \verb+'O'+.
\paragraph{}
After a player's move, if there is a row, column or a diagonal containing $4$ of that player's symbols, or containing $3$ of her symbols and the \verb+'T'+ symbol, she wins and the game ends. Otherwise the game continues with the other player's move. If all of the fields are filled with symbols and nobody won, the game ends in a draw. See the sample input for examples of various winning positions.
\paragraph{}
Given a $4\times4$ board description containing \verb+'X'+, \verb+'O'+, \verb+'T'+ and \verb+'.'+ characters (where \verb+'.'+ represents an empty square), describing the current state of a game, determine the status of the Tic-Tac-Toe-Tomek game going on. The statuses to choose from are:
\begin{enumerate}
 \item \verb+"X won"+ (the game is over, and X won)
 \item \verb+"O won"+ (the game is over, and O won)
 \item \verb+"Draw"+ (the game is over, and it ended in a draw)
 \item \verb+"Game has not completed"+ (the game is not over yet)
\end{enumerate}
If there are empty cells, and the game is not over, you should output \verb+"Game has not completed"+, even if the outcome of the game is inevitable.

\subsubsection{Input}
The first line of the input gives the number of test cases, $T$. $T$ test cases follow. Each test case consists of $4$ lines with $4$ characters each, with each character being \verb+'X'+, \verb+'O'+, \verb+'.'+ or \verb+'T'+ (quotes for clarity only). Each test case is followed by an empty line.

\subsubsection{Output}
For each test case, output one line containing \verb+"Case #x: y"+, where \verb+x+ is the case number (starting from $1$) and \verb+y+ is one of the statuses given above. Make sure to get the statuses exactly right. When you run your code on the sample input, it should create the sample output exactly, including the \verb+"Case #1: "+, the capital letter \verb+"O"+ rather than the number \verb+"0"+, and so on.

\subsubsection{Limits}
The game board provided will represent a valid state that was reached through play of the game Tic-Tac-Toe-Tomek as described above.
\paragraph{Small dataset} $1\leq T\leq 10$.
\paragraph{Large dataset} $1\leq T\leq 1000$.

\subsubsection{Sample}
\paragraph{Input}
\begin{verbatim} 
6
XXXT
....
OO..
....

XOXT
XXOO
OXOX
XXOO

XOX.
OX..
....
....

OOXX
OXXX
OX.T
O..O

XXXO
..O.
.O..
T...

OXXX
XO..
..O.
...O
\end{verbatim}
\paragraph{Output}
\begin{verbatim}
Case #1: X won
Case #2: Draw
Case #3: Game has not completed
Case #4: O won
Case #5: O won
Case #6: O won
\end{verbatim}

\subsection{Lawnmower}

\subsubsection{Problem}
Alice and Bob have a lawn in front of their house, shaped like an $N$ metre by $M$ metre rectangle. Each year, they try to cut the lawn in some interesting pattern. They used to do their cutting with shears, which was very time-consuming; but now they have a new automatic lawnmower with multiple settings, and they want to try it out.
\paragraph{}
The new lawnmower has a height setting - you can set it to any height $h$ between $1$ and $100$ millimetres, and it will cut all the grass higher than $h$ it encounters to height h. You run it by entering the lawn at any part of the edge of the lawn; then the lawnmower goes in a straight line, perpendicular to the edge of the lawn it entered, cutting grass in a swath $1\mbox{ m}$ wide, until it exits the lawn on the other side. The lawnmower's height can be set only when it is not on the lawn.
\paragraph{}
Alice and Bob have a number of various patterns of grass that they could have on their lawn. For each of those, they want to know whether it's possible to cut the grass into this pattern with their new lawnmower. Each pattern is described by specifying the height of the grass on each $1\mbox{ m}\times1\mbox{ m}$ square of the lawn.
\paragraph{}
The grass is initially $100\mbox{ mm}$ high on the whole lawn.

\subsubsection{Input}
The first line of the input gives the number of test cases, $T$. $T$ test cases follow. Each test case begins with a line containing two integers: $N$ and $M$. Next follow $N$ lines, with the $i$-th line containing $M$ integers $a_{i,j}$ each, the number $a_{i,j}$ describing the desired height of the grass in the $j$-th square of the $i$-th row.

\subsubsection{Output}
For each test case, output one line containing \verb+"Case #x: y"+, where \verb+x+ is the case number (starting from $1$) and \verb+y+ is either the word \verb+"YES"+ if it's possible to get the \verb+x+-th pattern using the lawnmower, or \verb+"NO"+, if it's impossible (quotes for clarity only).

\subsubsection{Limits}
$1\leq T\leq100$.
\paragraph{Small dataset}
$1\leq N,M\leq10$, $1\leq a_{i,j}\leq2$.
\paragraph{Large dataset}
$1\leq N,M\leq100$, $1\leq a_{i,j}\leq100$.

\subsubsection{Sample}
\paragraph{Input}
\begin{verbatim}
3
3 3
2 1 2
1 1 1
2 1 2
5 5
2 2 2 2 2
2 1 1 1 2
2 1 2 1 2
2 1 1 1 2
2 2 2 2 2
1 3
1 2 1
\end{verbatim}

\paragraph{Output}
\begin{verbatim}
Case #1: YES
Case #2: NO
Case #3: YES
\end{verbatim}

\subsection{Fair and Square}

\subsubsection{Problem}
Little John likes palindromes, and thinks them to be fair (which is a fancy word for nice). A palindrome is just an integer that reads the same backwards and forwards - so $6$, $11$ and $121$ are all palindromes, while $10$, $12$, $223$ and $2244$ are not (even though $010=10$, we don't consider leading zeroes when determining whether a number is a palindrome).
\paragraph{}
He recently became interested in squares as well, and formed the definition of a fair and square number - it is a number that is a palindrome and the square of a palindrome at the same time. For instance, $1$, $9$ and $121$ are fair and square (being palindromes and squares, respectively, of $1$, $3$ and $11$), while $16$, $22$ and $676$ are not fair and square: $16$ is not a palindrome, $22$ is not a square, and while $676$ is a palindrome and a square number, it is the square of $26$, which is not a palindrome.
\paragraph{}
Now he wants to search for bigger fair and square numbers. Your task is, given an interval Little John is searching through, to tell him how many fair and square numbers are there in the interval, so he knows when he has found them all.

\subsubsection{Input}
The first line of the input gives the number of test cases, $T$. $T$ lines follow. Each line contains two integers, $A$ and $B$ - the endpoints of the interval Little John is looking at.

\subsubsection{Output}
For each test case, output one line containing \verb+"Case #x: y"+, where \verb+x+ is the case number (starting from $1$) and \verb+y+ is the number of fair and square numbers greater or equal to $A$ and smaller or equal than $B$.

\subsubsection{Limits}

\paragraph{Small dataset}
$1\leq T\leq100$, $1\leq A\leq B\leq1000$.

\paragraph{First large dataset}
$1\leq T\leq10000$, $1\leq A\leq B\leq10^{14}$.

\paragraph{Second large dataset}
$1\leq T\leq1000$, $1\leq A\leq B\leq10^{100}$.

\subsubsection{Sample}

\paragraph{Input}
\begin{verbatim}
3
1 4
10 120
100 1000
\end{verbatim}

\paragraph{Output}
\begin{verbatim}
Case #1: 2
Case #2: 0
Case #3: 2
\end{verbatim}

\subsection{Treasure}

\subsubsection{Problem}
Following an old map, you have stumbled upon the Dread Pirate Larry's secret treasure trove!
\paragraph{}
The treasure trove consists of N locked chests, each of which can only be opened by a key of a specific type. Furthermore, once a key is used to open a chest, it can never be used again. Inside every chest, you will of course find lots of treasure, and you might also find one or more keys that you can use to open other chests. A chest may contain multiple keys of the same type, and you may hold any number of keys.
\paragraph{}
You already have at least one key and your map says what other keys can be found inside the various chests. With all this information, can you figure out how to unlock all the chests?
\paragraph{}
For example, suppose the treasure trove consists of four chests as described below, and you began with exactly one key of type $1$: see table \ref{tbl:treasure}.
\begin{table}[hbt]
\centering
\begin{tabular}{l|l|l}
Chest Number&Key Type To Open Chest&Key Types Inside\\\hline
1&1&None\\
2&1&1,3\\
3&2&None\\
4&3&2
\end{tabular}
\caption{Chest properties of the example.}
\label{tbl:treasure}
\end{table}
\paragraph{}
You can open all the chests in this example if you do them in the order $2$, $1$, $4$, $3$. If you start by opening chest \#1 first, then you will have used up your only key, and you will be stuck.

\subsubsection{Input}
The first line of the input gives the number of test cases, $T$. $T$ test cases follow. Each test case begins with a single line containing two positive integers $K$ and $N$, representing the number of keys you start with and the number of chests you need to open.
\paragraph{}
This is followed by a line containing $K$ integers, representing the types of the keys that you start with.
\paragraph{}
After that, there will be $N$ lines, each representing a single chest. Each line will begin with integers $T_i$ and $K_i$, indicating the key type needed to open the chest and the number of keys inside the chest. These two integers will be followed by Ki more integers, indicating the types of the keys contained within the chest.

\subsubsection{Output}
For each test case, output one line containing \verb+"Case #x: C1 C2 ... CN"+, where \verb+x+ is the case number (starting from $1$), and where \verb+Ci+ represents the index (starting from $1$) of the \verb+i+-th chest that you should open.
\paragraph{}
If there are multiple ways of opening all the chests, choose the "lexicographically smallest" way. In other words, you should choose to make \verb+C1+ as small as possible, and if there are multiple ways of making \verb+C1+ as small as possible, choose the one that makes \verb+C2+ as small as possible, and so on.
\paragraph{}
If there is no way to open all the chests, you should instead output one line containing \verb+"Case #x: IMPOSSIBLE"+.

\subsubsection{Limits}
$1\leq T\leq25$, $1\leq K$. All chest types and key types will be integers between $1$ and $200$ inclusive.

\paragraph{Small dataset}
$1\leq N\leq20$. In each test case, there will be at most $40$ keys altogether.

\paragraph{Large dataset}
$1\leq N\leq200$. In each test case, there will be at most $400$ keys altogether.

\subsubsection{Sample}
\paragraph{Input}
\begin{verbatim}
3
1 4
1
1 0
1 2 1 3
2 0
3 1 2
3 3
1 1 1
1 0
1 0
1 0
1 1
2
1 1 1
\end{verbatim}
\paragraph{Output}
\begin{verbatim}
Case #1: 2 1 4 3
Case #2: 1 2 3
Case #3: IMPOSSIBLE
\end{verbatim}
\end{document}