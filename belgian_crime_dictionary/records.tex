\documentclass[8pt,a5paper,twocolumn]{extbook}
\title{Belgisch Crimineel Woordenboek}
\author{BCW-redactie}
\date{2013}
\usepackage[dutch]{babel}
\usepackage{../SharedData/dictionary}
\usepackage[top=2cm, headsep=0.4cm, bottom=1.5cm, left=1.5cm, right=1.5cm, twocolumn]{geometry}
\usepackage[nonumberlist,xindy]{glossaries}
\usepackage{hyperref,makeidx}
\makeglossaries
\makeindex
\glsenablehyper
%Criminal
% -identifier
% -name
% -identity
% -description

\newcommand{\lazygls}[1]{\ifglsentryexists{#1}{\gls{#1}}{??}}

\newcommand{\entrybase}[6]{\entry[#1]{#3}{(#4) #5}\index{#3|textbf}\newglossaryentry{#2}{name={#3}}}
\newcommand{\individual}[4]{\entrybase{#1}{ind:#1}{#2}{#3}{#4}{#1}}
\newcommand{\caseentry}[3]{\entrybase{#1}{case:#1}{Zaak #2}{}{#3}{#1zaak}}
\newcommand{\evententry}[3]{\entrybase{#1}{event:#1}{#2}{}{#3}{#1}}
\newcommand{\groupentry}[3]{\entrybase{#1}{group:#1}{#2}{}{#3}{#1}}
\newcommand{\commisieentry}[3]{\entrybase{#1}{comm:#1}{Commissie #2}{}{#3}{#1commissie}}
\newcommand{\operationentry}[3]{\entrybase{#1}{oper:#1}{Operatie #2}{}{#3}{#1operatie}}
\newcommand{\annon}[3]{\individual{ann:#1}{Anoniem \##1}{#2}{#3}}
\newcommand{\indref}[1]{\index{\lazygls{ind:#1}}\lazygls{ind:#1}}
\newcommand{\groupref}[1]{\index{\lazygls{group:#1}}\lazygls{group:#1}}
\newcommand{\annref}[1]{}
\newcommand{\caseref}[1]{\index{\lazygls{case:#1}}\lazygls{case:#1}}
\newcommand{\commissieref}[1]{\index{\lazygls{comm:#1}}\lazygls{comm:#1}}
\newcommand{\operationref}[1]{\index{\lazygls{oper:#1}}\lazygls{oper:#1}}
\newcommand{\location}[1]{\index{#1}\emph{#1}}
\newcommand{\brand}[1]{\index{#1}\emph{#1}}
\newcommand{\institution}[1]{\index{#1}\emph{#1}}
\newcommand{\eventdate}[1]{\index{#1}#1}
\newcommand{\newchaplet}[2]{\clearpage\lettergroup{#2}\dictchar{#2 #1}\input{#1.tex}}
\newcommand{\newchapleta}[2]{\clearpage\lettergroup{#2}\dictchar{#2}\input{#1.tex}}
\begin{document}
\maketitle

\chapter*{Inleiding}
In dit woordenboek bespreken we de verschillende personen die op de \'e\'en of andere manier betrokken zijn in het Belgische misdaadmilieu. Dit zijn zowel daders, slachtoffers, getuigen of personen die eerder onrechtstreeks betrokken zijn. In belangrijke zaken (zoals de zaak Dutroux en de zaak rond de Bende van Nijvel) is niet altijd duidelijk hoe de feiten precies verlopen zijn. Daarom stellen we de getuigenissen in dit boek niet voor als feiten. De meeste zaken worden dan ook in genuanceerde vorm vertelt. Bij elk element worden bovendien de bronnen vermeld.
\paragraph{}
Een groot aantal elementen in het woordenboek hebben betrekking met de zogenaamde \emph{Ann\'ees des Plombes}, de moeilijke periode in de jaren '70 en '80 waarin Belgi\"e af te rekenen krijgt met de georganiseerde en politieke misdaad.
\paragraph{}
Personen die extra informatie willen leveren, of bestaande informatie willen corrigeren kunnen mailen naar \href{mailto:bcw@gmail.com}{bcw@gmail.com}.

\newchaplet{a}{A}
\newchaplet{b}{B}
\newchaplet{c}{C}
\newchaplet{d}{D}
\newchaplet{e}{E}
\newchaplet{f}{F}
\newchaplet{g}{G}
\newchaplet{h}{H}
\newchaplet{i}{I}
\newchaplet{j}{J}
\newchaplet{k}{K}
\newchaplet{l}{L}
\newchaplet{m}{M}
\newchaplet{n}{N}
\newchaplet{o}{O}
\newchaplet{p}{P}
\newchaplet{q}{Q}
\newchaplet{r}{R}
\newchaplet{s}{S}
\newchaplet{t}{T}
\newchaplet{u}{U}
\newchaplet{v}{V}
\newchaplet{w}{W}
\newchaplet{x}{X}
\newchaplet{y}{Y}
\newchaplet{z}{Z}
\newchaplet{annon}{?}
\newchapleta{197}{70's}
\newchapleta{198}{80's}
\newchapleta{199}{90's}
\newchapleta{200}{00's}

\appendix

\chapter{Transcripties}
\begin{quote}
Dames en heren\\
Op verzoek van onderzoeksrechter Willy Van Wezel van het parket van \location{Nijvel}, lezen wij volgend bericht.\\
Naar aanleiding van de bloedige hold-up gepleegd gisteren 30 september '82 te \location{Waver}, heeft de rijkswacht een robotfoto samengesteld van \'e\'en van de daders.\\
Beschrijving van de dader: 25 \`a 30 jaar, een normale tot zware lichaamsbouw, zwart haar en hij is 1 meter 70 tot 1 meter 75 lang.\\
Alle personen die inlichten kunnen verstrekken omtrent deze persoon: gelieve deze door te geven aan de rijkswacht van \location{Waver}. Het kan telefonisch op het volgende nummer (010) 22 38 15. Of aan de dichtstbijgelegen rijkswacht- of politie-post.\cite{opsporing19821001}
\end{quote}
\begin{quote}
In een bos in \location{Sint-Pieters Leeuw} hebben speurders van het onderzoek naar de \groupref{bendevannijvel} twee dagen lang opgravingen gedaan. Ze zochten naar het lichaam van een omgekomen bendelid of naar wapens die daar de bende eventueel zou hebben begraven. Zonder resultaat evenwel.
\cite{hetjournaalbendeopgravingenleverennietsop}
\end{quote}
\begin{quote}
Voor het eerst in zes jaar zijn er weer huiszoekingen gebeurd in het onderzoek naar de \groupref{bendevannijvel}. Op 21 plaatsen in het land vielen speurders binnen bij mensen die in de jaren '80 bekende of verborgen sympathie\"en koesterden voor voor extreem-rechts.\cite{hetjournaal21huiszoekingennaarbendevannijvel}
\end{quote}

\nocite{*}
\bibliographystyle{alpha}
\bibliography{biblio}
\printindex
\end{document}