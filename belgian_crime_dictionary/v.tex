\individual{vandermissenpatricia}{Patricia Vandermissen}{}{Vandermissen was de advocaat van \indref{loufregina} tijdens het proces rond de \caseref{dutroux}. Vandermissen argumenteert dat de getuigenis van Louf geloofwaardig is vanwege een onderzoek door specialisten onder leiding van \indref{igodtpaul} van de \institution{KU Leuven}. Vandermissen stelt dat indien de getuigenissen van Louf niet als juridisch bewijsmateriaal kunnen worden gebruikt, er verder onderzoek dient plaats te vinden op grond van deze verklaringen. De Belgische justitie besliste echter deze getuigenissen niet verder te onderzoeken.\cite{zdf20011028}}
\individual{vervloessemmarcel}{Marcel Vervloessem}{×}{Bekende van \indref{pardaensgina}. Vervloessem getuigde dat nadat Pardaens de bedreigingen kenbaar maakte aan de politie ze regelmatig met de dood wordt bedreigd. Verder wordt ze geschaduwd en gevolgd. Verder waren er merkwaardige problemen aan haar telefoontoestel. Een man belde haar en zei dat beter met haar werk kon stoppen, of anders niet lang meer zou leven.\cite{zdf20011028}}
\individual{vanessagerard}{G\'erard Vanessa}{×}{Getuige in de \caseref{dutroux}. De getuige was voor de aanvang van het proces reeds overleden.\cite{zdf20011028}}
\individual{vancampjacques}{Jacques Van Camp}{ - 2 oktober 1983; restaurantuitbater}{Uitbater van het restaurant \institution{Aux 3 Canard}. Werd op 2 oktober 1983 vermoord. Vermoedelijk door de \groupref{bendevannijvel}.\cite{panoramabvn95}}
\individual{vermaelenfreddy}{Freddy Vermaelen}{- 7 oktober 1983}{Filiaalhouder van de \institution{Delhaize} in \location{Beersel}. Werd vermoord tijdens de overval op dit warenhuis op 7 oktober 1983. De overval was vermoedelijk het werk van de \groupref{bendevannijvel}.\cite{panoramabvn95}}
\individual{vanparystony}{Tony Van Parys}{×}{Van Parys was volksvertegenwoordiger en verslaggever van de \commissieref{bendevannijvel}. Van Parys stelt dat het onderzoek vanaf het begin foutliep:\begin{quote}Het is van in den beginne fout gelopen. Men heeft sporen verwaarloosd, men heeft aanknopingspunten niet gezocht. Er zijn zoveel dingen spaak gelopen waardoor alle soorten hypothesen zich stellen, en zich de vraag fundamenteel stelt: waarom heeft men niet alles gezocht wat klaarblijkelijk moest onderzocht worden.\end{quote}Verder stelt Van Parys dat het feit dat de daders nooit gevonden geweest zijn een groot probleem geeft in een democratie als Belgi\"e:\begin{quote}Het feit dat dit mogelijk is geweest in een democratisch land, stelt een enorm probleem: 28 mensen die op een afschuwelijke wijze worden vermoord. Dit stelt enorme problemen. Plus de vaststelling dat dit onderzoek aaneenhangt van tal van incidenten, die bijna niet toevallig kunnen zijn. Het kan bijna niet anders dan dat er andere motieven hebben gespeeld dan het belang van het onderzoek. Het is een aaneenrijging van problemen.\end{quote}\cite{panoramabvn95}}
\individual{verhaegherik}{Rik Verhaeghe}{Getuige overval \institution{Delhaize} in \location{Aalst}}{\begin{quote}Wanneer er eindelijk politie aanwezig was, was het ongeveer 25 minuten en dan moest je zelf gaan vragen aan de rijkswachter: ``Meneer ik was er bij, kan ik vertellen wat ik tegengekomen ben?''.\end{quote}\cite{terzakegetuigenovervaldelhaizeoverijse}}
\individual{vanthielenpaul}{Paul Van Thielen}{rijkswacht Brussel}{Van Thielen was lid van de rijkswacht van \location{Brussel} en deed onder meer onderzoek naar de \groupref{haemers}. Over de situatie waarin \indref{haemerspatrick} opgroeide, getuigde hij als volgt:\begin{quote}Ouders die hun kinderen op dat moment wat verwenden: ``Grabbel maar in de pot. Heb je nodig, goed ik heb geen tijd voor jou maar ik heb wel geld voor jouw, en dat is misgegaan. Want die hebben zich een aantal dingen in het uitgaansleven eigen gemaakt als ze nog heel jong waren.\end{quote}\cite{bendehaemersdejeugdjaren}}
