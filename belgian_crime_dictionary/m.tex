\individual{marshalan}{An Marshal}{}{Slachtoffer van \indref{dutrouxmarc}.}
\individual{massahubert}{Hubert Massa}{hoofdofficier van justitie}{Hij werkt kort aan de \caseref{dutroux}. Op een dag treft men hem dood aan in zijn kantoor: neergeschoten. Hij liet geen brieven achter voor vrouw of kinderen. Het ministerie besluit zelfmoord.\cite{zdf20011028}}
\individual{melonfrancois}{Fran\c{c}ois Melon}{×}{Een bekende van \indref{reykensfrancois}. Melon getuigde dat op een dat Reykens hem heeft aangesproken. Er was iets ernstigs gebeurd en Reykens wilde erover praten. Melon stelde voor het toen te vertellen. Reykens antwoorde hierop dat hij over \indref{russomelissa} wilde praten. Melon verklaart dat Reykens iets weet over de ontvoering als volgt:\begin{quote}het zijn geen mensen zoals wij, uit ons milieu uit onze kringen, die over dergelijke informatie beschikken, dat kon alleen Fran\c{c}ois.\end{quote}.\cite{zdf20011028}}
\individual{mendezjuan}{Juan Mendez}{ - 7 januari 1986; Handelsingenieur}{Mendez was handelsingenieur voor \institution{FN Herstal} en wapenliefhebber. Hij werd op 7 januari 1986 vermoord in zijn wagen. Hij was verantwoordelijk voor wapenleveringen aan Latijns Amerika.\cite{panoramabvn95}}
\operationentry{mammoet}{Mammoet}{Een operatie uitgevoerd op \eventdate{19 oktober 1984}. De politie voert een grote razzia uit. Er wordt gezocht naar tientallen verdachten en er vinden honderden huiszoekingen plaats. Ook militanten van \groupref{ecolo} krijgen de politie op bezoek. In de \location{Albani\"estraat, Brussel} wordt de drukkerij van \indref{carettepierre} onderzocht. Er wordt allerhande materiaal meegenomen, maar Carette blijft onvindbaar. De operatie levert veel informatie op. Sommigen beweren echter dat de politie de aanslagen van de \groupref{cellullescommunistecombatantes} heeft misbruikt omwille van politieke motieven.\cite{dezaakdejachtopdeccc}}
