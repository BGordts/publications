\individual{lambreckxeefje}{Eefje Lambreckx}{}{Slachtoffer van \indref{dutrouxmarc}.}
\individual{lejeunejeandenis}{Jean-Denis Lejeune}{×}{Vader van \indref{lejeunejulie}. Lejeune stelt dat een significant aantal getuigen in de \caseref{dutroux} op mysterieuze wijze zijn verdwenen:\begin{quote}En heel toevallig sterven mensen op onverklaarbare wijze. Ze hebben bijvoorbeeld een auto ongeluk wanneer ze als getuige willen optreden. Of men vindt ze verbrand in hun huizen. Wonderbaarlijk verontrust dit justitie niet.\end{quote}\cite{zdf20011028}}
\individual{lejeunejulie}{Julie Lejeune}{}{Slachtoffer van \indref{dutrouxmarc}.}
\individual{loufregina}{Regina Louf}{}{Getuige X1 in de \caseref{dutroux}. Getrouwde vrouw met vier kinderen en uitbater van een hondenpension. Bij het uitbreken van de \caseref{dutroux} meldt ze zich en getuigt anoniem. Gedurende dagenlange verhoren vertelt zij over gruwelijke belevenissen met rijke en machtige mannen die bij seksfeesten jonge meisjes verkrachter, martelen en soms zelfs doden. De onderzoekers (\indref{debaetspatriek} en \indref{hupezphillipe}) controleren haar beweringen en vinden voldoende bewijzen om haar verklaringen ernstig te nemen. In de \caseref{dutroux} verklaart ze dat \indref{dutrouxmarc} de leverancier en \indref{nihoulmichel} de organisator van een kinderprostitutie-netwerk zijn:\begin{quote}Wel ik kende Nihoul (\indref{nihoulmichel}, red.) de tweede helft van de jaren zeventig en het begin van de jaren tachtig. In die zin is hij weinig verandert dus zijn uiterlijk is hetzelfde gebleven. Ik herkende hem dan ook onmiddellijk als een van de pooiers die kinderen misbruikte en voor seksfeesten opleidde, om misbruikt te worden.\end{quote}Louf werd onderzocht door het team van \indref{igodtpaul}. De onderzoekers stellen dat Louf tijdens haar jeugd massaal werd blootgesteld aan seksueel misbruik. Haar advocate, \indref{vandermissenpatricia} stelt op grond hiervan6 dat indien Louf haar getuigenis niet als juridisch bewijsmateriaal kan worden gebruikt, er verder onderzoek moet plaatsvinden op basis van deze getuigenissen. De Belgische justitie oordeelde dat de feiten niet verder moesten worden onderzocht.\cite{zdf20011028}}
\individual{lynafrancine}{Francine Lyna}{Onderzoeksrechter}{Lyna volgende het onderzoek van de \caseref{bendevannijvel} met veel belangstelling. Lyna stelt dat het onderzoek op een slechte manier werd gevoerd:\begin{quote}Het onderzoek ging van magistraat naar magistraat, hoorde in \location{Brussel} thuis, maar bleef in \location{Nijvel}, en daar volgende de magistraten elkaar op. Het onderzoek verliep op een manier die niet bijzonder gelukkig was, en niet erg doeltreffend.\end{quote}\cite{panoramabvn95}}
