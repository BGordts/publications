\individual{dutrouxmarc}{Marc Dutroux}{kinderverkrachter en moordenaar}{Wordt in augustus 1996 gearresteerd samen met \indref{lelievremichel} en \indref{martinmichel}. Na twee dagen te zijn verhoord leidt Dutroux de speurders naar een kelder. In deze kelder bevindt zich achter een rek een ruimte van twee vierkante meter groot. In deze ruimte worden \indref{delhezletitia} en \indref{dardennesabine} levend aangetroffen. Sabine bracht 80 dagen in de kelder door. In de volgende dagen worden alle huizen van Dutroux doorzocht. Men vindt de lijken van \indref{lejeunejulie} en \indref{russomelissa}. Er werd reeds 14 maanden naar deze meisjes gezocht. Waarschijnlijk zijn ze in de kelder omgekomen van honger. Ook het huis van een vermoorde medeplichtige van Dutroux wordt onderzocht: hier worden de lijken van \indref{marshalan} en \indref{lambreckxeefje} gevonden. In een ander huis van Dutroux ontdekte men bouwmateriaal en een ondergronds tunnelcomplex.\cite{zdf20011028}}.
\individual{coenraadschristiaan}{Christiaan Coenraads}{gevangene}{Zou in de Brusselse gevangenis worden gehoord over zijn relatie met \indref{dutrouxmarc} en medeverdachten. Een dag voor dit verhoor ontsnapt hij uit de gevangenis. Een maand later wordt hij in een voorstad van Brussel vermoord teruggevonden.\cite{zdf20011028}}
\individual{dardennesabine}{Sabine Dardenne}{}{Slachtoffer van \indref{dutrouxmarc}.}
\individual{delhezletitia}{Letitia Delhez}{}{Slachtoffer van \indref{dutrouxmarc}.}
\caseentry{dutroux}{Dutroux}{}
\individual{jenartbrigitte}{Brigitte Jenart}{tandarts}{Was bevriend met \indref{nihoulmichel}. Zij geldt als een belangrijke getuige. Een jaar later is ze dood. Men stelt dat dit zelfdoding is.\cite{zdf20011028}}
\individual{konjevodaanna}{Anna Konjevoda}{?? - voor 7 april 1998}{Deze vrouw werd op 7 april 1998 uit de waterzuiveringsinstallatie van de \location{Maas bij Luik} opgevist. Ze werd geslagen en gewurgd.\cite{zdf20011028}}
\individual{lambreckxeefje}{Eefje Lambreckx}{}{Slachtoffer van \indref{dutrouxmarc}.}
\individual{lejeunejeandenis}{Jean-Denis Lejeune}{×}{Vader van \indref{lejeunejulie}. Lejeune stelt dat een significant aantal getuigen in de \caseref{dutroux} op mysterieuze wijze zijn verdwenen:\begin{quote}En heel toevallig sterven mensen op onverklaarbare wijze. Ze hebben bijvoorbeeld een auto ongeluk wanneer ze als getuige willen optreden. Of men vindt ze verbrand in hun huizen. Wonderbaarlijk verontrust dit justitie niet.\end{quote}\cite{zdf20011028}}
\individual{lejeunejulie}{Julie Lejeune}{}{Slachtoffer van \indref{dutrouxmarc}.}
\individual{marshalan}{An Marshal}{}{Slachtoffer van \indref{dutrouxmarc}.}
\individual{massahubert}{Hubert Massa}{hoofdofficier van justitie}{Hij werkt kort aan de \caseref{dutroux}. Op een dag treft men hem dood aan in zijn kantoor: neergeschoten. Hij liet geen brieven achter voor vrouw of kinderen. Het ministerie besluit zelfmoord.\cite{zdf20011028}}
\individual{nihoulmichel}{Michel Nihoul}{}{Nihoul wordt in augustus 1996 voorgeleid voor de onderzoeksrechter. Hij is de enige belangrijke verdachte in de \caseref{dutroux} die tot de veroordeling vrij is. \indref{loufregina} stelde dat hij de organisator was van een kinderprostitutie-netwerk. Nihoul wijst de beschuldigingen af:\begin{quote}Ik heb deze vrouw nooit gezien. Ik zeg vrouw omdat zij nu getrouwd is en kinderen heeft. Daarnaast spreekt zijn geen woord Frans en ik geen woord Nederlands. U zult zich waarschijnlijk stellen dat men de taal niet moet kennen om iemand te verkrachten. Maar als men aan groepsseks deelneemt, dan spreek je toch de taal van degene die het organiseert.\end{quote}.\cite{zdf20011028}}
\individual{routmonbernard}{Bernard Routmon}{}{Film maker die handelt in pornocassettes. Hij wordt verdacht van het ontvoeren van een meisje en een mogelijke connectie met \indref{dutrouxmarc}. Bij een huiszoeking vindt men kinderkleding en speelgoed. Na enkele vruchteloze verhoren meldt hij zich plotseling telefonisch bij de politie. Op weg naar het verhoor rijdt hij met zijn auto tegen de pui van een huis. Hij komt om bij dit ongeluk.\cite{zdf20011028}}
\individual{russomelissa}{Melissa Russo}{}{Slachtoffer van \indref{dutrouxmarc}.}
\individual{russocarine}{Carine Russo}{}{Moeder van \indref{russomelissa}. Russo stelt dat wanneer \indref{dutrouxmarc} alleen handelde, het onderzoek sneller tot resultaten had moeten leiden. Ze stelt echter dat men op de belangrijkste vragen nog steeds geen antwoord is geformuleerd. Bijgevolg zou er een netwerk achter de \caseref{dutroux} schuilgaan. Daarnaast vertelt Russo omtrent het verhoor van \indref{reykensfrancois}:\begin{quote}Op de dag dat het verhoor van Fran\c{c}ois Reykens was gepland kwamen er rijkswachters bij ons thuis om te melden dat het verhoor niet doorging en dat Fran\c{c}ois Reykens onder de trein was omgekomen.\end{quote}\cite{zdf20011028}}
\annon{6209}{}{Crimineel in de buurt van \location{Charleroi}. Leerde \indref{dutrouxmarc} in de gevangenis kennen. Hij stelt dat vermits de kinderen niet kortstondig na hun verdwijning werden teruggevonden, Dutroux geen ordinaire verkrachter is. Hij stelt dat er nooit losgeld is gevraagd aan de ouders van de vermoorde kinderen. Nochtans bezat Dutroux vele huizen. Hij stelt dan ook dat Dutroux aan geld kwam door de meisjes uit te lenen.\cite{zdf20011028}}
\individual{weinsteinbernard}{Bernard Weinstein}{×}{Vriend (en mogelijke medeplichtige) van \indref{dutrouxmarc}. Woonde in de \location{Rue Daubresse, Charleroi}. De plaats waar tevens de lijken van \indref{marshalan} en \indref{lambreckxeefje} werden gevonden. Dutroux zou zijn medeplichtige rohypnol hebben toegediend (in een boterham met pat\'e\cite{xdossiers}) en vervolgens levend hebben begraven. Men vermoedt dat Weinstein eruit wilde stappen en Dutroux hem als een gevaarlijke getuige zag.\cite{zdf20011028}}
\individual{steppejose}{Jose Steppe}{1939-1997}{Jose Steppe was van eenvoudige komaf maar was gedurende zijn hele leven bekend in \location{Charleroi}. Hij was een belangrijk vertrouwenspersoon voor de medebewoners in zijn wijk. Enkele weken na de arrestatie van \indref{dutrouxmarc} meldt hij zich telefonisch bij een journalist (vermoedelijk \indref{nabercaspar}) en de rijkswacht: hij zou belangrijke informatie hebben over Dutroux. Twee dagen voor zijn afspraak met de politie overlijdt hij in een caf\'e. De lokale hoofdofficier van justitie oordeelde dat het overlijden niet verdacht was: er werd geen autopsie uitgevoerd.\cite{zdf20011028}}
\individual{nabercaspar}{Caspar Naber}{journalist}{Naber is een getuige bij de dood van \indref{steppejose}. Naber stelt dat Steppe last had van astma en hij bij zijn overlijden een beademingsapparaat bij hem lag. Naber inspecteerde het apparaat samen met de zoon van Steppe. In het apparaat bevond zich een hoeveelheid rohypnol. Rohypnol is een verdovingsmiddel en wordt nooit in een beademingsapparaat gebruikt. \indref{dutrouxmarc} gebruikte echter rohypnol om zijn slachtoffers te drogeren.\cite{zdf20011028}}
\individual{goebelsguy}{Guy Goebels}{rijkswachter}{Rijkswachter in \location{Gr\^ace-Hollogne} (de plaats waar \indref{lejeunejulie} en \indref{russomelissa} verdwenen). Goebels werkte vanaf het begin aan de zaak (wat later de \caseref{dutroux} werd). Zijn collegas treffen hem op een dag dood aan in zijn woning: met zijn dienstwapen neergeschoten. Men stel zelfmoord vast, hoewel sommigen hier aan twijfelen. Familieleden stellen dat hij geen grote problemen had en niet suicidaal was.\cite{zdf20011028}}
\individual{tagliaferrobruno}{Bruno Tagliaferro}{}{Oud ijzerhandelaar, sloper en bekende van \indref{dutrouxmarc}. Man van \indref{jauparifabienne}. Zijn vrouw beweerde dat Tagliaferro werd omgebracht omdat hij teveel wist. Ook zijn vader stelt dit:\begin{quote}Bruno stond in de weg, hij heeft rotzooi gemaakt, ik weet niet wat. Men chanteerde hem via zijn kinderen en heeft hem smerige zaakjes laten opknappen.\end{quote}Ook een vriendin van zijn vrouw stelt dit:\begin{quote}Fabienne heeft mij verteld dat dat Bruno een auto gesloopt heeft dat bij de ontvoering van twee kleine meisjes was gebruikt. Niks zal men daarvan terugvinden. Pas later begreep ik dat het ging om Julie (\indref{lejeunejulie}, red.) en Melissa (\indref{russomelissa}, red.).\end{quote} De politie zocht in \location{Keumi\'ee, Namur} naar sporen. Een autopsie toont aan dat Tagliaferro werd vergiftigd.\cite{zdf20011028}}
\individual{jauparifabienne}{Fabienne Jaupari}{17 juli 1964-15 december 1998}{vrouw van \indref{tagliaferrobruno}. Jaupari stelde dat haar man werd vermoord. Vlak voor zijn dood zou Tagliaferro tegen Jaupari gezegd hebben:\begin{quote}Dat alles voorbij is, dat hij teveel wist en op korte termijn zou sterven.\end{quote}. Een autopsie gaf Jaupari gelijk: Tagliaferro werd vergiftigd. Niet veel later werd Jaupari zelf bedreigd. Ze vroeg tevergeefs om politiebescherming. Ze dacht belangrijke bewijsstukken van haar man te hebben ontdekt. Kort daarna vindt haar zoon haar dood in de slaapkamer. Ze is half verbrand en haar matras is met methanol overgoten. De justitie stelt zelfmoord of een ongeluk. Een vriendin spreekt dit echter tegen:\begin{quote}Weet u, iemand die zelfmoord wil plegen laat niet de wasmachine lopen, zet geen pan met aardappelen en zout op het vuur. (...) Ze (\indref{jauparifabienne}, red.) was geen zwetser, ze wist precies wat ze deed. Ze was ook niet gek, ze wist alleen niet meer wie ze kon vertrouwen.\end{quote}Een ongeval is bovendien onwaarschijnlijk: op de foto's van het plaats delict staat de fles methanol dichtgedraaid op het nachtkastje.\cite{zdf20011028}}
\individual{thiraultclaude}{Claude Thirault}{}{Bekende van \indref{dutrouxmarc}. Thirault getuigt dat Dutroux het plan had opgevat om \indref{tagliaferrobruno} om te brengen:\begin{quote}Dutroux wilde hem (\indref{tagliaferrobruno}, red.) ombrengen, maar waarom vertelde hij niet. (...) Hij (\indref{dutrouxmarc}, red.) bood hiervoor 50'000 Belgische franken en een wapen. (...) Ik zat bij Dutroux in de auto, daar heeft hij het tegen mij verteld.\end{quote}.\cite{zdf20011028}}
\individual{russogino}{Gino Russo}{}{Vader van \indref{russomelissa}.}
\individual{loufregina}{Regina Louf}{}{Getuige X1 in de \caseref{dutroux}. Getrouwde vrouw met vier kinderen en uitbater van een hondenpension. Bij het uitbreken van de \caseref{dutroux} meldt ze zich en getuigt anoniem. Gedurende dagenlange verhoren vertelt zij over gruwelijke belevenissen met rijke en machtige mannen die bij seksfeesten jonge meisjes verkrachter, martelen en soms zelfs doden. De onderzoekers (\indref{debaetspatriek} en \indref{hupezphillipe}) controleren haar beweringen en vinden voldoende bewijzen om haar verklaringen ernstig te nemen. In de \caseref{dutroux} verklaart ze dat \indref{dutrouxmarc} de leverancier en \indref{nihoulmichel} de organisator van een kinderprostitutie-netwerk zijn:\begin{quote}Wel ik kende Nihoul (\indref{nihoulmichel}, red.) de tweede helft van de jaren zeventig en het begin van de jaren tachtig. In die zin is hij weinig verandert dus zijn uiterlijk is hetzelfde gebleven. Ik herkende hem dan ook onmiddellijk als een van de pooiers die kinderen misbruikte en voor seksfeesten opleidde, om misbruikt te worden.\end{quote}Louf werd onderzocht door het team van \indref{igodtpaul}. De onderzoekers stellen dat Louf tijdens haar jeugd massaal werd blootgesteld aan seksueel misbruik. Haar advocate, \indref{vandermissenpatricia} stelt op grond hiervan6 dat indien Louf haar getuigenis niet als juridisch bewijsmateriaal kan worden gebruikt, er verder onderzoek moet plaatsvinden op basis van deze getuigenissen. De Belgische justitie oordeelde dat de feiten niet verder moesten worden onderzocht.\cite{zdf20011028}}
\individual{debaetspatriek}{Patriek De Baets}{×}{Rijkswachtadjudant en ondervrager bij de eerste verhoren van \indref{loufregina}.\cite{xdossiers}}
\individual{hupezphillipe}{Philippe Hupez}{×}{Eerste wachtmeester van de rijkswacht en ondervrager bij de eerste verhoren van \indref{loufregina}.\cite{xdossiers}}
\individual{vandermissenpatricia}{Patricia Vandermissen}{}{Vandermissen was de advocaat van \indref{loufregina} tijdens het proces rond de \caseref{dutroux}. Vandermissen argumenteert dat de getuigenis van Louf geloofwaardig is vanwege een onderzoek door specialisten onder leiding van \indref{igodtpaul} van de \institution{KU Leuven}. Vandermissen stelt dat indien de getuigenissen van Louf niet als juridisch bewijsmateriaal kunnen worden gebruikt, er verder onderzoek dient plaats te vinden op grond van deze verklaringen. De Belgische justitie besliste echter deze getuigenissen niet verder te onderzoeken.\cite{zdf20011028}}
\individual{igodtpaul}{Paul Igodt}{gewoon hoogleraar}{Igodt is gewoon hoogleraar aan de \institution{KU Leuven} en onderzocht de geloofwaardigheid van de getuigenissen van \indref{loufregina} in de \caseref{dutroux}. Igodt stelde dat Louf tijdens haar jeugd aan massaal seksueel misbruik is blootgesteld. Verder stelt Igodt dat het zich loont te onderzoeken wat Louf vertelt.\cite{zdf20011028}}
\individual{pardaensgina}{Gina Pardaens}{-mei 1998}{Sociaal maatschappelijk werker. Deze vrouw steunde slachtoffers van een kinderprostitutie-netwerk. Ze overleed na een ongeluk waarbij ze s'nachts met 80 kilometer per uur tegen de railing van een brug knalde op weg naar huis. Het openbaar ministerie stelde een ongeluk vast. Sommige vrienden (zoals \indref{vervloessemmarcel} en \annref{2167}) stellen echter dat er sabotage in het spel is. Enkele dagen voor haar dood schreef ze een brief. Hierin verklaart ze dat ze in het kader van haar werk doodsbedreigingen kreeg en voor haar leven vreesde. Ze vertelde de politie dat ze haar bedreigd hadden met een auto-ongeluk. Vlak voor haar dood vertelde Pardaens enkele vrienden over een videoband van een seksfeest waarop meisjes werden vermoord. Ze herkende \'e\'en van de deelnemers als \indref{nihoulmichel}.\cite{zdf20011028}}
\individual{vervloessemmarcel}{Marcel Vervloessem}{×}{Bekende van \indref{pardaensgina}. Vervloessem getuigde dat nadat Pardaens de bedreigingen kenbaar maakte aan de politie ze regelmatig met de dood wordt bedreigd. Verder wordt ze geschaduwd en gevolgd. Verder waren er merkwaardige problemen aan haar telefoontoestel. Een man belde haar en zei dat beter met haar werk kon stoppen, of anders niet lang meer zou leven.\cite{zdf20011028}}
\annon{2167}{}{Vriendin van \indref{pardaensgina}. De getuige stelt dat Pardaens teveel met de dood is bedreigd om dergelijk ongeval nog geloofwaardig te laten lijken. Voor de getuige is een ongeval dan ook uitgesloten.\cite{zdf20011028}}
\individual{taminiaujeanpaul}{Jean-Paul Taminiau}{×}{Taminiau verdween kort nadat hij een vriend had vertelt dat hij over belangrijke informatie beschikte. Taminiau huurde een garagebox waarbij de hangar ertegenover door \indref{dutrouxmarc} werd gebruikt voor illegale praktijken. Een jaar na zijn verdwijning vindt een visser een voet van Taminiau in het nabijgelegen kanaal. De rest van zijn lijk wordt nooit gevonden.\cite{zdf20011028}}
\individual{deulinjeanine}{Jeanin Deulin}{×}{Moeder van \indref{taminiaujeanpaul}. Deulin getuigt het volgende: \begin{quote}Die vriend die hem een nog een dag voor zijn verdwijning heeft gezien kwam een aantal weken later bij mij en vertelde: ``Voor mij is hij niet dood: hij verstopt zich. Hij is bang voor iets. De politie maakt het hem moeilijk, hij heeft angst.\end{quote} Deulin denkt dat indien er geen priv\'eredenen zijn om haar zoon om te brengen, haar zoon vermoord is door de mensen die \indref{dutrouxmarc} proberen te beschermen.\cite{zdf20011028}}
\individual{reykensfrancois}{Fran\c{c}ois Reykens}{×}{Op 28 jarige leeftijd wenst Reykens een getuigenverklaring af te leggen. De rijkswacht wil weten wat hij over \indref{russomelissa} weet. Het verhoor vindt nooit plaats: spoormedewerkers vinden zijn lijk langs het spoor. Hij werd door een trein overleden. Reykens was actief in het drugsmilieu en de criminaliteit. En kende reeds voor de bekendmakigng details over de \caseref{dutroux}.\cite{zdf20011028}}
\individual{melonfrancois}{Fran\c{c}ois Melon}{×}{Een bekende van \indref{reykensfrancois}. Melon getuigde dat op een dat Reykens hem heeft aangesproken. Er was iets ernstigs gebeurd en Reykens wilde erover praten. Melon stelde voor het toen te vertellen. Reykens antwoorde hierop dat hij over \indref{russomelissa} wilde praten. Melon verklaart dat Reykens iets weet over de ontvoering als volgt:\begin{quote}het zijn geen mensen zoals wij, uit ons milieu uit onze kringen, die over dergelijke informatie beschikken, dat kon alleen Fran\c{c}ois.\end{quote}.\cite{zdf20011028}}
\individual{reykensjean}{Jean Reykens}{×}{Vader van \indref{reykensfrancois}. Reykens had een nogal moeilijke relatie met zijn zoon. Hij weet niet of er bij het ongeval van zijn zoon geen kwaad opzet was.\cite{zdf20011028}}
\individual{priomichel}{Michel Piro}{×}{Piro was een bordeel- en nachtclub-bezitter. Drie maanden na de arrestatie van \indref{dutrouxmarc} wordt Piro koelbloedig op een parkeerplaats neergeschoten. Piro is een bekende in de onderwereld van \location{Charleroi}. Kort voor zijn dood neemt hij contact op met de ouders van \indref{lejeunejulie} en \indref{russomelissa} met een verzoek hen te ontmoeten. Hij wilde een avond organiseren om iets te onthullen maar waarover het precies ging werd niet onthult.}
\individual{ponceletguy}{Guy Poncelet}{×}{Oud hoofd van justitie. Vader van \indref{ponceletsimon}. Sinds de dood van zijn zoon probeert hij de \caseref{dutroux} zelf op te lossen. Poncelet is bezorgd dat veel getuigen onder verdachte omstandigheden zijn omgekomen en anderen reeds oud geworden zijn. Hierdoor zijn de getuigenissen minder scherp. Daarom vindt Guy Poncelet dat het proces sneller moest worden gevoerd in de \caseref{dutroux}.\cite{zdf20011028}}
\individual{ponceletsimon}{Simon Poncelet}{1955-1996}{Poncelet was een politieagent die onderzoek deed in de kringen van \indref{dutrouxmarc}. Hij werd tijdens een nachtdienst in zijn bureau neergeschoten. De moord is tot op vandaag niet opgehelderd.\cite{zdf20011028}}
\individual{henrottemarielouise}{Marie-Louise Henrotte}{}{Henrotte is vermoedelijk de enige getuige die heeft gezien hoe \indref{lejeunejulie} en \indref{russomelissa} werden ontvoerd:\begin{quote}Het was een donkere auto met vier deuren. (...) De meisjes zijn ingestapt. (...) Hij (\indref{dutrouxmarc}, red.) deed niks. Hij heeft alleen de deur open gedaan en de meisjes zijn achterin gaan zitten en hij is weggereden.\end{quote}Henrotte kon tijdens het proces niet getuigen: ze leed aan Alzheimer op dat moment.\cite{zdf20011028}}
\individual{renardadeche}{Adeche Renard}{×}{Getuige in de \caseref{dutroux}. De getuige was voor de aanvang van het proces reeds overleden.\cite{zdf20011028}}
\individual{gosselinalexandre}{Alexandre Gosselin}{×}{Getuige in de \caseref{dutroux}. De getuige was voor de aanvang van het proces reeds overleden.\cite{zdf20011028}}
\individual{dinantmichel}{Michel Dinant}{×}{Getuige in de \caseref{dutroux}. De getuige was voor de aanvang van het proces reeds overleden.\cite{zdf20011028}}
\individual{gregoirechristof}{Christof Gregoire}{}{Getuige in de \caseref{dutroux}. De getuige was voor de aanvang van het proces reeds overleden.\cite{zdf20011028}}
\individual{toussaintjoseph}{Joseph Toussaint}{×}{Getuige in de \caseref{dutroux}. De getuige was voor de aanvang van het proces reeds overleden.\cite{zdf20011028}}
\individual{antipinegregory}{Gr\'egory Antipine}{×}{Getuige in de \caseref{dutroux}. De getuige was voor de aanvang van het proces reeds overleden.\cite{zdf20011028}}
\individual{purshhenriette}{Henriette Push}{×}{Getuige in de \caseref{dutroux}. De getuige was voor de aanvang van het proces reeds overleden.\cite{zdf20011028}}
\individual{bahrklaus}{Klaus Bahr}{×}{Getuige in de \caseref{dutroux}. De getuige was voor de aanvang van het proces reeds overleden.\cite{zdf20011028}}
\individual{ferontjeanjacquesg}{Jean-Jacques Feront}{×}{Getuige in de \caseref{dutroux}. De getuige was voor de aanvang van het proces reeds overleden.\cite{zdf20011028}}
\individual{derijckepierrepol}{Pierre-Pol de Rijcke}{×}{Getuige in de \caseref{dutroux}. De getuige was voor de aanvang van het proces reeds overleden.\cite{zdf20011028}}
\individual{claeyssandra}{Sandra Claeys}{×}{Getuige in de \caseref{dutroux}. De getuige was voor de aanvang van het proces reeds overleden.\cite{zdf20011028}}
\individual{pinonvirginie}{Virginie Pinon}{×}{Getuige in de \caseref{dutroux}. De getuige was voor de aanvang van het proces reeds overleden.\cite{zdf20011028}}
\individual{vanessagerard}{G\'erard Vanessa}{×}{Getuige in de \caseref{dutroux}. De getuige was voor de aanvang van het proces reeds overleden.\cite{zdf20011028}}
\individual{deleuzephilippe}{Philippe Deleuze}{×}{Getuige in de \caseref{dutroux}. De getuige was voor de aanvang van het proces reeds overleden.\cite{zdf20011028}}
\caseentry{bendevannijvel}{Bende Van Nijvel}{×}
\groupentry{bendevannijvel}{Bende Van Nijvel}{Officieel heeft de bende 28 mensen vermoord en wordt ze in het bijzonder verdacht van een reeks overvallen op de \institution{Delhaize}'s en \institution{Colruyt}'s van ondermeer \location{Genval}, \location{Halle}, \location{Nijvel}, \location{Beersel}, \location{Eigenbrakel}, \location{Overijse} en \location{Aalst}. Van volgende feiten wordt vermoed dat ze gepleegd werden door de \groupref{bendevannijvel}:\begin{enumerate}\item Diefstal van een \brand{Austin} in \location{Ixelles}, \eventdate{10 september 1982}. Deze wagen werd op \date{11 mei 1982} teruggevonden in \location{Lembeek} op \eventdate{11 mei 1982}.\item Diefstal van een \brand{Santana} in \location{Lembeek}, \eventdate{11 mei 1982}\item \end{enumerate}.\cite{panoramabvn95,vtmlaat19980411}}
\individual{dekaisedaniel}{Daniel Dekaise}{Wapenhandelaar}{Werd op 30 september 1983 overvallen in zijn zaak in \location{Waver}. Vermoedelijk was dit door de \groupref{bendevannijvel}.\cite{panoramabvn95}}
\evententry{19820930Waver}{Waver 30 september 1982}{Op \eventdate{30 september 1982} werd een overval gepleegd op \institution{Wapenhandel Dekaise} in \location{Waver}. De overval wordt toegeschreven aan de \groupref{bendevannijvel}. De eigenaar (\indref{dekaisedaniel}) en een klant raken gewond. Een toegesnelde politieman wordt gedood.\cite{panoramabvn95} Een dag nadien werd een robotfoto op de televisie verspreid.\cite{opsporing19821001}}
\evententry{19830211genval}{Genval 11 februari 1983}{Op \eventdate{11 februari 1983} werd een \institution{Delhaize} in \location{Genval} overvallen. Dit is vermoedelijk het werk van de \groupref{bendevannijvel}. De overval leverde ongeveer 700'000 Belgische franken op.\cite{panoramabvn95}}
\evententry{19830303halle}{Halle 3 maart 1983}{Op die dag werd een \institution{Colruyt} in \location{Halle} overvallen. Dit is vermoedelijk het werk van de \groupref{bendevannijvel}. De filiaalhouder werd vermoordt en \'e\'en werknemer werd verwond.\cite{panoramabvn95}}
\evententry{19830910temse}{Temse 10 september 1983}{Op die dag werden zeven kogelvesten gestolen in \location{Temse}. Het waren prototypes waarvan bijna niemand het bestaan kende. Dit is vermoedelijk het werk van de \groupref{bendevannijvel}. Een nachtwaker werd vermoord, zijn echtgenote zwaar verwond.\cite{panoramabvn95}}
\evententry{19830917nijvel}{Nijvel 17 september 1983}{Op \eventdate{17 september 1983} werd een \institution{Colruyt} in \location{Nijvel} overvallen. Dit is vermoedelijk het werk van de \groupref{bendevannijvel}. De buit bestond uit enkele etenswaren. Er vielen drie doden: \'e\'en rijkswachter en het echtpaar Fourrier-Dewitte, vermoord aangetroffen onder de winkelwagentjes. De vluchtende gangsters namen een achtervolgende politiewagen op bijna militaire wijze in de tang. Hierbij raakte \'e\'en politieman gewond. \'E\'en vluchtauto werd teruggevonden. Daarop werden vingerafdrukken ontdekt, maar deze zijn zoekgeraakt.\cite{panoramabvn95}}
\evententry{19831002ohain}{Ohain 2 oktober 1983}{Op \eventdate{2 oktober 1983} werd \indref{vancampjacques}, de uitbater van het restaurant \institution{Aux 3 Canard}, vermoord in \location{Ohain}. Dit is vermoedelijk het werk van de \groupref{bendevannijvel}. Ook werd zijn wagen gestolen.\cite{panoramabvn95}}
\individual{vancampjacques}{Jacques Van Camp}{ - 2 oktober 1983; restaurantuitbater}{Uitbater van het restaurant \institution{Aux 3 Canard}. Werd op 2 oktober 1983 vermoord. Vermoedelijk door de \groupref{bendevannijvel}.\cite{panoramabvn95}}
\individual{vermaelenfreddy}{Freddy Vermaelen}{- 7 oktober 1983}{Filiaalhouder van de \institution{Delhaize} in \location{Beersel}. Werd vermoord tijdens de overval op dit warenhuis op 7 oktober 1983. De overval was vermoedelijk het werk van de \groupref{bendevannijvel}.\cite{panoramabvn95}}
\evententry{19831007beersel}{Beersel 7 oktober 1983}{Op \eventdate{7 oktober 1983} vond een overval plaats op de \institution{Delhaize} in \location{Beersel}. Dit is vermoedelijk het werk van de \groupref{bendevannijvel}. De filiaalhouder (\indref{vermaelenfreddy}) werd vermoord. De buit bedroeg ongeveer 1'300'000 Belgische franken.\cite{panoramabvn95}}
\commisieentry{bendevannijvel}{Bende Van Nijvel}{Deze commissie werd georganiseerd door de kamer.\cite{panoramabvn95}}
\individual{vanparystony}{Tony Van Parys}{×}{Van Parys was volksvertegenwoordiger en verslaggever van de \commissieref{bendevannijvel}. Van Parys stelt dat het onderzoek vanaf het begin foutliep:\begin{quote}Het is van in den beginne fout gelopen. Men heeft sporen verwaarloosd, men heeft aanknopingspunten niet gezocht. Er zijn zoveel dingen spaak gelopen waardoor alle soorten hypothesen zich stellen, en zich de vraag fundamenteel stelt: waarom heeft men niet alles gezocht wat klaarblijkelijk moest onderzocht worden.\end{quote}Verder stelt Van Parys dat het feit dat de daders nooit gevonden geweest zijn een groot probleem geeft in een democratie als Belgi\"e:\begin{quote}Het feit dat dit mogelijk is geweest in een democratisch land, stelt een enorm probleem: 28 mensen die op een afschuwelijke wijze worden vermoord. Dit stelt enorme problemen. Plus de vaststelling dat dit onderzoek aaneenhangt van tal van incidenten, die bijna niet toevallig kunnen zijn. Het kan bijna niet anders dan dat er andere motieven hebben gespeeld dan het belang van het onderzoek. Het is een aaneenrijging van problemen.\end{quote}\cite{panoramabvn95}}
\individual{lynafrancine}{Francine Lyna}{Onderzoeksrechter}{Lyna volgende het onderzoek van de \caseref{bendevannijvel} met veel belangstelling. Lyna stelt dat het onderzoek op een slechte manier werd gevoerd:\begin{quote}Het onderzoek ging van magistraat naar magistraat, hoorde in \location{Brussel} thuis, maar bleef in \location{Nijvel}, en daar volgende de magistraten elkaar op. Het onderzoek verliep op een manier die niet bijzonder gelukkig was, en niet erg doeltreffend.\end{quote}\cite{panoramabvn95}}
\evententry{19850927eigenbrakel}{Eigenbrakel 27 september 1985}{Op \eventdate{27 september 1985} wordt de \institution{Delhaize} in \location{Eigenbrakel} overvallen. Dit is vermoedelijk het werk van de \groupref{bendevannijvel}. Drie mensen worden vermoord. De buit is ongeveer 200'000 Belgische franken. Een half uur later vindt ook een overval in \location{Overijse plaats}.\cite{panoramabvn95}}
\individual{finneleon}{Leon Finne}{- 27 september 1985; Bankier}{Finne kwam vermoedelijk om bij de overval op een \institution{Delhaize} in \location{Overijse}. Deze overval is vermoedelijk gepleegd door de \groupref{bendevannijvel}.\cite{panoramabvn95}}
\evententry{19850927overijse}{Overijse 27 september 1985}{Op \eventdate{27 september 1985} vindt een overval op een \institution{Delhaize} in \location{Overijse} plaats. Dit is vermoedelijk het werk van de \groupref{bendevannijvel}. Er worden vijf mensen vermoord. \'E\'en van de slachtoffers is de bankier \indref{finneleon}. De buit bedraagt ongeveer 1'000'000 Belgische franken. Een half uur eerder vond een overval op een \institution{Delhaize} in \location{Eigenbrakel} plaats. Vermoedelijk is dit ook het werk van de \groupref{bendevannijvel}.\cite{panoramabvn95}}
\evententry{19850927eigenbrakel}{Eigenbrakel 27 september 1985}{Dit is vermoedelijk het werk van de \groupref{bendevannijvel}.\cite{panoramabvn95}}
\evententry{19851109aalst}{Aalst 9 november 1985}{Op \eventdate{9 november 1985} werd een \institution{Delhaize} in \location{Aalst} overvallen. Dit is vermoedelijk het werk van de \groupref{bendevannijvel}. Er werden acht mensen vermoord en zeven verwond. De buit bedroeg ongeveer 700'000 Belgische franken. \location{Aalst} was de laatste in een reeks bloedige overvallen op \institution{Delhaize} superwarenhuizen. Toch denkt men dat de \groupref{bendevannijvel} ook nog betrokken is bij de moord op \indref{mendezjuan}.\cite{panoramabvn95}}
\individual{mendezjuan}{Juan Mendez}{ - 7 januari 1986; Handelsingenieur}{Mendez was handelsingenieur voor \institution{FN Herstal} en wapenliefhebber. Hij werd op 7 januari 1986 vermoord in zijn wagen. Hij was verantwoordelijk voor wapenleveringen aan Latijns Amerika.\cite{panoramabvn95}}
\evententry{19860107rosieres}{Rosieres 7 januari 1986}{Op \eventdate{7 januari 1986} werd \indref{mendezjuan} vermoord. Dit is vermoedelijk het werk van de \groupref{bendevannijvel}.\cite{panoramabvn95}}
\individual{goljean}{Jean Gol}{×}{Een voormalig minister van justitie. Overleefde een bomaanslag die vermoedelijk het werd georganiseerd door de \groupref{cellullescommunistecombatantes}.\cite{dezaakdejachtopdeccc}}
\groupentry{ecolo}{Ecolo}{×}
\groupentry{cellullescommunistecombatantes}{Cellules Communiste Combatantes (CCC)}{De CCC wordt verdacht en is veroordeeld voor volgende feiten:\begin{enumerate}\item Diefstal van wapens en explosieven uit een Waalse steengroeve en legerkazerne, \eventdate{Voorjaar, 1984}.\item Drie bomaanslagen op bedrijven in \location{Brussel}, \eventdate{begin oktober 1984}.\item Twee bomaanslagen op politieke gebouwen. \indref{decrooherman} en \indref{goljean} overleven de aanslag.\location{Elsene}, \eventdate{15 oktober 1984}.\item Bomaanslag bij het hoofdkantoor van de \groupref{christelijkevolkspartij} in \location{Gent}, \eventdate{17 oktober 1984}.\item Opblazen van de \institution{NAVO} pijpleidingen, \eventdate{11 december 1984}.\end{enumerate}\cite{dezaakdejachtopdeccc}}
\groupentry{frontrevolutionairdactionproletariat}{Front Revolutionair d'Action Proletariat (FRAP)}{De CCC wordt verdacht en is veroordeeld voor volgende feiten:\begin{enumerate}\item \location{Brussel}, \eventdate{20 april 1984}.\item \location{Ukkel}, \eventdate{21 april 1984}.\end{enumerate}\cite{dezaakdejachtopdeccc}}
\individual{decrooherman}{Herman Decroo}{×}{Overleefde een bomaanslag die vermoedelijk het werd georganiseerd door de \groupref{cellullescommunistecombatantes}.\cite{dezaakdejachtopdeccc}}
\groupentry{christelijkevolkspartij}{Christelijke Volkspartij (CVP)}{×}
\individual{carettepierre}{Pierre Carette}{×}{Carette was lid van de \groupref{cellullescommunistecombatantes}.\cite{dezaakdejachtopdeccc}}
\groupentry{frontdelajeunesse}{Front de la Jeunesse}{×}
\groupentry{westlandnewpost}{Westland New Post (WNP)}{×}
\annon{7840}{getuige overval \institution{Delhaize} in \location{Overijse}}{\begin{quote}Ik stond met twee Engelstalige vriendinnen te wachten aan de kassa toen we schoten hoorden. Pas toen de flessen naast mij aan stukken vlogen realiseerde ik mij dat het om een overval ging. Iedereen ging namelijk op de grond liggen. Ze liepen onmiddellijk door naar het kantoor van de zaakvoerder. Toen ze buitenkwamen bleef \'e\'en van de overvallers staan bij onze kassa. Hij wilde het geld en zei tegen de kassierster: ``Ouvre ta casse, o\`u tu va cr\^eve''. Omdat ik bang was dat hij zou schieten, hielp ik haar recht achter haar kassa, maar ze kreeg ze niet open. De overvaller bleef aandringen en zei nogmaals: ``Ouvre ta casse, o\`u tu va cr\^eve''. Hij sprak Frans met een Brussels accent. Al die tijd, zeker meer dan \'e\'en minuut, zagen we vanop de grond hoe de overvaller, heel koelbloedig, bleef wachten op dat geld. Hij droeg een oudemannenmasker, een lange mantel en daaronder legerlaarzen. Zijn wapen droeg hij heel de tijd links, wellicht was hij dus linkshandig.
\end{quote}\cite{terzakegetuigenovervaldelhaizeoverijse}}
\individual{verhaegherik}{Rik Verhaeghe}{Getuige overval \institution{Delhaize} in \location{Aalst}}{\begin{quote}Wanneer er eindelijk politie aanwezig was, was het ongeveer 25 minuten en dan moest je zelf gaan vragen aan de rijkswachter: ``Meneer ik was er bij, kan ik vertellen wat ik tegengekomen ben?''.\end{quote}\cite{terzakegetuigenovervaldelhaizeoverijse}}
\operationentry{mammoet}{Mammoet}{Een operatie uitgevoerd op \eventdate{19 oktober 1984}. De politie voert een grote razzia uit. Er wordt gezocht naar tientallen verdachten en er vinden honderden huiszoekingen plaats. Ook militanten van \groupref{ecolo} krijgen de politie op bezoek. In de \location{Albani\"estraat, Brussel} wordt de drukkerij van \indref{carettepierre} onderzocht. Er wordt allerhande materiaal meegenomen, maar Carette blijft onvindbaar. De operatie levert veel informatie op. Sommigen beweren echter dat de politie de aanslagen van de \groupref{cellullescommunistecombatantes} heeft misbruikt omwille van politieke motieven.\cite{dezaakdejachtopdeccc}}
\individual{brabantjeanmarie}{Jean-Marie Brabant}{\groupref{bijzondereopsporingsbrigade}}{\cite{dezaakdejachtopdeccc}}
\groupentry{bijzondereopsporingsbrigade}{Bijzondere Opsporingsbrigade (B.O.B.)}{\cite{dezaakdejachtopdeccc}}
\individual{dislairelucien}{Lucien Dislaire}{paracommando}{\begin{quote}Ik kom uit het noorden van de provincie Luxemburg. Op dat moment was ik manager van een bank en ex-paracommando. Op een dag kwamen enkele mensen bij mij thuis, en vroegen hulp bij het uitvoeren van enkele speciale manoeuvres in co\"ordinatie met Amerikaanse speciale eenheden. De Belgische commando's werd gevraagd om Amerikaanse paratroepers terug te halen die waren geland in het bos. Na deze operatie moesten ze zich naar vooraf bepaalde plaatsen begeven, en de barakken aanvallen van de rijkswacht.\\Ik had de voorzieningen, de wapens en de radio's bij mij, om het allemaal te co\"ordineren.\\Een paar dagen voorheen waren er problemen: de Amerikanen waren te ver gegaan.\end{quote}\cite{timewatchoperationgladiothefootsoldier}}
\individual{haemerspatrick}{Patrick Haemers}{}{Haemers was getrouwd met \indref{tyackdenise}. Haemers werd opgevoed door zijn overgrootouders waar hij erg verwend werd. Hij was zeer gekend in het \location{Brussel}se nachtleven. Hij had onder meer relaties met \indref{dewitmichele}.\cite{bendehaemersdejeugdjaren,bendehaemersdeeerstedoden,bendehaemersdeontsnapping}}
\individual{haemerskevin}{Kevin Haemers}{}{Haemers is de zoon van \indref{haemerspatrick}. Hij baat een caf\'e uit in \location{Waver}.\cite{bendehaemersdejeugdjaren}}
\individual{haemerslilianne}{Lilianne Haemers}{×}{Haemers is de moeder van \indref{haemerspatrick}. Na de zelfmoord van haar zoon is ze verhuisd naar \location{Perpinion}.\cite{bendehaemersdejeugdjaren}}
\individual{hamersachiel}{Achiel Haemers}{-2001}{Haemers was de vader van \indref{hamerspatrick}. Na de zelfmoord van zijn zoon is hij verhuisd naar \location{Perpinion}. Haemers minimaliseerde altijd de feiten van zijn zoon:\begin{quote}Dat is een jongen, hij kan geen vlieg kwaad doen, hij kan het niet. Het is geen geweldenaar. Hij zit boordevol sentimentele. Hij ziet de wereld mooi in. Hij droomt van een mooie wereld. Enfin het is geen gangster. Maar ze maken er nu al een supergangster van, vijand nummer \'e\'en. Ik kan het niet begrijpen.\end{quote}.\cite{bendehaemersdejeugdjaren}}
\individual{tyackdenise}{Denise Tyack}{}{Tyack is de weduwe van \indref{haemerspatrick}. Ze heeft nu een caf\'e in \location{Waver} waar ze samenwoont met haar zoon \indref{haemerskevin}.\cite{bendehaemersdejeugdjaren}}
\groupentry{hamers}{Bende van Haemers}{De \groupref{haemers} was een criminele bende die tijdens de jaren '80 vermoedelijk 30 tot 50 overvallen pleegde, vooral op geldtransporten. De \groupref{hamers} bestond ondermeer uit \indref{haemerspatrick}, .}
\individual{hauteradaniel}{Daniel Hautera}{rijkswachter Leuven}{Hautera was een rijkswachter die onder meer onderzoek deed naar de \groupref{hamers}. Hij getuigde over \indref{haemerspatrick}:\begin{quote}Patrick Haemers was eigenlijk een rijkeluiszoontje die uit een zeer eerbare familie kwam. Vader Achiel Haemers had een kledingzaak, sportkleding Montex. (...) En dan zijn ze begonnen met het plegen van kleine feiten, zoals bromfietsdiefstallen en autodiefstallen. Een beetje de flikken voor de aap houden door er van weg te rijden zodat ze niet konden opgepakt worden. (...) Eind jaren '70 heeft Patrick Haemers een eerste veroordeling opgelopen, voor een groepsverkrachting in de buurt van de vijvers van \location{Woluwe}.\end{quote}.\cite{bendehaemersdejeugdjaren}}
\individual{dejongherudy}{Rudy De Jonghe}{B.O.B. Leuven}{De Jonghe was een lid van de B.O.B. \location{Leuven} die onder meer onderzoek deed naar de \groupref{haemers}. De Jonghe getuigde over de situatie waarin \indref{haemerspatrick} opgroeide:\begin{quote}Ze hadden toen twintig kledingwinkels (...) in het Brusselse.\end{quote}. Over de groepsverkrachting van Haemers in de jaren '70 stelt De Jonghe dat Haemers onschuldig pleitte.\cite{bendehaemersdejeugdjaren}}
\individual{vanthielenpaul}{Paul Van Thielen}{rijkswacht Brussel}{Van Thielen was lid van de rijkswacht van \location{Brussel} en deed onder meer onderzoek naar de \groupref{haemers}. Over de situatie waarin \indref{haemerspatrick} opgroeide, getuigde hij als volgt:\begin{quote}Ouders die hun kinderen op dat moment wat verwenden: ``Grabbel maar in de pot. Heb je nodig, goed ik heb geen tijd voor jou maar ik heb wel geld voor jouw, en dat is misgegaan. Want die hebben zich een aantal dingen in het uitgaansleven eigen gemaakt als ze nog heel jong waren.\end{quote}\cite{bendehaemersdejeugdjaren}}
\individual{deraedtfernandemotte}{Fernande Motte de Raedt}{advocate}{de Raedt was de advocate van \indref{haemerspatrick}. Over Haemers stelt ze het volgende:\begin{quote}Het is echt zonde. Voor mij heeft hij zijn leven vergooid. Hij was in de wieg gelegd om een gelukkig leven te leiden, zeg maar. En hij heeft dat leven vergooid. Dat is duidelijk.\end{quote}Over de veroordeling van Haemers voor de groepsverkrachting eind jaren '70 zegt de Raedt het volgende:\begin{quote}Ze hadden een prostituee opgezocht en nadien geweigerd te betalen voor haar diensten. Het liep fout en hij werd beschuldigd van groepsverkrachting. Hij is effectief veroordeeld en de rechter heeft hem schuldig bevonden.\end{quote}\cite{bendehaemersdejeugdjaren}}
\individual{dewitmichele}{Mich\`ele Dewit}{×}{Bijgenaamd chouchou. Ze had een relatie met \indref{haemerspatrick}. Ze is de dochter van de peetvader van het \location{Brussel}se misdaadmilieu. Ze woont nu in \location{Perpignan}. Ze is altijd contact blijven houden met familieleden van Haemers.\cite{bendehaemersdejeugdjaren}}
\individual{cloetensbritta}{Britta Cloetens}{}{Britta Cloetens was verdween op \date{23 april 2011} tijdens een bezoek aan de \institution{Honda-garage} in \location{Wilrijk}. Cloetens had een afspraak gemaakt voor een proefrit. De verkoper van de garage verklaarde later dat hij avances had gemaakt tegenover Cloetens, maar dat deze hem had afgewezen. Hierop had de verkoper het kofferdeksel dichtgeklapt en overleefde Cloetens de klap niet. De verkoper stelt dat hij niet meer weet waar hij het lichaam heeft gedumpt. De familie van Cloetens stelt echter dat de verkoper dit enkel zegt omdat op het lichaam tekenen op zwaardere fysieke agresse zouden wijzen. De verkoper is in \date{2009} al veroordeeld voor de aanranding van twee vrouwen. Op \date{16 februari 2014} werd de zoektocht naar haar lichaam stopgezet.}
\individual{hofkenskurt}{Kurt Hofkens}{}{
Kurt Hofkens heeft in \date{september 2012} zijn ex-vriendin \indref{steyaerterna} gewurgd. Op \date{20 februari 2014} werd hij veroordeeld door het \institution{Gentse hof van assisen} tot 27 jaar cel. Alleen zijn ongelukkige jeugd was een verzachtende omstandigheid. Het hof en de jury hielden in de eerste plaats rekening met zijn zwaar strafblad met elf veroordelingen, het vele intrafamiliale geweld en de gruwelijke feiten.\cite{standaard20140220hofkens}}
\individual{steyaerterna}{Erna Steyaert}{}{Slachtoffer van \indref{hofkenskurt}, vermoord in \date{september 2012} door wurging.\cite{standaard20140220hofkens}}