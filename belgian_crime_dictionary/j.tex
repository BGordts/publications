\individual{jenartbrigitte}{Brigitte Jenart}{tandarts}{Was bevriend met \indref{nihoulmichel}. Zij geldt als een belangrijke getuige. Een jaar later is ze dood. Men stelt dat dit zelfdoding is.\cite{zdf20011028}}
\individual{jauparifabienne}{Fabienne Jaupari}{17 juli 1964-15 december 1998}{vrouw van \indref{tagliaferrobruno}. Jaupari stelde dat haar man werd vermoord. Vlak voor zijn dood zou Tagliaferro tegen Jaupari gezegd hebben:\begin{quote}Dat alles voorbij is, dat hij teveel wist en op korte termijn zou sterven.\end{quote}. Een autopsie gaf Jaupari gelijk: Tagliaferro werd vergiftigd. Niet veel later werd Jaupari zelf bedreigd. Ze vroeg tevergeefs om politiebescherming. Ze dacht belangrijke bewijsstukken van haar man te hebben ontdekt. Kort daarna vindt haar zoon haar dood in de slaapkamer. Ze is half verbrand en haar matras is met methanol overgoten. De justitie stelt zelfmoord of een ongeluk. Een vriendin spreekt dit echter tegen:\begin{quote}Weet u, iemand die zelfmoord wil plegen laat niet de wasmachine lopen, zet geen pan met aardappelen en zout op het vuur. (...) Ze (\indref{jauparifabienne}, red.) was geen zwetser, ze wist precies wat ze deed. Ze was ook niet gek, ze wist alleen niet meer wie ze kon vertrouwen.\end{quote}Een ongeval is bovendien onwaarschijnlijk: op de foto's van het plaats delict staat de fles methanol dichtgedraaid op het nachtkastje.\cite{zdf20011028}}
