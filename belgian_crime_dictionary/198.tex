\evententry{19820930Waver}{Waver 30 september 1982}{Op \eventdate{30 september 1982} werd een overval gepleegd op \institution{Wapenhandel Dekaise} in \location{Waver}. De overval wordt toegeschreven aan de \groupref{bendevannijvel}. De eigenaar (\indref{dekaisedaniel}) en een klant raken gewond. Een toegesnelde politieman wordt gedood.\cite{panoramabvn95} Een dag nadien werd een robotfoto op de televisie verspreid.\cite{opsporing19821001}}
\evententry{19830211genval}{Genval 11 februari 1983}{Op \eventdate{11 februari 1983} werd een \institution{Delhaize} in \location{Genval} overvallen. Dit is vermoedelijk het werk van de \groupref{bendevannijvel}. De overval leverde ongeveer 700'000 Belgische franken op.\cite{panoramabvn95}}
\evententry{19830303halle}{Halle 3 maart 1983}{Op die dag werd een \institution{Colruyt} in \location{Halle} overvallen. Dit is vermoedelijk het werk van de \groupref{bendevannijvel}. De filiaalhouder werd vermoordt en \'e\'en werknemer werd verwond.\cite{panoramabvn95}}
\evententry{19830910temse}{Temse 10 september 1983}{Op die dag werden zeven kogelvesten gestolen in \location{Temse}. Het waren prototypes waarvan bijna niemand het bestaan kende. Dit is vermoedelijk het werk van de \groupref{bendevannijvel}. Een nachtwaker werd vermoord, zijn echtgenote zwaar verwond.\cite{panoramabvn95}}
\evententry{19830917nijvel}{Nijvel 17 september 1983}{Op \eventdate{17 september 1983} werd een \institution{Colruyt} in \location{Nijvel} overvallen. Dit is vermoedelijk het werk van de \groupref{bendevannijvel}. De buit bestond uit enkele etenswaren. Er vielen drie doden: \'e\'en rijkswachter en het echtpaar Fourrier-Dewitte, vermoord aangetroffen onder de winkelwagentjes. De vluchtende gangsters namen een achtervolgende politiewagen op bijna militaire wijze in de tang. Hierbij raakte \'e\'en politieman gewond. \'E\'en vluchtauto werd teruggevonden. Daarop werden vingerafdrukken ontdekt, maar deze zijn zoekgeraakt.\cite{panoramabvn95}}
\evententry{19831002ohain}{Ohain 2 oktober 1983}{Op \eventdate{2 oktober 1983} werd \indref{vancampjacques}, de uitbater van het restaurant \institution{Aux 3 Canard}, vermoord in \location{Ohain}. Dit is vermoedelijk het werk van de \groupref{bendevannijvel}. Ook werd zijn wagen gestolen.\cite{panoramabvn95}}
\evententry{19831007beersel}{Beersel 7 oktober 1983}{Op \eventdate{7 oktober 1983} vond een overval plaats op de \institution{Delhaize} in \location{Beersel}. Dit is vermoedelijk het werk van de \groupref{bendevannijvel}. De filiaalhouder (\indref{vermaelenfreddy}) werd vermoord. De buit bedroeg ongeveer 1'300'000 Belgische franken.\cite{panoramabvn95}}
\evententry{19850927eigenbrakel}{Eigenbrakel 27 september 1985}{Op \eventdate{27 september 1985} wordt de \institution{Delhaize} in \location{Eigenbrakel} overvallen. Dit is vermoedelijk het werk van de \groupref{bendevannijvel}. Drie mensen worden vermoord. De buit is ongeveer 200'000 Belgische franken. Een half uur later vindt ook een overval in \location{Overijse plaats}.\cite{panoramabvn95}}
\evententry{19850927overijse}{Overijse 27 september 1985}{Op \eventdate{27 september 1985} vindt een overval op een \institution{Delhaize} in \location{Overijse} plaats. Dit is vermoedelijk het werk van de \groupref{bendevannijvel}. Er worden vijf mensen vermoord. \'E\'en van de slachtoffers is de bankier \indref{finneleon}. De buit bedraagt ongeveer 1'000'000 Belgische franken. Een half uur eerder vond een overval op een \institution{Delhaize} in \location{Eigenbrakel} plaats. Vermoedelijk is dit ook het werk van de \groupref{bendevannijvel}.\cite{panoramabvn95}}
\evententry{19850927eigenbrakel}{Eigenbrakel 27 september 1985}{Dit is vermoedelijk het werk van de \groupref{bendevannijvel}.\cite{panoramabvn95}}
\evententry{19851109aalst}{Aalst 9 november 1985}{Op \eventdate{9 november 1985} werd een \institution{Delhaize} in \location{Aalst} overvallen. Dit is vermoedelijk het werk van de \groupref{bendevannijvel}. Er werden acht mensen vermoord en zeven verwond. De buit bedroeg ongeveer 700'000 Belgische franken. \location{Aalst} was de laatste in een reeks bloedige overvallen op \institution{Delhaize} superwarenhuizen. Toch denkt men dat de \groupref{bendevannijvel} ook nog betrokken is bij de moord op \indref{mendezjuan}.\cite{panoramabvn95}}
\evententry{19860107rosieres}{Rosieres 7 januari 1986}{Op \eventdate{7 januari 1986} werd \indref{mendezjuan} vermoord. Dit is vermoedelijk het werk van de \groupref{bendevannijvel}.\cite{panoramabvn95}}
