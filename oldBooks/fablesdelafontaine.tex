\documentclass{book}
\usepackage[frenchb]{babel}
\title{Fables de la Fontaine}
\date{Paris\\Garnier Fr\`eres, Libraires-\'Editeurs\\6, Rue des Saints-P\`eres -- Palais Royal, 215\\M DCCC LXVIII}
\newenvironment{fable}[1]{\section{#1}\begin{verse}}{\end{verse}}
\begin{document}
\frontmatter
\begin{titlepage}
\maketitle
\end{titlepage}
\chapter*{A Monseigneur le Dauphin}
Monseigneur,
\paragraph{}
Il y a quelque chose d'ing\'enieux dans la r\'epublique des lettres, on peut dire que c'est la mani\`ere dont \'Esope a d\'ebit\'e sa morale. Il seroit v\'eritablement \`a souhaiter que d'autres mains que les miennes y eussent ajout\'e les ornements de la po\'esie, puisque le plus sage des acines a jug\'e qu'ils n'y \'etoient pas inutiles. J'ose, monseigneur, vous en pr\'esenter quelques essais. C'est un entretien convenable \`a vos premi\`eres ann\'ees. Vous \^etes en un \^age o\`u l'amusement et les jeux sont permis aux princes; mais en m\`eme temps vous devez donner quelques-unes de vos pens\'ees \`a des r\'eflexions s\'erieuses. Tout cela se rencontre aux fables que nous devons \`a \'Esope. L'apparence en est pu\'erile, je le confesse; mais ces pu\'erilit\'es servent d'enveloppe \`a v\'erit\'es importantes.
\paragraph{}
Je ne doute point, monseigneur, que vous ne regardiez favorablement des inventions si utiles et tous ensemble si agr\'eables: car que peut-on souhaiter davantage que ces deux points? Ce sont eux qui ont introduit les sciences parmi les hommes. \'Esope a trouv\'e un art singulier de les joindre, l'un avec l'autre. La lecture de son ouvrage r\'epand insensiblement dans un \^ame les semences de la vertu, et lui apprend \`a se conno\^itre sans qu'elle s'aper\c coive de cette \'etude, et tandis qu'elle croit faire toute autre chose. C'est une adresse dont s'est servi tr\`es-heureusement celui sur lequel Sa Majest\'e a jet\'e les yeux pour vous donner des instructions. Il fait en sorte que vous appreniez sans peine, ou pour mieux parler, avec plaisir, tout ce qu'il est n\'ecessaire qu'un prince sache. Nous esp\'erons beaucoup de cette conduite. Mais, \`a dire la v\'erit\'e, il y a des choses dont nous esp\'erons infiniment davantage: ce sont, monseigneur, les qualit\'es que notre invincible monarque vous a donn\'ees avec la naissance; c'est l'exemple que tous les jours il vous donne. Quand vous le voyez former de si grands desseins; quand vous le consid\'erez qui regarde sons s'\'etonner l'agitation de l'Europe et les machines qu'elle remue pour le d\'etourner de son entreprise; quand il pr\'en\`etre d\`es sa premi\`ere d\'emarche jusque dans le c\oe ur d'une province o\`u l'on trouve \`a chaque pas des barri\`eres insurmontables, et qu'il en subjugue une autre en huit jours, pendant la saison la plus ennemie de la guerre, lorsque le repos et les plaisirs r\`egnent dans les cours des autres princes; quand, non content de dompter les hommes, il veut aussi triompher des \'el\'ement; et quand, au retour de cette exp\'edition o\`u il a vaincu comme un Alexandre, vous le voyez gouverner ses peuples comme un Auguste: avouez le vrai, monseigneur, vous soupirez pour la gloire aussi bien que lui, malgr\'e l'impuissance de vos ann\'ees: vous attendez avec impatience le temps o\`u vous pourrez vous d\'eclarer son rival dans l'amour de cette divine ma\^itresse. Vous ne l'attendez pas, monseigneur, vous le pr\'evenez. Je n'en veux pour t\'emoignage que ces nobles inqui\'etudes, cette vivacit\'e, cette ardeur, ces marques d'esprit, de courage et de grandeur d'\^ame, que vous faites paro\^itre \`a tous les moments. Certainement c'est une joie bien sensible \`a notre monarque; mais c'est un spectacle bien agr\'eable pour l'univers, que de voir ainsi cro\^itre une jeune plante qui couvrira un jour de son ombre tant de peuples et de nations.
\paragraph{}
Je devrois m'\'etendre sur ce sujet: mais comme le dessein que j'ai de vous divertir est plus proportionn\'e \`a mes forces que celui de vous louer, je me h\^ate de venir aux fables, et n'ajouterai aux v\'erit\'es que je vous ai dites que celle-ci: c'est, monseigneur, que je suis, avec un z\`ele respectueux,
\paragraph{}
Votre tr\`es-humble, tr\`es-ob\'eissant et tr\`es-fid\`ele serviteur,
De La Fontaine.
\chapter*{Pr\'eface}
'Indulgence que l'on a eue pour quelque-unes de mes fables me donne leu d'esp\'erer la m\^eme gr\`ace pour ce recueil. Ce n'est pas qu'un des ma\^itres de notre \'eloquence n'ait d\'esapprouv\'e le dessein de les mettre en vers: il a cru que leur principal ornement est de n'en avoir aucun, que d'ailleurs la contrainte de la po\'esie, jointe \`a la s\'ev\'erit\'e de notre langue, m'embarrasseroient en beaucoup d'endroits, et banniroient de la plupart de ces r\'ecits la br\`evet\'e, qu'on peut fort bien appeler l'\^ame du conte, puisque sans elle il faut n\'ecessairement qu'il languisse. Cette opinion ne sauroit partir que d'un homme d'excellent go\^ut; je demanderois seulement qu'il en rel\^ach\^at quelque peu, et qu'il cr\^ut que les Gr\^aces lac\'ed\'emoniennes ne sont pas tellement ennemies des Muses fran\c coises, que l'on ne puisse souvent les faire marcher de compagnie.
\paragraph{}
Apres tous je n'ai entrepris la chose que sur l'example, je ne veux pas dire des anciens, qui ne tire point \`a cons\'equence pour moi, mais sur celui des modernes. C'est tous temps, et chez tous les peuples qui font profession po\'esie, que le Parnasse a jug\'e ceci de son apanage. A peine les fables que l'on attribue \`a \'Esope virent le jour, que Socrate trouvera \`a propos de les habiller des livr\'ees des Muses. Ce que Platon en rapporte est si agr\'eable, que je ne puis m'enp\^echer d'en faire un des ornements des cette pr\'eface. Il dit que Socrate \'etant condamn\'e au dernier supplice, l'on remit l'ex\'ecution 
\mainmatter
\begin{fable}{La Cigale et la Fourmi}
La cigale, ayant chant\'e\\
Tout l'\'et\'e,\\
Se trouva fort d\'epourvu\\
Quand la bise fut venue:\\
Pas un seul petit morceau\\
De mouche ou de vermisseau.\\
Elle alla crier famine\\
Chez la fourmi sa voisine,\\
La priant de lui pr\^eter\\

Quelque grain pour subsister\\
Jusqu'\`a la saison nouvelle:\\
Je vous pa\^irai, lui dit-elle,\\
Avant l'o\^ut, foi d'animal,\\
Int\'er\^et et principal.\\
La fourmi n'est pas pr\^eteuse;\\
C'est l\`a son moindre d\'efaut:\\
Que faisiez-vous au temps chaud?\\
Dit-elle \`a cette emprunteuse. --\\
Nuit et jour \`a tout venant\\
Je chantois, ne vous d\'eplaise. --\\
Vous chantiez! J'en suis fort aise.\\
H\'e bien! dansez maintenant.
\end{fable}

\backmatter
\chapter*{Table des Fables}
\end{document}